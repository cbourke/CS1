%!TEX root = ComputerScienceOne.tex

%%Chapter: Loops in Java

Java supports while loops, for loops, and do-while loops using the keywords
\mintinline{java}{while}, \mintinline{java}{for}, and \mintinline{java}{do} (along with 
another \mintinline{java}{while}).  Continuation conditions for loops are 
enclosed in parentheses, \mintinline{java}{(...)} and the blocks of code
associated with the loop are enclosed in curly brackets.  

\section{While Loops}

Code Sample \ref{code:java:whileLoop} contains an example of a basic
while loop in C.  Just as with conditional statements, our code styling
places the opening curly bracket on the same line as the \mintinline{java}{while}
keyword and continuation condition.  The inner block of code is also
indented and all lines in the block are indented to the same level.

\begin{listing}
\begin{minted}{java}
int i = 1; //Initialization
while(i <= 10) {  //continuation condition
  //perform some action
  i++; //iteration
}
\end{minted}
  \caption{While Loop in Java}
  \label{code:java:whileLoop}
\end{listing}

In addition, the continuation condition does \emph{not} contain 
a semicolon since it is not an executable statement.  Just as with
an if-statement, if we \emph{had} placed a semicolon it would 
have led to unintended results.  Consider the following:

\begin{minted}{java}
while(i <= 10); {
  //perform some action
  i++; //iteration
}
\end{minted}

A similar problem occurs: the \mintinline{java}{while} keyword and
continuation condition bind to the next executable statement or
code block.  As a consequence of the semicolon, the executable
statement that gets bound to the while loop is \emph{empty}.  What
happens is even worse: the program will enter an infinite loop.  To
see this, the code is essentially equivalent to the following:

\begin{minted}{java}
while(i <= 10) {
}
{
  //perform some action
  i++; //iteration
}
\end{minted}

In the while loop, we never increment the counter variable \mintinline{java}{i}, 
the loop does nothing, and so the computation will continue on
forever!  Some compilers will warn you about this, others will not.  It
is valid Java and it will compile and run, but obviously won't work as intended.
Avoid this problem by using proper syntax.

Another common use case for a while loop is a flag-controlled loop in which
we use a Boolean flag rather than an expression to determine if a loop
should continue or not.  An example
can be found in Code Sample \ref{code:java:flagControlledWhileLoop}.

\begin{listing}
\begin{minted}{java}
int i = 1;
boolean flag = true;
while(flag) {
  //perform some action
  i++; //iteration  
  if(i>10) {
    flag = false;
  }
}
\end{minted}
  \caption{Flag-controlled While Loop in Java}
  \label{code:java:flagControlledWhileLoop}
\end{listing}

\section{For Loops}

For loops in Java use the familiar syntax of placing the initialization, continuation
condition, and iteration on the same line as the keyword \mintinline{java}{for}.
An example can be found in Code Sample \ref{code:java:forLoop}.

\begin{listing}[H]
\begin{minted}{java}
for(int i=1; i<=10; i++) {
  //perform some action
}
\end{minted}
  \caption{For Loop in Java}
  \label{code:java:forLoop}
\end{listing}

Again, note the syntax: semicolons are placed at the end of the initialization and
continuation condition, but \emph{not} the iteration statement.  Just as with while
loops, the opening curly bracket is placed on the same line as the \mintinline{java}{for}
keyword.  Code within the loop body is indented, all at the same indentation level.

Another observation: the declaration of the counter variable \mintinline{java}{i} 
was done in the initialization statement.  This scopes the variable to the loop itself.
The variable \mintinline{java}{i} is valid inside the loop body, but will be out-of-scope
\emph{after} the loop body.  It is possible to declare the variable prior to the loop, but
the variable \mintinline{java}{i} would have a much larger scope.  It is best practice 
to limit the scope of variables only to where they are needed.  Thus, we will
write our loops as above.

\section{Do-While Loops}

Finally, Java does support do-while loops.  Recall that the difference between a
while loop and a do-while loop is when the continuation condition is checked.
For a while loop it is \emph{prior} to the beginning of the loop body and in
a do-while loop it is at the \emph{end} of the loop.  This means that a do-while 
always executes \emph{at least once}.  An example can be found in Code
Sample \ref{code:java:doWhileLoop}.

\begin{listing}
\begin{minted}{java}
int i;
do {
  //perform some action
  i++;
} while(i <= 10);
\end{minted}
  \caption{Do-While Loop in Java}
  \label{code:java:doWhileLoop}
\end{listing}

Note the syntax and style: the opening curly bracket is again on the same
line as the keyword \mintinline{java}{do}.  The \mintinline{java}{while} keyword and
continuation condition are on the same line as the closing curly bracket.
In a slight departure from consistent syntax, a semicolon \emph{does} appear
at the end of the continuation condition even though it is not an
executable statement.

\section{Enhanced For Loops}

Java also supports foreach loops (which were introduced in JDK 1.5.0) which
Java refers to as ``Enhanced For Loops''.  Foreach loops allow you to 
iterate over each element in a collection without having to define an index
variable or otherwise ``get'' each element.  We'll revisit these concepts
in detail in Chapter \ref{chapter:java:arrays}, but let's take a look 
at a couple of examples.

An enhanced for loop in Java still uses the keyword \mintinline{java}{for} 
but uses different syntax for its control.  The example in Code Sample 
\ref{code:java:enhancedForLoopExample1} illustrates this syntax:
\mintinline{java}{(int a : arr)}.  The last element of this syntax is
a reference to the collection that we want to iterate over.  The first
part is the \emph{type} and local reference variable that the loop
will use.  

\begin{listing}[H]
\begin{minted}{java}
int arr[] = {10, 20, 8, 42};
int sum = 0;
for(int a : arr) {
  sum += a;
}
\end{minted}
  \caption{Enhanced For Loops in Java Example 1}
  \label{code:java:enhancedForLoopExample1}
\end{listing}

The code \mintinline{java}{(int a : arr)} should be read as ``for each
integer element \mintinline{java}{a} in the collection \mintinline{java}{arr}...''
Within the enhanced for loop, the variable \mintinline{java}{a} will
be automatically updated for you on each iteration.  Outside the
loop body, the variable \mintinline{java}{a} is out-of-scope.  

Java allows you to use an enhanced for loop with any array or collection
(technically, anything that implements the \mintinline{java}{Iterable} 
interface).  One example is a \mintinline{java}{List}, an ordered
collection of elements.  Code Sample \ref{code:java:enhancedForLoopExample2} 
contains an example.

\begin{listing}[H]
\begin{minted}{java}
List<Integer> list = Arrays.asList(10, 20, 8, 42);
int sum = 0;
for(Integer a : list) {
  sum += a;
}
\end{minted}
  \caption{Enhanced For Loops in Java Example 2}
  \label{code:java:enhancedForLoopExample2}
\end{listing}

\section{Examples}

\subsection{Normalizing a Number}

Let's revisit the example from Section \ref{subsection:whileLoopExample} in which 
we \emph{normalize} a number by continually dividing it by 10 until it is less 
than 10.  The code in Code Sample \ref{code:java:normalizeWhileLoop} specifically
refers to the value $32145.234$ but would work equally well with any value of 
\mintinline{java}{x}.

\begin{listing}[H]
\begin{minted}{java}
double x = 32145.234;
int k = 0;
while(x > 10) {
  x = x / 10; //or: x /= 10;
  k++;
}
\end{minted}
  \caption{Normalizing a Number with a While Loop in Java}
  \label{code:java:normalizeWhileLoop}
\end{listing}

\subsection{Summation}

Let's revisit the example from Section \ref{subsection:summationExample} in which
we computed the sum of integers $1 + 2 + \cdots + 10$.  The code is presented in
Code Sample \ref{code:java:summationForLoop}

\begin{listing}[H]
\begin{minted}{java}
int i;
int sum = 0;
for(i=1; i<=10; i++) {
  sum += i;
}
\end{minted}
  \caption{Summation of Numbers using a For Loop in Java}
  \label{code:java:summationForLoop}
\end{listing}

Of course we could easily have generalized the code somewhat.  Instead of computing
a sum up to a particular number, we could have written it to sum up to another
variable \mintinline{java}{n}, in which case the for loop would instead look like the
following.

\begin{minted}{java}
for(i=1; i<=n; i++) {
  sum += i;
}
\end{minted}

\subsection{Nested Loops}

Recall that you can write loops within loops.  The inner loop will execute fully 
\emph{for each} iteration of the outer loop.  An example of two nested of
loops in Java can be found in Code Sample \ref{code:java:nestedForLoops}.

\begin{listing}[H]
\begin{minted}{java}
int i, j;
int n = 10;
int m = 20;
for(i=0; i<n; i++) {
  for(j=0; j<m; j++) {
    System.out.printf("(i, j) = (%d, %d)\n", i, j);
  }
}
\end{minted}
  \caption{Nested For Loops in Java}
  \label{code:java:nestedForLoops}
\end{listing}

The inner loop execute for $j = 0, 1, 2, \ldots, 19 < m = 20$ for a total
of 20 times.  However, it executes 20 times \emph{for each} iteration of
the outer loop.  Since the outer loop execute for $i = 0, 1, 2, \ldots, 9 < n = 10$, 
the total number of times the \mintinline{java}{System.out.printf} statement execute is
$10 \times 20 = 200$.  In this example, the sequence $(0, 0), (0, 1), (0, 2), \ldots, (0,19), (1, 0), \ldots, (9, 19)$
will be printed.

\subsection{Paying the Piper}

Let's adapt the solution for the loan amortization schedule we developed in 
Section \ref{subsection:loanAmortization}.  First, we'll read the principle, 
terms, and interest as command line inputs.

Adapting the formula for the monthly payment and using the
math library's \mintinline{java}{Math.pow()} function, we get

\begin{minted}{java}
double monthlyPayment = (monthlyInterestRate * principle) / 
  (1 - pow( (1 + monthlyInterestRate), -n));
\end{minted}

However, recall that we may have problems due to accuracy.  The monthly
payment could come out to be a fraction of a cent, say \$43.871.  For 
accuracy, we need to ensure that all of the figures for currency are rounded
to the nearest cent.  The standard math library does have a \mintinline{java}{Math.round()}
function, but it only rounds to the nearest whole number, not the nearest
100th.

However, we can \emph{adapt} the ``off-the-shelf'' solution to fit our needs.  
If we take the number, multiply it by 100, we get (say) 4387.1 which we can
now round to the nearest whole number, giving us 4387.  We can then 
divide by 100 to get a number that has been rounded to the nearest 100th!
In Java, we could simply do the following.

\mintinline{java}{monthlyPayment = Math.round(monthlyPayment * 100.0) / 100.0;}

We can use the same trick to round the monthly interest payment and any
other number expected to be whole cents.  To output our numbers, we use
\mintinline{java}{System.out.printf} and take care to align our columns to make make it look 
nice.  To finish our adaptation, we handle the final month separately to account
for an over/under payment due to rounding.  The full solution can be found
in Code Sample \ref{code:java:loanAmortization}.

\begin{listing}
\begin{minted}[fontsize=\footnotesize]{c}
public class LoanAmortization {

  public static void main(String args[]) {

    if(args.length != 4) {
      System.err.println("Usage: principle apr terms");
      System.exit(1);
    }

    double principle = Double.parseDouble(args[0]);
    double apr = Double.parseDouble(args[1]);
    int n = Integer.parseInt(args[2]);
    
    double balance = principle;
    double monthlyInterestRate = apr / 12.0;

    //monthly payment  
    double monthlyPayment = (monthlyInterestRate * principle) / 
      (1 - Math.pow( (1 + monthlyInterestRate), -n));
    //round to the nearest cent
    monthlyPayment = Math.round(monthlyPayment * 100.0) / 100.0;

    System.out.printf("Principle: $%.2f\n", principle);
    System.out.printf("APR: %.4f%%\n", apr*100.0);
    System.out.printf("Months: %d\n", n);
    System.out.printf("Monthly Payment: $%.2f\n", monthlyPayment);

    //for the first n-1 payments in a loop:
    for(int i=1; i<n; i++) {  
      //  compute the monthly interest, rounded:
      double monthlyInterest = 
        Math.round( (balance * monthlyInterestRate) * 100.0) / 100.0;
      //  compute the monthly principle payment
      double monthlyPrinciplePayment = monthlyPayment - monthlyInterest;
      //  update the balance
      balance = balance - monthlyPrinciplePayment;
      //  print i, monthly interest, monthly principle, new balance
      System.out.printf("%d\t$%10.2f  $%10.2f  $%10.2f\n", i, monthlyInterest, 
        monthlyPrinciplePayment, balance);
    }

    //handle the last month and last payment separately
    double lastInterest = Math.round( 
    	(balance * monthlyInterestRate) * 100.0) / 100.0;
    double lastPayment = balance + lastInterest;

    System.out.printf("Last payment = $%.2f\n", lastPayment);

  }

}
\end{minted}
\caption{Loan Amortization Program in Java}
\label{code:java:loanAmortization}
\end{listing}



