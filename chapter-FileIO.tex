%!TEX root = ComputerScienceOne.tex

%%Chapter: File IO

A \gls{file} is a block of data used for storing information.
Normally, we think of a file as something that is stored on 
a hard drive (or memory stick or other disk media), but the
concept of a file is much more general.  For example, when
a file is loaded (``read'') by a program it then exists in 
main memory.  An executable program itself is a file 
(containing instructions to be executed), both stored on 
the hard drive and run in memory.

In a typical unix-based system, \emph{everything} is a
file.  Directories are files, executables are files, 
running processes are files, etc.  Even the familiar 
standard input and standard output are buffers that
are treated as files that can be read from or written to.

Files may be stored as binary data or as plaintext files.
Granted, plaintext files are still stored as binary data, 
but are stored as an encoding using the \gls{asciiLabel}
text values.  Binary files will also have structure, but
it depends on the application that produced the file to
give meaning to the data.  For example, an image file
in a \gls{jpegLabel} format is essentially just binary
data but it has a very specific format that an
image viewer would be able to process, but, say, a text
editor would not.  Further, if the binary format is
corrupted, even the image viewer might not be able to
display the image correctly

Files provide \gls{persistence} of data.  Typical programs
are short lived, anywhere from a few milliseconds to maybe
a user ``session.''  However, we often want data to be
saved across multiple runs we need to save it or \emph{persist}
it in some durable storage medium (disk).  

\section{Processing Files}

In general, processing data in a file involves three basic
steps:
\begin{enumerate}
  \item Open (or create) the file
  \item Read from (or write to) the file
  \item Close the file
\end{enumerate}

Depending on the language, the act of opening a file may
determine if it will be read from or written to.  When
read from, the file is referred to as an \emph{input file}
while a file  that is written to is an \emph{output file}.
Languages may also have different a different \gls{apiLabel}
or functions to read/write a file.

A file may be processed line by line until the end of the
end of the file has been reached.  Languages usually support
this by a special \gls{eofLabel} flag or value to indicate
the end of a file has been reached.

We need to properly manage a resource like a file just as
we would with, say, memory and properly close it once we 
are finished processing it.  Depending on the language
and other factors such as the operating system, 
failure to close a file may result in corrupted data.  
Though a file may be closed automatically for us when
the program terminates, its still best practice to 
properly close it.

\subsection{Paths}

When opening a file on a file system, it is necessary
to specify \emph{which} file you want to open.  This
is typically done by specifying at least the name of
the file.  Often files will have ``extensions'' which
indicate the type of file it is such as 
\mintinline{text}{.txt} for text files or \mintinline{text}{.html}
for \gls{htmlLabel} files.  However, in general file
extensions are only for organizational purposes and
have no real bearing on what data is stored in the file.

More important is the \emph{path} of the file.  Usually,
if no path is specified, then implicitly we are opening
the file in the \gls{cwdLabel}.
For example, if we open the file \mintinline{text}{data.txt}
then we are opening the file in the same directory in
which our program is executing.  

When specifying a path we can either specify an \emph{absolute}
path or a \emph{relative} path.  An absolute path specifies
each and every subdirectory in the file system from the
\emph{root} to the directory that the file is located in.
The root directory is the top-most directory in the file system.
Each subdirectory is separated by some \emph{delimiter}.

Windows systems usually use a backslash as a directory 
delimiter while the root directory is specified using a 
``volume'' name such as \mintinline{text}{C:\}.  For example, 
an absolute path on a Windows system may look something like:

\mintinline{text}{C:\applications\users\data\data.txt}

On a Unix-based system, a forward slash is used as
a directory delimiter and the root directory is simply a
single forward slash.  The same directory structure in
a Unix-based system would look like the following.

\mintinline{text}{/applications/users/data/data.txt}

A path may also be relative to the current working 
directory.  In most systems (Windows and Unix-based)
the current directory is denoted using a single period,
\mintinline{text}{.}.  You can use this to specify 
directories deeper in the directory tree from the current
directory.  For example (in Unix), 

\mintinline{text}{./app/data/data.txt}

would refer to the directory \mintinline{text}{app} in
the current working directory, the directory \mintinline{text}{data}
within that, and finally the file \mintinline{text}{data.txt}
within that directory.

We can also refer to files further up the directory 
tree using the ``parent'' directory symbol which is 
two periods, \mintinline{text}{..}.  For example, 

\mintinline{text}{../../system/data.txt}

would refer to a file two levels up in the subdirectory
\mintinline{text}{system}.

\begin{figure}
\centering
\begin{minted}{text}
/proc/self/
|-- attr
|-- cwd -> /proc
|-- fd
|   `-- 3 -> /proc/15589/fd
|-- fdinfo
|-- net
|   |-- dev_snmp6
|   |-- netfilter
|   |-- rpc
|   |   |-- auth.rpcsec.context
|   |   |-- auth.rpcsec.init
|   |   |-- auth.unix.gid
|   |   |-- auth.unix.ip
|   |   |-- nfs4.idtoname
|   |   |-- nfs4.nametoid
|   |   |-- nfsd.export
|   |   `-- nfsd.fh
|   `-- stat
|-- root -> /
`-- task
    `-- 15589
        |-- attr
        |-- cwd -> /proc
        |-- fd
        | `-- 3 -> /proc/15589/task/15589/fd
        |-- fdinfo
        `-- root -> /
\end{minted}
\caption[Linux Tree Directory Structure]{The Linux file system (like most
file systems), defines a tree directory structure.  Each file and directory
is contained in subdirectories all contained within the root directory, 
\mintinline{text}{/}.  This diagram was generated by the Tree command, 
\url{http://mama.indstate.edu/users/ice/tree/}.}
\end{figure}

\subsection{Error Handling}

When dealing with files there are many potential error
conditions that may be anticipated and may need to be
dealt with.  Some types of errors that can occur include:

\begin{itemize}
  \item Permission errors -- Your program does not have
  	the proper permissions to read or write to a particular
	directory or file.  For example, user-level processes
	typically do not have permissions to read system-level
	files (such as password files) nor can they write to them
	lest they corrupt critical system data.  
  \item File Not Found errors -- Your program may attempt
  	to open a file that does not exist.  Sometimes, particularly
	when writing, the file will be created if it does not 
	exist.  However, for reading, this may be a more
	serious error.
  \item I/O Errors -- We may be able to successfully find
  	and open a file, but during processing something might
	go wrong with the file system that results in a general
	input/output error.
  \item Formatting errors -- As previously mentioned, the
  	format of a file is highly dependent on the application
	that created it (though there are universal data formats
	such as \gls{xmlLabel} or \gls{jsonLabel}).  If the
	format is not as expected the file may be corrupted and
	the program may not be able to successfully read 
	data from it.
\end{itemize}

Depending on the language, these errors may be fatal or may
result in error codes or error values that can be dealt with.
Or they may result in exceptions that can be caught and handled.

\subsection{Buffered and Unbuffered}

When processing files the input/output may be either 
\emph{buffered} or \emph{unbuffered}.  A buffered input or
output ``stream'' is one in which data that is read/written
is actually stored in memory in a ``buffer'' until such a 
time as the buffer is ``flushed'' and the accumulated data
is passed to/from the actual file.  

For example, in a buffered output file, our program could
write several kilobytes of data to the output file, but it
might not actually be written to the file right away.  Instead,
those kilobytes of data are stored in memory until the
buffer fills up or some other event takes place to cause the
buffer to be flushed.  At that point, the data stored in the
buffer is emptied and written to the file.

Buffered input/output is used because, in general, I/O operations
are expensive in terms of system resources and can slow the system
down.  Because of this, it is better to keep I/O operations
as infrequent as possible.  Buffers help to reduce the number
of I/O operations performed by a program.  

There are some instances in which we want unbuffered I/O.
When error messages are written to the standard error output
for example, we would prefer to know about errors as soon 
as possible rather than waiting for error messages to accumulate
in a buffer.  Using an unbuffered output means that data
is written to the standard error (which \emph{is} a file) 
immediately.  However, because errors are (hopefully) infrequent
and (likely) fatal, this is not a general performance issue.

\subsection{Binary vs Text Files}

As previously mentioned, files can be stored as pure binary
data or as plaintext (\gls{asciiLabel}).  Depending on our
application and the nature of the data being written to files, 
the choice of which to use may be clear.  If we want the
data in our files to be human-readable, then we need
to store them as plaintext.  

However, in general, we should prefer storing data in a 
binary format.  The reason for this is that binary generally
requires less space and is more efficient to process.

Consider as an example, storing a collection of integers
in a file.  Each integer requires 4 bytes when represented
in binary.  However, when represented as a string, it
requires as many bytes as there are digits in the number.
For example, the maximum representable value of such an
integer would be $2^{31}-1 = 2,147,483,647$, requiring
10 ASCII characters and thus 10 bytes to represent, two
and a half times as much memory as with the binary 
representation.  Further, if a lot of numbers are stored, 
each number (as a string) would need to be delimited by
yet another character.  With a binary representation, 
no delimiter would be necessary as we would know that 
each 4 byte block represents a single number.

This disparity can even more stark with other types 
such as floating-point numbers.  There are additional
performance issues when reading/writing the data and
converting binary numbers to their string representations.
With binary data no such parsing is necessary. As long 
as the data does not need to be human-readable, binary
formats should be used.

