%!TEX root = ComputerScienceOne.tex

%%Chapter: Recursion in Java
\index{recursion!in Java}
\index{Java!recursion}

Java supports recursion with no special syntax necessary.  
However, as an object-oriented language, recursion is generally 
expensive and iterative or
other non-recursive solutions are generally preferred.  We present
a few examples to demonstrate how to write recursive methods in Java.
The first example of a recursive method we gave was the toy count down
example.  In Java it could be implemented as follows.

\begin{minted}{java}
public static void countDown(int n) {
  if(n == 0) {
    System.out.println("Happy New Year!");
  } else {
    System.out.println(n);
    countDown(n-1);
  }
}
\end{minted}

As another example that actually does something useful, consider the
following recursive summation method that takes an array 
and an index variable.  The recursion works as follows: if the index
variable has reached the size of the array, it stops and returns zero
(the base case).  Otherwise, it makes a recursive call to 
\mintinline{java}{recSum()}, incrementing the index variable by 1.  When
the method returns, it adds its result to the $i$-th element
in the array.  To invoke this method we would call it with an initial
value of 0 for the index variable: \mintinline{java}{recSum(arr, 0)}.

\begin{minted}{java}
public static int recSum(int arr[], int i) {
  if(i == arr.length) {
    return 0;
  } else {
    return recSum(arr, i+1) + arr[i];
  }
}
\end{minted}

This example was not tail-recursive as the recursive call was not the
final operation (the sum was the final operation).  To make this method
tail recursive, we can carry the summation through to each method call
ensuring that the summation is done prior to the recursive method call.

\begin{minted}{java}
public static int recSumTail(int arr[], int i, int sum) {
  if(i == arr.length) {
    return sum;
  } else {
    return recSumTail(arr, i+1, sum + arr[i]);
  }
}
\end{minted}

As another example, consider the following Java implementation of the 
naive recursive Fibonacci sequence.  An additional condition has been
included to check for ``invalid'' negative values of $n$ for which
an exception is thrown.

\begin{minted}{java}
public static int fibonacci(int n) {
  if(n < 0) {
    throw new IllegalArgumentException("Undefined for n < 0");
  } else if(n <= 1) {
    return 1;
  } else {
    return fibonacci(n-1) + fibonacci(n-2);
  }
}
\end{minted}

Java is not a language that provides implicit memoization.  Instead, 
we need to explicitly keep track of values using a table.  In the 
following example, we use a \mintinline{java}{Map} data structure
to store previously computed values.

\begin{minted}{java}
public static int fibMemoization(int n, Map<Integer, Integer> m) {
  if(n < 0) {
    throw new IllegalArgumentException("Undefined for n < 0");
  } else if(n <= 1) {
    return 1;
  } else {
    Integer result = m.get(n);
    if(result == null) {
      Integer a = fibMemoization(n-1, m);
      Integer b = fibMemoization(n-2, m);
      result = a + b;
      m.put(n, result);
    }
    return result;
  }
}
\end{minted}

Java provides an arbitrary precision data type, \mintinline{java}{BigInteger}
that can be used to compute arbitrarily large integer values.  Since
Fibonacci numbers grow exponentially, using an \mintinline{java}{int} we could
only represent up to to $F_{45}$.  Using \mintinline{java}{BigInteger} we
can support much larger values.  An example:

\begin{minted}{java}
public static BigInteger fibMem(int n, Map<Integer, BigInteger> m) {
  if(n < 0) {
    throw new IllegalArgumentException("Undefined for n < 0");
  } else if(n <= 1) {
    return BigInteger.ONE;
  } else {
    BigInteger result = m.get(n);
    if(result == null) {
      BigInteger a = fibMem(n-1, m);
      BigInteger b = fibMem(n-2, m);
      result = a.add(b);
      m.put(n, result);
    }
    return result;
  }
}  
\end{minted}




