%!TEX root = ComputerScienceOne.tex

%%Chapter: Arrays in PHP

PHP allows you to use arrays, but PHP arrays are actually
\emph{associative arrays}.  Though you can treat them
as regular arrays and use contiguous integer indices, they
are more flexible.  You can also use strings as indices for
example.  In addition, since PHP is dynamically typed, PHP 
arrays allow mixed types.  An array has no fixed type and 
you can place different mixed types into the same array.
Moreover, PHP arrays are dynamic, so there is no memory
management or allocation/deallocation of memory space.
Arrays will grow and shrink automatically as you add and 
remove elements.

\section{Creating Arrays}

Since PHP is dynamically typed, you do not need to 
declare an array or specify a particular type
of variable that it holds.  However, there are several
ways that you can initialize an array.

To create an empty array, you can call the \mintinline{java}{array()}
function.  Optionally, you can provide an initial list of elements
to insert into the array by providing the list of elements as 
arguments to the function.

\begin{minted}{php}
//create an empty array:
$arr = array();

//create an array with elements 10, 20, 30:
$arr = array(10, 20, 30);

//create an array with mixed types:
$arr = array(10, 3.14, "Hello");
\end{minted}

\section{Indexing}

By default, when inserting elements, 0-indexing is used.
In the three examples above, each element would be
located at indices 0, 1, and 2 respectively.  The usual
square bracket syntax can be used to access and 
assign elements.

\begin{minted}{php}
//create an array with elements 10, 20, 30:
$arr = array(10, 20, 30);

//get the first element:
$x = $arr[0]; //x has value 10

//change the 3rd element to 5:
$arr[2] = 5;

//print the 2nd element:
printf("$arr[1] = %d\n", $arr[1]);
\end{minted}

Attempting to access an element at an invalid index does
not actually result in an error or an exception (though a
\emph{warning} may be issued depending on how PHP is
setup).  Instead, if you attempt to access an invalid element, 
it will be treated as a \mintinline{php}{null} value.  

\begin{minted}{php}
$arr = array(10, 20, 30);
$x = $arr[10]; //x is null
\end{minted}

However, an array could have \mintinline{php}{null} values
in it as elements.  How do we distinguish whether or not an
accessed value was \emph{actually} \mintinline{php}{null} or if
it is not part of the array?  PHP provides a function, 
\mintinline{php}{array_key_exists()} to distinguish between 
these two cases.  It returns \mintinline{php}{true} 
or \mintinline{php}{false} depending on whether or not a 
particular index has been set or not.

\begin{minted}{php}
$arr = array(10, null, 30);
if(array_key_exists(1, $arr)) {
  print "index 1 contains an element\n";
}
if(!array_key_exists(10, $arr)) {
  print "index 10 does not contain an element\n";
}

if($arr[1] === $arr[10]) {
  print "but they are both null\n";
}
\end{minted}

\subsection{Strings as Indices}

Since arrays in PHP are associative arrays, keys are not
limited to integers.  You can also use strings as keys to
index elements.  

\begin{minted}{php}
$arr = array();
$arr[0] = 5;
$arr['foo'] = 10;
$arr['hello'] = 'world';

print "value = " . $arr["hello"];
\end{minted}

Note that strings that contain integer values will be 
type-juggled into their numeric values.  For example, 
\mintinline{php}{$arr["10"] = 3;} will be equivalent to
\mintinline{php}{$arr[10] = 3;}.  However, strings containing
floating-point values will not be coerced but will remain
as strings, \mintinline{php}{$arr["3.15"] = 7;} for example.

\subsection{Non-Contiguous Indices}

Another consequence of PHP arrays being associative
arrays is that indices need not be contiguous.  For example,

\begin{minted}{php}
$arr = array();
$arr[0] = 10;
$arr[5] = 20;
\end{minted}

The values at indices 1 through 4 are undefined and the
array contains some ``holes'' in its indices.  

\subsection{Key-Value Initialization}

Since associative arrays in PHP can be indexed by
either integers or strings and need not be ordered or
contiguous, we can use a special key-value initialization
syntax to define not only the values, but the keys that
map to those values when we initialize an array.  The
``double arrow'', \mintinline{php}{=>} is used to denote
the mapping.

\begin{minted}{php}
$arr = array(
  "foo" => 5,
  4 => "bar",
  0 => 3.14,
  "baz" => "ten"
);
\end{minted}

\section{Useful Functions}

There are dozens of useful functions that PHP defines that can
be used with arrays.  We'll only highlight a few of the more
useful ones. 

First, the \mintinline{php}{count()} function can be used to
compute how many elements are stored in the array.

\begin{minted}{php}
$arr = array(10, 20, 30);
$n = count($arr); //n is 3
\end{minted}

For convenience and debugging, a special function, 
\mintinline{php}{print_r()} allows you to print the
contents of an array in a human-readable format 
that resembles the key-value initialization syntax
above.  For example, 

\begin{minted}{php}
$arr = array(
  "foo" => 5,
  4 => "bar",
  0 => 3.14,
  "baz" => "ten"
);
print_r($arr);  
\end{minted}

would end up printing

\begin{minted}[fontsize=\scriptsize]{text}
Array
(
    [foo] => 5
    [4] => bar
    [0] => 3.14
    [baz] => ten
)
\end{minted}

Two other functions, \mintinline{php}{array_keys()} and 
\mintinline{php}{array_values()} return new zero-indexed
arrays containing the keys and the values of an array 
respectively.  Reusing the example above, 

\begin{minted}{php}
$keys = array_keys($arr);
$vals = array_values($arr);
print_r($keys);
print_r($vals);
\end{minted}

would print

\begin{minted}[fontsize=\scriptsize]{text}
Array
(
    [0] => foo
    [1] => 4
    [2] => 0
    [3] => baz
)
Array
(
    [0] => 5
    [1] => bar
    [2] => 3.14
    [3] => ten
)
\end{minted}

%SPLICE
%Another useful function is the \mintinline{php}{splice()}
%function that allows you to remove or replace arbitrary
%elements or ranges of elements in an array.

Finally, you can use the equality operators, 
\mintinline{php}{==} and \mintinline{php}{===}
to compare arrays.  The first is the loose equality
operator and evaluates to true if the two compared
arrays have the same key-value pairs while the
second is the strict equality operator and is true
only if the arrays have the same key/value pairs in
the same order and are of the same type.

\section{Iteration}

If we have an array in PHP that we \emph{know} is
0-indexed and all elements are contiguous, we can 
use a normal for-loop to iterate over its elements
by incrementing an index variable.

\begin{minted}{php}
for($i=0; $i<count($arr); $i++) {
  print $arr[$i] . "\n";
}
\end{minted}

This fails, however, when we have an associative array
that has a mix of integer and string keys or ``holes'' in
the indexing of integer keys.  For this reason, it is more
reliable to use \mintinline{php}{foreach} loops.  There
are several ways that we can use a \mintinline{php}{foreach}
loop.  The most general usage is to use the double arrow
notation to iterate over each key-value pair.

\begin{minted}{php}
//for each key value pair:
foreach($arr as $key => $val) {
  print "$key maps to $val \n";
}
\end{minted}

This syntax gives you access to both the key and the
value for each element in the array \mintinline{php}{$arr}.
The keyword \mintinline{php}{as} is used to denote
the variable names \mintinline{php}{$key} and 
\mintinline{php}{$val} that will be changed on each
iteration of the loop.  You need not use the identifiers
\mintinline{php}{$key} and \mintinline{php}{$val}; you 
can use any legal variable names for the key/value 
variables.

If you do not need the keys when iterating, you can
use the following shorthand syntax.

\begin{minted}{php}
//for each value:
foreach($arr as $val) {
  print "$val \n";
}
\end{minted}

\section{Adding Elements}

You can easily add elements to an array by simply
providing an index and using the assignment operator
as in the previous examples.  There are also several
functions that PHP defines that can insert elements to
the beginning (\mintinline{php}{array_unshift()}), end
(\mintinline{php}{array_push()}) or at an arbitrary position
(\mintinline{php}{array_splice()}).

Another, simpler way of adding elements is to use the
following syntax:

\begin{minted}{php}
$arr = array(10, 20, 30);
$arr[] = 5;
$arr[] = 15;
$arr[] = 25;
print_r($arr);
\end{minted}

By using the assignment operator but not specifying
the index, the element will be added to the next available
integer index.  Since there were already 3 elements in the
array, each subsequent element is inserted at index 3, 4, and
finally 5.  In general, the element will be inserted at the
maximum index value already used plus one.  The
example above results in the following.

\begin{minted}[fontsize=\scriptsize]{text}
Array
(
    [0] => 10
    [1] => 20
    [2] => 30
    [3] => 5
    [4] => 15
    [5] => 25
)
\end{minted}

\section{Removing Elements}

You can remove elements from an array using the 
\mintinline{php}{unset()} function.  This function only
removes the element from the array, it does not
shift other elements down to fill in the unused
index. 

\begin{minted}{php}
$arr = array(10, 20, 30);
unset($arr[1]);
print_r($arr);
\end{minted}

This example would result in the following.

\begin{minted}{text}
Array
(
    [0] => 10
    [2] => 30
)
\end{minted}

Further, you can use \mintinline{php}{unset()}
to destroy all elements in the array:

\mintinline{php}{unset($arr);} 

destroys the entire array.  It does not merely 
empty the array, but it unsets the variable \mintinline{php}{$arr}
itself.  

\section{Using Arrays in Functions}

By default, all arguments to a function in PHP are
passed by value; this includes arrays.  Thus, if you 
make any changes to an array passed to a function, 
the changes are not realized in the calling function.
You can explicitly specify that the array parameter
is passed by reference so that any changes to the
array are realized in the calling function.

To illustrate, consider the following example.

\begin{minted}{php}
function setFirst($a) {
  $a[0] = 5;
}

$arr = array(10, 20, 30);
print_r($arr);
setFirst($arr);
print_r($arr);
\end{minted}

This example results in the following.

\begin{minted}[fontsize=\scriptsize]{text}
Array
(
    [0] => 10
    [1] => 20
    [2] => 30
)
Array
(
    [0] => 10
    [1] => 20
    [2] => 30
)
\end{minted}

That is, the change to the first element does not affect the
original array.  However, if we specify that the array is passed
by reference, then the change is realized.  For example, 

\begin{minted}{php}
function setFirst(&$a) {
  $a[0] = 5;
}

$arr = array(10, 20, 30);
print_r($arr);
setFirst($arr);
print_r($arr);
\end{minted}

This now results in the original array being changed:

\begin{minted}[fontsize=\scriptsize]{text}
Array
(
    [0] => 10
    [1] => 20
    [2] => 30
)
Array
(
    [0] => 5
    [1] => 20
    [2] => 30
)
\end{minted}

\section{Multidimensional Arrays}

PHP supports multidimensional arrays in the sense that
elements in an array can be of any type, including other 
arrays.

We can use all the same syntax and operations for single
dimensional arrays.  For example, we can use the double
arrow syntax and assign arrays as values to create a 2-dimensional
array.

\begin{minted}{php}
$mat = array(
  0 => array(10, 20, 30),
  1 => array(40, 50, 60),
  2 => array(70, 80, 90)
);
print_r($mat);
\end{minted}

Which results in the following:

\begin{minted}[fontsize=\scriptsize]{text}
Array
(
    [0] => Array
        (
            [0] => 10
            [1] => 20
            [2] => 30
        )

    [1] => Array
        (
            [0] => 40
            [1] => 50
            [2] => 60
        )

    [2] => Array
        (
            [0] => 70
            [1] => 80
            [2] => 90
        )

)
\end{minted}

Alternatively, you can use two indices to get and set values 
from a 2-dimensional array.

\begin{minted}{php}
for($i=0; $i<3; $i++) {
  for($j=0; $j<4; $j++) {
    $mat[$i][$j] = ($i+$j)*3;
  }
}
\end{minted}

which results in:

\begin{minted}[fontsize=\scriptsize]{text}
Array
(
    [0] => Array
        (
            [0] => 0
            [1] => 3
            [2] => 6
            [3] => 9
        )

    [1] => Array
        (
            [0] => 3
            [1] => 6
            [2] => 9
            [3] => 12
        )

    [2] => Array
        (
            [0] => 6
            [1] => 9
            [2] => 12
            [3] => 15
        )

)
\end{minted}

