%!TEX root = ComputerScienceOne.tex

%%Chapter: Encapsulation

One reason we prefer to write programs in high-level programming
languages is that we can write programs using syntax that is closer 
to plain English.  Granted, programming language syntax is a far cry 
from ``natural'' language, but it is far closer than lower level 
languages such as assembly or binary machine code.  However, from
what we've seen so far, when writing programs we are still forced
to utilize the primitive variable types built-in to the language
we're using.  This is still quite limiting.

As a motivating example, suppose we were to write a program that
involved organizing the enrollment of students into courses.  To
model a particular student, we would need a collection of variables, 
say a first name, last name, GPA, and a unique identification number
(likely a lot more, but let's keep it simple).  Each of these
pieces of data could be modeled by strings, a floating-point number
and perhaps an integer.\footnote{Depending on the identification
number, it may be more appropriately modeled with a string.  Social
Security Numbers for example are not purely numeric: they include
dashes and may begin with zeros.}  Each of these pieces of 
data are stored in separate, unrelated variables even though they
represent a single \emph{entity}.  

Even worse, suppose that we needed to keep track of a collection
of students.  Each piece of data would need to be stored in 
a separate array.  If we wanted to rearrange the data (say, sort
it), we would need to do a lot of manual bookkeeping to make sure
that the separate pieces of data that represented a single entity
were all aligned at the same index in each of the arrays.  If we
wanted to pass the data around to functions, we'd be forced to
pass multiple arrays to each function.  This becomes all the more
complex when we attempt to model entities with more pieces of 
data.  

The solution is to \emph{encapsulate} the pieces of data into
one logical entity, sometimes referred to as an \emph{object}.  
More formally, \emph{encapsulation} is a mechanism by which
pieces of data can be grouped together along with the functions
that operate on that data.  Encapsulation may also provides a 
means to \emph{protect} that data.

TODO: Contrast with array (also a collection of data); difference: 
columns vs rows in a table

\section{Objects}

OOP: design paradigm; some non-OOP languages support some aspects, 
not all OOP languages support all aspects or in the same way, etc.
Big four: Abstraction, Encapsulation, Inheritance, Polymorphism; 
for now we'll only focus on Encapsulation.

Collection

Protection

Inclusion of methods

\subsection{Structures}

Only the collection of data

\subsection{Construction}

constructors or factory methods

\subsection{Usage}

variable types, 

accessing member fields; getters/setters 

passing as functions

\subsection{Composition}

ownership of other objects, collections of objects (aggregation) 

\section{Design Considerations}

bottom-up

anytime you need to associate pieces of data, it may be appropriate to 
collect it into an object



