%!TEX root = ComputerScienceOne.tex

%%Chapter: Encapsulation

One reason we prefer to write programs in high-level programming
languages is that we can write programs using syntax that is closer 
to plain English.  Granted, programming language syntax is a far cry 
from ``natural'' language, but it is far closer than lower level 
languages such as assembly or binary machine code.  However, from
what we've seen so far, when writing programs we are still forced
to utilize the primitive variable types built-in to the language
we're using.  This is still quite limiting.

As a motivating example, suppose we were to write a program that
involved organizing the enrollment of students into courses.  To
model a particular student, we would need a collection of variables, 
say a first name, last name, GPA, and a unique identification number
(likely a lot more, but let's keep it simple).  Each of these
pieces of data could be modeled by strings, a floating-point number
and perhaps an integer.\footnote{Depending on the identification
number, it may be more appropriately modeled with a string.  Social
Security Numbers for example are not purely numeric: they include
dashes and may begin with zeros.}  Each of these pieces of 
data are stored in separate, unrelated variables even though they
represent a single \emph{entity}.  

Even worse, suppose that we needed to keep track of a collection
of students.  Each piece of data would need to be stored in 
a separate array.  If we wanted to rearrange the data (say, sort
it), we would need to do a lot of manual bookkeeping to make sure
that the separate pieces of data that represented a single entity
were all aligned at the same index in each of the arrays.  If we
wanted to pass the data around to functions, we'd be forced to
pass multiple arrays to each function.  This becomes all the more
complex when we attempt to model entities with more pieces of 
data.  

The solution is to \emph{encapsulate} the pieces of data into
one logical entity, sometimes referred to as an \emph{object}.  
More formally, \emph{encapsulation} is a mechanism by which
pieces of data can be grouped together along with the functions
that operate on that data.  Encapsulation may also provides a 
means to \emph{protect} that data by controlling the visibility
of that data from code outside the object.

Contrast this with an array which is also a collection of data.
However, an array usually contains pieces of similar data while
an object may collect pieces of dissimilar data that make up a
larger entity.  Its kind of like the difference between rows
and columns in a table.  Consider the student data in Table
\ref{table:studentData}.  Each row represents a record while each
column represents a collection of data from each record.  A
single column is comparable to an array while each row is comparable
to an object.  In this example, each object has four pieces of data
encapsulated in it, a first name, last name, an ID, and a GPA.

\begin{table}
\centering
\begin{tabular}{|l|l|l|l|}
\hline
First Name & Last Name & ID & GPA \\
\hline\hline
Tom & Baker & 74 & 3.75 \\
\hline
Christopher & Eccleston & 5 & 3.5 \\
\hline
David & Tennant & 10 & 4.0 \\
\hline
Matt & Smith & 29 & 3.2 \\
\hline
Peter & Capaldi & 13 & 2.9 \\
\hline
\end{tabular}
\caption{Student Data}
\label{table:studentData}
\end{table}

Without objects, to represent this data in code we would need at least
4 separate arrays, more if we wanted to model more data for a student.  
Moreover, data in separate arrays or collections have no real logical
relationship to each other.  The solution that most programming languages
provide is allowing you to \emph{define} an object or structure that
collects pieces of data into one logical unit, allowing you to \emph{name}
the object (say ``Student'') so that you can deal with the data in
a more abstract way.  With objects, we can treat each row in the table
as a single, distinct entity allowing us to collect student objects
into a single array or collection rather than many separate ones.

\section{Objects}

Though languages will differ in how they support objects, they all have
some commonalities.  A language needs to provide ways to define objects,
create \emph{instances} of objects, and to use them in code.

\subsection{Defining}

Most object oriented programming languages such as C++ and Java are 
\emph{class-based} languages.  Meaning that they allow you to define
objects by declaring and defining a ``class.''  A class is essentially
a blue print for what the object \emph{is} and how it is defined.  
Generally, a class declaration allows you to specify member variables
and member methods.  Further, full encapsulation is achieved by using
\emph{visibility} keywords such as \mintinline{java}{public} or
\mintinline{java}{private} to either allow or restrict access to variables
and methods from code \emph{outside} the object.  

Some languages (such as C) do not support full encapsulation, rather they
allow you to define \emph{structures} which allow for the grouping of data, 
but make it difficult or impossible to achieve the other two aspects of
encapsulation (the grouping of methods that act on that data and the
protection of data).

In either case, a language usually allows you to define the member variables
and to \emph{name} the class or structure so that instances can be referred
to by that \emph{type}.  Built-in types such as numbers or strings already
have a type name defined by the language.  However, an object is a 
\emph{user-defined} type that is not built in to the language.  Once defined,
however, the class or structure \emph{can} be referred to just like any
built-in variable type.

Often, it is usual to create objects that are made of other objects.  For
example, a student object may be defined by using two strings for its first
and last name.  In the language, a string itself may be an object.  As a
more complex example, suppose that we wanted an additional member variable
to model a student's date of birth.  A date may itself be an object as it
consists of several pieces of information (a year, month, and date at least).
When an object ``owns'' an instance of another object it is referred to
as \emph{composition} \index{composition} as the object is \emph{composed} of 
other objects.  Further, an object may consist of a \emph{collection}
of other objects (suppose that a student object owned an array of 
course objects representing their schedule).  This is a form of composition
known as \emph{aggregation} (multiple objects have been aggregated
by the object).

\subsection{Creating}

Once a blueprint for an object (or structure) has been declared and defined, 
the second element that a language usually provides for is a way to 
\emph{create} instances of the object.  The concept of an ``object'' is
general and abstract.  Its like the \emph{idea} of a student.  Only once
we have created an actual instance that lives in memory do we have an 
actual \emph{instance}.  Creating instances of an object is usually
referred to as \emph{instantiation}.  

Languages may be able to automatically create instances of your object
with \emph{default} values.  After all, your object is likely 
\emph{composed} of built-in types.  The student example above for
example could be modeled with two strings, an integer, and a floating
point number.  The language/compiler/interpreter ``knows'' how to deal
with these built-in types, so it can extend that knowledge to create
instances of your object which are essentially just collections of 
types that it already knows how to deal with.

Object oriented languages usually provide a special method for you to
be able to give more specifics about how an object is created.  These
are called \emph{constructor} methods.  Sometimes you can define 
multiple constructors methods that take different number(s) of arguments
and/or have different behavior.  Constructor methods typically have
special syntax or have the same name as the class.

In other languages that do not fully support object-oriented programming, 
you must define utility functions that can be used to create instances 
of your object.  Sometimes these are referred to as \emph{factory} functions 
as they are responsible for ``manufacturing'' instances of your object.

\subsection{Using Objects}

After defining and then creating an object, you can usually use it like
any regular variable.  In a strongly typed language you would declare a
variable whose type matches the declared class or structure.  The variable
type can usually be passed and returned from functions, assigned to 
other variables, etc.

A language can also provide a means to access the member variables or
methods that are \emph{visible} to the outside world.  Languages usually
allow you to do this through the ``dot operator'' or the ``arrow operator''.
Suppose we have an instance of a student object stored in a variable
\mintinline{c}{s}.  To access the first name of this instance, we may
be able to use either \mintinline{c}{s.firstName} or 
\mintinline{c}{s->firstName}.  We can access and invoke visible methods 
likewise.  

The dot/arrow operators are how code \emph{outside} the object interacts
with the object.  Outside code is able to do this because it holds a 
reference, \mintinline{c}{s} to the object.  However, \emph{inside} the
object, we may not have a reference (the variable \mintinline{c}{s} was
ostensibly declared and used outside the object and so is not in
scope inside the object).  However, we still have need to reference
member variables or methods from inside the object.  Many languages
provide a means to do this using \gls{open recursion} \index{open recursion}
using a keyword such as \mintinline{java}{this} or \mintinline{php}{self}
or something similar.  The keyword is essentially a self-reference to
the object itself so that you can refer to ``this'' object from within
the object.

\section{Design Principles \& Best Practices}

Using objects in your code follows more of a bottom-up design rather than
a top-down design approach.  In a top-down design, a program is designed
by breaking a program or problem down into smaller and smaller components.
A bottom-up design approaches a problem differently.  First, real-world
entities involved with the problem are modeled by defining objects.  
Then objects are used as building blocks that can be combined and made
to interact to solve a problem.

Object design is usually a straightforward process.  Typically an object
is modeling a real-world entity, so it is simple enough to decompose the
entity into its constituent components.  We do this until the component
can either be modeled by a built-in type such as a string or number or
by an existing object.  In general, you want to keep things as simple
as possible.  Any time you need to associate pieces of data together into
one logical unit, it is likely appropriate to encapsulate them
into an object.

A good design principle is to utilize composition as much as possible.
If you have multiple pieces of data that define a logical entity or 
unit, it is good design to create another object.  For example, if a
student object needs to model a mailing address; think about what an
address is: it is a street address, city, state, zip, etc.  Rather
than having these as member fields to your object, it is probably
more appropriate to define an ``address'' object, especially if such
an object would be useful elsewhere in a program.

