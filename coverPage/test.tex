\documentclass[12pt]{scrbook}
\usepackage{fullpage}
\usepackage{hyperref}
\usepackage{tikz}
\usetikzlibrary{calc}

%\usetikzlibrary{decorations.pathmorphing} % LATEX and plain TEX when using Tik Z

\definecolor{left}{HTML}{041832}
\definecolor{secondary}{HTML}{241024}

\title{Computer Science I}
\subtitle{Programming in C, Java, and Beyond}
\author{Dr.\ Chris Bourke\\
        \href{mailto:cbourke@cse.unl.edu}{cbourke@cse.unl.edu} \\
        Department of Computer Science \& Engineering\\
        University of Nebraska---Lincoln\\
        Lincoln, NE 68588, USA
}

\date{DATE}

\begin{document}

  \begin{tikzpicture}[remember picture,overlay,block/.style={
  text width=0.95*\columnwidth, 
  anchor=west,
  minimum height=1cm
  }, 
font=\small
]
    \node [shading = axis,rectangle, left color=left!80!white, right color=left!20!white,shading angle=120, anchor=north, minimum width=\paperwidth, minimum height=\paperheight,opacity=.70] (box) at (current page.north){  ~ };

    \node [shading = axis,rectangle, left color=secondary!80!white, right color=secondary!20!white,shading angle=45, anchor=north, minimum width=\paperwidth, minimum height=\paperheight,opacity=.60] (box) at (current page.north){  ~ };

    \node[block,align=center] at (0, -8) {\maketitle};


  %\node[draw,fill,circle] at (current page.south east) {$\alpha$};
  %\node[draw,circle,opacity=1,fill=green] at (7.75, -15) {$\alpha$};

\node[draw,circle,opacity=0.1] (a1) at (37.75, -15) {$\phi$}; 
\node[draw,circle,opacity=0.1] (a2) at (37.513441039434, -11.240002993071) {$\varepsilon$}; 
\node[draw,circle,opacity=0.1] (a3) at (36.807494833859, -7.5393033850544) {$\psi$}; 
\node[draw,circle,opacity=0.1] (a4) at (35.643294576648, -3.9562634194597) {$\varepsilon$}; 
\node[draw,circle,opacity=0.1] (a5) at (34.039200401316, -0.54738977694854) {$\Upsilon$}; 
\node[draw,circle,opacity=0.1] (a6) at (32.020509831248, 2.6335575687742) {$\delta$}; 
\node[draw,circle,opacity=0.1] (a7) at (29.619058822642, 5.5364131778607) {$\gamma$}; 
\node[draw,circle,opacity=0.1] (a8) at (26.872719692461, 8.1153972832737) {$\gamma$}; 
\node[draw,circle,opacity=0.1] (a9) at (23.82480384937, 10.32983776506) {$\Omega$}; 
\node[draw,circle,opacity=0.1] (a10) at (20.523378746952, 12.144811573981) {$\Psi$}; 
\node[draw,circle,opacity=0.1] (a11) at (17.020509831248, 13.531695488855) {$\Gamma$}; 
\node[draw,circle,opacity=0.1] (a12) at (13.371439437572, 14.468617521861) {$\phi$}; 
\node[draw,circle,opacity=0.1] (a13) at (9.6337155858794, 14.940801852848) {$\Gamma$}; 
\node[draw,circle,opacity=0.1] (a14) at (5.8662844141206, 14.940801852848) {$\kappa$}; 
\node[draw,circle,opacity=0.1] (a15) at (2.1285605624283, 14.468617521861) {$\xi$}; 
\node[draw,circle,opacity=0.1] (a16) at (-1.5205098312484, 13.531695488855) {$\gamma$}; 
\node[draw,circle,opacity=0.1] (a17) at (-5.0233787469522, 12.144811573981) {$\chi$}; 
\node[draw,circle,opacity=0.1] (a18) at (-8.3248038493699, 10.32983776506) {$\psi$}; 
\node[draw,circle,opacity=0.1] (a19) at (-11.372719692461, 8.1153972832737) {$\Xi$}; 
\node[draw,circle,opacity=0.1] (a20) at (-14.119058822642, 5.5364131778607) {$\theta$}; 
\node[draw,circle,opacity=0.1] (a21) at (-16.520509831248, 2.6335575687742) {$\rho$}; 
\node[draw,circle,opacity=0.1] (a22) at (-18.539200401316, -0.54738977694854) {$\psi$}; 
\node[draw,circle,opacity=0.1] (a23) at (-20.143294576648, -3.9562634194597) {$\vartheta$}; 
\node[draw,circle,opacity=0.1] (a24) at (-21.307494833859, -7.5393033850544) {$\Psi$}; 
\node[draw,circle,opacity=0.1] (a25) at (-22.013441039434, -11.240002993071) {$\Upsilon$}; 
\node[draw,circle,opacity=0.1] (a26) at (-22.25, -15) {$\epsilon$}; 
\node[draw,circle,opacity=0.1] (a27) at (-22.013441039434, -18.759997006929) {$\Delta$}; 
\node[draw,circle,opacity=0.1] (a28) at (-21.307494833859, -22.460696614946) {$\gamma$}; 
\node[draw,circle,opacity=0.1] (a29) at (-20.143294576648, -26.04373658054) {$\kappa$}; 
\node[draw,circle,opacity=0.1] (a30) at (-18.539200401316, -29.452610223051) {$\sigma$}; 
\node[draw,circle,opacity=0.1] (a31) at (-16.520509831248, -32.633557568774) {$\Gamma$}; 
\node[draw,circle,opacity=0.1] (a32) at (-14.119058822642, -35.536413177861) {$\pi$}; 
\node[draw,circle,opacity=0.1] (a33) at (-11.372719692461, -38.115397283274) {$\psi$}; 
\node[draw,circle,opacity=0.1] (a34) at (-8.3248038493699, -40.32983776506) {$\chi$}; 
\node[draw,circle,opacity=0.1] (a35) at (-5.0233787469522, -42.144811573981) {$\Gamma$}; 
\node[draw,circle,opacity=0.1] (a36) at (-1.5205098312484, -43.531695488855) {$\omega$}; 
\node[draw,circle,opacity=0.1] (a37) at (2.1285605624283, -44.468617521861) {$\chi$}; 
\node[draw,circle,opacity=0.1] (a38) at (5.8662844141206, -44.940801852848) {$\beta$}; 
\node[draw,circle,opacity=0.1] (a39) at (9.6337155858794, -44.940801852848) {$\Delta$}; 
\node[draw,circle,opacity=0.1] (a40) at (13.371439437572, -44.468617521861) {$\pi$}; 
\node[draw,circle,opacity=0.1] (a41) at (17.020509831248, -43.531695488855) {$\Lambda$}; 
\node[draw,circle,opacity=0.1] (a42) at (20.523378746952, -42.144811573981) {$\Psi$}; 
\node[draw,circle,opacity=0.1] (a43) at (23.82480384937, -40.32983776506) {$\upsilon$}; 
\node[draw,circle,opacity=0.1] (a44) at (26.872719692461, -38.115397283274) {$\Delta$}; 
\node[draw,circle,opacity=0.1] (a45) at (29.619058822642, -35.536413177861) {$\varpi$}; 
\node[draw,circle,opacity=0.1] (a46) at (32.020509831248, -32.633557568774) {$\chi$}; 
\node[draw,circle,opacity=0.1] (a47) at (34.039200401316, -29.452610223051) {$\Xi$}; 
\node[draw,circle,opacity=0.1] (a48) at (35.643294576648, -26.04373658054) {$\varrho$}; 
\node[draw,circle,opacity=0.1] (a49) at (36.807494833859, -22.460696614946) {$\vartheta$}; 
\node[draw,circle,opacity=0.1] (a50) at (37.513441039434, -18.759997006929) {$\varphi$}; 
\draw[opacity=0.1] (a1) -- (a2);
\draw[opacity=0.1] (a1) -- (a3);
\draw[opacity=0.1] (a1) -- (a4);
\draw[opacity=0.1] (a1) -- (a5);
\draw[opacity=0.1] (a1) -- (a6);
\draw[opacity=0.1] (a1) -- (a7);
\draw[opacity=0.1] (a1) -- (a8);
\draw[opacity=0.1] (a1) -- (a9);
\draw[opacity=0.1] (a1) -- (a10);
\draw[opacity=0.1] (a1) -- (a11);
\draw[opacity=0.1] (a1) -- (a12);
\draw[opacity=0.1] (a1) -- (a13);
\draw[opacity=0.1] (a1) -- (a14);
\draw[opacity=0.1] (a1) -- (a15);
\draw[opacity=0.1] (a1) -- (a16);
\draw[opacity=0.1] (a1) -- (a17);
\draw[opacity=0.1] (a1) -- (a18);
\draw[opacity=0.1] (a1) -- (a19);
\draw[opacity=0.1] (a1) -- (a20);
\draw[opacity=0.1] (a1) -- (a21);
\draw[opacity=0.1] (a1) -- (a22);
\draw[opacity=0.1] (a1) -- (a23);
\draw[opacity=0.1] (a1) -- (a24);
\draw[opacity=0.1] (a1) -- (a25);
\draw[opacity=0.1] (a1) -- (a26);
\draw[opacity=0.1] (a1) -- (a27);
\draw[opacity=0.1] (a1) -- (a28);
\draw[opacity=0.1] (a1) -- (a29);
\draw[opacity=0.1] (a1) -- (a30);
\draw[opacity=0.1] (a1) -- (a31);
\draw[opacity=0.1] (a1) -- (a32);
\draw[opacity=0.1] (a1) -- (a33);
\draw[opacity=0.1] (a1) -- (a34);
\draw[opacity=0.1] (a1) -- (a35);
\draw[opacity=0.1] (a1) -- (a36);
\draw[opacity=0.1] (a1) -- (a37);
\draw[opacity=0.1] (a1) -- (a38);
\draw[opacity=0.1] (a1) -- (a39);
\draw[opacity=0.1] (a1) -- (a40);
\draw[opacity=0.1] (a1) -- (a41);
\draw[opacity=0.1] (a1) -- (a42);
\draw[opacity=0.1] (a1) -- (a43);
\draw[opacity=0.1] (a1) -- (a44);
\draw[opacity=0.1] (a1) -- (a45);
\draw[opacity=0.1] (a1) -- (a46);
\draw[opacity=0.1] (a1) -- (a47);
\draw[opacity=0.1] (a1) -- (a48);
\draw[opacity=0.1] (a1) -- (a49);
\draw[opacity=0.1] (a1) -- (a50);
\draw[opacity=0.1] (a2) -- (a3);
\draw[opacity=0.1] (a2) -- (a4);
\draw[opacity=0.1] (a2) -- (a5);
\draw[opacity=0.1] (a2) -- (a6);
\draw[opacity=0.1] (a2) -- (a7);
\draw[opacity=0.1] (a2) -- (a8);
\draw[opacity=0.1] (a2) -- (a9);
\draw[opacity=0.1] (a2) -- (a10);
\draw[opacity=0.1] (a2) -- (a11);
\draw[opacity=0.1] (a2) -- (a12);
\draw[opacity=0.1] (a2) -- (a13);
\draw[opacity=0.1] (a2) -- (a14);
\draw[opacity=0.1] (a2) -- (a15);
\draw[opacity=0.1] (a2) -- (a16);
\draw[opacity=0.1] (a2) -- (a17);
\draw[opacity=0.1] (a2) -- (a18);
\draw[opacity=0.1] (a2) -- (a19);
\draw[opacity=0.1] (a2) -- (a20);
\draw[opacity=0.1] (a2) -- (a21);
\draw[opacity=0.1] (a2) -- (a22);
\draw[opacity=0.1] (a2) -- (a23);
\draw[opacity=0.1] (a2) -- (a24);
\draw[opacity=0.1] (a2) -- (a25);
\draw[opacity=0.1] (a2) -- (a26);
\draw[opacity=0.1] (a2) -- (a27);
\draw[opacity=0.1] (a2) -- (a28);
\draw[opacity=0.1] (a2) -- (a29);
\draw[opacity=0.1] (a2) -- (a30);
\draw[opacity=0.1] (a2) -- (a31);
\draw[opacity=0.1] (a2) -- (a32);
\draw[opacity=0.1] (a2) -- (a33);
\draw[opacity=0.1] (a2) -- (a34);
\draw[opacity=0.1] (a2) -- (a35);
\draw[opacity=0.1] (a2) -- (a36);
\draw[opacity=0.1] (a2) -- (a37);
\draw[opacity=0.1] (a2) -- (a38);
\draw[opacity=0.1] (a2) -- (a39);
\draw[opacity=0.1] (a2) -- (a40);
\draw[opacity=0.1] (a2) -- (a41);
\draw[opacity=0.1] (a2) -- (a42);
\draw[opacity=0.1] (a2) -- (a43);
\draw[opacity=0.1] (a2) -- (a44);
\draw[opacity=0.1] (a2) -- (a45);
\draw[opacity=0.1] (a2) -- (a46);
\draw[opacity=0.1] (a2) -- (a47);
\draw[opacity=0.1] (a2) -- (a48);
\draw[opacity=0.1] (a2) -- (a49);
\draw[opacity=0.1] (a2) -- (a50);
\draw[opacity=0.1] (a3) -- (a4);
\draw[opacity=0.1] (a3) -- (a5);
\draw[opacity=0.1] (a3) -- (a6);
\draw[opacity=0.1] (a3) -- (a7);
\draw[opacity=0.1] (a3) -- (a8);
\draw[opacity=0.1] (a3) -- (a9);
\draw[opacity=0.1] (a3) -- (a10);
\draw[opacity=0.1] (a3) -- (a11);
\draw[opacity=0.1] (a3) -- (a12);
\draw[opacity=0.1] (a3) -- (a13);
\draw[opacity=0.1] (a3) -- (a14);
\draw[opacity=0.1] (a3) -- (a15);
\draw[opacity=0.1] (a3) -- (a16);
\draw[opacity=0.1] (a3) -- (a17);
\draw[opacity=0.1] (a3) -- (a18);
\draw[opacity=0.1] (a3) -- (a19);
\draw[opacity=0.1] (a3) -- (a20);
\draw[opacity=0.1] (a3) -- (a21);
\draw[opacity=0.1] (a3) -- (a22);
\draw[opacity=0.1] (a3) -- (a23);
\draw[opacity=0.1] (a3) -- (a24);
\draw[opacity=0.1] (a3) -- (a25);
\draw[opacity=0.1] (a3) -- (a26);
\draw[opacity=0.1] (a3) -- (a27);
\draw[opacity=0.1] (a3) -- (a28);
\draw[opacity=0.1] (a3) -- (a29);
\draw[opacity=0.1] (a3) -- (a30);
\draw[opacity=0.1] (a3) -- (a31);
\draw[opacity=0.1] (a3) -- (a32);
\draw[opacity=0.1] (a3) -- (a33);
\draw[opacity=0.1] (a3) -- (a34);
\draw[opacity=0.1] (a3) -- (a35);
\draw[opacity=0.1] (a3) -- (a36);
\draw[opacity=0.1] (a3) -- (a37);
\draw[opacity=0.1] (a3) -- (a38);
\draw[opacity=0.1] (a3) -- (a39);
\draw[opacity=0.1] (a3) -- (a40);
\draw[opacity=0.1] (a3) -- (a41);
\draw[opacity=0.1] (a3) -- (a42);
\draw[opacity=0.1] (a3) -- (a43);
\draw[opacity=0.1] (a3) -- (a44);
\draw[opacity=0.1] (a3) -- (a45);
\draw[opacity=0.1] (a3) -- (a46);
\draw[opacity=0.1] (a3) -- (a47);
\draw[opacity=0.1] (a3) -- (a48);
\draw[opacity=0.1] (a3) -- (a49);
\draw[opacity=0.1] (a3) -- (a50);
\draw[opacity=0.1] (a4) -- (a5);
\draw[opacity=0.1] (a4) -- (a6);
\draw[opacity=0.1] (a4) -- (a7);
\draw[opacity=0.1] (a4) -- (a8);
\draw[opacity=0.1] (a4) -- (a9);
\draw[opacity=0.1] (a4) -- (a10);
\draw[opacity=0.1] (a4) -- (a11);
\draw[opacity=0.1] (a4) -- (a12);
\draw[opacity=0.1] (a4) -- (a13);
\draw[opacity=0.1] (a4) -- (a14);
\draw[opacity=0.1] (a4) -- (a15);
\draw[opacity=0.1] (a4) -- (a16);
\draw[opacity=0.1] (a4) -- (a17);
\draw[opacity=0.1] (a4) -- (a18);
\draw[opacity=0.1] (a4) -- (a19);
\draw[opacity=0.1] (a4) -- (a20);
\draw[opacity=0.1] (a4) -- (a21);
\draw[opacity=0.1] (a4) -- (a22);
\draw[opacity=0.1] (a4) -- (a23);
\draw[opacity=0.1] (a4) -- (a24);
\draw[opacity=0.1] (a4) -- (a25);
\draw[opacity=0.1] (a4) -- (a26);
\draw[opacity=0.1] (a4) -- (a27);
\draw[opacity=0.1] (a4) -- (a28);
\draw[opacity=0.1] (a4) -- (a29);
\draw[opacity=0.1] (a4) -- (a30);
\draw[opacity=0.1] (a4) -- (a31);
\draw[opacity=0.1] (a4) -- (a32);
\draw[opacity=0.1] (a4) -- (a33);
\draw[opacity=0.1] (a4) -- (a34);
\draw[opacity=0.1] (a4) -- (a35);
\draw[opacity=0.1] (a4) -- (a36);
\draw[opacity=0.1] (a4) -- (a37);
\draw[opacity=0.1] (a4) -- (a38);
\draw[opacity=0.1] (a4) -- (a39);
\draw[opacity=0.1] (a4) -- (a40);
\draw[opacity=0.1] (a4) -- (a41);
\draw[opacity=0.1] (a4) -- (a42);
\draw[opacity=0.1] (a4) -- (a43);
\draw[opacity=0.1] (a4) -- (a44);
\draw[opacity=0.1] (a4) -- (a45);
\draw[opacity=0.1] (a4) -- (a46);
\draw[opacity=0.1] (a4) -- (a47);
\draw[opacity=0.1] (a4) -- (a48);
\draw[opacity=0.1] (a4) -- (a49);
\draw[opacity=0.1] (a4) -- (a50);
\draw[opacity=0.1] (a5) -- (a6);
\draw[opacity=0.1] (a5) -- (a7);
\draw[opacity=0.1] (a5) -- (a8);
\draw[opacity=0.1] (a5) -- (a9);
\draw[opacity=0.1] (a5) -- (a10);
\draw[opacity=0.1] (a5) -- (a11);
\draw[opacity=0.1] (a5) -- (a12);
\draw[opacity=0.1] (a5) -- (a13);
\draw[opacity=0.1] (a5) -- (a14);
\draw[opacity=0.1] (a5) -- (a15);
\draw[opacity=0.1] (a5) -- (a16);
\draw[opacity=0.1] (a5) -- (a17);
\draw[opacity=0.1] (a5) -- (a18);
\draw[opacity=0.1] (a5) -- (a19);
\draw[opacity=0.1] (a5) -- (a20);
\draw[opacity=0.1] (a5) -- (a21);
\draw[opacity=0.1] (a5) -- (a22);
\draw[opacity=0.1] (a5) -- (a23);
\draw[opacity=0.1] (a5) -- (a24);
\draw[opacity=0.1] (a5) -- (a25);
\draw[opacity=0.1] (a5) -- (a26);
\draw[opacity=0.1] (a5) -- (a27);
\draw[opacity=0.1] (a5) -- (a28);
\draw[opacity=0.1] (a5) -- (a29);
\draw[opacity=0.1] (a5) -- (a30);
\draw[opacity=0.1] (a5) -- (a31);
\draw[opacity=0.1] (a5) -- (a32);
\draw[opacity=0.1] (a5) -- (a33);
\draw[opacity=0.1] (a5) -- (a34);
\draw[opacity=0.1] (a5) -- (a35);
\draw[opacity=0.1] (a5) -- (a36);
\draw[opacity=0.1] (a5) -- (a37);
\draw[opacity=0.1] (a5) -- (a38);
\draw[opacity=0.1] (a5) -- (a39);
\draw[opacity=0.1] (a5) -- (a40);
\draw[opacity=0.1] (a5) -- (a41);
\draw[opacity=0.1] (a5) -- (a42);
\draw[opacity=0.1] (a5) -- (a43);
\draw[opacity=0.1] (a5) -- (a44);
\draw[opacity=0.1] (a5) -- (a45);
\draw[opacity=0.1] (a5) -- (a46);
\draw[opacity=0.1] (a5) -- (a47);
\draw[opacity=0.1] (a5) -- (a48);
\draw[opacity=0.1] (a5) -- (a49);
\draw[opacity=0.1] (a5) -- (a50);
\draw[opacity=0.1] (a6) -- (a7);
\draw[opacity=0.1] (a6) -- (a8);
\draw[opacity=0.1] (a6) -- (a9);
\draw[opacity=0.1] (a6) -- (a10);
\draw[opacity=0.1] (a6) -- (a11);
\draw[opacity=0.1] (a6) -- (a12);
\draw[opacity=0.1] (a6) -- (a13);
\draw[opacity=0.1] (a6) -- (a14);
\draw[opacity=0.1] (a6) -- (a15);
\draw[opacity=0.1] (a6) -- (a16);
\draw[opacity=0.1] (a6) -- (a17);
\draw[opacity=0.1] (a6) -- (a18);
\draw[opacity=0.1] (a6) -- (a19);
\draw[opacity=0.1] (a6) -- (a20);
\draw[opacity=0.1] (a6) -- (a21);
\draw[opacity=0.1] (a6) -- (a22);
\draw[opacity=0.1] (a6) -- (a23);
\draw[opacity=0.1] (a6) -- (a24);
\draw[opacity=0.1] (a6) -- (a25);
\draw[opacity=0.1] (a6) -- (a26);
\draw[opacity=0.1] (a6) -- (a27);
\draw[opacity=0.1] (a6) -- (a28);
\draw[opacity=0.1] (a6) -- (a29);
\draw[opacity=0.1] (a6) -- (a30);
\draw[opacity=0.1] (a6) -- (a31);
\draw[opacity=0.1] (a6) -- (a32);
\draw[opacity=0.1] (a6) -- (a33);
\draw[opacity=0.1] (a6) -- (a34);
\draw[opacity=0.1] (a6) -- (a35);
\draw[opacity=0.1] (a6) -- (a36);
\draw[opacity=0.1] (a6) -- (a37);
\draw[opacity=0.1] (a6) -- (a38);
\draw[opacity=0.1] (a6) -- (a39);
\draw[opacity=0.1] (a6) -- (a40);
\draw[opacity=0.1] (a6) -- (a41);
\draw[opacity=0.1] (a6) -- (a42);
\draw[opacity=0.1] (a6) -- (a43);
\draw[opacity=0.1] (a6) -- (a44);
\draw[opacity=0.1] (a6) -- (a45);
\draw[opacity=0.1] (a6) -- (a46);
\draw[opacity=0.1] (a6) -- (a47);
\draw[opacity=0.1] (a6) -- (a48);
\draw[opacity=0.1] (a6) -- (a49);
\draw[opacity=0.1] (a6) -- (a50);
\draw[opacity=0.1] (a7) -- (a8);
\draw[opacity=0.1] (a7) -- (a9);
\draw[opacity=0.1] (a7) -- (a10);
\draw[opacity=0.1] (a7) -- (a11);
\draw[opacity=0.1] (a7) -- (a12);
\draw[opacity=0.1] (a7) -- (a13);
\draw[opacity=0.1] (a7) -- (a14);
\draw[opacity=0.1] (a7) -- (a15);
\draw[opacity=0.1] (a7) -- (a16);
\draw[opacity=0.1] (a7) -- (a17);
\draw[opacity=0.1] (a7) -- (a18);
\draw[opacity=0.1] (a7) -- (a19);
\draw[opacity=0.1] (a7) -- (a20);
\draw[opacity=0.1] (a7) -- (a21);
\draw[opacity=0.1] (a7) -- (a22);
\draw[opacity=0.1] (a7) -- (a23);
\draw[opacity=0.1] (a7) -- (a24);
\draw[opacity=0.1] (a7) -- (a25);
\draw[opacity=0.1] (a7) -- (a26);
\draw[opacity=0.1] (a7) -- (a27);
\draw[opacity=0.1] (a7) -- (a28);
\draw[opacity=0.1] (a7) -- (a29);
\draw[opacity=0.1] (a7) -- (a30);
\draw[opacity=0.1] (a7) -- (a31);
\draw[opacity=0.1] (a7) -- (a32);
\draw[opacity=0.1] (a7) -- (a33);
\draw[opacity=0.1] (a7) -- (a34);
\draw[opacity=0.1] (a7) -- (a35);
\draw[opacity=0.1] (a7) -- (a36);
\draw[opacity=0.1] (a7) -- (a37);
\draw[opacity=0.1] (a7) -- (a38);
\draw[opacity=0.1] (a7) -- (a39);
\draw[opacity=0.1] (a7) -- (a40);
\draw[opacity=0.1] (a7) -- (a41);
\draw[opacity=0.1] (a7) -- (a42);
\draw[opacity=0.1] (a7) -- (a43);
\draw[opacity=0.1] (a7) -- (a44);
\draw[opacity=0.1] (a7) -- (a45);
\draw[opacity=0.1] (a7) -- (a46);
\draw[opacity=0.1] (a7) -- (a47);
\draw[opacity=0.1] (a7) -- (a48);
\draw[opacity=0.1] (a7) -- (a49);
\draw[opacity=0.1] (a7) -- (a50);
\draw[opacity=0.1] (a8) -- (a9);
\draw[opacity=0.1] (a8) -- (a10);
\draw[opacity=0.1] (a8) -- (a11);
\draw[opacity=0.1] (a8) -- (a12);
\draw[opacity=0.1] (a8) -- (a13);
\draw[opacity=0.1] (a8) -- (a14);
\draw[opacity=0.1] (a8) -- (a15);
\draw[opacity=0.1] (a8) -- (a16);
\draw[opacity=0.1] (a8) -- (a17);
\draw[opacity=0.1] (a8) -- (a18);
\draw[opacity=0.1] (a8) -- (a19);
\draw[opacity=0.1] (a8) -- (a20);
\draw[opacity=0.1] (a8) -- (a21);
\draw[opacity=0.1] (a8) -- (a22);
\draw[opacity=0.1] (a8) -- (a23);
\draw[opacity=0.1] (a8) -- (a24);
\draw[opacity=0.1] (a8) -- (a25);
\draw[opacity=0.1] (a8) -- (a26);
\draw[opacity=0.1] (a8) -- (a27);
\draw[opacity=0.1] (a8) -- (a28);
\draw[opacity=0.1] (a8) -- (a29);
\draw[opacity=0.1] (a8) -- (a30);
\draw[opacity=0.1] (a8) -- (a31);
\draw[opacity=0.1] (a8) -- (a32);
\draw[opacity=0.1] (a8) -- (a33);
\draw[opacity=0.1] (a8) -- (a34);
\draw[opacity=0.1] (a8) -- (a35);
\draw[opacity=0.1] (a8) -- (a36);
\draw[opacity=0.1] (a8) -- (a37);
\draw[opacity=0.1] (a8) -- (a38);
\draw[opacity=0.1] (a8) -- (a39);
\draw[opacity=0.1] (a8) -- (a40);
\draw[opacity=0.1] (a8) -- (a41);
\draw[opacity=0.1] (a8) -- (a42);
\draw[opacity=0.1] (a8) -- (a43);
\draw[opacity=0.1] (a8) -- (a44);
\draw[opacity=0.1] (a8) -- (a45);
\draw[opacity=0.1] (a8) -- (a46);
\draw[opacity=0.1] (a8) -- (a47);
\draw[opacity=0.1] (a8) -- (a48);
\draw[opacity=0.1] (a8) -- (a49);
\draw[opacity=0.1] (a8) -- (a50);
\draw[opacity=0.1] (a9) -- (a10);
\draw[opacity=0.1] (a9) -- (a11);
\draw[opacity=0.1] (a9) -- (a12);
\draw[opacity=0.1] (a9) -- (a13);
\draw[opacity=0.1] (a9) -- (a14);
\draw[opacity=0.1] (a9) -- (a15);
\draw[opacity=0.1] (a9) -- (a16);
\draw[opacity=0.1] (a9) -- (a17);
\draw[opacity=0.1] (a9) -- (a18);
\draw[opacity=0.1] (a9) -- (a19);
\draw[opacity=0.1] (a9) -- (a20);
\draw[opacity=0.1] (a9) -- (a21);
\draw[opacity=0.1] (a9) -- (a22);
\draw[opacity=0.1] (a9) -- (a23);
\draw[opacity=0.1] (a9) -- (a24);
\draw[opacity=0.1] (a9) -- (a25);
\draw[opacity=0.1] (a9) -- (a26);
\draw[opacity=0.1] (a9) -- (a27);
\draw[opacity=0.1] (a9) -- (a28);
\draw[opacity=0.1] (a9) -- (a29);
\draw[opacity=0.1] (a9) -- (a30);
\draw[opacity=0.1] (a9) -- (a31);
\draw[opacity=0.1] (a9) -- (a32);
\draw[opacity=0.1] (a9) -- (a33);
\draw[opacity=0.1] (a9) -- (a34);
\draw[opacity=0.1] (a9) -- (a35);
\draw[opacity=0.1] (a9) -- (a36);
\draw[opacity=0.1] (a9) -- (a37);
\draw[opacity=0.1] (a9) -- (a38);
\draw[opacity=0.1] (a9) -- (a39);
\draw[opacity=0.1] (a9) -- (a40);
\draw[opacity=0.1] (a9) -- (a41);
\draw[opacity=0.1] (a9) -- (a42);
\draw[opacity=0.1] (a9) -- (a43);
\draw[opacity=0.1] (a9) -- (a44);
\draw[opacity=0.1] (a9) -- (a45);
\draw[opacity=0.1] (a9) -- (a46);
\draw[opacity=0.1] (a9) -- (a47);
\draw[opacity=0.1] (a9) -- (a48);
\draw[opacity=0.1] (a9) -- (a49);
\draw[opacity=0.1] (a9) -- (a50);
\draw[opacity=0.1] (a10) -- (a11);
\draw[opacity=0.1] (a10) -- (a12);
\draw[opacity=0.1] (a10) -- (a13);
\draw[opacity=0.1] (a10) -- (a14);
\draw[opacity=0.1] (a10) -- (a15);
\draw[opacity=0.1] (a10) -- (a16);
\draw[opacity=0.1] (a10) -- (a17);
\draw[opacity=0.1] (a10) -- (a18);
\draw[opacity=0.1] (a10) -- (a19);
\draw[opacity=0.1] (a10) -- (a20);
\draw[opacity=0.1] (a10) -- (a21);
\draw[opacity=0.1] (a10) -- (a22);
\draw[opacity=0.1] (a10) -- (a23);
\draw[opacity=0.1] (a10) -- (a24);
\draw[opacity=0.1] (a10) -- (a25);
\draw[opacity=0.1] (a10) -- (a26);
\draw[opacity=0.1] (a10) -- (a27);
\draw[opacity=0.1] (a10) -- (a28);
\draw[opacity=0.1] (a10) -- (a29);
\draw[opacity=0.1] (a10) -- (a30);
\draw[opacity=0.1] (a10) -- (a31);
\draw[opacity=0.1] (a10) -- (a32);
\draw[opacity=0.1] (a10) -- (a33);
\draw[opacity=0.1] (a10) -- (a34);
\draw[opacity=0.1] (a10) -- (a35);
\draw[opacity=0.1] (a10) -- (a36);
\draw[opacity=0.1] (a10) -- (a37);
\draw[opacity=0.1] (a10) -- (a38);
\draw[opacity=0.1] (a10) -- (a39);
\draw[opacity=0.1] (a10) -- (a40);
\draw[opacity=0.1] (a10) -- (a41);
\draw[opacity=0.1] (a10) -- (a42);
\draw[opacity=0.1] (a10) -- (a43);
\draw[opacity=0.1] (a10) -- (a44);
\draw[opacity=0.1] (a10) -- (a45);
\draw[opacity=0.1] (a10) -- (a46);
\draw[opacity=0.1] (a10) -- (a47);
\draw[opacity=0.1] (a10) -- (a48);
\draw[opacity=0.1] (a10) -- (a49);
\draw[opacity=0.1] (a10) -- (a50);
\draw[opacity=0.1] (a11) -- (a12);
\draw[opacity=0.1] (a11) -- (a13);
\draw[opacity=0.1] (a11) -- (a14);
\draw[opacity=0.1] (a11) -- (a15);
\draw[opacity=0.1] (a11) -- (a16);
\draw[opacity=0.1] (a11) -- (a17);
\draw[opacity=0.1] (a11) -- (a18);
\draw[opacity=0.1] (a11) -- (a19);
\draw[opacity=0.1] (a11) -- (a20);
\draw[opacity=0.1] (a11) -- (a21);
\draw[opacity=0.1] (a11) -- (a22);
\draw[opacity=0.1] (a11) -- (a23);
\draw[opacity=0.1] (a11) -- (a24);
\draw[opacity=0.1] (a11) -- (a25);
\draw[opacity=0.1] (a11) -- (a26);
\draw[opacity=0.1] (a11) -- (a27);
\draw[opacity=0.1] (a11) -- (a28);
\draw[opacity=0.1] (a11) -- (a29);
\draw[opacity=0.1] (a11) -- (a30);
\draw[opacity=0.1] (a11) -- (a31);
\draw[opacity=0.1] (a11) -- (a32);
\draw[opacity=0.1] (a11) -- (a33);
\draw[opacity=0.1] (a11) -- (a34);
\draw[opacity=0.1] (a11) -- (a35);
\draw[opacity=0.1] (a11) -- (a36);
\draw[opacity=0.1] (a11) -- (a37);
\draw[opacity=0.1] (a11) -- (a38);
\draw[opacity=0.1] (a11) -- (a39);
\draw[opacity=0.1] (a11) -- (a40);
\draw[opacity=0.1] (a11) -- (a41);
\draw[opacity=0.1] (a11) -- (a42);
\draw[opacity=0.1] (a11) -- (a43);
\draw[opacity=0.1] (a11) -- (a44);
\draw[opacity=0.1] (a11) -- (a45);
\draw[opacity=0.1] (a11) -- (a46);
\draw[opacity=0.1] (a11) -- (a47);
\draw[opacity=0.1] (a11) -- (a48);
\draw[opacity=0.1] (a11) -- (a49);
\draw[opacity=0.1] (a11) -- (a50);
\draw[opacity=0.1] (a12) -- (a13);
\draw[opacity=0.1] (a12) -- (a14);
\draw[opacity=0.1] (a12) -- (a15);
\draw[opacity=0.1] (a12) -- (a16);
\draw[opacity=0.1] (a12) -- (a17);
\draw[opacity=0.1] (a12) -- (a18);
\draw[opacity=0.1] (a12) -- (a19);
\draw[opacity=0.1] (a12) -- (a20);
\draw[opacity=0.1] (a12) -- (a21);
\draw[opacity=0.1] (a12) -- (a22);
\draw[opacity=0.1] (a12) -- (a23);
\draw[opacity=0.1] (a12) -- (a24);
\draw[opacity=0.1] (a12) -- (a25);
\draw[opacity=0.1] (a12) -- (a26);
\draw[opacity=0.1] (a12) -- (a27);
\draw[opacity=0.1] (a12) -- (a28);
\draw[opacity=0.1] (a12) -- (a29);
\draw[opacity=0.1] (a12) -- (a30);
\draw[opacity=0.1] (a12) -- (a31);
\draw[opacity=0.1] (a12) -- (a32);
\draw[opacity=0.1] (a12) -- (a33);
\draw[opacity=0.1] (a12) -- (a34);
\draw[opacity=0.1] (a12) -- (a35);
\draw[opacity=0.1] (a12) -- (a36);
\draw[opacity=0.1] (a12) -- (a37);
\draw[opacity=0.1] (a12) -- (a38);
\draw[opacity=0.1] (a12) -- (a39);
\draw[opacity=0.1] (a12) -- (a40);
\draw[opacity=0.1] (a12) -- (a41);
\draw[opacity=0.1] (a12) -- (a42);
\draw[opacity=0.1] (a12) -- (a43);
\draw[opacity=0.1] (a12) -- (a44);
\draw[opacity=0.1] (a12) -- (a45);
\draw[opacity=0.1] (a12) -- (a46);
\draw[opacity=0.1] (a12) -- (a47);
\draw[opacity=0.1] (a12) -- (a48);
\draw[opacity=0.1] (a12) -- (a49);
\draw[opacity=0.1] (a12) -- (a50);
\draw[opacity=0.1] (a13) -- (a14);
\draw[opacity=0.1] (a13) -- (a15);
\draw[opacity=0.1] (a13) -- (a16);
\draw[opacity=0.1] (a13) -- (a17);
\draw[opacity=0.1] (a13) -- (a18);
\draw[opacity=0.1] (a13) -- (a19);
\draw[opacity=0.1] (a13) -- (a20);
\draw[opacity=0.1] (a13) -- (a21);
\draw[opacity=0.1] (a13) -- (a22);
\draw[opacity=0.1] (a13) -- (a23);
\draw[opacity=0.1] (a13) -- (a24);
\draw[opacity=0.1] (a13) -- (a25);
\draw[opacity=0.1] (a13) -- (a26);
\draw[opacity=0.1] (a13) -- (a27);
\draw[opacity=0.1] (a13) -- (a28);
\draw[opacity=0.1] (a13) -- (a29);
\draw[opacity=0.1] (a13) -- (a30);
\draw[opacity=0.1] (a13) -- (a31);
\draw[opacity=0.1] (a13) -- (a32);
\draw[opacity=0.1] (a13) -- (a33);
\draw[opacity=0.1] (a13) -- (a34);
\draw[opacity=0.1] (a13) -- (a35);
\draw[opacity=0.1] (a13) -- (a36);
\draw[opacity=0.1] (a13) -- (a37);
\draw[opacity=0.1] (a13) -- (a38);
\draw[opacity=0.1] (a13) -- (a39);
\draw[opacity=0.1] (a13) -- (a40);
\draw[opacity=0.1] (a13) -- (a41);
\draw[opacity=0.1] (a13) -- (a42);
\draw[opacity=0.1] (a13) -- (a43);
\draw[opacity=0.1] (a13) -- (a44);
\draw[opacity=0.1] (a13) -- (a45);
\draw[opacity=0.1] (a13) -- (a46);
\draw[opacity=0.1] (a13) -- (a47);
\draw[opacity=0.1] (a13) -- (a48);
\draw[opacity=0.1] (a13) -- (a49);
\draw[opacity=0.1] (a13) -- (a50);
\draw[opacity=0.1] (a14) -- (a15);
\draw[opacity=0.1] (a14) -- (a16);
\draw[opacity=0.1] (a14) -- (a17);
\draw[opacity=0.1] (a14) -- (a18);
\draw[opacity=0.1] (a14) -- (a19);
\draw[opacity=0.1] (a14) -- (a20);
\draw[opacity=0.1] (a14) -- (a21);
\draw[opacity=0.1] (a14) -- (a22);
\draw[opacity=0.1] (a14) -- (a23);
\draw[opacity=0.1] (a14) -- (a24);
\draw[opacity=0.1] (a14) -- (a25);
\draw[opacity=0.1] (a14) -- (a26);
\draw[opacity=0.1] (a14) -- (a27);
\draw[opacity=0.1] (a14) -- (a28);
\draw[opacity=0.1] (a14) -- (a29);
\draw[opacity=0.1] (a14) -- (a30);
\draw[opacity=0.1] (a14) -- (a31);
\draw[opacity=0.1] (a14) -- (a32);
\draw[opacity=0.1] (a14) -- (a33);
\draw[opacity=0.1] (a14) -- (a34);
\draw[opacity=0.1] (a14) -- (a35);
\draw[opacity=0.1] (a14) -- (a36);
\draw[opacity=0.1] (a14) -- (a37);
\draw[opacity=0.1] (a14) -- (a38);
\draw[opacity=0.1] (a14) -- (a39);
\draw[opacity=0.1] (a14) -- (a40);
\draw[opacity=0.1] (a14) -- (a41);
\draw[opacity=0.1] (a14) -- (a42);
\draw[opacity=0.1] (a14) -- (a43);
\draw[opacity=0.1] (a14) -- (a44);
\draw[opacity=0.1] (a14) -- (a45);
\draw[opacity=0.1] (a14) -- (a46);
\draw[opacity=0.1] (a14) -- (a47);
\draw[opacity=0.1] (a14) -- (a48);
\draw[opacity=0.1] (a14) -- (a49);
\draw[opacity=0.1] (a14) -- (a50);
\draw[opacity=0.1] (a15) -- (a16);
\draw[opacity=0.1] (a15) -- (a17);
\draw[opacity=0.1] (a15) -- (a18);
\draw[opacity=0.1] (a15) -- (a19);
\draw[opacity=0.1] (a15) -- (a20);
\draw[opacity=0.1] (a15) -- (a21);
\draw[opacity=0.1] (a15) -- (a22);
\draw[opacity=0.1] (a15) -- (a23);
\draw[opacity=0.1] (a15) -- (a24);
\draw[opacity=0.1] (a15) -- (a25);
\draw[opacity=0.1] (a15) -- (a26);
\draw[opacity=0.1] (a15) -- (a27);
\draw[opacity=0.1] (a15) -- (a28);
\draw[opacity=0.1] (a15) -- (a29);
\draw[opacity=0.1] (a15) -- (a30);
\draw[opacity=0.1] (a15) -- (a31);
\draw[opacity=0.1] (a15) -- (a32);
\draw[opacity=0.1] (a15) -- (a33);
\draw[opacity=0.1] (a15) -- (a34);
\draw[opacity=0.1] (a15) -- (a35);
\draw[opacity=0.1] (a15) -- (a36);
\draw[opacity=0.1] (a15) -- (a37);
\draw[opacity=0.1] (a15) -- (a38);
\draw[opacity=0.1] (a15) -- (a39);
\draw[opacity=0.1] (a15) -- (a40);
\draw[opacity=0.1] (a15) -- (a41);
\draw[opacity=0.1] (a15) -- (a42);
\draw[opacity=0.1] (a15) -- (a43);
\draw[opacity=0.1] (a15) -- (a44);
\draw[opacity=0.1] (a15) -- (a45);
\draw[opacity=0.1] (a15) -- (a46);
\draw[opacity=0.1] (a15) -- (a47);
\draw[opacity=0.1] (a15) -- (a48);
\draw[opacity=0.1] (a15) -- (a49);
\draw[opacity=0.1] (a15) -- (a50);
\draw[opacity=0.1] (a16) -- (a17);
\draw[opacity=0.1] (a16) -- (a18);
\draw[opacity=0.1] (a16) -- (a19);
\draw[opacity=0.1] (a16) -- (a20);
\draw[opacity=0.1] (a16) -- (a21);
\draw[opacity=0.1] (a16) -- (a22);
\draw[opacity=0.1] (a16) -- (a23);
\draw[opacity=0.1] (a16) -- (a24);
\draw[opacity=0.1] (a16) -- (a25);
\draw[opacity=0.1] (a16) -- (a26);
\draw[opacity=0.1] (a16) -- (a27);
\draw[opacity=0.1] (a16) -- (a28);
\draw[opacity=0.1] (a16) -- (a29);
\draw[opacity=0.1] (a16) -- (a30);
\draw[opacity=0.1] (a16) -- (a31);
\draw[opacity=0.1] (a16) -- (a32);
\draw[opacity=0.1] (a16) -- (a33);
\draw[opacity=0.1] (a16) -- (a34);
\draw[opacity=0.1] (a16) -- (a35);
\draw[opacity=0.1] (a16) -- (a36);
\draw[opacity=0.1] (a16) -- (a37);
\draw[opacity=0.1] (a16) -- (a38);
\draw[opacity=0.1] (a16) -- (a39);
\draw[opacity=0.1] (a16) -- (a40);
\draw[opacity=0.1] (a16) -- (a41);
\draw[opacity=0.1] (a16) -- (a42);
\draw[opacity=0.1] (a16) -- (a43);
\draw[opacity=0.1] (a16) -- (a44);
\draw[opacity=0.1] (a16) -- (a45);
\draw[opacity=0.1] (a16) -- (a46);
\draw[opacity=0.1] (a16) -- (a47);
\draw[opacity=0.1] (a16) -- (a48);
\draw[opacity=0.1] (a16) -- (a49);
\draw[opacity=0.1] (a16) -- (a50);
\draw[opacity=0.1] (a17) -- (a18);
\draw[opacity=0.1] (a17) -- (a19);
\draw[opacity=0.1] (a17) -- (a20);
\draw[opacity=0.1] (a17) -- (a21);
\draw[opacity=0.1] (a17) -- (a22);
\draw[opacity=0.1] (a17) -- (a23);
\draw[opacity=0.1] (a17) -- (a24);
\draw[opacity=0.1] (a17) -- (a25);
\draw[opacity=0.1] (a17) -- (a26);
\draw[opacity=0.1] (a17) -- (a27);
\draw[opacity=0.1] (a17) -- (a28);
\draw[opacity=0.1] (a17) -- (a29);
\draw[opacity=0.1] (a17) -- (a30);
\draw[opacity=0.1] (a17) -- (a31);
\draw[opacity=0.1] (a17) -- (a32);
\draw[opacity=0.1] (a17) -- (a33);
\draw[opacity=0.1] (a17) -- (a34);
\draw[opacity=0.1] (a17) -- (a35);
\draw[opacity=0.1] (a17) -- (a36);
\draw[opacity=0.1] (a17) -- (a37);
\draw[opacity=0.1] (a17) -- (a38);
\draw[opacity=0.1] (a17) -- (a39);
\draw[opacity=0.1] (a17) -- (a40);
\draw[opacity=0.1] (a17) -- (a41);
\draw[opacity=0.1] (a17) -- (a42);
\draw[opacity=0.1] (a17) -- (a43);
\draw[opacity=0.1] (a17) -- (a44);
\draw[opacity=0.1] (a17) -- (a45);
\draw[opacity=0.1] (a17) -- (a46);
\draw[opacity=0.1] (a17) -- (a47);
\draw[opacity=0.1] (a17) -- (a48);
\draw[opacity=0.1] (a17) -- (a49);
\draw[opacity=0.1] (a17) -- (a50);
\draw[opacity=0.1] (a18) -- (a19);
\draw[opacity=0.1] (a18) -- (a20);
\draw[opacity=0.1] (a18) -- (a21);
\draw[opacity=0.1] (a18) -- (a22);
\draw[opacity=0.1] (a18) -- (a23);
\draw[opacity=0.1] (a18) -- (a24);
\draw[opacity=0.1] (a18) -- (a25);
\draw[opacity=0.1] (a18) -- (a26);
\draw[opacity=0.1] (a18) -- (a27);
\draw[opacity=0.1] (a18) -- (a28);
\draw[opacity=0.1] (a18) -- (a29);
\draw[opacity=0.1] (a18) -- (a30);
\draw[opacity=0.1] (a18) -- (a31);
\draw[opacity=0.1] (a18) -- (a32);
\draw[opacity=0.1] (a18) -- (a33);
\draw[opacity=0.1] (a18) -- (a34);
\draw[opacity=0.1] (a18) -- (a35);
\draw[opacity=0.1] (a18) -- (a36);
\draw[opacity=0.1] (a18) -- (a37);
\draw[opacity=0.1] (a18) -- (a38);
\draw[opacity=0.1] (a18) -- (a39);
\draw[opacity=0.1] (a18) -- (a40);
\draw[opacity=0.1] (a18) -- (a41);
\draw[opacity=0.1] (a18) -- (a42);
\draw[opacity=0.1] (a18) -- (a43);
\draw[opacity=0.1] (a18) -- (a44);
\draw[opacity=0.1] (a18) -- (a45);
\draw[opacity=0.1] (a18) -- (a46);
\draw[opacity=0.1] (a18) -- (a47);
\draw[opacity=0.1] (a18) -- (a48);
\draw[opacity=0.1] (a18) -- (a49);
\draw[opacity=0.1] (a18) -- (a50);
\draw[opacity=0.1] (a19) -- (a20);
\draw[opacity=0.1] (a19) -- (a21);
\draw[opacity=0.1] (a19) -- (a22);
\draw[opacity=0.1] (a19) -- (a23);
\draw[opacity=0.1] (a19) -- (a24);
\draw[opacity=0.1] (a19) -- (a25);
\draw[opacity=0.1] (a19) -- (a26);
\draw[opacity=0.1] (a19) -- (a27);
\draw[opacity=0.1] (a19) -- (a28);
\draw[opacity=0.1] (a19) -- (a29);
\draw[opacity=0.1] (a19) -- (a30);
\draw[opacity=0.1] (a19) -- (a31);
\draw[opacity=0.1] (a19) -- (a32);
\draw[opacity=0.1] (a19) -- (a33);
\draw[opacity=0.1] (a19) -- (a34);
\draw[opacity=0.1] (a19) -- (a35);
\draw[opacity=0.1] (a19) -- (a36);
\draw[opacity=0.1] (a19) -- (a37);
\draw[opacity=0.1] (a19) -- (a38);
\draw[opacity=0.1] (a19) -- (a39);
\draw[opacity=0.1] (a19) -- (a40);
\draw[opacity=0.1] (a19) -- (a41);
\draw[opacity=0.1] (a19) -- (a42);
\draw[opacity=0.1] (a19) -- (a43);
\draw[opacity=0.1] (a19) -- (a44);
\draw[opacity=0.1] (a19) -- (a45);
\draw[opacity=0.1] (a19) -- (a46);
\draw[opacity=0.1] (a19) -- (a47);
\draw[opacity=0.1] (a19) -- (a48);
\draw[opacity=0.1] (a19) -- (a49);
\draw[opacity=0.1] (a19) -- (a50);
\draw[opacity=0.1] (a20) -- (a21);
\draw[opacity=0.1] (a20) -- (a22);
\draw[opacity=0.1] (a20) -- (a23);
\draw[opacity=0.1] (a20) -- (a24);
\draw[opacity=0.1] (a20) -- (a25);
\draw[opacity=0.1] (a20) -- (a26);
\draw[opacity=0.1] (a20) -- (a27);
\draw[opacity=0.1] (a20) -- (a28);
\draw[opacity=0.1] (a20) -- (a29);
\draw[opacity=0.1] (a20) -- (a30);
\draw[opacity=0.1] (a20) -- (a31);
\draw[opacity=0.1] (a20) -- (a32);
\draw[opacity=0.1] (a20) -- (a33);
\draw[opacity=0.1] (a20) -- (a34);
\draw[opacity=0.1] (a20) -- (a35);
\draw[opacity=0.1] (a20) -- (a36);
\draw[opacity=0.1] (a20) -- (a37);
\draw[opacity=0.1] (a20) -- (a38);
\draw[opacity=0.1] (a20) -- (a39);
\draw[opacity=0.1] (a20) -- (a40);
\draw[opacity=0.1] (a20) -- (a41);
\draw[opacity=0.1] (a20) -- (a42);
\draw[opacity=0.1] (a20) -- (a43);
\draw[opacity=0.1] (a20) -- (a44);
\draw[opacity=0.1] (a20) -- (a45);
\draw[opacity=0.1] (a20) -- (a46);
\draw[opacity=0.1] (a20) -- (a47);
\draw[opacity=0.1] (a20) -- (a48);
\draw[opacity=0.1] (a20) -- (a49);
\draw[opacity=0.1] (a20) -- (a50);
\draw[opacity=0.1] (a21) -- (a22);
\draw[opacity=0.1] (a21) -- (a23);
\draw[opacity=0.1] (a21) -- (a24);
\draw[opacity=0.1] (a21) -- (a25);
\draw[opacity=0.1] (a21) -- (a26);
\draw[opacity=0.1] (a21) -- (a27);
\draw[opacity=0.1] (a21) -- (a28);
\draw[opacity=0.1] (a21) -- (a29);
\draw[opacity=0.1] (a21) -- (a30);
\draw[opacity=0.1] (a21) -- (a31);
\draw[opacity=0.1] (a21) -- (a32);
\draw[opacity=0.1] (a21) -- (a33);
\draw[opacity=0.1] (a21) -- (a34);
\draw[opacity=0.1] (a21) -- (a35);
\draw[opacity=0.1] (a21) -- (a36);
\draw[opacity=0.1] (a21) -- (a37);
\draw[opacity=0.1] (a21) -- (a38);
\draw[opacity=0.1] (a21) -- (a39);
\draw[opacity=0.1] (a21) -- (a40);
\draw[opacity=0.1] (a21) -- (a41);
\draw[opacity=0.1] (a21) -- (a42);
\draw[opacity=0.1] (a21) -- (a43);
\draw[opacity=0.1] (a21) -- (a44);
\draw[opacity=0.1] (a21) -- (a45);
\draw[opacity=0.1] (a21) -- (a46);
\draw[opacity=0.1] (a21) -- (a47);
\draw[opacity=0.1] (a21) -- (a48);
\draw[opacity=0.1] (a21) -- (a49);
\draw[opacity=0.1] (a21) -- (a50);
\draw[opacity=0.1] (a22) -- (a23);
\draw[opacity=0.1] (a22) -- (a24);
\draw[opacity=0.1] (a22) -- (a25);
\draw[opacity=0.1] (a22) -- (a26);
\draw[opacity=0.1] (a22) -- (a27);
\draw[opacity=0.1] (a22) -- (a28);
\draw[opacity=0.1] (a22) -- (a29);
\draw[opacity=0.1] (a22) -- (a30);
\draw[opacity=0.1] (a22) -- (a31);
\draw[opacity=0.1] (a22) -- (a32);
\draw[opacity=0.1] (a22) -- (a33);
\draw[opacity=0.1] (a22) -- (a34);
\draw[opacity=0.1] (a22) -- (a35);
\draw[opacity=0.1] (a22) -- (a36);
\draw[opacity=0.1] (a22) -- (a37);
\draw[opacity=0.1] (a22) -- (a38);
\draw[opacity=0.1] (a22) -- (a39);
\draw[opacity=0.1] (a22) -- (a40);
\draw[opacity=0.1] (a22) -- (a41);
\draw[opacity=0.1] (a22) -- (a42);
\draw[opacity=0.1] (a22) -- (a43);
\draw[opacity=0.1] (a22) -- (a44);
\draw[opacity=0.1] (a22) -- (a45);
\draw[opacity=0.1] (a22) -- (a46);
\draw[opacity=0.1] (a22) -- (a47);
\draw[opacity=0.1] (a22) -- (a48);
\draw[opacity=0.1] (a22) -- (a49);
\draw[opacity=0.1] (a22) -- (a50);
\draw[opacity=0.1] (a23) -- (a24);
\draw[opacity=0.1] (a23) -- (a25);
\draw[opacity=0.1] (a23) -- (a26);
\draw[opacity=0.1] (a23) -- (a27);
\draw[opacity=0.1] (a23) -- (a28);
\draw[opacity=0.1] (a23) -- (a29);
\draw[opacity=0.1] (a23) -- (a30);
\draw[opacity=0.1] (a23) -- (a31);
\draw[opacity=0.1] (a23) -- (a32);
\draw[opacity=0.1] (a23) -- (a33);
\draw[opacity=0.1] (a23) -- (a34);
\draw[opacity=0.1] (a23) -- (a35);
\draw[opacity=0.1] (a23) -- (a36);
\draw[opacity=0.1] (a23) -- (a37);
\draw[opacity=0.1] (a23) -- (a38);
\draw[opacity=0.1] (a23) -- (a39);
\draw[opacity=0.1] (a23) -- (a40);
\draw[opacity=0.1] (a23) -- (a41);
\draw[opacity=0.1] (a23) -- (a42);
\draw[opacity=0.1] (a23) -- (a43);
\draw[opacity=0.1] (a23) -- (a44);
\draw[opacity=0.1] (a23) -- (a45);
\draw[opacity=0.1] (a23) -- (a46);
\draw[opacity=0.1] (a23) -- (a47);
\draw[opacity=0.1] (a23) -- (a48);
\draw[opacity=0.1] (a23) -- (a49);
\draw[opacity=0.1] (a23) -- (a50);
\draw[opacity=0.1] (a24) -- (a25);
\draw[opacity=0.1] (a24) -- (a26);
\draw[opacity=0.1] (a24) -- (a27);
\draw[opacity=0.1] (a24) -- (a28);
\draw[opacity=0.1] (a24) -- (a29);
\draw[opacity=0.1] (a24) -- (a30);
\draw[opacity=0.1] (a24) -- (a31);
\draw[opacity=0.1] (a24) -- (a32);
\draw[opacity=0.1] (a24) -- (a33);
\draw[opacity=0.1] (a24) -- (a34);
\draw[opacity=0.1] (a24) -- (a35);
\draw[opacity=0.1] (a24) -- (a36);
\draw[opacity=0.1] (a24) -- (a37);
\draw[opacity=0.1] (a24) -- (a38);
\draw[opacity=0.1] (a24) -- (a39);
\draw[opacity=0.1] (a24) -- (a40);
\draw[opacity=0.1] (a24) -- (a41);
\draw[opacity=0.1] (a24) -- (a42);
\draw[opacity=0.1] (a24) -- (a43);
\draw[opacity=0.1] (a24) -- (a44);
\draw[opacity=0.1] (a24) -- (a45);
\draw[opacity=0.1] (a24) -- (a46);
\draw[opacity=0.1] (a24) -- (a47);
\draw[opacity=0.1] (a24) -- (a48);
\draw[opacity=0.1] (a24) -- (a49);
\draw[opacity=0.1] (a24) -- (a50);
\draw[opacity=0.1] (a25) -- (a26);
\draw[opacity=0.1] (a25) -- (a27);
\draw[opacity=0.1] (a25) -- (a28);
\draw[opacity=0.1] (a25) -- (a29);
\draw[opacity=0.1] (a25) -- (a30);
\draw[opacity=0.1] (a25) -- (a31);
\draw[opacity=0.1] (a25) -- (a32);
\draw[opacity=0.1] (a25) -- (a33);
\draw[opacity=0.1] (a25) -- (a34);
\draw[opacity=0.1] (a25) -- (a35);
\draw[opacity=0.1] (a25) -- (a36);
\draw[opacity=0.1] (a25) -- (a37);
\draw[opacity=0.1] (a25) -- (a38);
\draw[opacity=0.1] (a25) -- (a39);
\draw[opacity=0.1] (a25) -- (a40);
\draw[opacity=0.1] (a25) -- (a41);
\draw[opacity=0.1] (a25) -- (a42);
\draw[opacity=0.1] (a25) -- (a43);
\draw[opacity=0.1] (a25) -- (a44);
\draw[opacity=0.1] (a25) -- (a45);
\draw[opacity=0.1] (a25) -- (a46);
\draw[opacity=0.1] (a25) -- (a47);
\draw[opacity=0.1] (a25) -- (a48);
\draw[opacity=0.1] (a25) -- (a49);
\draw[opacity=0.1] (a25) -- (a50);
\draw[opacity=0.1] (a26) -- (a27);
\draw[opacity=0.1] (a26) -- (a28);
\draw[opacity=0.1] (a26) -- (a29);
\draw[opacity=0.1] (a26) -- (a30);
\draw[opacity=0.1] (a26) -- (a31);
\draw[opacity=0.1] (a26) -- (a32);
\draw[opacity=0.1] (a26) -- (a33);
\draw[opacity=0.1] (a26) -- (a34);
\draw[opacity=0.1] (a26) -- (a35);
\draw[opacity=0.1] (a26) -- (a36);
\draw[opacity=0.1] (a26) -- (a37);
\draw[opacity=0.1] (a26) -- (a38);
\draw[opacity=0.1] (a26) -- (a39);
\draw[opacity=0.1] (a26) -- (a40);
\draw[opacity=0.1] (a26) -- (a41);
\draw[opacity=0.1] (a26) -- (a42);
\draw[opacity=0.1] (a26) -- (a43);
\draw[opacity=0.1] (a26) -- (a44);
\draw[opacity=0.1] (a26) -- (a45);
\draw[opacity=0.1] (a26) -- (a46);
\draw[opacity=0.1] (a26) -- (a47);
\draw[opacity=0.1] (a26) -- (a48);
\draw[opacity=0.1] (a26) -- (a49);
\draw[opacity=0.1] (a26) -- (a50);
\draw[opacity=0.1] (a27) -- (a28);
\draw[opacity=0.1] (a27) -- (a29);
\draw[opacity=0.1] (a27) -- (a30);
\draw[opacity=0.1] (a27) -- (a31);
\draw[opacity=0.1] (a27) -- (a32);
\draw[opacity=0.1] (a27) -- (a33);
\draw[opacity=0.1] (a27) -- (a34);
\draw[opacity=0.1] (a27) -- (a35);
\draw[opacity=0.1] (a27) -- (a36);
\draw[opacity=0.1] (a27) -- (a37);
\draw[opacity=0.1] (a27) -- (a38);
\draw[opacity=0.1] (a27) -- (a39);
\draw[opacity=0.1] (a27) -- (a40);
\draw[opacity=0.1] (a27) -- (a41);
\draw[opacity=0.1] (a27) -- (a42);
\draw[opacity=0.1] (a27) -- (a43);
\draw[opacity=0.1] (a27) -- (a44);
\draw[opacity=0.1] (a27) -- (a45);
\draw[opacity=0.1] (a27) -- (a46);
\draw[opacity=0.1] (a27) -- (a47);
\draw[opacity=0.1] (a27) -- (a48);
\draw[opacity=0.1] (a27) -- (a49);
\draw[opacity=0.1] (a27) -- (a50);
\draw[opacity=0.1] (a28) -- (a29);
\draw[opacity=0.1] (a28) -- (a30);
\draw[opacity=0.1] (a28) -- (a31);
\draw[opacity=0.1] (a28) -- (a32);
\draw[opacity=0.1] (a28) -- (a33);
\draw[opacity=0.1] (a28) -- (a34);
\draw[opacity=0.1] (a28) -- (a35);
\draw[opacity=0.1] (a28) -- (a36);
\draw[opacity=0.1] (a28) -- (a37);
\draw[opacity=0.1] (a28) -- (a38);
\draw[opacity=0.1] (a28) -- (a39);
\draw[opacity=0.1] (a28) -- (a40);
\draw[opacity=0.1] (a28) -- (a41);
\draw[opacity=0.1] (a28) -- (a42);
\draw[opacity=0.1] (a28) -- (a43);
\draw[opacity=0.1] (a28) -- (a44);
\draw[opacity=0.1] (a28) -- (a45);
\draw[opacity=0.1] (a28) -- (a46);
\draw[opacity=0.1] (a28) -- (a47);
\draw[opacity=0.1] (a28) -- (a48);
\draw[opacity=0.1] (a28) -- (a49);
\draw[opacity=0.1] (a28) -- (a50);
\draw[opacity=0.1] (a29) -- (a30);
\draw[opacity=0.1] (a29) -- (a31);
\draw[opacity=0.1] (a29) -- (a32);
\draw[opacity=0.1] (a29) -- (a33);
\draw[opacity=0.1] (a29) -- (a34);
\draw[opacity=0.1] (a29) -- (a35);
\draw[opacity=0.1] (a29) -- (a36);
\draw[opacity=0.1] (a29) -- (a37);
\draw[opacity=0.1] (a29) -- (a38);
\draw[opacity=0.1] (a29) -- (a39);
\draw[opacity=0.1] (a29) -- (a40);
\draw[opacity=0.1] (a29) -- (a41);
\draw[opacity=0.1] (a29) -- (a42);
\draw[opacity=0.1] (a29) -- (a43);
\draw[opacity=0.1] (a29) -- (a44);
\draw[opacity=0.1] (a29) -- (a45);
\draw[opacity=0.1] (a29) -- (a46);
\draw[opacity=0.1] (a29) -- (a47);
\draw[opacity=0.1] (a29) -- (a48);
\draw[opacity=0.1] (a29) -- (a49);
\draw[opacity=0.1] (a29) -- (a50);
\draw[opacity=0.1] (a30) -- (a31);
\draw[opacity=0.1] (a30) -- (a32);
\draw[opacity=0.1] (a30) -- (a33);
\draw[opacity=0.1] (a30) -- (a34);
\draw[opacity=0.1] (a30) -- (a35);
\draw[opacity=0.1] (a30) -- (a36);
\draw[opacity=0.1] (a30) -- (a37);
\draw[opacity=0.1] (a30) -- (a38);
\draw[opacity=0.1] (a30) -- (a39);
\draw[opacity=0.1] (a30) -- (a40);
\draw[opacity=0.1] (a30) -- (a41);
\draw[opacity=0.1] (a30) -- (a42);
\draw[opacity=0.1] (a30) -- (a43);
\draw[opacity=0.1] (a30) -- (a44);
\draw[opacity=0.1] (a30) -- (a45);
\draw[opacity=0.1] (a30) -- (a46);
\draw[opacity=0.1] (a30) -- (a47);
\draw[opacity=0.1] (a30) -- (a48);
\draw[opacity=0.1] (a30) -- (a49);
\draw[opacity=0.1] (a30) -- (a50);
\draw[opacity=0.1] (a31) -- (a32);
\draw[opacity=0.1] (a31) -- (a33);
\draw[opacity=0.1] (a31) -- (a34);
\draw[opacity=0.1] (a31) -- (a35);
\draw[opacity=0.1] (a31) -- (a36);
\draw[opacity=0.1] (a31) -- (a37);
\draw[opacity=0.1] (a31) -- (a38);
\draw[opacity=0.1] (a31) -- (a39);
\draw[opacity=0.1] (a31) -- (a40);
\draw[opacity=0.1] (a31) -- (a41);
\draw[opacity=0.1] (a31) -- (a42);
\draw[opacity=0.1] (a31) -- (a43);
\draw[opacity=0.1] (a31) -- (a44);
\draw[opacity=0.1] (a31) -- (a45);
\draw[opacity=0.1] (a31) -- (a46);
\draw[opacity=0.1] (a31) -- (a47);
\draw[opacity=0.1] (a31) -- (a48);
\draw[opacity=0.1] (a31) -- (a49);
\draw[opacity=0.1] (a31) -- (a50);
\draw[opacity=0.1] (a32) -- (a33);
\draw[opacity=0.1] (a32) -- (a34);
\draw[opacity=0.1] (a32) -- (a35);
\draw[opacity=0.1] (a32) -- (a36);
\draw[opacity=0.1] (a32) -- (a37);
\draw[opacity=0.1] (a32) -- (a38);
\draw[opacity=0.1] (a32) -- (a39);
\draw[opacity=0.1] (a32) -- (a40);
\draw[opacity=0.1] (a32) -- (a41);
\draw[opacity=0.1] (a32) -- (a42);
\draw[opacity=0.1] (a32) -- (a43);
\draw[opacity=0.1] (a32) -- (a44);
\draw[opacity=0.1] (a32) -- (a45);
\draw[opacity=0.1] (a32) -- (a46);
\draw[opacity=0.1] (a32) -- (a47);
\draw[opacity=0.1] (a32) -- (a48);
\draw[opacity=0.1] (a32) -- (a49);
\draw[opacity=0.1] (a32) -- (a50);
\draw[opacity=0.1] (a33) -- (a34);
\draw[opacity=0.1] (a33) -- (a35);
\draw[opacity=0.1] (a33) -- (a36);
\draw[opacity=0.1] (a33) -- (a37);
\draw[opacity=0.1] (a33) -- (a38);
\draw[opacity=0.1] (a33) -- (a39);
\draw[opacity=0.1] (a33) -- (a40);
\draw[opacity=0.1] (a33) -- (a41);
\draw[opacity=0.1] (a33) -- (a42);
\draw[opacity=0.1] (a33) -- (a43);
\draw[opacity=0.1] (a33) -- (a44);
\draw[opacity=0.1] (a33) -- (a45);
\draw[opacity=0.1] (a33) -- (a46);
\draw[opacity=0.1] (a33) -- (a47);
\draw[opacity=0.1] (a33) -- (a48);
\draw[opacity=0.1] (a33) -- (a49);
\draw[opacity=0.1] (a33) -- (a50);
\draw[opacity=0.1] (a34) -- (a35);
\draw[opacity=0.1] (a34) -- (a36);
\draw[opacity=0.1] (a34) -- (a37);
\draw[opacity=0.1] (a34) -- (a38);
\draw[opacity=0.1] (a34) -- (a39);
\draw[opacity=0.1] (a34) -- (a40);
\draw[opacity=0.1] (a34) -- (a41);
\draw[opacity=0.1] (a34) -- (a42);
\draw[opacity=0.1] (a34) -- (a43);
\draw[opacity=0.1] (a34) -- (a44);
\draw[opacity=0.1] (a34) -- (a45);
\draw[opacity=0.1] (a34) -- (a46);
\draw[opacity=0.1] (a34) -- (a47);
\draw[opacity=0.1] (a34) -- (a48);
\draw[opacity=0.1] (a34) -- (a49);
\draw[opacity=0.1] (a34) -- (a50);
\draw[opacity=0.1] (a35) -- (a36);
\draw[opacity=0.1] (a35) -- (a37);
\draw[opacity=0.1] (a35) -- (a38);
\draw[opacity=0.1] (a35) -- (a39);
\draw[opacity=0.1] (a35) -- (a40);
\draw[opacity=0.1] (a35) -- (a41);
\draw[opacity=0.1] (a35) -- (a42);
\draw[opacity=0.1] (a35) -- (a43);
\draw[opacity=0.1] (a35) -- (a44);
\draw[opacity=0.1] (a35) -- (a45);
\draw[opacity=0.1] (a35) -- (a46);
\draw[opacity=0.1] (a35) -- (a47);
\draw[opacity=0.1] (a35) -- (a48);
\draw[opacity=0.1] (a35) -- (a49);
\draw[opacity=0.1] (a35) -- (a50);
\draw[opacity=0.1] (a36) -- (a37);
\draw[opacity=0.1] (a36) -- (a38);
\draw[opacity=0.1] (a36) -- (a39);
\draw[opacity=0.1] (a36) -- (a40);
\draw[opacity=0.1] (a36) -- (a41);
\draw[opacity=0.1] (a36) -- (a42);
\draw[opacity=0.1] (a36) -- (a43);
\draw[opacity=0.1] (a36) -- (a44);
\draw[opacity=0.1] (a36) -- (a45);
\draw[opacity=0.1] (a36) -- (a46);
\draw[opacity=0.1] (a36) -- (a47);
\draw[opacity=0.1] (a36) -- (a48);
\draw[opacity=0.1] (a36) -- (a49);
\draw[opacity=0.1] (a36) -- (a50);
\draw[opacity=0.1] (a37) -- (a38);
\draw[opacity=0.1] (a37) -- (a39);
\draw[opacity=0.1] (a37) -- (a40);
\draw[opacity=0.1] (a37) -- (a41);
\draw[opacity=0.1] (a37) -- (a42);
\draw[opacity=0.1] (a37) -- (a43);
\draw[opacity=0.1] (a37) -- (a44);
\draw[opacity=0.1] (a37) -- (a45);
\draw[opacity=0.1] (a37) -- (a46);
\draw[opacity=0.1] (a37) -- (a47);
\draw[opacity=0.1] (a37) -- (a48);
\draw[opacity=0.1] (a37) -- (a49);
\draw[opacity=0.1] (a37) -- (a50);
\draw[opacity=0.1] (a38) -- (a39);
\draw[opacity=0.1] (a38) -- (a40);
\draw[opacity=0.1] (a38) -- (a41);
\draw[opacity=0.1] (a38) -- (a42);
\draw[opacity=0.1] (a38) -- (a43);
\draw[opacity=0.1] (a38) -- (a44);
\draw[opacity=0.1] (a38) -- (a45);
\draw[opacity=0.1] (a38) -- (a46);
\draw[opacity=0.1] (a38) -- (a47);
\draw[opacity=0.1] (a38) -- (a48);
\draw[opacity=0.1] (a38) -- (a49);
\draw[opacity=0.1] (a38) -- (a50);
\draw[opacity=0.1] (a39) -- (a40);
\draw[opacity=0.1] (a39) -- (a41);
\draw[opacity=0.1] (a39) -- (a42);
\draw[opacity=0.1] (a39) -- (a43);
\draw[opacity=0.1] (a39) -- (a44);
\draw[opacity=0.1] (a39) -- (a45);
\draw[opacity=0.1] (a39) -- (a46);
\draw[opacity=0.1] (a39) -- (a47);
\draw[opacity=0.1] (a39) -- (a48);
\draw[opacity=0.1] (a39) -- (a49);
\draw[opacity=0.1] (a39) -- (a50);
\draw[opacity=0.1] (a40) -- (a41);
\draw[opacity=0.1] (a40) -- (a42);
\draw[opacity=0.1] (a40) -- (a43);
\draw[opacity=0.1] (a40) -- (a44);
\draw[opacity=0.1] (a40) -- (a45);
\draw[opacity=0.1] (a40) -- (a46);
\draw[opacity=0.1] (a40) -- (a47);
\draw[opacity=0.1] (a40) -- (a48);
\draw[opacity=0.1] (a40) -- (a49);
\draw[opacity=0.1] (a40) -- (a50);
\draw[opacity=0.1] (a41) -- (a42);
\draw[opacity=0.1] (a41) -- (a43);
\draw[opacity=0.1] (a41) -- (a44);
\draw[opacity=0.1] (a41) -- (a45);
\draw[opacity=0.1] (a41) -- (a46);
\draw[opacity=0.1] (a41) -- (a47);
\draw[opacity=0.1] (a41) -- (a48);
\draw[opacity=0.1] (a41) -- (a49);
\draw[opacity=0.1] (a41) -- (a50);
\draw[opacity=0.1] (a42) -- (a43);
\draw[opacity=0.1] (a42) -- (a44);
\draw[opacity=0.1] (a42) -- (a45);
\draw[opacity=0.1] (a42) -- (a46);
\draw[opacity=0.1] (a42) -- (a47);
\draw[opacity=0.1] (a42) -- (a48);
\draw[opacity=0.1] (a42) -- (a49);
\draw[opacity=0.1] (a42) -- (a50);
\draw[opacity=0.1] (a43) -- (a44);
\draw[opacity=0.1] (a43) -- (a45);
\draw[opacity=0.1] (a43) -- (a46);
\draw[opacity=0.1] (a43) -- (a47);
\draw[opacity=0.1] (a43) -- (a48);
\draw[opacity=0.1] (a43) -- (a49);
\draw[opacity=0.1] (a43) -- (a50);
\draw[opacity=0.1] (a44) -- (a45);
\draw[opacity=0.1] (a44) -- (a46);
\draw[opacity=0.1] (a44) -- (a47);
\draw[opacity=0.1] (a44) -- (a48);
\draw[opacity=0.1] (a44) -- (a49);
\draw[opacity=0.1] (a44) -- (a50);
\draw[opacity=0.1] (a45) -- (a46);
\draw[opacity=0.1] (a45) -- (a47);
\draw[opacity=0.1] (a45) -- (a48);
\draw[opacity=0.1] (a45) -- (a49);
\draw[opacity=0.1] (a45) -- (a50);
\draw[opacity=0.1] (a46) -- (a47);
\draw[opacity=0.1] (a46) -- (a48);
\draw[opacity=0.1] (a46) -- (a49);
\draw[opacity=0.1] (a46) -- (a50);
\draw[opacity=0.1] (a47) -- (a48);
\draw[opacity=0.1] (a47) -- (a49);
\draw[opacity=0.1] (a47) -- (a50);
\draw[opacity=0.1] (a48) -- (a49);
\draw[opacity=0.1] (a48) -- (a50);
\draw[opacity=0.1] (a49) -- (a50);


\draw[black,fill=black,opacity=0.1] (5.609, -17.405) circle (3pt);
\draw[black,fill=black,opacity=0.1] (7.203, -16.023) circle (3pt);
\draw[black,fill=black,opacity=0.1] (7.408, -13.436) circle (3pt);
\draw[black,fill=black,opacity=0.1] (8.201, -15.903) circle (3pt);
\draw[black,fill=black,opacity=0.1] (9.413, -14.479) circle (3pt);
\draw[black,fill=black,opacity=0.1] (8.034, -14.671) circle (3pt);
\draw[black,fill=black,opacity=0.1] (5.782, -16.197) circle (3pt);
\draw[black,fill=black,opacity=0.1] (8.3, -15.502) circle (3pt);
\draw[black,fill=black,opacity=0.1] (6.82, -13.235) circle (3pt);
\draw[black,fill=black,opacity=0.1] (6.948, -14.783) circle (3pt);
\draw[black,fill=black,opacity=0.1] (7.929, -16.233) circle (3pt);
\draw[black,fill=black,opacity=0.1] (9.146, -15.252) circle (3pt);
\draw[black,fill=black,opacity=0.1] (8.229, -13.525) circle (3pt);
\draw[black,fill=black,opacity=0.1] (6.792, -15.951) circle (3pt);
\draw[black,fill=black,opacity=0.1] (5.454, -14.928) circle (3pt);
\draw[black,fill=black,opacity=0.1] (7.895, -14.563) circle (3pt);
\draw[black,fill=black,opacity=0.1] (6.593, -14.481) circle (3pt);
\draw[black,fill=black,opacity=0.1] (7.864, -14.199) circle (3pt);
\draw[black,fill=black,opacity=0.1] (7.898, -15.333) circle (3pt);
\draw[black,fill=black,opacity=0.1] (6.716, -15.242) circle (3pt);
\draw[black,fill=black,opacity=0.1] (9.07, -13.942) circle (3pt);
\draw[black,fill=black,opacity=0.1] (7.67, -15.752) circle (3pt);
\draw[black,fill=black,opacity=0.1] (5.521, -17.166) circle (3pt);
\draw[black,fill=black,opacity=0.1] (7.994, -14.642) circle (3pt);
\draw[black,fill=black,opacity=0.1] (7.482, -15.1) circle (3pt);
\draw[black,fill=black,opacity=0.1] (6.717, -14.949) circle (3pt);
\draw[black,fill=black,opacity=0.1] (7.289, -16.768) circle (3pt);
\draw[black,fill=black,opacity=0.1] (6.032, -16.07) circle (3pt);
\draw[black,fill=black,opacity=0.1] (7.081, -16.859) circle (3pt);
\draw[black,fill=black,opacity=0.1] (7.035, -14.118) circle (3pt);
\draw[black,fill=black,opacity=0.1] (6.653, -12.862) circle (3pt);
\draw[black,fill=black,opacity=0.1] (9, -14.443) circle (3pt);
\draw[black,fill=black,opacity=0.1] (6.601, -14.942) circle (3pt);
\draw[black,fill=black,opacity=0.1] (9.755, -17.123) circle (3pt);
\draw[black,fill=black,opacity=0.1] (7.614, -14.466) circle (3pt);
\draw[black,fill=black,opacity=0.1] (8.457, -16.344) circle (3pt);
\draw[black,fill=black,opacity=0.1] (8.348, -16.783) circle (3pt);
\draw[black,fill=black,opacity=0.1] (7.41, -14.015) circle (3pt);
\draw[black,fill=black,opacity=0.1] (8.028, -14.532) circle (3pt);
\draw[black,fill=black,opacity=0.1] (6.421, -15.57) circle (3pt);
\draw[black,fill=black,opacity=0.1] (7.742, -15.346) circle (3pt);
\draw[black,fill=black,opacity=0.1] (8.233, -15.285) circle (3pt);
\draw[black,fill=black,opacity=0.1] (9.689, -15.173) circle (3pt);
\draw[black,fill=black,opacity=0.1] (6.103, -16.629) circle (3pt);
\draw[black,fill=black,opacity=0.1] (8.932, -15.64) circle (3pt);
\draw[black,fill=black,opacity=0.1] (5.917, -16.148) circle (3pt);
\draw[black,fill=black,opacity=0.1] (5.99, -13.768) circle (3pt);
\draw[black,fill=black,opacity=0.1] (7.006, -15.963) circle (3pt);
\draw[black,fill=black,opacity=0.1] (9.282, -14.643) circle (3pt);
\draw[black,fill=black,opacity=0.1] (7.104, -15.938) circle (3pt);
\draw[black,fill=black,opacity=0.1] (6.649, -14.144) circle (3pt);
\draw[black,fill=black,opacity=0.1] (7.906, -14.464) circle (3pt);
\draw[black,fill=black,opacity=0.1] (8.904, -15.347) circle (3pt);
\draw[black,fill=black,opacity=0.1] (5.916, -14.809) circle (3pt);
\draw[black,fill=black,opacity=0.1] (6.088, -15.075) circle (3pt);
\draw[black,fill=black,opacity=0.1] (6.425, -16.642) circle (3pt);
\draw[black,fill=black,opacity=0.1] (7.571, -13.062) circle (3pt);
\draw[black,fill=black,opacity=0.1] (6.167, -14.949) circle (3pt);
\draw[black,fill=black,opacity=0.1] (8.923, -14.62) circle (3pt);
\draw[black,fill=black,opacity=0.1] (8.229, -15.665) circle (3pt);
\draw[black,fill=black,opacity=0.1] (8.936, -14.705) circle (3pt);
\draw[black,fill=black,opacity=0.1] (6.163, -15.855) circle (3pt);
\draw[black,fill=black,opacity=0.1] (7.232, -16.998) circle (3pt);
\draw[black,fill=black,opacity=0.1] (7.614, -14.389) circle (3pt);
\draw[black,fill=black,opacity=0.1] (5.495, -14.624) circle (3pt);
\draw[black,fill=black,opacity=0.1] (8.718, -14.242) circle (3pt);
\draw[black,fill=black,opacity=0.1] (6.9, -15.114) circle (3pt);
\draw[black,fill=black,opacity=0.1] (7.606, -13.225) circle (3pt);
\draw[black,fill=black,opacity=0.1] (8.704, -15.738) circle (3pt);
\draw[black,fill=black,opacity=0.1] (6.61, -14.374) circle (3pt);
\draw[black,fill=black,opacity=0.1] (9.751, -15.433) circle (3pt);
\draw[black,fill=black,opacity=0.1] (8.394, -13.81) circle (3pt);
\draw[black,fill=black,opacity=0.1] (7.675, -14.922) circle (3pt);
\draw[black,fill=black,opacity=0.1] (8.201, -17.086) circle (3pt);
\draw[black,fill=black,opacity=0.1] (8.429, -16.06) circle (3pt);
\draw[black,fill=black,opacity=0.1] (8.784, -13.069) circle (3pt);
\draw[black,fill=black,opacity=0.1] (8.262, -12.956) circle (3pt);
\draw[black,fill=black,opacity=0.1] (7.908, -14.267) circle (3pt);
\draw[black,fill=black,opacity=0.1] (8.131, -15.522) circle (3pt);
\draw[black,fill=black,opacity=0.1] (6.76, -14.169) circle (3pt);
\draw[black,fill=black,opacity=0.1] (7.996, -15.595) circle (3pt);
\draw[black,fill=black,opacity=0.1] (6.294, -14.337) circle (3pt);
\draw[black,fill=black,opacity=0.1] (8.977, -14.343) circle (3pt);
\draw[black,fill=black,opacity=0.1] (8.59, -15.208) circle (3pt);
\draw[black,fill=black,opacity=0.1] (7.96, -14.678) circle (3pt);
\draw[black,fill=black,opacity=0.1] (7.088, -16.005) circle (3pt);
\draw[black,fill=black,opacity=0.1] (6.919, -14.907) circle (3pt);
\draw[black,fill=black,opacity=0.1] (5.542, -15.621) circle (3pt);
\draw[black,fill=black,opacity=0.1] (7.969, -15.384) circle (3pt);
\draw[black,fill=black,opacity=0.1] (7.974, -16.629) circle (3pt);
\draw[black,fill=black,opacity=0.1] (7.542, -16.002) circle (3pt);
\draw[black,fill=black,opacity=0.1] (7.721, -14.538) circle (3pt);
\draw[black,fill=black,opacity=0.1] (7.107, -15.972) circle (3pt);
\draw[black,fill=black,opacity=0.1] (7.038, -15.518) circle (3pt);
\draw[black,fill=black,opacity=0.1] (6.593, -15.791) circle (3pt);
\draw[black,fill=black,opacity=0.1] (7.608, -15.1) circle (3pt);
\draw[black,fill=black,opacity=0.1] (8.604, -13.534) circle (3pt);
\draw[black,fill=black,opacity=0.1] (8.397, -13.543) circle (3pt);
\draw[black,fill=black,opacity=0.1] (8.152, -14.236) circle (3pt);
\draw[black,fill=black,opacity=0.1] (8.166, -16.603) circle (3pt);
\draw[black,fill=black,opacity=0.1] (5.887, -14.406) circle (3pt);
\draw[black,fill=black,opacity=0.1] (4.248, -16.827) circle (3pt);
\draw[black,fill=black,opacity=0.1] (7.09, -14.238) circle (3pt);
\draw[black,fill=black,opacity=0.1] (9.37, -16.7) circle (3pt);
\draw[black,fill=black,opacity=0.1] (8.026, -13.353) circle (3pt);
\draw[black,fill=black,opacity=0.1] (5.604, -13.119) circle (3pt);
\draw[black,fill=black,opacity=0.1] (7.544, -14.438) circle (3pt);
\draw[black,fill=black,opacity=0.1] (6.613, -17.007) circle (3pt);
\draw[black,fill=black,opacity=0.1] (8.303, -16.521) circle (3pt);
\draw[black,fill=black,opacity=0.1] (5.527, -14.177) circle (3pt);
\draw[black,fill=black,opacity=0.1] (7.27, -15.637) circle (3pt);
\draw[black,fill=black,opacity=0.1] (5.706, -13.355) circle (3pt);
\draw[black,fill=black,opacity=0.1] (8.256, -17.992) circle (3pt);
\draw[black,fill=black,opacity=0.1] (6.488, -14.913) circle (3pt);
\draw[black,fill=black,opacity=0.1] (6.951, -13.608) circle (3pt);
\draw[black,fill=black,opacity=0.1] (8.265, -14.224) circle (3pt);
\draw[black,fill=black,opacity=0.1] (6.887, -14.848) circle (3pt);
\draw[black,fill=black,opacity=0.1] (7.344, -15.441) circle (3pt);
\draw[black,fill=black,opacity=0.1] (8.305, -15.862) circle (3pt);
\draw[black,fill=black,opacity=0.1] (8.913, -14.062) circle (3pt);
\draw[black,fill=black,opacity=0.1] (6.477, -15.721) circle (3pt);
\draw[black,fill=black,opacity=0.1] (6.358, -17.288) circle (3pt);
\draw[black,fill=black,opacity=0.1] (7.51, -14.336) circle (3pt);
\draw[black,fill=black,opacity=0.1] (8.288, -14.616) circle (3pt);
\draw[black,fill=black,opacity=0.1] (8.997, -16.462) circle (3pt);
\draw[black,fill=black,opacity=0.1] (8.024, -15.776) circle (3pt);
\draw[black,fill=black,opacity=0.1] (8.038, -14.944) circle (3pt);
\draw[black,fill=black,opacity=0.1] (7.281, -14.756) circle (3pt);
\draw[black,fill=black,opacity=0.1] (8.678, -15.281) circle (3pt);
\draw[black,fill=black,opacity=0.1] (8.78, -13.925) circle (3pt);
\draw[black,fill=black,opacity=0.1] (9.639, -14.684) circle (3pt);
\draw[black,fill=black,opacity=0.1] (7.264, -13.588) circle (3pt);
\draw[black,fill=black,opacity=0.1] (7.208, -12.033) circle (3pt);
\draw[black,fill=black,opacity=0.1] (9.041, -14.15) circle (3pt);
\draw[black,fill=black,opacity=0.1] (8.477, -16.611) circle (3pt);
\draw[black,fill=black,opacity=0.1] (6.798, -15.536) circle (3pt);
\draw[black,fill=black,opacity=0.1] (8.513, -15.066) circle (3pt);
\draw[black,fill=black,opacity=0.1] (9.202, -15.459) circle (3pt);
\draw[black,fill=black,opacity=0.1] (7.785, -15.34) circle (3pt);
\draw[black,fill=black,opacity=0.1] (7.729, -15.734) circle (3pt);
\draw[black,fill=black,opacity=0.1] (7.412, -15.004) circle (3pt);
\draw[black,fill=black,opacity=0.1] (7.563, -14.873) circle (3pt);
\draw[black,fill=black,opacity=0.1] (6.392, -14.425) circle (3pt);
\draw[black,fill=black,opacity=0.1] (7.201, -15.421) circle (3pt);
\draw[black,fill=black,opacity=0.1] (9.033, -15.541) circle (3pt);
\draw[black,fill=black,opacity=0.1] (7.82, -13.673) circle (3pt);
\draw[black,fill=black,opacity=0.1] (5.049, -13.945) circle (3pt);
\draw[black,fill=black,opacity=0.1] (9.623, -15.587) circle (3pt);
\draw[black,fill=black,opacity=0.1] (7.992, -11.772) circle (3pt);
\draw[black,fill=black,opacity=0.1] (7.069, -14.427) circle (3pt);
\draw[black,fill=black,opacity=0.1] (6.135, -15.392) circle (3pt);
\draw[black,fill=black,opacity=0.1] (9.113, -15.696) circle (3pt);
\draw[black,fill=black,opacity=0.1] (9.108, -14.855) circle (3pt);
\draw[black,fill=black,opacity=0.1] (8.028, -15.939) circle (3pt);
\draw[black,fill=black,opacity=0.1] (6.915, -14.978) circle (3pt);
\draw[black,fill=black,opacity=0.1] (8.851, -12.984) circle (3pt);
\draw[black,fill=black,opacity=0.1] (8.399, -14.083) circle (3pt);
\draw[black,fill=black,opacity=0.1] (8.665, -14.858) circle (3pt);
\draw[black,fill=black,opacity=0.1] (8.025, -14.102) circle (3pt);
\draw[black,fill=black,opacity=0.1] (8.182, -15.559) circle (3pt);
\draw[black,fill=black,opacity=0.1] (7.419, -17.658) circle (3pt);
\draw[black,fill=black,opacity=0.1] (6.737, -13.073) circle (3pt);
\draw[black,fill=black,opacity=0.1] (6.949, -16.416) circle (3pt);
\draw[black,fill=black,opacity=0.1] (7.309, -14.577) circle (3pt);
\draw[black,fill=black,opacity=0.1] (8.401, -15.162) circle (3pt);
\draw[black,fill=black,opacity=0.1] (5.737, -16.68) circle (3pt);
\draw[black,fill=black,opacity=0.1] (8.007, -15.148) circle (3pt);
\draw[black,fill=black,opacity=0.1] (7.079, -14.453) circle (3pt);
\draw[black,fill=black,opacity=0.1] (8.714, -15.71) circle (3pt);
\draw[black,fill=black,opacity=0.1] (7.377, -15.826) circle (3pt);
\draw[black,fill=black,opacity=0.1] (6.347, -15.479) circle (3pt);
\draw[black,fill=black,opacity=0.1] (10.153, -14.874) circle (3pt);
\draw[black,fill=black,opacity=0.1] (8.225, -14.313) circle (3pt);
\draw[black,fill=black,opacity=0.1] (8.368, -13.649) circle (3pt);
\draw[black,fill=black,opacity=0.1] (7.264, -14.512) circle (3pt);
\draw[black,fill=black,opacity=0.1] (8.405, -17.018) circle (3pt);
\draw[black,fill=black,opacity=0.1] (7.709, -17.226) circle (3pt);
\draw[black,fill=black,opacity=0.1] (7.625, -16.528) circle (3pt);
\draw[black,fill=black,opacity=0.1] (6.464, -14.963) circle (3pt);
\draw[black,fill=black,opacity=0.1] (7.899, -15.325) circle (3pt);
\draw[black,fill=black,opacity=0.1] (7.524, -13.958) circle (3pt);
\draw[black,fill=black,opacity=0.1] (7.25, -15.144) circle (3pt);
\draw[black,fill=black,opacity=0.1] (10.396, -14.661) circle (3pt);
\draw[black,fill=black,opacity=0.1] (7.579, -15.613) circle (3pt);
\draw[black,fill=black,opacity=0.1] (7.89, -14.934) circle (3pt);
\draw[black,fill=black,opacity=0.1] (9.452, -15.805) circle (3pt);
\draw[black,fill=black,opacity=0.1] (7.292, -15.359) circle (3pt);
\draw[black,fill=black,opacity=0.1] (8.036, -14.199) circle (3pt);
\draw[black,fill=black,opacity=0.1] (6.754, -15.163) circle (3pt);
\draw[black,fill=black,opacity=0.1] (8.893, -14.108) circle (3pt);
\draw[black,fill=black,opacity=0.1] (8.469, -11.617) circle (3pt);
\draw[black,fill=black,opacity=0.1] (6.665, -13.636) circle (3pt);
\draw[black,fill=black,opacity=0.1] (8.119, -14.33) circle (3pt);
\draw[black,fill=black,opacity=0.1] (6.939, -17.621) circle (3pt);
\draw[black,fill=black,opacity=0.1] (8.361, -14.86) circle (3pt);
\draw[black,fill=black,opacity=0.1] (7.488, -13.481) circle (3pt);
\draw[black,fill=black,opacity=0.1] (7.344, -16.199) circle (3pt);
\draw[black,fill=black,opacity=0.1] (7.265, -16.174) circle (3pt);
\draw[black,fill=black,opacity=0.1] (8.069, -13.928) circle (3pt);
\draw[black,fill=black,opacity=0.1] (6.508, -14.421) circle (3pt);
\draw[black,fill=black,opacity=0.1] (8.751, -16.064) circle (3pt);
\draw[black,fill=black,opacity=0.1] (6.168, -13.984) circle (3pt);
\draw[black,fill=black,opacity=0.1] (6.078, -15.683) circle (3pt);
\draw[black,fill=black,opacity=0.1] (9.61, -17.813) circle (3pt);
\draw[black,fill=black,opacity=0.1] (7.375, -16.437) circle (3pt);
\draw[black,fill=black,opacity=0.1] (7.88, -14.058) circle (3pt);
\draw[black,fill=black,opacity=0.1] (7.53, -15.383) circle (3pt);
\draw[black,fill=black,opacity=0.1] (7.89, -13.135) circle (3pt);
\draw[black,fill=black,opacity=0.1] (7.738, -16.677) circle (3pt);
\draw[black,fill=black,opacity=0.1] (7.906, -13.878) circle (3pt);
\draw[black,fill=black,opacity=0.1] (6.799, -13.454) circle (3pt);
\draw[black,fill=black,opacity=0.1] (9.301, -16.476) circle (3pt);
\draw[black,fill=black,opacity=0.1] (6.62, -15.887) circle (3pt);
\draw[black,fill=black,opacity=0.1] (8.222, -13.377) circle (3pt);
\draw[black,fill=black,opacity=0.1] (7.545, -14.417) circle (3pt);
\draw[black,fill=black,opacity=0.1] (9.49, -14.556) circle (3pt);
\draw[black,fill=black,opacity=0.1] (8.969, -12.637) circle (3pt);
\draw[black,fill=black,opacity=0.1] (6.531, -15.789) circle (3pt);
\draw[black,fill=black,opacity=0.1] (7.789, -15.215) circle (3pt);
\draw[black,fill=black,opacity=0.1] (7.754, -14.057) circle (3pt);
\draw[black,fill=black,opacity=0.1] (7.878, -16.88) circle (3pt);
\draw[black,fill=black,opacity=0.1] (7.231, -15.469) circle (3pt);
\draw[black,fill=black,opacity=0.1] (5.832, -14.247) circle (3pt);
\draw[black,fill=black,opacity=0.1] (7.522, -17.475) circle (3pt);
\draw[black,fill=black,opacity=0.1] (9.823, -16.467) circle (3pt);
\draw[black,fill=black,opacity=0.1] (7.081, -16.567) circle (3pt);
\draw[black,fill=black,opacity=0.1] (8.656, -15.138) circle (3pt);
\draw[black,fill=black,opacity=0.1] (8.341, -13.149) circle (3pt);
\draw[black,fill=black,opacity=0.1] (6.742, -16.9) circle (3pt);
\draw[black,fill=black,opacity=0.1] (4.466, -14.77) circle (3pt);
\draw[black,fill=black,opacity=0.1] (9.123, -17.255) circle (3pt);
\draw[black,fill=black,opacity=0.1] (8.525, -14.266) circle (3pt);
\draw[black,fill=black,opacity=0.1] (7.631, -14.385) circle (3pt);
\draw[black,fill=black,opacity=0.1] (6.957, -16.958) circle (3pt);
\draw[black,fill=black,opacity=0.1] (7.062, -15.354) circle (3pt);
\draw[black,fill=black,opacity=0.1] (7.743, -15.925) circle (3pt);
\draw[black,fill=black,opacity=0.1] (7.54, -15.422) circle (3pt);
\draw[black,fill=black,opacity=0.1] (7.492, -15.668) circle (3pt);
\draw[black,fill=black,opacity=0.1] (10.832, -14.228) circle (3pt);
\draw[black,fill=black,opacity=0.1] (8.579, -15.4) circle (3pt);
\draw[black,fill=black,opacity=0.1] (7.839, -16.092) circle (3pt);
\draw[black,fill=black,opacity=0.1] (7.762, -15.612) circle (3pt);
\draw[black,fill=black,opacity=0.1] (7.015, -15.351) circle (3pt);
\draw[black,fill=black,opacity=0.1] (8.023, -14.75) circle (3pt);
\draw[black,fill=black,opacity=0.1] (8.29, -15.553) circle (3pt);
\draw[black,fill=black,opacity=0.1] (7.112, -14.381) circle (3pt);
\draw[black,fill=black,opacity=0.1] (5.777, -14.768) circle (3pt);
\draw[black,fill=black,opacity=0.1] (6.124, -13.472) circle (3pt);
\draw[black,fill=black,opacity=0.1] (6.389, -15.916) circle (3pt);
\draw[black,fill=black,opacity=0.1] (7.53, -16.28) circle (3pt);
\draw[black,fill=black,opacity=0.1] (8.383, -13.799) circle (3pt);
\draw[black,fill=black,opacity=0.1] (7.045, -13.694) circle (3pt);
\draw[black,fill=black,opacity=0.1] (6.146, -14.702) circle (3pt);
\draw[black,fill=black,opacity=0.1] (8.315, -14.924) circle (3pt);
\draw[black,fill=black,opacity=0.1] (8.719, -15.014) circle (3pt);
\draw[black,fill=black,opacity=0.1] (9.274, -16.708) circle (3pt);
\draw[black,fill=black,opacity=0.1] (7.115, -17.874) circle (3pt);
\draw[black,fill=black,opacity=0.1] (8.046, -14.693) circle (3pt);
\draw[black,fill=black,opacity=0.1] (8.181, -16.416) circle (3pt);
\draw[black,fill=black,opacity=0.1] (6.781, -17.662) circle (3pt);
\draw[black,fill=black,opacity=0.1] (7.264, -14.968) circle (3pt);
\draw[black,fill=black,opacity=0.1] (7.313, -14.742) circle (3pt);
\draw[black,fill=black,opacity=0.1] (7.367, -13.564) circle (3pt);
\draw[black,fill=black,opacity=0.1] (5.693, -16.378) circle (3pt);
\draw[black,fill=black,opacity=0.1] (9.046, -15.134) circle (3pt);
\draw[black,fill=black,opacity=0.1] (7.008, -14.285) circle (3pt);
\draw[black,fill=black,opacity=0.1] (7.678, -16.097) circle (3pt);
\draw[black,fill=black,opacity=0.1] (6.582, -11.895) circle (3pt);
\draw[black,fill=black,opacity=0.1] (9.489, -12.246) circle (3pt);
\draw[black,fill=black,opacity=0.1] (6.108, -15.972) circle (3pt);
\draw[black,fill=black,opacity=0.1] (8.663, -15.114) circle (3pt);
\draw[black,fill=black,opacity=0.1] (8.73, -14.354) circle (3pt);
\draw[black,fill=black,opacity=0.1] (6.405, -14.887) circle (3pt);
\draw[black,fill=black,opacity=0.1] (7.541, -12.676) circle (3pt);
\draw[black,fill=black,opacity=0.1] (8.666, -14.873) circle (3pt);
\draw[black,fill=black,opacity=0.1] (7.134, -13.573) circle (3pt);
\draw[black,fill=black,opacity=0.1] (7.617, -15.375) circle (3pt);
\draw[black,fill=black,opacity=0.1] (5.693, -14.986) circle (3pt);
\draw[black,fill=black,opacity=0.1] (10.112, -15.233) circle (3pt);
\draw[black,fill=black,opacity=0.1] (10.323, -13.233) circle (3pt);
\draw[black,fill=black,opacity=0.1] (9.477, -14.329) circle (3pt);
\draw[black,fill=black,opacity=0.1] (7.618, -14.872) circle (3pt);
\draw[black,fill=black,opacity=0.1] (6.865, -15.395) circle (3pt);
\draw[black,fill=black,opacity=0.1] (6.849, -14.942) circle (3pt);
\draw[black,fill=black,opacity=0.1] (7.728, -13.533) circle (3pt);
\draw[black,fill=black,opacity=0.1] (7.794, -13.413) circle (3pt);
\draw[black,fill=black,opacity=0.1] (7.56, -15.47) circle (3pt);
\draw[black,fill=black,opacity=0.1] (9.573, -14.59) circle (3pt);
\draw[black,fill=black,opacity=0.1] (7.272, -15.676) circle (3pt);
\draw[black,fill=black,opacity=0.1] (9.57, -16.333) circle (3pt);
\draw[black,fill=black,opacity=0.1] (7.445, -14.224) circle (3pt);
\draw[black,fill=black,opacity=0.1] (9.755, -14.887) circle (3pt);
\draw[black,fill=black,opacity=0.1] (7.492, -14.039) circle (3pt);
\draw[black,fill=black,opacity=0.1] (6.374, -12.578) circle (3pt);
\draw[black,fill=black,opacity=0.1] (5.673, -14.258) circle (3pt);
\draw[black,fill=black,opacity=0.1] (6.508, -15.397) circle (3pt);
\draw[black,fill=black,opacity=0.1] (8.462, -14.965) circle (3pt);
\draw[black,fill=black,opacity=0.1] (5.932, -15.672) circle (3pt);
\draw[black,fill=black,opacity=0.1] (8.589, -15.643) circle (3pt);
\draw[black,fill=black,opacity=0.1] (6.662, -14.117) circle (3pt);
\draw[black,fill=black,opacity=0.1] (9.308, -15.668) circle (3pt);
\draw[black,fill=black,opacity=0.1] (7.04, -14.719) circle (3pt);
\draw[black,fill=black,opacity=0.1] (4.927, -14.644) circle (3pt);
\draw[black,fill=black,opacity=0.1] (5.421, -13.844) circle (3pt);
\draw[black,fill=black,opacity=0.1] (7.547, -14.711) circle (3pt);
\draw[black,fill=black,opacity=0.1] (7.102, -16.001) circle (3pt);
\draw[black,fill=black,opacity=0.1] (6.153, -15.737) circle (3pt);
\draw[black,fill=black,opacity=0.1] (8.562, -16.598) circle (3pt);
\draw[black,fill=black,opacity=0.1] (8.553, -15.224) circle (3pt);
\draw[black,fill=black,opacity=0.1] (7.341, -14.342) circle (3pt);
\draw[black,fill=black,opacity=0.1] (6.757, -16.054) circle (3pt);
\draw[black,fill=black,opacity=0.1] (7.607, -14.668) circle (3pt);
\draw[black,fill=black,opacity=0.1] (7.942, -14.014) circle (3pt);
\draw[black,fill=black,opacity=0.1] (9.544, -14.393) circle (3pt);
\draw[black,fill=black,opacity=0.1] (6.79, -14.11) circle (3pt);
\draw[black,fill=black,opacity=0.1] (10.425, -15.876) circle (3pt);
\draw[black,fill=black,opacity=0.1] (8.122, -15.394) circle (3pt);
\draw[black,fill=black,opacity=0.1] (6.148, -17.769) circle (3pt);
\draw[black,fill=black,opacity=0.1] (8.473, -16.237) circle (3pt);
\draw[black,fill=black,opacity=0.1] (8.377, -12.401) circle (3pt);
\draw[black,fill=black,opacity=0.1] (9.81, -14.834) circle (3pt);
\draw[black,fill=black,opacity=0.1] (5.788, -14.516) circle (3pt);
\draw[black,fill=black,opacity=0.1] (6.357, -13.934) circle (3pt);
\draw[black,fill=black,opacity=0.1] (8.185, -14.265) circle (3pt);
\draw[black,fill=black,opacity=0.1] (5.419, -15.921) circle (3pt);
\draw[black,fill=black,opacity=0.1] (7.316, -14.627) circle (3pt);
\draw[black,fill=black,opacity=0.1] (7.639, -16.071) circle (3pt);
\draw[black,fill=black,opacity=0.1] (7.569, -14.282) circle (3pt);
\draw[black,fill=black,opacity=0.1] (8.03, -15.176) circle (3pt);
\draw[black,fill=black,opacity=0.1] (6.837, -15.038) circle (3pt);
\draw[black,fill=black,opacity=0.1] (8.98, -13.031) circle (3pt);
\draw[black,fill=black,opacity=0.1] (7.814, -15.566) circle (3pt);
\draw[black,fill=black,opacity=0.1] (8.009, -13.907) circle (3pt);
\draw[black,fill=black,opacity=0.1] (7.543, -14.211) circle (3pt);
\draw[black,fill=black,opacity=0.1] (7.264, -15.032) circle (3pt);
\draw[black,fill=black,opacity=0.1] (8.943, -14.951) circle (3pt);
\draw[black,fill=black,opacity=0.1] (8.034, -16.568) circle (3pt);
\draw[black,fill=black,opacity=0.1] (7.323, -14.522) circle (3pt);
\draw[black,fill=black,opacity=0.1] (6.089, -14.579) circle (3pt);
\draw[black,fill=black,opacity=0.1] (6.267, -13.547) circle (3pt);
\draw[black,fill=black,opacity=0.1] (7.803, -15.021) circle (3pt);
\draw[black,fill=black,opacity=0.1] (9.653, -13.187) circle (3pt);
\draw[black,fill=black,opacity=0.1] (5.684, -16.244) circle (3pt);
\draw[black,fill=black,opacity=0.1] (7.315, -13.734) circle (3pt);
\draw[black,fill=black,opacity=0.1] (8.864, -16.321) circle (3pt);
\draw[black,fill=black,opacity=0.1] (7.435, -17.143) circle (3pt);
\draw[black,fill=black,opacity=0.1] (9.489, -15.884) circle (3pt);
\draw[black,fill=black,opacity=0.1] (8.737, -13.954) circle (3pt);
\draw[black,fill=black,opacity=0.1] (6.127, -13.141) circle (3pt);
\draw[black,fill=black,opacity=0.1] (7.555, -14.865) circle (3pt);
\draw[black,fill=black,opacity=0.1] (7.845, -13.746) circle (3pt);
\draw[black,fill=black,opacity=0.1] (8.35, -14.546) circle (3pt);
\draw[black,fill=black,opacity=0.1] (7.606, -14.287) circle (3pt);
\draw[black,fill=black,opacity=0.1] (6.186, -14.742) circle (3pt);
\draw[black,fill=black,opacity=0.1] (9.878, -16.451) circle (3pt);
\draw[black,fill=black,opacity=0.1] (5.941, -15.601) circle (3pt);
\draw[black,fill=black,opacity=0.1] (7.603, -14.554) circle (3pt);
\draw[black,fill=black,opacity=0.1] (6.941, -16.35) circle (3pt);
\draw[black,fill=black,opacity=0.1] (6.81, -13.772) circle (3pt);
\draw[black,fill=black,opacity=0.1] (9.923, -15.756) circle (3pt);
\draw[black,fill=black,opacity=0.1] (7.468, -15.284) circle (3pt);
\draw[black,fill=black,opacity=0.1] (9.184, -15.439) circle (3pt);
\draw[black,fill=black,opacity=0.1] (8.884, -16.305) circle (3pt);
\draw[black,fill=black,opacity=0.1] (8.202, -11.837) circle (3pt);
\draw[black,fill=black,opacity=0.1] (6.575, -14.704) circle (3pt);
\draw[black,fill=black,opacity=0.1] (8.911, -14.393) circle (3pt);
\draw[black,fill=black,opacity=0.1] (7.126, -15.446) circle (3pt);
\draw[black,fill=black,opacity=0.1] (7.205, -17.124) circle (3pt);
\draw[black,fill=black,opacity=0.1] (8.894, -16.609) circle (3pt);
\draw[black,fill=black,opacity=0.1] (5.876, -16.654) circle (3pt);
\draw[black,fill=black,opacity=0.1] (8.133, -14.12) circle (3pt);
\draw[black,fill=black,opacity=0.1] (9.29, -14.592) circle (3pt);
\draw[black,fill=black,opacity=0.1] (8.557, -15.903) circle (3pt);
\draw[black,fill=black,opacity=0.1] (9.222, -18.037) circle (3pt);
\draw[black,fill=black,opacity=0.1] (6.624, -13.673) circle (3pt);
\draw[black,fill=black,opacity=0.1] (6.719, -15.037) circle (3pt);
\draw[black,fill=black,opacity=0.1] (8.573, -16.077) circle (3pt);
\draw[black,fill=black,opacity=0.1] (7.57, -16.927) circle (3pt);
\draw[black,fill=black,opacity=0.1] (8.812, -15.723) circle (3pt);
\draw[black,fill=black,opacity=0.1] (6.243, -13.563) circle (3pt);
\draw[black,fill=black,opacity=0.1] (7.364, -15.253) circle (3pt);
\draw[black,fill=black,opacity=0.1] (7.398, -13.397) circle (3pt);
\draw[black,fill=black,opacity=0.1] (8.27, -15.249) circle (3pt);
\draw[black,fill=black,opacity=0.1] (7.696, -13.639) circle (3pt);
\draw[black,fill=black,opacity=0.1] (9.323, -16.162) circle (3pt);
\draw[black,fill=black,opacity=0.1] (6.72, -14.699) circle (3pt);
\draw[black,fill=black,opacity=0.1] (8.081, -14.185) circle (3pt);
\draw[black,fill=black,opacity=0.1] (7.804, -13.87) circle (3pt);
\draw[black,fill=black,opacity=0.1] (8.062, -14.506) circle (3pt);
\draw[black,fill=black,opacity=0.1] (8.908, -15.139) circle (3pt);
\draw[black,fill=black,opacity=0.1] (7.268, -13.98) circle (3pt);
\draw[black,fill=black,opacity=0.1] (7.827, -14.7) circle (3pt);
\draw[black,fill=black,opacity=0.1] (7.379, -14.903) circle (3pt);
\draw[black,fill=black,opacity=0.1] (8.807, -17.657) circle (3pt);
\draw[black,fill=black,opacity=0.1] (9.517, -15.118) circle (3pt);
\draw[black,fill=black,opacity=0.1] (7.557, -14.999) circle (3pt);
\draw[black,fill=black,opacity=0.1] (8.95, -14.366) circle (3pt);
\draw[black,fill=black,opacity=0.1] (9.716, -15.26) circle (3pt);
\draw[black,fill=black,opacity=0.1] (7.398, -13.515) circle (3pt);
\draw[black,fill=black,opacity=0.1] (8.156, -14.082) circle (3pt);
\draw[black,fill=black,opacity=0.1] (8.76, -14.111) circle (3pt);
\draw[black,fill=black,opacity=0.1] (5.76, -13.2) circle (3pt);
\draw[black,fill=black,opacity=0.1] (8.372, -15.121) circle (3pt);
\draw[black,fill=black,opacity=0.1] (8.965, -16.595) circle (3pt);
\draw[black,fill=black,opacity=0.1] (6.85, -15.945) circle (3pt);
\draw[black,fill=black,opacity=0.1] (6.433, -15.484) circle (3pt);
\draw[black,fill=black,opacity=0.1] (5.996, -16.889) circle (3pt);
\draw[black,fill=black,opacity=0.1] (9.639, -14.332) circle (3pt);
\draw[black,fill=black,opacity=0.1] (6.874, -14.319) circle (3pt);
\draw[black,fill=black,opacity=0.1] (6.185, -15.325) circle (3pt);
\draw[black,fill=black,opacity=0.1] (9.133, -14.834) circle (3pt);
\draw[black,fill=black,opacity=0.1] (7.656, -11.738) circle (3pt);
\draw[black,fill=black,opacity=0.1] (9.61, -15.277) circle (3pt);
\draw[black,fill=black,opacity=0.1] (7.827, -17.641) circle (3pt);
\draw[black,fill=black,opacity=0.1] (9.055, -13.724) circle (3pt);
\draw[black,fill=black,opacity=0.1] (8.895, -16.577) circle (3pt);
\draw[black,fill=black,opacity=0.1] (9.019, -13.646) circle (3pt);
\draw[black,fill=black,opacity=0.1] (5.59, -14.789) circle (3pt);
\draw[black,fill=black,opacity=0.1] (7.02, -16.375) circle (3pt);
\draw[black,fill=black,opacity=0.1] (7.074, -14.029) circle (3pt);
\draw[black,fill=black,opacity=0.1] (6.745, -14.658) circle (3pt);
\draw[black,fill=black,opacity=0.1] (9.81, -16.548) circle (3pt);
\draw[black,fill=black,opacity=0.1] (6.826, -15.969) circle (3pt);
\draw[black,fill=black,opacity=0.1] (6.585, -15.442) circle (3pt);
\draw[black,fill=black,opacity=0.1] (7.371, -13.964) circle (3pt);
\draw[black,fill=black,opacity=0.1] (9.866, -15.51) circle (3pt);
\draw[black,fill=black,opacity=0.1] (8.651, -15.386) circle (3pt);
\draw[black,fill=black,opacity=0.1] (6.591, -14.347) circle (3pt);
\draw[black,fill=black,opacity=0.1] (7.456, -16.093) circle (3pt);
\draw[black,fill=black,opacity=0.1] (8.444, -16.94) circle (3pt);
\draw[black,fill=black,opacity=0.1] (8.143, -15.536) circle (3pt);
\draw[black,fill=black,opacity=0.1] (9.112, -14.333) circle (3pt);
\draw[black,fill=black,opacity=0.1] (6.265, -16.352) circle (3pt);
\draw[black,fill=black,opacity=0.1] (7.732, -14.543) circle (3pt);
\draw[black,fill=black,opacity=0.1] (5.353, -14.862) circle (3pt);
\draw[black,fill=black,opacity=0.1] (4.907, -15.733) circle (3pt);
\draw[black,fill=black,opacity=0.1] (7.561, -14.93) circle (3pt);
\draw[black,fill=black,opacity=0.1] (7.036, -15.599) circle (3pt);
\draw[black,fill=black,opacity=0.1] (8.281, -14.923) circle (3pt);
\draw[black,fill=black,opacity=0.1] (7.813, -16.801) circle (3pt);
\draw[black,fill=black,opacity=0.1] (5.638, -17.563) circle (3pt);
\draw[black,fill=black,opacity=0.1] (8.044, -15.809) circle (3pt);
\draw[black,fill=black,opacity=0.1] (7.779, -16.146) circle (3pt);
\draw[black,fill=black,opacity=0.1] (8.442, -15.225) circle (3pt);
\draw[black,fill=black,opacity=0.1] (7.616, -15.242) circle (3pt);
\draw[black,fill=black,opacity=0.1] (6.948, -16.197) circle (3pt);
\draw[black,fill=black,opacity=0.1] (6.626, -12.901) circle (3pt);
\draw[black,fill=black,opacity=0.1] (8.066, -14.188) circle (3pt);
\draw[black,fill=black,opacity=0.1] (7.955, -12.451) circle (3pt);
\draw[black,fill=black,opacity=0.1] (8.835, -14.1) circle (3pt);
\draw[black,fill=black,opacity=0.1] (7.26, -16.323) circle (3pt);
\draw[black,fill=black,opacity=0.1] (8.875, -16.991) circle (3pt);
\draw[black,fill=black,opacity=0.1] (8.808, -14.592) circle (3pt);
\draw[black,fill=black,opacity=0.1] (7.57, -14.004) circle (3pt);
\draw[black,fill=black,opacity=0.1] (6.781, -16.288) circle (3pt);
\draw[black,fill=black,opacity=0.1] (8.139, -17.621) circle (3pt);
\draw[black,fill=black,opacity=0.1] (6.508, -14.703) circle (3pt);
\draw[black,fill=black,opacity=0.1] (8.846, -14.72) circle (3pt);
\draw[black,fill=black,opacity=0.1] (5.816, -14.804) circle (3pt);
\draw[black,fill=black,opacity=0.1] (8.486, -17.538) circle (3pt);
\draw[black,fill=black,opacity=0.1] (7.084, -13.167) circle (3pt);
\draw[black,fill=black,opacity=0.1] (8.482, -16.782) circle (3pt);
\draw[black,fill=black,opacity=0.1] (8.346, -14.495) circle (3pt);
\draw[black,fill=black,opacity=0.1] (8.571, -13.808) circle (3pt);
\draw[black,fill=black,opacity=0.1] (7.739, -13.83) circle (3pt);
\draw[black,fill=black,opacity=0.1] (6.888, -14.828) circle (3pt);
\draw[black,fill=black,opacity=0.1] (7.801, -15.937) circle (3pt);
\draw[black,fill=black,opacity=0.1] (7.034, -14.537) circle (3pt);
\draw[black,fill=black,opacity=0.1] (8.312, -15.4) circle (3pt);
\draw[black,fill=black,opacity=0.1] (6.701, -15.156) circle (3pt);
\draw[black,fill=black,opacity=0.1] (7.97, -14.144) circle (3pt);
\draw[black,fill=black,opacity=0.1] (8.63, -15.393) circle (3pt);
\draw[black,fill=black,opacity=0.1] (9.234, -15.83) circle (3pt);
\draw[black,fill=black,opacity=0.1] (5.816, -13.723) circle (3pt);
\draw[black,fill=black,opacity=0.1] (8.224, -14.033) circle (3pt);
\draw[black,fill=black,opacity=0.1] (7.045, -14.199) circle (3pt);
\draw[black,fill=black,opacity=0.1] (8.905, -14.628) circle (3pt);
\draw[black,fill=black,opacity=0.1] (7.439, -13.82) circle (3pt);
\draw[black,fill=black,opacity=0.1] (8.381, -15.471) circle (3pt);
\draw[black,fill=black,opacity=0.1] (7.932, -15.029) circle (3pt);
\draw[black,fill=black,opacity=0.1] (10.631, -15.816) circle (3pt);
\draw[black,fill=black,opacity=0.1] (7.456, -16.928) circle (3pt);
\draw[black,fill=black,opacity=0.1] (8.174, -15.807) circle (3pt);
\draw[black,fill=black,opacity=0.1] (8.829, -14.481) circle (3pt);
\draw[black,fill=black,opacity=0.1] (8.846, -14.905) circle (3pt);
\draw[black,fill=black,opacity=0.1] (7.659, -16.379) circle (3pt);
\draw[black,fill=black,opacity=0.1] (7.815, -12.426) circle (3pt);
\draw[black,fill=black,opacity=0.1] (6.463, -16.279) circle (3pt);
\draw[black,fill=black,opacity=0.1] (7.437, -13.786) circle (3pt);
\draw[black,fill=black,opacity=0.1] (8.285, -17) circle (3pt);
\draw[black,fill=black,opacity=0.1] (7.674, -14.897) circle (3pt);
\draw[black,fill=black,opacity=0.1] (6.627, -13.523) circle (3pt);
\draw[black,fill=black,opacity=0.1] (10.003, -13.567) circle (3pt);
\draw[black,fill=black,opacity=0.1] (9.06, -14.962) circle (3pt);
\draw[black,fill=black,opacity=0.1] (7.925, -15.032) circle (3pt);
\draw[black,fill=black,opacity=0.1] (10.034, -14.076) circle (3pt);
\draw[black,fill=black,opacity=0.1] (9.776, -15.038) circle (3pt);
\draw[black,fill=black,opacity=0.1] (8.462, -15.715) circle (3pt);
\draw[black,fill=black,opacity=0.1] (6.098, -13.495) circle (3pt);
\draw[black,fill=black,opacity=0.1] (7.886, -16.554) circle (3pt);


  \end{tikzpicture}
  
  \newpage
  ~\\

~\\

foo bar

\end{document}


\end{document}