%!TEX root = ComputerScienceOne.tex

%%Chapter: C Basics

The C programming language is a relatively old language, but still widely 
used.  It is universal in that nearly every system, platform, and operating system
has a C compiler that produces machine code for that system.  C is used extensively
in systems programming for operating system kernels, embedded systems, microcontrollers, 
and supercomputers.  It is generally an ``imperative'' language which is a paradigm that
characterizes computation in terms of executable statements and functions that change 
a program's state (variable values).  C has been highly influential in 
the design and syntax of other
languages including C++, Objective-C, C\#, Java, and PHP, among many others.  
Many languages have adopted the basic syntactic elements and structured programming 
approach of the C language.

C was originally developed by Dennis Ritchie while at AT\&T Bell Labs 1969--1972.  C was
born out of the need for a new language for the PDP-11 minicomputer that used the 
Unix operating system (written by Ken Thompson).  From its inception, C has had a close relation to Unix; in fact the
operating system was subsequently rewritten in C, making it the first OS to be written in 
a language other than assembly.  The language was dubbed ``C'' as its predecessor was
named ``B,'' a simplified version of BCPL (Basic Combined Programming Language).  
The first formal specification was published as \emph{The C Programming Language} by
Kernighan and Ritchie (1978) \cite{Kernighan:1988:CPL:576122} often referred to as 
``The K\&R Book'' which would later become the \gls{ansiLabel} C standard.

C gained in popularity and directly influenced object-oriented variations of it.  
Bjarne Stroustrup developed C++ while at Bell Labs from 1979--1983.  
Brad Cox and Tom Love developed Objective-C from 1981--1983 at their
company Stepstone.  Subsequent standards of the C language have added
and extended features.  In 1990, the \gls{isoLabel}/\gls{iecLabel} 9899:1990 
standard, referred to as C89 or C90, was adopted.  About every 10 years 
since, a new standard has been adopted; ISO/IEC 9899:1999 (referred
to as C99) in 1999 and ISO/IEC 9899:2011 (C11) in 2011.

\section{Getting Started: Hello World}
\index{Hello World!C}

The hallmark of an introduction to a new programming language is 
the \emph{Hello World!} program.  It consists of a simple program whose 
only purpose is to print out the message ``Hello World!'' to the user 
in some manner.  The simplicity of the program allows the
focus to be on the basic syntax of the language.  It is also typically used to ensure that 
your development environment, compiler, runtime environment, etc.\ are functioning 
properly with a minimal example.  The Hello World! program is generally attributed to 
Brian Kernighan who used it as an example of programming in C in 1974 \cite{Kernighan:1974}.
A basic Hello World! program in C can be found in Code Sample \ref{code:helloWorldInC}.

\begin{listing}
\begin{minted}{c}
#include<stdlib.h>
#include<stdio.h>

/**
 * Basic Hello World program in C
 * Prints "Hello World" to the standard output and exits
 */
int main(int argc, char **argv) {

  printf("Hello World\n");

  return 0;
}
\end{minted}
\caption{Hello World Program in C}
\label{code:helloWorldInC}
\end{listing}

We will not focus on any particular development environment, code editor, or any 
particular operating system, compiler, or ancillary standards in our presentation.  However,
as a first step, you should be able to write, compile, and run the above program on the
environment you intend to use for the rest of this book.  This may require that you download
and install a basic C compiler/development environment (such as GCC, the GNU Compiler 
Collection on OSX/Unix/Linux, cygwin or MinGW for Windows) or a full IDE (such as Xcode 
for OSX, or Code::Blocks, \url{http://www.codeblocks.org/} for Windows).

\section{Basic Elements}

Using the Hello World! program as a starting point, we will now examine the basic
elements of the C language.

\subsection{Basic Syntax Rules}

C is a highly influential programming language.  Many modern programming
languages have adopted syntactic elements that originated in C.  Usually
such languages are referred to as ``C-style syntax'' languages.  These
elements include the following.
\begin{itemize}
  \item C is a statically typed language so variables must be declared along with
  	their types before using them.
  \item Strings are delimited with double quotes.  Single characters, including
  	special escaped characters are delimited by single quotes; \mintinline{c}{"this is a string"}, 
	and these are characters: \mintinline{c}{'A'}, \mintinline{c}{'4'}, \mintinline{c}{'$'} and
	\mintinline{c}{'\n'}
  \item Executable statements are terminated by a semicolon, \mintinline{c}{;}
  \item Code blocks are defined by using opening and closing curly brackets, 
  	\mintinline{c}{{ ... }}.  Moreover, code blocks can be \emph{nested}: code
	blocks can be defined within other code blocks.  
  \item Variables are \glslink{scope}{scoped} to the code block in which they are declared
  	and are only valid within that code block.
  \item In general, whitespace between coding elements is ignored.
\end{itemize}

Though not a syntactic requirement, the proper use of whitespace is important for
good, readable code.  Code inside code blocks is indented at the same indentation level.
Nested code blocks are indented further.  Think of a typical table of contents or
the outline of a formal paper or essay.  Sections and subsections or points and 
subpoints all follow proper indentation with elements at the same level at the same
indentation.  This convention is used to organize code and make it more readable.
 
\subsection{Preprocessor Directives}
\index{preprocessor directive}

The lines, 

\begin{minted}{c}
#include<stdlib.h>
#include<stdio.h>
\end{minted}

are \emph{preprocessor directives}.  Preprocessor directives are instructions to
the compiler to \emph{modify} the source code before it starts to compile it.  These
particular lines are ``including'' standard libraries of functions so that the program can
use functionality that has already been implemented for us.  

The first, \mintinline{c}{stdlib.h} represents the C standard (\mintinline{c}{std}) 
library (\mintinline{c}{lib}).  This library is
so essential that many compilers will automatically include it even if you do not
explicitly do so in your program.  Still, it is best practice to include it in your code.

The second, \mintinline{c}{stdio.h} is the standard (\mintinline{c}{std}) input/output 
(\mintinline{c}{io}) library which contains basic input/output (I/O) functions that we can use.  
In particular, the standard output function \mintinline{c}{printf()} is part of this library.

Failure to include a library means that you will not be able to use the functions
it provides in your program.  Using the functions without including the library may
result in a compiler error.  The \mintinline{c}{.h} in the library names stands for 
``header''; function declarations are typically contained in a header file while 
their definitions are placed in a source file of the same name.  We'll explore
this convention in detail when we look at functions in C (Chapter \ref{chapter:c:functions}).

There are many other important standard libraries that we'll touch on as 
needed; of immediate interest is the standard
mathematics library, \mintinline{c}{math.h}.  It includes many useful functions
to compute common mathematical functions such as the square root and 
natural logarithm.  Table \ref{table:cMathFunctions} highlights several of 
these functions.  To use them you'd include the math library in your source
file \mintinline{c}{#include<math.h>} and then ``call'' them by providing input 
and assigning the output value to a variable.  For example:

\begin{minted}{c}
double x = 1.5;
double y, z;
y = sqrt(x); //y now has the value $\sqrt{x} = \sqrt{1.5}$
z = sin(x); //z now has the value $\sin{(x)} = \sin{(1.5)}$
\end{minted}

In both of the function calls above, the value of the variable \mintinline{c}{x} is
``passed'' to the math function which computes and ``returns'' the
result which then gets assigned to another variable.

\begin{table}
\centering
\begin{tabular}{l|l}
\hline
Function & Description \\
\hline
\mintinline{c}{abs(x)}  & Absolute value for \mintinline{c}{int} variables, $|x|$\textsuperscript{a}\\
\mintinline{c}{fabs(x)} & Absolute value for \mintinline{c}{double} variables \\
\\
\mintinline{c}{ceil(x)} & Ceiling function, $\lceil 46.3\rceil = 47.0$\\
\mintinline{c}{floor(x)} & Floor function, $\lfloor 46.3 \rfloor =46.0$\\
\\
\mintinline{c}{cos(x)} & Cosine function\textsuperscript{b}\\
\mintinline{c}{sin(x)} & Sine function\textsuperscript{b}\\
\mintinline{c}{tan(x)} & Tangent function\textsuperscript{b}\\
\\
\mintinline{c}{exp(x)} & Exponential function, $e^x$, $e = 2.71828\ldots$ \\
\mintinline{c}{log(x)}  & Natural logarithm, $\ln{(x)}$\textsuperscript{c} \\
\mintinline{c}{log10(x)} & Logarithm base 10, $\log_{10}{(x)}$\textsuperscript{c} \\
\\
\mintinline{c}{pow(x,y)} & The power function, computes $x^y$\\
\mintinline{c}{sqrt(x)} & Square root function\textsuperscript{c}\\

\hline
\end{tabular}
\caption[Several functions defined in the C standard math library]{Several functions defined in the C standard math library.  \textsuperscript{a}The absolute value function is actually in the standard library, \mintinline{c}{stdlib.h}.  \textsuperscript{b}all trigonometric functions assume input is in \emph{radians}, \textbf{not} degrees. \textsuperscript{c}Input is assumed to be positive, $x > 0$.}
\label{table:cMathFunctions}
\end{table}

\subsubsection{Macros}

Another preprocessor directive establishes \emph{macros} using the \mintinline{c}{#define}
keyword.  A macro is a single instruction that specifies a more complex set of instructions.
The macro can be used to define constants to be used throughout your program.  To 
illustrate, consider the following example.

\begin{minted}{c}
#define MILES_PER_KM 1.609
\end{minted}

The macro defines an ``alias'' for the \mintinline{c}{MILES_PER_KM} identifier as the
value $1.609$.  Essentially, the C preprocessor will go through the code and any
instance of \mintinline{c}{MILES_PER_KM} will be replaced with \mintinline{c}{1.609}.
The advantage of using a macro like this is that we can use the identifier \mintinline{c}{MILES_PER_KM}
throughout our program instead of \glspl{magic number} whose meaning and intent
may not be immediately clear.  Moreover, if we want to change the definition (say make it more
precise, using $1.60934$ instead) then we only need to change the macro instead of making the
same change throughout our program.

As a stylistic note: macro constants in C are usually associated with uppercase 
underscore casing as in our example.  Also, the math standard library defines several
macros for common mathematical constants such as $\pi$, $e$, and $\sqrt{2}$ 
(\mintinline{c}{M_PI}, \mintinline{c}{M_E}, and \mintinline{c}{M_SQRT2} 
respectively) among others.

\subsection{Comments}

Comments can be written in a C program either as a single line using
two forward slashes, \mintinline{c}{//comment} or as a multiline comment using
a combination of forward slash and asterisk: \mintinline{c}{/* comment */}.  
With a single line comment, everything on the line \emph{after} the forward
slashes is ignored.  With a multiline comment, everything between the forward
slash/asterisk is ignored.  Comments are ultimately ignored by the compiler so
the amount of comments do not have an effect on the final executable code.
Consider the following example.

\begin{minted}{c}
//this is a single line comment
int x;  //this is also a single line comment, but after some code

/*
  This is a comment that can 
  span multiple lines to format the comment
  message more clearly
*/
double y;
\end{minted}

Most code editors and \glspl{ideLabel} will present comments in a special color or
font to distinguish them from the rest of the code (just as our example above does).
Failure to close a multiline comment will likely result in a compiler error but with
color-coded comments its easy to see the mistake visually.

\subsection{The \mintinline{c}{main()} Function}

Every executable program starts its execution somewhere. In C, 
the starting point is the \mintinline{c}{main()}
function.  When a program is compiled to an executable and the program
is invoked, the code in the \mintinline{c}{main()} function starts executing.  In our example, the code between the two curly brackets defines the
\mintinline{c}{main()} function.
At the end of the function we have a \mintinline{c}{return} statement.  We'll
examine \mintinline{c}{return} statements in detail when we examine functions. 
The program is ``returning'' a zero value to the operating system.
The convention in C is that zero indicates ``no error occurred'' while a 
non-zero value is used to indicate that ``some'' error occurred (the 
value used is determined by system standards such as the \gls{posixLabel}
standard).

In addition, our \mintinline{c}{main()} function has two \emph{arguments}:
\mintinline{c}{argc} and \mintinline{c}{argv} which serve to communicate
any command line arguments provided to the program (review Section
\ref{subsection:commandLineInput} for details).  The first, \mintinline{c}{argc} is an integer that indicates
the \emph{number} of arguments provided \emph{including} the executable
file name itself.  The second, \mintinline{c}{argv} actually stores the 
arguments as strings.  We will understand the syntax later on, but
for now we can at least understand how we might convert these arguments
to different types such as integers and floating point numbers.

Recall that \mintinline{c}{argv} is the argument \emph{vector}: it 
is an array (see Chapter \ref{chapter:c:arrays}) of the command line arguments.  
To access them, you can \emph{index} them starting at zero, the first being \mintinline{c}{argv[0]}, the
second \mintinline{c}{argv[1]}, etc. (the last one of course would be at
\mintinline{c}{argv[argc-1]}).  The first one is always the name of the
executable file being run.  The remaining are the command line 
arguments provided by the user.

To convert them you can use two different functions, \mintinline{c}{atoi()}
and \mintinline{c}{atof()} which are short for \textbf{a}lphanumeric \textbf{to}
\textbf{i}nteger and \textbf{f}loafing-point number respectively.  An 
example:

\begin{minted}{c}
//prints the first command line argument:
printf("%s\n", argv[0]);
//converts the "second" command line argument to an integer
int x = atoi(argv[1]);  
//converts the "third" command line argument to a double:
double y = atof(argv[2]);
\end{minted}

\section{Variables}
\index{variables!C}

As previewed, the three primary primitive types supported in C are \mintinline{c}{int},
\mintinline{c}{double}, and \mintinline{c}{char} which support integers, floating point
numbers, and single \gls{asciiLabel} characters.  

Integer (\mintinline{c}{int}) types are only guaranteed to ``be at least'' 16 bytes by
the C standard but are usually 32-bit signed integers on most modern 
systems.\footnote{You may have to deal with 16-bit \mintinline{c}{int} types 
in legacy systems/compilers or in modern embedded systems.}  With a  32-bit signed 
\mintinline{c}{int} we can represent integers in the range $-2,147,483,648$ to 
2,147,483,647.  Doubles (\mintinline{c}{double}) types are double-precision 
floating point numbers as per the \gls{ieeeLabel} 754 standard and provide about 16 digits
of precision.  

C provides a \mintinline{c}{float} (single precision floating point 
number) type and there are various modifiers such as \mintinline{c}{short}, \mintinline{c}{long}, 
\mintinline{c}{unsigned} and \mintinline{c}{signed} that can be used, but these
are either system-dependent or rely on later versions of the C standard (such as
C99).  We will restrict our focus to more portable, interoperable code and stick
with the basic \mintinline{c}{int} and \mintinline{c}{double} types in 
our code.

Finally, the \mintinline{c}{char} type is typically a single byte that represents a
single ASCII character.  For all intents and purposes a \mintinline{c}{char} can 
be treated as an integer in the range 0 to 127 (or 255) as defined by the
ASCII text table (see Table \ref{table:asciiTable}).

\subsection{Declaration \& Assignment}

C is a statically typed language meaning that all variables must be declared
before you can use them or refer to them.  In addition, when declaring a variable, 
you must specify both its type and its identifier.  For example:

\begin{minted}{c}
int numUnits;
double costPerUnit;
char firstInitial;
\end{minted}

Each declaration specifies the variable's type followed by the identifier and ending
with a semicolon.  The identifier rules are fairly standard: a name can consist
of lower and uppercase alphabetic characters, numbers, and underscores but
may \emph{not} begin with a numeric character.  We adopt the modern camelCasing
naming convention for variables in our code.

The assignment operator is a single equal sign, \mintinline{c}{=} and is a right-to-left
assignment.  The variable that we wish to assign the value to appears on the
left-hand-side while the value (literal, variable or expression) is on the right-hand-size.
Using our variables from before, we can assign them values:

\begin{minted}{c}
numUnits = 42;
costPerUnit = 32.79;
firstInitial = 'C';
\end{minted}

An important thing to understand and to keep in mind is: if you 
declare a variable but do not assign it a value, its value is \emph{undefined}.  
If we code something like \mintinline{c}{int a;}, the value
of the variable \mintinline{c}{a} is \emph{not} necessarily zero; depending
on the system, it could contain a special value that indicates ``uninitialized
memory'' or it could contain garbage, or it \emph{could} have the value zero.
The C standard does \emph{not} specify default values for variables.  The
default value of variables is highly system dependent--on the compiler, the
libraries, and even the operating system.  You should not make any assumptions on 
the initial or default values of variables.  If you need such assumptions, then 
values must be assigned.

For brevity, C allows you to declare a variable and immediately assign it a value on the same
line.  Our example could be written more compactly written as follows.

\begin{minted}{c}
int numUnits = 42;
double costPerUnit = 32.79;
char firstInitial = 'C';
\end{minted}

As another shorthand, we can declare multiple variables on the same line by delimiting
them with a comma.  However, they \emph{must} be of the same type.  We can also 
use an assignment with them. 

\begin{minted}{c}
int numOrders, numUnits = 42, numCustomers = 10, numItems;
double costPerUnit = 32.79, salesTaxRate;
\end{minted}

Another convenient keyword is \mintinline{c}{const}, short for ``constant.''  We can 
apply it to any variable to indicate that it is a read-only variable.  Of course, we
\emph{must} assign it a value at declaration.  For example:

\begin{minted}{c}
const int secret = 42;
const double salesTaxRate = 0.075;
\end{minted}

Any attempt to reassign the values of \mintinline{c}{const} variables will result
in a compiler error.

\section{Operators}
\index{operators!C}

C supports the standard arithmetic operators for addition, subtraction, multiplication, and
division using \mintinline{c}{+}, \mintinline{c}{-}, \mintinline{c}{*}, and
\mintinline{c}{/} respectively.  Each of these operators is a binary operator that
acts on two operands which can be literals, variables or expressions and 
follow the usual rules of arithmetic when it comes to order of precedence (multiplication 
and division before addition and subtraction).

\begin{minted}{c}
int a = 10, b = 20, c = 30, d;
d = a + 5;
d = a + b;
d = a - b;
d = a + b * c;
d = a * b;
d = a / b; //integer division and truncation!  See below

double x = 1.5, y = 3.4, z = 10.5, w;
w = x + 5.0;
w = x + y;
w = x - y;
w = x + y * z;
w = x * y;
w = x / y;

//you can do arithmetic with both types:
w = a + x;
d = b + y; //truncation also occurs here!
\end{minted}

Special care must be taken when dealing with \mintinline{c}{int} types.  
For all four operators, if both operands are integers, the result will be an
integer.  For addition, subtraction, and multiplication this isn't a big deal, but
for division it means that when we divide, say \mintinline{c}{10 / 20}, the result
is not $0.5$ as expected.  The number 0.5 is a floating point number.  As
such, the fractional part gets \glslink{truncation}{truncated} (cut off and thrown out) leaving
only zero.  In the code above, \mintinline{c}{d = a / b;} the variable \mintinline{c}{d}
ends up getting the value zero because of this.

Similarly, attempting to assign a floating point number to an integer also
results in truncation because an \mintinline{c}{int} type cannot handle
the fractional part.  In the line \mintinline{c}{d = b + y} above, \mintinline{c}{b + y}
correctly evaluates as $20 + 3.4 = 23.4$, but when assigned to the \mintinline{c}{int}
variable \mintinline{c}{d} the $.4$ gets truncated and \mintinline{c}{d} is
assigned the value 23.  Assigning an \mintinline{c}{int} value to a 
\mintinline{c}{double} variable is
not a problem as the integer $2$ implicitly becomes the floating point number $2.0$.

A solution to this problem is to use explicit \index{type casting}
\gls{type casting} to force at
least one of the operands in an integer division to become a \mintinline{c}{double}
type.  For example:

\begin{minted}{c}
int a = 10, b = 20;
double x;

x = (double) a / b; 
\end{minted}

results in \mintinline{c}{x} getting the ``correct'' value of $0.5$.  This works because
the \mintinline{c}{(double)} code forces the \mintinline{c}{int} variable \mintinline{c}{a}
to \emph{temporarily} be treated as a double variable (in this case $10.0$) for the
purposes of division (so that truncation does \emph{not} occur).  

C also supports the integer remainder operator using the \mintinline{c}{%} symbol.
This operator gives the remainder of the result of dividing two integers.  Examples:

\begin{minted}{c}
int x;

x = 10 % 5; //x is 0
x = 10 % 3; //x is 1
x = 29 % 5; //x is 4
\end{minted}

\section{Basic I/O}

The standard I/O library (\mintinline{c}{stdio.h}) contains many functions that facilitate
input and output including \mintinline{c}{printf()} for standard output and \mintinline{c}{scanf()}
(\textbf{scan f}ormatted input) for standard input.

The \mintinline{c}{printf()} function works exactly as discussed in Section \ref{subsection:printfStyleFormatting}.
The \mintinline{c}{scanf()} function works using similar placeholders as \mintinline{c}{printf()}.
To illustrate how it works, consider the following lines of code:

\begin{minted}{c}
int a;
printf("Please enter a number: ");
scanf("%d", &a);
\end{minted}

The \mintinline{c}{printf()} statement prompts the user for an input.  The \mintinline{c}{scanf()} then
executes and the program \emph{waits} for the user to enter input.  The user is free to 
start typing.  When the user is done, they hit the enter key at which point the program resumes
and reads the input from the standard input buffer, converts the value entered by the user into
an integer and places the result in the variable \mintinline{c}{a} where it can now be used
by the remainder of the program.

A few points of interest.  First, the same placeholder as \mintinline{c}{printf()} was used, \mintinline{c}{%d}
for \mintinline{c}{int} values.  However, when we ``passed'' the variable \mintinline{c}{a} to \mintinline{c}{scanf()}
we placed an ampersand, \mintinline{c}{&} in front of it.  This is passing the variable \emph{by reference}
and we'll explore that concept further in Chapter \ref{chapter:c:functions}, but for now just know that
when using variables with \mintinline{c}{scanf()}, an ampersand is required.  Failure to place an
ampersand in front of a variable with \mintinline{c}{scanf()} will likely result in a \emph{segmentation fault} \index{segmentation fault}
(an illegal memory access).

You can use the same placeholder, \mintinline{c}{%c} with \mintinline{c}{scanf()} to read in single 
characters as well.  However, for floating point numbers, in particular \mintinline{c}{double} types, 
the placeholder \mintinline{c}{%lf} must be used (which stands for long float, a double precision
number).  Failure to use the correct placeholder may result in garbage results as the input will
be interpreted incorrectly.  Another example:

\begin{minted}{c}
double x;
printf("Please enter a fractional number: ");
scanf("%lf", &x);
\end{minted}

Another potential problem is that \mintinline{c}{scanf()} expects a certain \emph{format} (thus its 
name).  If we prompt the user for a number but they just start mashing the keyboard giving 
non-numerical input, we may get incorrect results.  \mintinline{c}{scanf()} will interpret
the input as zero.  It may be very difficult to distinguish between the case of a user actually
entering in zero as a legitimate input versus bad input.  In general, \mintinline{c}{scanf()} is
not a good mechanism for reading input (and in fact can be very dangerous), but it does provide a good starting point.

\section{Examples}

\subsection{Converting Units}

Let's write a program that will prompt the user to enter a temperature
in degrees Fahrenheit and convert it to degrees Celsius using the formula
  $$C = (F - 32) \cdot \frac{5}{9}$$

We begin with the basic program outline which will include preprocessor directives to bring
in the standard library and the standard input/output library (we'll need to prompt
for input and print the result as output to the user).  Further, we want our program to 
be executable, so we need to put our code into the \mintinline{c}{main()} function.  Finally, we'll document
our program to indicate its purpose.

\begin{minted}{c}
#include<stdlib.h>
#include<stdio.h>

/**
 * This program converts Fahrenheit temperatures to 
 * Celsius
 */
int main(int argc, char **argv) {

  //TODO: implement this
  
  return 0;
}
\end{minted}

It is common for programmers to use a comment along with a \mintinline{c}{TODO} note to
themselves as a reminder of things that they still need to implement in a program.  

Let's first outline the basic steps that our program will go through:
\begin{enumerate}
  \item We'll first prompt the user for input, asking them for a temperature in Fahrenheit
  \item Next we'll read the user's input, likely into a floating point number as degrees can be fractional
  \item Once we have the input, we can calculate the degrees Celsius by using the formula above
  \item Lastly, we will want to print the result to the user to inform them of the value
\end{enumerate}
Sometimes its helpful to write an outline of such a program directly in the code using
comments to provide a step-by-step process.  For example:

\begin{minted}{c}
#include<stdlib.h>
#include<stdio.h>

/**
 * This program converts Fahrenheit temperatures to 
 * Celsius
 */
int main(int argc, char **argv) {

  //TODO: implement this
  //1. Prompt the user for input in Fahrenheit
  //2. Read the Fahrenheit value from the standard input
  //3. Compute the degrees Celsius
  //4. Print the result to the user
  
  return 0;
}
\end{minted}

As we read each step it becomes apparent that we'll need a couple of variables:
one to hold the Fahrenheit (input) value and one for the Celsius (output) value.  It also
makes sense that each of these should be \mintinline{c}{double} variables as we
want to support fractional values.  So at the top of our \mintinline{c}{main()} function, we'll add
the variable declarations:

\mintinline{c}{double fahrenheit, celsius;}

Each of the steps is now straightforward; we'll use a \mintinline{c}{printf()} statement in the
first step to prompt the user for input:

\mintinline{c}{printf("Please enter degrees in Fahrenheit: ");}

In the second step, we'll use the standard input to read the \mintinline{c}{fahrenheit} 
variable value from the user.  Recall that we use the placeholder \mintinline{c}{%lf} 
for reading \mintinline{c}{double} values and use an ampersand when using \mintinline{c}{scanf()}:

\mintinline{c}{scanf("%lf", &fahrenheit);}

We can now compute \mintinline{c}{celsius} using the formula provided:

\mintinline{c}{celsius = (fahrenheit - 32) * (5 / 9);}

Finally, we use \mintinline{c}{printf()} again to output the result to the user:

\mintinline{c}{printf("%f Fahrenheit is %f Celsius\n", fahrenheit, celsius);}  

Try typing and running the program as defined above and you'll find that
you don't get correct answers.  In fact, you'll find that no matter what
values you enter, you get zero.  This is because of the calculation using
\mintinline{c}{5 / 9}: recall what happens with integer division: truncation! 
This will \emph{always} end up being zero.  

One way we could fix it would be to pull out our calculators and find that
$\frac{5}{9} = 0.55555\ldots$ and replace \mintinline{c}{5/9} with \mintinline{c}{0.555555}.
But, how many fives?  It may be difficult to tell how accurate we can make
this floating point number by hardcoding it ourselves.  A much better approach
would be to let the compiler take care of the optimal computation for us by
making at least one of the numbers a \mintinline{c}{double} to prevent
integer truncation.  That is, we should instead use \mintinline{c}{5.0 / 9}.
The full program can be found in Code Sample \ref{code:c:fahrenheitToCelsiusProgram}.

\begin{listing}[H]
\begin{minted}{c}
#include<stdlib.h>
#include<stdio.h>

/**
 * This program converts Fahrenheit temperatures to 
 * Celsius
 */
int main(int argc, char **argv) {

  double fahrenheit, celsius;
  
  //1. Prompt the user for input in Fahrenheit
  printf("Please enter degrees in Fahrenheit: ");
  
  //2. Read the Fahrenheit value from the standard input
  scanf("%lf", &fahrenheit);
  
  //3. Compute the degrees Celsius
  celsius = (fahrenheit - 32) * 5.0/9;
  
  //4. Print the result to the user
  printf("%f Fahrenheit is %f Celsius\n", fahrenheit, celsius);
  
  return 0;
}
\end{minted}
\caption{Fahrenheit-to-Celsius Conversion Program in C}
\label{code:c:fahrenheitToCelsiusProgram}
\end{listing}

\subsection{Computing Quadratic Roots}

Some programs require the user to enter multiple inputs.  The 
prompt-input process can be repeated.  In this example, consider asking
the user for the coefficients, $a, b, c$ to a quadratic polynomial, 
  $$ax^2 + bx + c$$
and computing its roots using the quadratic formula, 
  $$x = \frac{-b \pm \sqrt{b^2 - 4ac}}{2a}$$
As before, we can create a basic program with a \mintinline{c}{main()}
function and start filling in the details.  In particular, we'll need to prompt
for the input $a$, then read it in; then prompt for $b$, read it in and
repeat for $c$.  We'll also need several variables: three for the coefficients
$a, b, c$ and \emph{two} more; one for each root.  Thus, we have 

\begin{minted}{c}
double a, b, c, root1, root2;

printf("Please enter a: ");
scanf("%lf", &a);
printf("Please enter b: ");
scanf("%lf", &b);
printf("Please enter c: ");
scanf("%lf", &c);
\end{minted}

Now to compute the roots: we need to take care that we correctly adapt the
formula so it accurately reflects the order of operations.  We also need to use
the standard math library's square root function (unless you want to write 
your own!\footnote{We \emph{will} write several square root methods as
exercises later!}  Carefully adapting the formula leads to 

\begin{minted}{c}
root1 = (-b + sqrt(b*b - 4*a*c) ) / (2*a);
root1 = (-b - sqrt(b*b - 4*a*c) ) / (2*a);
\end{minted}

Finally, we print the output using \mintinline{c}{printf()}.  The full program 
can be found in Code Sample \ref{code:c:quadraticRootsProgram}.

\begin{listing}[h]
\begin{minted}{c}
#include<stdlib.h>
#include<stdio.h>
#include<math.h>

/**
 * This program computes the roots to a quadratic equation
 * using the quadratic formula.
 */
int main(int argc, char **argv) {

  double a, b, c, root1, root2;

  printf("Please enter a: ");
  scanf("%lf", &a);
  printf("Please enter b: ");
  scanf("%lf", &b);
  printf("Please enter c: ");
  scanf("%lf", &c);
  
  root1 = (-b + sqrt(b*b - 4*a*c) ) / (2*a);
  root2 = (-b - sqrt(b*b - 4*a*c) ) / (2*a);
  
  printf("The roots of %fx^2 + %fx + %f are: \n", a, b, c);
  printf("  root1 = %f\n", root1);
  printf("  root2 = %f\n", root2);
  return 0;
}
\end{minted}
\caption{Quadratic Roots Program in C}
\label{code:c:quadraticRootsProgram}
\end{listing}

This program was interactive.  As an alternative, we could have read 
all three of the inputs as command line arguments, converting them 
to floating point numbers.  Lines 12--17 in the
program could have been changed to 

\begin{minted}{c}
a = atof(argv[1]);
b = atof(argv[2]);
c = atof(argv[3]);
\end{minted}

Finally, think about the possible inputs a user could provide that may cause problems
for this program.  For example:
\begin{itemize}
  \item What if the user entered zero for $a$?
  \item What if the user entered some combination such that $b^2 < 4ac$?
  \item What if the user entered non-numeric values?
  \item For the command line argument version, what if the user provided 
  	less than three arguments?  Or more?
\end{itemize}
How might we prevent the consequences of such bad inputs?   
How might we handle the event that a user enters bad input and
how do we communicate these errors to the user?  To begin to 
resolve these issues, we'll need conditionals.




