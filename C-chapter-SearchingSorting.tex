%!TEX root = ComputerScienceOne.tex

%%Chapter: Searching & Sorting in C

The standard C library provides several functions to search and sort
arrays of \emph{any} type of element including \mintinline{c}{int}, 
\mintinline{c}{double}, or even user defined structures such as our
\mintinline{c}{Student} example from Chapter \ref{chapter:c:structs}.  
These functions are able to operate on any type of array because they
take generic \mintinline{c}{void} pointers.  However, these functions
still need a way to \emph{order} the generic elements.  To understand
how this all works we need to understand both \emph{comparator} functions
and \emph{function pointers}.

\section{Comparator Functions}

Let's consider a ``generic'' Quick Sort algorithm as was presented in
Algorithm \ref{algo:quickSort}.  The algorithm itself specifies how to 
sort elements, but
it doesn't specify how they are \emph{ordered}.  The difference is subtle
but important.  Essentially, Quick Sort needs to know when two elements, 
$a, b$ are in order, out of order, or equivalent in order to decide which
partition each element goes in.  However, it doesn't ``know'' anything about
the elements $a$ and $b$ themselves.  They could be numbers, they could
be strings, they could be user-defined objects.  

A sorting algorithm still needs to be able to determine the proper ordering 
in order to sort.  In C this is achieved through a \emph{comparator function}, 
which is a function that is responsible for comparing two elements and
determining their proper order.  A comparator function has the following 
signature and behavior:

\mintinline{c}{int cmp(const void *a, const void *b);}

\begin{itemize}
  \item The function takes two generic \mintinline{c}{void} pointers which
  	point to the elements being compared.  
  \item Moreover, the \mintinline{c}{const} keyword is used to indicate that
  	no changes will be made to the elements.
  \item The function returns an integer indicating the relative ordering of
  	the two elements:
	\begin{itemize}
	  \item It returns something negative, \mintinline{c}{< 0} if \mintinline{c}{a}
	  comes before \mintinline{c}{b} (that is, $a < b$)
	  \item It returns zero if \mintinline{c}{a} and \mintinline{c}{b} are
	  equal ($a = b$)
	  \item It returns something positive, \mintinline{c}{> 0} if \mintinline{c}{a}
	  comes after \mintinline{c}{b} (that is, $a > b$)
	\end{itemize}
\end{itemize}

Note that there is no guarantee on the value's magnitude, it does \emph{not}
necessarily return $-1$ or $+1$; it just returns \emph{something} negative or
positive.  We've previously seen this pattern when comparing strings.  The
standard string library provides a function, \mintinline{c}{strcmp()} that
has the same basic contract: it takes two strings and returns something
negative, zero or something positive depending no the lexicographic ordering
of the two strings.  Strictly speaking, however, \mintinline{c}{strcmp()} is
\emph{not} a comparator function.  It is defined to take two 
\mintinline{c}{const char *} parameters, not \mintinline{c}{const void *}
pointers.

The C language (and thus the compiler) ``knows'' how to compare built-in
primitive types like \mintinline{c}{int} and \mintinline{c}{double} using the
built-in comparison operators.  However, to generalize the comparison operation, 
we use \mintinline{c}{void *} pointers so that we can write and use general
comparator functions that can be used in generic searching and sorting functions
as we'll see below.

As an example, let's write a comparator function that orders integers in
ascending order.  We'll call our function \mintinline{c}{cmpInt()} and use
the signature above.

\begin{minted}{c}
int cmpInt(const void *a, const void *b) {

}
\end{minted}

But how we implement this function?  Obviously we want to compare the values
stored in the pointer variables, so we need to dereference them.  However, 
the comparison, \mintinline{c}{(*a < *b)} for example is not comparing integer
values.  The variables \mintinline{c}{a} and \mintinline{c}{b} are 
\mintinline{c}{void} pointers, not \mintinline{c}{int} pointers.  Thus, the 
first step is to \emph{make} them \mintinline{c}{int} pointers by doing an
explicit type cast:

\begin{minted}{c}
const int *x = (const int *) a;
const int *y = (const int *) b;
\end{minted}

Now the variables \mintinline{c}{x} and \mintinline{c}{y} \emph{are} 
\mintinline{c}{int} pointers which can be dereferenced and compared (we 
preserve the keyword \mintinline{c}{const} to ensure we do not make changes
to the variables).  Altogether, we have the full comparator function.

\begin{minted}{c}
int cmpInt(const void *a, const void *b) {
  const int *x = (const int *) a;
  const int *y = (const int *) b;
  if(*x < *y) {
    return -1;
  } else if(*x == *y) {
    return 0;
  } else {
    return -1;
  }
}
\end{minted}

What if we wanted to order integers in the opposite order?  We could write
another comparator in which the comparisons or values are reversed.  Even
simpler, we could reuse the comparator above and ``flip'' the sign by 
multiplying by $-1$ (that is, after one of the purposes of writing functions:
code reuse).  Even simpler, we could flip the arguments we pass to 
\mintinline{c}{cmpInt()} to reverse the order.

\begin{minted}{c}
int cmpIntDesc(const void *a, const void *b) {
  return cmpInt(b, a);
}
\end{minted}

The examples above illustrate the standard implementation pattern of 
comparator functions:
\begin{enumerate}
  \item Make the general \mintinline{c}{const void *} pointers into a 
    \mintinline{c}{const} pointer to the specific type you want to compare
    by making an explicit type cast.
  \item Use the state of the type (one or more components if you are comparing
  	structures) to determine their order.
  \item Return an integer that expresses the desired order of elements.
\end{enumerate}

The cautious observer may be asking: what happens if I call a comparator
and pass it pointers to mismatched elements?  For example, what if I pass
two \mintinline{c}{double} pointers to \mintinline{c}{cmpInt()}?  Obviously, the
code will not work as intended (it will end up comparing the upper 32 bits of the
\mintinline{c}{double} values, which leads to unintended behavior).  However, 
the function was designed with certain expectations (namely, that a user would
only ever provide it valid, pointers to \mintinline{c}{int} values).  If
a user violates these expectations, its their fault, not ours.  This is no
different from many other functions in the standard library.  Passing non-null
terminated strings to most string functions for example is undefined behavior.

To illustrate some more examples, consider the \mintinline{c}{Student} structure
we defined in Code Sample \ref{code:c:studentStructure}.  The following Code
Samples demonstrate various ways of ordering \mintinline{c}{Student} structures based on one or more of their components.

\begin{minted}{c}
/**
 * A comparator function to order Student structures by 
 * last name/first name in alphabetic order
 */
int studentByNameCmp(const void *s1, const void *s2) {

  const Student *a = (const Student *)s1;
  const Student *b = (const Student *)s2;
  int result = strcmp(a->lastName, b->lastName);
  if(result == 0) {
    return strcmp(a->firstName, b->firstName);
  } else {
    return result;
  }
}
\end{minted}

\begin{minted}{c}
/**
 * A comparator function to order Student structures by 
 * last name/first name in reverse alphabetic order
 */
int studentByNameCmpDesc(const void *s1, const void *s2) {

  return studentByNameCmp(s2, s1);
}
\end{minted}

\begin{minted}{c}
/**
 * A comparator function to order Student structures by 
 * id in ascending numerical order
 */
int studentIdCmp(const void *s1, const void *s2) {

  const Student *a = (const Student *)s1;
  const Student *b = (const Student *)s2;
  if(a->nuid < b->nuid) {
    return -1;
  } else if(a->nuid == b->nuid) {
    return 0;
  } else {
    return 1;
  }
}
\end{minted}

\begin{minted}{c}
/**
 * A comparator function to order Student structures by 
 * GPA in descending order
 */
int studentGpaCmp(const void *s1, const void *s2) {

  const Student *a = (const Student *)s1;
  const Student *b = (const Student *)s2;
  if(a->gpa > b->gpa) {
    return -1;
  } else if(a->gpa == b->gpa) {
    return 0;
  } else {
    return 1;
  }
}
\end{minted}

\section{Function Pointers}

Now that we have comparator functions to order elements, we need a way
of passing a comparator to a generic search or sort function so that it
can be \emph{used} by that function.  To achieve this in C, we use
\emph{function pointers}.  

Recall that a pointer is simply a reference to some memory location.
As we've already seen, pointers can point to variables of simple
data types such as \mintinline{c}{int} or \mintinline{c}{double}, 
arrays of these types or even user-defined structures.  We've also
seen that pointers can be generic using the \mintinline{c}{void} keyword.
The \mintinline{c}{malloc()} function for example, returned a generic
void pointer, \mintinline{c}{(void *)}, to allow it to be used to allocate
memory for \emph{any} type.

Generic void pointers simply point to the start of a memory block, not 
necessarily to a memory location of any particular type of variable.  As
we've also already seen, the program code itself lives in memory (the
stack).  It thus makes sense that we could reference a memory location
that, instead of containing variables, contains executable code, in 
particular a \emph{function}.  This is what a function pointer does:
it points to a memory location where the code for the function is
stored.

To declare a function pointer, we need to specify more information than a
typical \mintinline{c}{int *} or \mintinline{c}{double *} pointer.  Since
it points to a function, we need to specify the function's signature: its
return type and parameter list.  Consider the following example:

\mintinline{c}{int (*ptrToFunc)(int, double, char) = NULL;}

In this declaration, we've created a function pointer named \mintinline{c}{ptrToFunc}.
This pointer is capable of pointing to any function whose return type is
\mintinline{c}{int} (indicated by the first keyword) and which takes 
\emph{three} parameters: an \mintinline{c}{int}, a \mintinline{c}{double} 
and a \mintinline{c}{char}.  The assignment operation has initialized this
pointer to \mintinline{c}{NULL}.

Consider another example: let's declare a function pointer that is capable
of pointing to the math library's \mintinline{c}{sqrt()} function.  The
\mintinline{c}{sqrt()} function returns a \mintinline{c}{double} and takes
a single \mintinline{c}{double} parameter.  Adapting the syntax above, 
we would create this pointer as follows.

\mintinline{c}{double (*ptrToSqrt)(double) = NULL;}

Again, we have initialized it to \mintinline{c}{NULL}; it does not yet
reference the \mintinline{c}{sqrt()} function.  To make it reference this
function, we need to assign it a value.  Recall that to get the memory
location of a regular variable, say \mintinline{c}{int x;}, we use the
referencing operator, \mintinline{c}{&x}.  Similarly, with functions we
\emph{can} use the referencing function, in this case \mintinline{c}{&sqrt}
to get the memory address of the \mintinline{c}{sqrt()} function, however
this is not necessary.  Similar to arrays, the identifier (name) of a
function serves as its memory address!  That is, \mintinline{c}{sqrt} by
itself is the memory location of the function.  Thus to assign 
\mintinline{c}{ptrToSqrt} to point to \mintinline{c}{sqrt()}, we simply
need to do the following:

\mintinline{c}{ptrToSqrt = sqrt;}

Note the difference: usually we \emph{invoke} \mintinline{c}{sqrt()} by 
writing parentheses and providing an argument.  Additional examples 
of this syntax can be found in Code Sample \ref{code:c:moreFunctionPointerSyntax}.

\subsubsection{Callbacks}

The main use for function pointers is so that references to functions can be passed 
as parameters to other functions.  The passed function is known as a 
\gls{callback}.  This gives us the ability to write a more abstract and
generic function.  

For example, in \gls{guiLabel} programming, we frequently need to associate
a particular function with a particular \emph{event}.  For example, suppose
we create a button; we need to be able to specify what happens when that
button gets clicked.  We do so by providing a function as a callback to
a \emph{registration} function that associates the ``click'' event with the
provided function.  Thus, whenever a user clicks the button, the callback
is invoked (``called back'').  

Let's illustrate some syntax usage with another example.  Suppose we want to 
create a \mintinline{c}{getMax()} function.  We \emph{could} write one 
function for arrays of integers, another for arrays of \mintinline{c}{double}s, 
another for \mintinline{c}{Student} arrays that gets the student with the
maximum GPA, then another for the ID, then another for the name and so on.  
Or, we could program a \emph{generic} \mintinline{c}{getMax()} function
that could be used for \emph{any} type by taking a comparator function as
a callback.  To illustrate, consider the following non-generic function for
integers.

\begin{minted}{c}
int getMax(const int *arr, int n) {
  int i, max_index = 0;
  for(i=1; i<n; i++) {
    if(arr[max_index} < arr[i] {
      //we've found something larger, update the max_index:
      max_index = i;
    }
  }
  return max_index;
}
\end{minted}

This simple function iterates through the array, keeping track of the maximum
value found ``so far'' and updating it when it finds something larger.  Because
we've specified that \mintinline{c}{arr} is an array if \mintinline{c}{int}
values, we can use the less-than comparison operator (line 4).  Now let's make
it more generic: rather than taking an array of \mintinline{c}{int} values, 
it will now take a generic \mintinline{c}{void} array.  Further, we will
pass a function pointer to this function that references a generic comparator
function that can be used to replace the less-than comparison operator.

\begin{minted}{c}
int getMax(const void *arr, int n, 
           int(*cmp)(const void *, const void *)) {
  int i, max_index = 0;
  for(i=1; i<n; i++) {
    ...
  }
  return max_index;
}
\end{minted}

There are a couple of issues here that we have to deal with here.  When
working with generic \mintinline{c}{void *} pointers in C and using arrays, 
you cannot simply index using the usual 0, 1, 2, etc. indices.  Recall that 
when elements are stored in an array, the index represents an \emph{offset} 
of a memory address.  If the array is an array of integers or 
\mintinline{c}{double} or some other build-in type, the compiler knows how 
large each one is and is able to compute the appropriate offset given the 
usual 0, 1, 2, etc. indices.
	
However, when dealing with \mintinline{c}{void *} elements, a function must 
be told how many bytes each element takes, say \mintinline{c}{size}.  Then 
each element can be indexed by multiplying an index by the size.  That is, 

\begin{minted}{c}
arr[0 * size] //first element
arr[1 * size] //second element
arr[2 * size] //third element
...
arr[i * size] //i-th element
\end{minted}

Thus, we also need to provide this to our \mintinline{c}{getMax()} function.
Once we do, we can use the comparator to determine which is the larger of
two elements in the array (indexed using the scheme above).  To do this, we
call the comparator on the maximum element we've found so far and the 
$i$-th element in the loop.  If it returns something negative, then we know
that the ``max'' element is less than the $i$-th element and so update our 
\mintinline{c}{max_index} variable.  Making these changes results in the
this final version.

\begin{minted}{c}
int getMax(const void *arr, int n, size_t size, 
           int(*cmp)(const void *, const void *)) {
  int i, max_index = 0;
  for(i=1; i<n; i++) {
    if(cmp(arr[max_index * size], arr[i * size]) < 0) {
      //we've found something larger, update the max_index:
      max_index = i;
    }
  }
  return max_index;
}
\end{minted}

Now to illustrate how we can use this generic function.  Suppose we have
an array of integers and the \mintinline{c}{cmpInt()} comparator function
above.  

\begin{minted}{c}
int arr[] = {8, 2, 9, 10, 4, 2, 2, 5, 6, 7};
int max_index = getMax(arr, 10, sizeof(int), cmpInt);
printf("maximum value: %d\n", arr[max_index]);
\end{minted}

In this example, the \mintinline{c}{getMax()} function would return
the index 3 and print the maximum value stored there, 10.  The 
\mintinline{c}{getMax()} function returns the index corresponding to the
``maximum'' element \emph{according to the comparator} used.  If instead
we had used \mintinline{c}{cmpIntDesc()} the ``maximum'' element would have
been the least element, (2 in the example above) because it would have
been the element ordered ``last'' by the \emph{descending} comparator.

Consider another example with our \mintinline{c}{Student} structure.

\begin{minted}{c}
int n = 10;
Student *roster = (Student *) malloc(sizeof(Student) * n);

...

int max_index = getMax(roster, n, sizeof(Student), studentByNameCmp);
\end{minted}

Since \mintinline{c}{studentByNameCmp()} orders by lexicographic ordering, 
a student with the last name ``Zadora'' would have been ``larger'' than 
someone with a last name ``Anderson.''  Thus the code above will return 
an index corresponding to the \emph{last} student in lexicographic ordering
of their name.

Similarly, if we had used the \mintinline{c}{studentGpaCmp()} comparator 
instead, \mintinline{c}{getMax()} would have returned an index for the
student with the \emph{lowest} GPA as this comparator ordered highest to
lowest.

Another variation on this function would be to return a \emph{pointer}
to the maximum element rather than an index.  The same logic would have
applied, but the return type would be a \mintinline{c}{void *} pointer
and the return statement would return the memory address of the maximum
element instead.

\begin{minted}{c}
void * getMax(const void *arr, int n, size_t size, 
              int(*cmp)(const void *, const void *)) {
  int i, max_index = 0;
  for(i=1; i<n; i++) {
    if(cmp(arr[max_index * size], arr[i * size]) < 0) {
      //we've found something larger, update the max_index:
      max_index = i;
    }
  }
  return &arr[max_index * size];
}
\end{minted}

\subsubsection{Full Example}

\begin{listing}[H]
\inputminted[fontsize=\scriptsize]{c}{code/functionPointers.c}
\caption{C Function Pointer Syntax Examples}
\label{code:c:moreFunctionPointerSyntax}
\end{listing}

\section{Searching \& Sorting}

We now turn our attention to the search and sorting functions provided by
the standard library.  As you can imagine, both are generic implementations
that take advantage of function pointers and comparator functions.

\subsection{Searching}

\subsubsection{Linear Search}

The C search library, \mintinline{c}{search.h} provides a linear search 
function to search arrays, named \mintinline{c}{lfind()} (linear find).  This
function does not require that the array be sorted and performs a linear
search algorithm, returning a \emph{pointer} to the first element such that
the comparator returns 0 (indicating equality).  The full signature of the
function is as follows.

\begin{minted}{c}
void *lfind(const void *key, 
            const void *base, 
            size_t *nmemb, 
            size_t size, 
            int(*compar)(const void *, const void *));
\end{minted}

Each argument represents:
\begin{itemize}
  \item \mintinline{c}{key} -- a ``dummy'' structure instance matching the
    element you are searching for.  For example, if you are searching for a
    \mintinline{c}{Student} with the last name \mintinline{c}{"Smith"} then
    you can construct an instance with the same last name, ignoring the value
    of all other components.  
  \item \mintinline{c}{base} -- a pointer to the array to be searched
  \item \mintinline{c}{nmemb} -- the size of the array (number of ``members'')
  \item \mintinline{c}{size} -- the size, in bytes, of each element in the array
   	(generally you use use \mintinline{c}{sizeof()} to determine this)
  \item \mintinline{c}{compar} -- a pointer to a comparator function to use
    in the search.
\end{itemize}

As with our \mintinline{c}{getMax()} example, \mintinline{c}{lfind()} returns
a pointer to the element it finds.  If no such element is found, this function
will return \mintinline{c}{NULL}.  

In the same library, there is another linear search function named 
\mintinline{c}{lsearch()} with the same parameter types (except that the array
is not \mintinline{c}{const} as we'll see why).  However, it differs
in that if it does not find a matching element, it still returns \mintinline{c}{NULL}
but also attempts to \emph{insert} the element being searched for at the
end of the array.  That is, the function will assume there is room at the end
of the array to accommodate the inserted element (if not, the behavior is undefined).
The function will also increment the \mintinline{c}{nmemb} variable to reflect
this new element.  Great care should be exercised when using this version to ensure
that there is enough valid memory to accommodate any inserts.

\subsubsection{Binary Search}

In the standard library (\mintinline{c}{stdlib.h}) there is an additional binary
search function that can be used to more efficiently search a \emph{sorted}
array.  The prototype:

\begin{minted}{c}
void *bsearch(const void *key, 
              const void *base, 
              size_t nmemb, 
              size_t size, 
              int (*compar)(const void *, const void *));
\end{minted}

All parameters are exactly as with \mintinline{c}{lfind()} as is the behavior:
it returns a pointer to the first element that it finds (though first does not  
necessarily mean the first in the order of the array) and \mintinline{c}{NULL} if 
no matching element is found.

It is an important requirement that the array be sorted \emph{with the same
comparator} as was used to sort the array or \mintinline{c}{NULL} may be 
returned erroneously.

\subsection{Sorting}

The standard library (\mintinline{c}{stdlib.h}) also provides a generic sorting
function, \mintinline{c}{qsort()}.  Though the name suggests a Quick Sort implementation, it does not necessarily have to be (it was when the function was
designed).  Modern implementations of \mintinline{c}{qsort()} may implement 
alternatives such as Merge Sort or non-recursive hybrid Quick Sort algorithms.

The prototype and parameters are similar to the search functions but do not
include a key.  The array is also not \mintinline{c}{const} indicating that
it \emph{will} be changed (which is the whole point of calling the function).

\begin{minted}{c}
void qsort(void *base, 
           size_t nmemb, 
           size_t size, 
           int(*compar)(const void *, const void *))
\end{minted}           

\begin{itemize}
  \item \mintinline{c}{base} -- pointer an array of elements
  \item \mintinline{c}{nmemb} -- the size of the array (number of members)
  \item \mintinline{c}{size} -- the size (in bytes) of each element (use \mintinline{c}{sizeof()})
  \item \mintinline{c}{compar} -- a comparator function used to order elements
  \item Sorts in ascending order according to the provided comparator function
\end{itemize}

The advantages to using \mintinline{c}{qsort()} (as well as \mintinline{c}{lfind()} 
and \mintinline{c}{bsearch()}) should be clear.  There is no need to write a 
new function that reimplements the same algorithm for every possible ordering of
every possible user-defined structure.  We need only to create a comparator
function and pass it to \mintinline{c}{qsort()}.  There is less code, and less
chance of bugs.  The \mintinline{c}{qsort()} function is well-designed, 
optimized, and most importantly well-tested and proven.

These functions represent a sort of ``weak'' form of polymorphic behavior found
in more modern object-oriented programming languages and other languages that
support ``generic programming.''  Polymorphism is the characteristic that
the same code can be executed on different types, greatly reducing the need
for duplicate code.

\subsection{Examples}

We illustrate the usage of these functions in Code Samples 
\ref{code:c:searchExamples} and \ref{code:c:sortExamples}.

\begin{listing}[H]
\inputminted[fontsize=\scriptsize]{c}{code/searchDemo.c}.
\caption{C Search Examples}
\label{code:c:searchExamples}
\end{listing}

\begin{listing}[H]
\inputminted[]{c}{code/sortDemo.c}.
\caption{C Sort Examples}
\label{code:c:sortExamples}
\end{listing}

\section{Other Considerations}

\subsection{Sorting Pointers to Elements}

Recall that it is sometimes preferable to maintain an array of pointers
to structures rather than an array of structures.  Sorting is a scenario
where this is particularly true.  When sorting an array of structure
elements, the entire structure is copied back and forth as elements
are swapped.  Depending on the number of bytes of a structure, this can
be quite expensive.  It is generally more efficient to sort an array of 
pointers to structures instead.  A prime example of this is sorting an
array of strings.

An array of strings can be thought of as a 2-dimensional array of 
\mintinline{c}{char}s.  Specifically, an array of strings is a 
\mintinline{c}{char **} type.  That is, an array of pointers to 
pointers of \mintinline{c}{char}s.  We may be tempted to use 
\mintinline{c}{strcmp} in the standard string library, passing it 
to \mintinline{c}{qsort()}.  
Unfortunately this will not work.  \mintinline{c}{qsort()} requires 
two \mintinline{c}{const void *} types, while \mintinline{c}{strcmp} 
takes two \mintinline{c}{const char *} types.  This difference is 
subtle but important.\footnote{A full discussion can be found on the 
c-faq, \url{http://c-faq.com/lib/qsort1.html}.}

The recommended way of doing this is to define a different comparator 
function as follows.

\begin{listing}[H]
\begin{minted}{c}
/* compare strings via pointers */
int pstrcmp(const void *p1, const void *p2)
{
  return strcmp(*(char * const *)p1, *(char * const *)p2);
}
\end{minted}
\caption{C Comparator Function for Strings}
\end{listing}

Observe the behavior of this function: it uses the standard 
\mintinline{c}{strcmp} function, but makes the proper explicit type 
casting before doing so.  The \mintinline{c}{*(char * const *)} casts 
the generic void pointers as pointers to strings (or pointers to 
pointers to characters), then dereferences it to be compatible
with \mintinline{c}{strcmp}.

Another case is when we wish to sort user-defined structures.  The 
\mintinline{c}{Student} structure presented earlier is ``small'' in 
that it only has a few fields.  When structures are stored in an 
array and sorted, there may be many \emph{swaps} of individual 
elements which involves a lot of memory copying.  If the structures 
are small this is not too bad, but for ``larger'' structures this 
could be potentially expensive.  Instead, it may be preferred to 
have an array of \emph{pointers} to structures.  Swapping elements 
involves only swapping pointers instead of the entire structure.  
This is far cheaper as a memory address is likely to be far smaller 
than the actual structure it points to.  This is essentially 
equivalent to the string scenario: we have an array of pointers 
to be sorted, our comparator function then needs to deal with pointers 
to pointers.\footnote{Again, a full discussion can be found on c-faq,
\url{http://c-faq.com/lib/qsort2.html}.}  An example appears in Code 
Sample \ref{code:SortingStructures}.

\begin{listing}[H]
\begin{minted}{c}
Student **roster = (Student **) malloc(sizeof(Student *) * n);
qsort(roster, n, sizeof(Student *), studentPtrLastNameCmp);

...

int studentPtrLastNameCmp(const void *s1, const void *s2) {
  //we receive a pointer to an individual element in the array
  //but individual elements are POINTERS to students thus we cast 
  //them as (const Student **) then dereference to get a pointer 
  //to a Student!
  const Student *a = *(const Student **)s1;
  const Student *b = *(const Student **)s2;
  int result = strcmp(a->lastName, b->lastName);
  if(result == 0) {
    return strcmp(a->firstName, b->firstName);
  } else {
    return result;
  }
}
\end{minted}
\caption{Sorting Structures via Pointers}
\label{code:SortingStructures}
\end{listing}

Another issue when sorting arrays of pointers is that we may now 
have to deal with \mintinline{c}{NULL} elements.  When sorting 
arrays of elements this is not an issue as a properly initialized 
array will contain non-null elements (though elements could still
be uninitialized, the memory space is still valid).

How we handle \mintinline{c}{NULL} pointers is more of a design 
decision.  We could ignore it and any attempt to access a 
\mintinline{c}{NULL} structure will result in undefined
behavior (or segmentation faults, etc.).  Or we could give 
\mintinline{c}{NULL} values an explicit ordering with respect to 
other elements.  That is, we could order all \mintinline{c}{NULL}
pointers \emph{before} non-\mintinline{c}{NULL} elements (and consider 
all \mintinline{c}{NULL} pointers to be equal).  An example with respect 
to our \mintinline{c}{Student} structure is given in Code Snippet 
\ref{code:SortingStructuresWithNulls}.

\begin{listing}[H]
\begin{minted}{c}
int studentPtrLastNameCmpWithNulls(const void *s1, const void *s2) {
  const Student *a = *(const Student **)s1;
  const Student *b = *(const Student **)s2;
  if(a == NULL && b == NULL) {
    return 0;
  } else if(a == NULL && b != NULL) {
    return -1;
  } else if (a != NULL && b == NULL) {
    return 1;
  }
  int result = strcmp(a->lastName, b->lastName);
  if(result == 0) {
    return strcmp(a->firstName, b->firstName);
  } else {
    return result;
  }
}
\end{minted}
\caption{Handling Null Values}
\label{code:SortingStructuresWithNulls}
\end{listing}

