%!TEX root = ComputerScienceOne.tex

\usepackage{xcolor}
\definecolor{darkred}{rgb}{0.75,0,0}
\definecolor{darkblue}{rgb}{0,0,0.5}
\definecolor{darkgreen}{rgb}{0,0.5,0}
\definecolor{darkergreen}{rgb}{0,0.75,0}
\definecolor{darkmagenta}{rgb}{0.55,0,0.55}
\definecolor{left}{HTML}{041832}
\definecolor{secondary}{HTML}{241024}

\usepackage[colorlinks=true,
		     urlcolor=darkblue,
		     citecolor=darkergreen,
		     linkcolor=darkblue,
		     plainpages=false,
		     pdfpagelabels]{hyperref}
\hypersetup{
  pdfpagemode=none,
  pdftitle={Computer Science One},
  pdfauthor={Christopher M. Bourke},
  pdfsubject={Computer Science, Programming},
  pdfkeywords={Computer Science, Programming, C, Java, PHP}
}


%\usepackage{fullpage}
\usepackage{wrapfig}
\usepackage{titling}
\usepackage{multirow}
%\newcommand{\subtitle}[1]{%
%  \posttitle{%
%    \par\end{center}
%    \begin{center}\large#1\end{center}
%    \vskip0.5em}%
%}

%used to make the floating option, [h!] force floats 
% to typeset where we want them
\usepackage{float}

%Used so that we can add captions to "floats" that
%are broken over multiple pages
\usepackage{caption}

%as per online advice
\usepackage{microtype}
%\usepackage[T1]{fontenc}
%\usepackage{lmodern}
\usepackage[ascii]{inputenc}

%http://en.wikibooks.org/wiki/LaTeX/Glossary
%must create the glossaries by using command:
%makeglossaries ComputerScienceOne
\newcommand{\ignore}[1]{}
\usepackage[toc,acronym]{glossaries}
\makeglossaries
%!TEX root = ComputerScienceOne.tex

\newglossaryentry{abstraction}
{
  name=abstraction,
  description={a technique for managing complexity whereby levels of complexity are established so that higher levels do not see or have to worry about details at lower levels}
}

\newglossaryentry{algorithm}
{
  name=algorithm,
  description={a process or method that consists of a specified step-by-step set of operations}
}

\newglossaryentry{anonymous class}
{
  name=anonymous class,
  description={a class that is defined ``inline'' without declaring a named class; typically created because the instance has a single use and there is no reason to create multiple instances},
  plural=anonymous classes
}

\newglossaryentry{anonymous function}
{
  name=anonymous function,
  description={a function that has no identifier or name, typically created so that it can be passed as an argument to another function as a callback},
  plural=anonymous functions
}

\newglossaryentry{anti-pattern}
{
  name=anti-pattern,
  description={a common software pattern that is used as a solution to recurring problems that is usually ineffective in solving the problem or introduces risks and other problems; a technical term for common ``bad-habits'' that can be found in software}
}

\newglossaryentry{array}
{
  name=array,
  description={an ordered collection of pieces of data, usually of the same type}
}

\newglossaryentry{assignment operator}
{
  name=assignment operator,
  description={an operator that allows a user to assign a value to a variable}
}

\newglossaryentry{backward compatible}
{
  name=backward compatible,
  description={a program, code, library, or standard that is compatible with previous versions so that current
  	and older versions of it can coexist and successfully operate without breaking anything}
}

\newglossaryentry{bit}
{
  name=bit,
  description={the basic unit of information in a digital computer.  A bit can be either 1 or 0 (alternatively, \True/\False, 
  	on/off, high voltage/low voltage, etc.).  Originally a portmanteau (mash up) of \textbf{b}inary dig\textbf{it}}
}

\newglossaryentry{Boolean}
{
  name=Boolean,
  description={a data type that represents the truth value of a logical statement.  Booleans typically have only two 
  	values: \True or \False}
}

\newglossaryentry{bug}
{
  name=bug,
  description={A flaw or mistake in a computer program that results in incorrect behavior that may have unintended such as errors or failure.  The
  	term predates modern computer systems but was popularized by Grace Hopper who, when working with the Mark II computer in 1946 traced
  	a system failure to a moth stuck in a relay}
}

\newglossaryentry{byte}
{
  name=byte,
  description={a unit of information in a digital computer consisting of 8 bits}
}

\newglossaryentry{cache}
{
  name=cache,
  description={a component or data structure that stores data in an efficiently retrievable manner so that future requests for the data are fast}
}

\newglossaryentry{callback}
{
  name=callback,
  description={a function or executable unit of code that is passed as an argument to another function with the intention that the function that it is passed to will execute or ``call back'' the passed function at some point}
}

\newglossaryentry{call by reference}
{
  name=call by reference,
  description={when a variable's memory address is passed as a parameter to a function, enabling the function to manipulate the contents of the memory address and change the original variable's value}
}

\newglossaryentry{call by value}
{
  name=call by value,
  description={when a \emph{copy} of a variable's value is passed as a parameter to a function; the function has no reference to the original variable and thus changes to the copy inside the function have no effect on the original variable}
}

\newglossaryentry{case sensitive}
{
  name=case sensitive,
  description={a language is case sensitive if it recognizes differences between lower and upper case characters in 
  	identifier names.  A language is case insensitive if it does not}
}

\newglossaryentry{chomp}
{
  name=chomp,
  description={the operation of removing any endline characters from a string (especially when read from a file); may also refer more generally to removing leading and trailing whitespace from a string or ``trimming'' it}
}

\newglossaryentry{closure}
{
  name=closure,
  description={a function with its own environment in which variables exist}
}

\newglossaryentry{code smell}
{
  name=code smell,
  description={a symptom or common pattern in source code that is usually indicative of a deeper problem or design flaw; smells are usually not bugs and may not cause problems in and of themselves, but instead indicate a pattern of carelessness or low quality of software design or implementation}
}

\newglossaryentry{comparator}
{
  name=comparator,
  description={a function or object that allows you to pass in two elements $a, b$ for comparison and returns an integer indicating their relative order: something negative, zero, or something positive if $a < b$, $a = b$ or $a > b$ respectively}
}

\newglossaryentry{compile}
{
  name=compile,
  description={the process of translating code in a high-level programming language to a low level language such as assembly or machine code}
}

\newglossaryentry{computer engineering}
{
  name=computer engineering,
  description={a discipline integrating electrical engineering and computer science that tends to focus on the development of hardware and its interaction with software}
}

\newglossaryentry{computer science}
{
  name=computer science,
  description={the mathematical modeling and scientific study of computation}
}

\newglossaryentry{concatenation}
{
  name=concatenation,
  description={the process of combining two (or more) strings to create a new string by appending one of them to the end of the other}
}

\newglossaryentry{contradiction}
{
  name=contradiction,
  description={a logical statement that is always \False regardless of the truth values of the statement's variables}
}

\newglossaryentry{constant}
{
  name=constant,
  description={a variable whose value cannot be changed once set}
}

\newglossaryentry{control flow}
{
  name=control flow,
  description={the order in which individual statements in a program are executed or evaluated}
}

\newglossaryentry{cruft}
{
  name=cruft,
  description={anything that is left over, redundant or getting in the way; in the context of code cruft is code that is no longer needed, legacy or simply poorly written source code}
}

\newglossaryentry{dangling pointer}
{
  name=dangling pointer,
  description={when a reference to dynamically allocated memory is lost and the memory can no longer be deallocated, resulting in a memory leak.  Alternatively, when a reference points to memory that gets deallocated or reallocated but the pointer remains unmodified, still referencing the deallocated memory}
}

\newglossaryentry{dead code}
{
  name=dead code,
  description={a code segment that has no effect on a program either because it is unused or unreachable (the 
	conditions involving the code will never be satisfied)}
}

\newglossaryentry{debug}
{
  name=debug,
  description={the process of analyzing a program to find a fault or error with the code that leads to bad or unexpected results}
}

\newglossaryentry{debugger}
{
  name=debugger,
  description={a software tool that facilitates debugging; usually a debugger simulates the execution of a program allowing a developer to view the contents of a program as it executes and to ``walk'' through the execution step by step}
}

\newglossaryentry{deep copy}
{
  name=deep copy,
  description={in contrast to a shallow copy, a deep copy is a copy of an array or other piece of data that is distinct from the original.  Changes to one copy do not affect the other}
}

\newglossaryentry{defensive programming}
{
  name=defensive programming,
  description={an approach to programming in which error conditions are checked and handled, preventing undefined or
  	erroneous operations from happening in a program}
}

\newglossaryentry{dynamic programming}
{
  name=dynamic programming,
  description={a technique for solving problems that involves iteratively computing values to subproblems, storing them in a table so that they can be used to solve larger versions of the problem}
}

\newglossaryentry{dynamic typing}
{
  name=dynamic typing,
  description={a variable whose type can change during runtime based on the value it is assigned}
}

\newglossaryentry{encapsulation}
{
  name=encapsulation,
  description={the grouping and protection of data together into one logical entity along with the functionality (functions or methods)
  	that act on that data}
}

\newglossaryentry{enumerated type}
{
  name=enumerated type,
  description={a data type (usually user defined) that consists of a list of named values}
}

\newglossaryentry{exception}
{
  name=exception,
  description={an event or occurrence of an erroneous or ``exceptional'' condition that interrupts the normal flow of control in a program, handing control over to exception handler(s)}
}

\newglossaryentry{expression}
{
  name=expression,
  description={a combination of values, constants, literals, variables, operators and possibly function calls such that when evaluated, produce a resulting value}
}

\newglossaryentry{file}
{
  name=file,
  description={a resource on a computer stored in memory that holds data}
}

\newglossaryentry{flowchart}
{
  name=flowchart,
  description={a diagram that represents an algorithm or process, showing steps as boxes connected by arrows which establish an
  order or flow}
}

\newglossaryentry{function}
{
  name=function,
  description={a sequence of program instructions that perform a specific task, packaged as a unit, also
  known as a \emph{subroutine}}
}

\newglossaryentry{function overloading}
{
  name=function overloading,
  description={the ability to define multiple functions with the same name but with with a different number of or different types of parameters}
}

\newglossaryentry{garbage collection}
{
  name=garbage collection,
  description={automated memory management in which a garbage collector attempts to reclaim memory (garbage) that is no
	longer being used by a program so that it can be reallocated for other purposes}
}

\newglossaryentry{global scope}
{
  name=global scope,
  description={a variable, function, or other element in a program has global scope if it is visible or has effect throughout
  	the entire program}
}

\newglossaryentry{grok}
{
  name=grok,
  description={slang, meaning to understand something}
}

\newglossaryentry{hardware}
{
  name=hardware,
  description={(computer hardware) the physical components that make up a computer system such as the processor, motherboard, storage devices, input and output devices, etc.}
}

\newglossaryentry{hexadecimal}
{
  name=hexadecimal,
  description={base-16 number system using the symbols 0, 1, \ldots, 9, A, B, C, D, E, F; usually denoted with a prefix \mintinline{text}{0x} such as
\mintinline{text}{0xff1321ab01}}
}


\newglossaryentry{hoisting}
{
  name=hoisting,
  description={usually used in interpreted languages, hoisting involves processing code to find variable or function declarations and processing them before actually executing the code or script}
}

\newglossaryentry{identifier}
{
  name=identifier,
  description={a symbol, token, or label that is used to refer to a variable.  Essentially, a variable's name}
}

\newglossaryentry{idiom}
{
  name=idiom,
  description={in the context of programming, an idiom is a commonly used pattern, expression or way of structuring code that is well-understood for users of the language.  For example, a for-loop structure that iterates over elements in an array.  May also refer to a programming design pattern.}
}

\newglossaryentry{inheritance}
{
  name=inheritance,
  description={an object oriented programming principle that allows you to derive an object from another object, usually to allow for more specificity}
}

\newglossaryentry{immutable}
{
  name=immutable,
  description={an object whose internal state cannot be changed once created, alternatively, one whose internal state
  cannot be \emph{observably} changed once created}
}

\newglossaryentry{input}
{
  name=input,
  description={data or information that is provided to a computer program for processing}
}

\newglossaryentry{interactive}
{
  name=interactive,
  description={a program that is designed to interface with humans by prompting them for input and displaying output directly to them}
}

\newglossaryentry{keyword}
{
  name=keyword,
  description={a word in a programming language with a special meaning in a particular context.  In
	contrast to a reserved word, a keyword \emph{may} be used for an identifier (variable or function name)
	but it is strongly discouraged to do so as the keyword already has an intended meaning}
}

\newglossaryentry{kilobyte}
{
  name=kilobyte,
  description={a unit of information in a digital computer consisting of 1024 bytes (equivalently, $2^{10}$ bytes), KB for short}
}

\newglossaryentry{kludge}
{
  name=kludge,
  description={a poorly designed or ``thrown-together'' solution; a design that is a collection of ill-fitting parts that may be functional, but is fragile and not easily maintained}
}

\newglossaryentry{lambda expression}
{
  name=lambda expression,
  description={another term for anonymous functions},
  plural=lambda expressions
}

\newglossaryentry{lexicographic}
{
  name=lexicographic,
  description={a generalization of the usual dictionary order as codified with the ASCII character table}
}

\newglossaryentry{linking}
{
  name=linking,
  description={the process of generating an executable file from (multiple) object files}
}

\newglossaryentry{lint}
{
  name=lint,
  description={(or linter) a static code analysis tool that analyzes code for suspicious or error-prone code that is likely to cause problems}
}

\newglossaryentry{literal}
{
  name=literal,
  description={in a programming language, a literal is notation for specifying a value such as a number of string that can be 
  	directly assigned to a variable}
}

\newglossaryentry{map}
{
  name=map,
  description={a data structure that allows you to store elements as key-value pairs with the key mapping to a value}
}

\newglossaryentry{magic number}
{
  name=magic number,
  description={a value used in a program with unexplained, undocumented, or ambiguous meaning, usually making the code less understandable}
}

\newglossaryentry{mantissa}
{
  name=mantissa,
  description={the part of a floating-point number consisting of its significant digits (called a significand in scientific notation)}
}

\newglossaryentry{memoization}
{
  name=memoization,
  description={a technique which uses a table to store previously computed values of a function so that they do not need to be recomputed, essentially the table serves as a cache}
}

\newglossaryentry{memory leak}
{
  name=memory leak,
  description={the gradual loss of memory when a program fails to deallocate or free up unused memory, degrading performance and possibly resulting in the termination of the program when memory runs out}
}

\newglossaryentry{naming convention}
{
  name=naming convention,
  description={a set of guidelines for choosing identifier names for variables, functions, etc. in a programming language.  Conventions may be generally accepted by all developers of a particular language or they may be established for use in a particular library, framework, or organization}
}

\newglossaryentry{network}
{
  name=network,
  description={a collection of two or more computer systems linked together through a physical connection over which data can be transmitted using some protocol}
}

\newglossaryentry{octal}
{
  name=octal,
  description={base-8 number system using the symbols 0, 1, 2, \ldots, 6, 7; usually denoted with a single leading zero such as
\mintinline{text}{0123742}}
}

\newglossaryentry{open recursion}
{
  name=open recursion,
  description={a mechanism by which an object is able to refer to itself
  usually using a keyword such as \mintinline{text}{this} or 
  \mintinline{text}{self}.}
}

\newglossaryentry{operand}
{
  name=operand,
  description={the arguments that an operator applies to}
}

\newglossaryentry{operator}
{
  name=operator,
  description={a symbol used to denote some transformation that combines or changes the operands it is applied to to produce a new value}
}

\newglossaryentry{order of precedence}
{
  name=order of precedence,
  description={the order in which operators are evaluated, multiplication is performed before addition for example}
}

\newglossaryentry{output}
{
  name=output,
  description={data or information that is produced as the result of the execution of a program}
}

\newglossaryentry{overflow}
{
  name=overflow,
  description={when an arithmetic operation results in a number that is larger than the specified type can represent overflow occurs resulting in an invalid result}
}

\newglossaryentry{parse}
{
  name=parse,
  description={to process data to identify its individual components or elements}
}

\newglossaryentry{persistence}
{
  name=persistence,
  description={the characteristic of data that outlives the process or program that created it; the saving of data across multiple runs of a program}
}

\newglossaryentry{pointer}
{
  name=pointer,
  description={a reference to a particular memory location in a computer}
}

\newglossaryentry{polymorphism}
{
  name=polymorphism,
  description={an object oriented programming concept that allows you to treat a variable, method, or object as different types}
}

\newglossaryentry{primitive}
{
  name=primitive,
  description={a basic data type that is defined and provided by a programming language.  Typically numeric and character types are primitive types in a language for example.  Generally, the user doesn't need to define the operations involving primitive types as they are defined by the language.  Primitive data types are used as the basic building blocks in a program and used to \emph{compose} more complex user-defined types}
}

\newglossaryentry{procedural abstraction}
{
  name=procedural abstraction,
  description={the concept that a procedure or sequence of operations can be encapsulated into one logical unit (function, subroutine, etc.) so that a user need not concern themselves with the low-level details of how it operates}
}

\newglossaryentry{program stack}
{
  name=program stack,
  description={also referred to as a \emph{call stack}, it is an area of memory where stack frames are stored for each function call containing memory for arguments, local variables and return values/addresses}
}

\newglossaryentry{prompt}
{
  name=interactive,
  description={an informal, abstract, high-level description of a process or algorithm}
}

\newglossaryentry{protocol}
{
  name=protocol,
  description={a set of rules or procedures that define how communication takes place}
}

\newglossaryentry{pseudocode}
{
  name=pseudocode,
  description={the act of a program asking a user to enter input and subsequently waiting for the user to enter data}
}

\newglossaryentry{queue}
{
  name=queue,
  description={a data structure that store elements in a FIFO (First-In First-Out) manner; elements can be added to the end of a queue by an \emph{enqueue} operation and removed from the start of a queue by a \emph{dequeue} operation}
}

\newglossaryentry{query}
{
  name=query,
  description={an operation that retrieves data, typically from a database}
}


\newglossaryentry{radix}
{
  name=radix,
  description={the base of a number system.  Binary, octal, decimal, hexadecimal would be base 2, 8, 10, and 16 respectively}
}

\newglossaryentry{reentrant}
{
  name=reentrant,
  description={a function that can be interrupted during its execution while another thread can successfully invoke the function without the two functions interfering with the data used in either function call}
}

\newglossaryentry{regular expression}
{
  name=regular expression,
  description={a sequence of characters in which special characters and directives can be used to define a complex pattern that can be searched and matched in another string or data}
}

\newglossaryentry{refactor}
{
  name=refactor,
  description={the process of modifying, updating or restructuring code without changing its external behavior; refactoring may be done to make code more efficient, more readable, more reliable, or simply to bring it into compliance with style or coding conventions}
}

\newglossaryentry{reference}
{
  name=reference,
  description={a reference in a computer program is a variable that refers to an object or function in memory}
}

\newglossaryentry{reserved word}
{
  name=reserved word,
  description={a word or identifier in a language that has a special meaning to the syntax of the language and 
  	therefore cannot be used as an identifier in variables, functions, etc.}
}

\newglossaryentry{scope}
{
  name=scope,
  description={the \emph{scope} of a variable, method, or other entity in a program
  	is the part of the program in which the name or reference of the entity is bound.
	That is, the part of the program that ``knows'' about the variable in which the variable
	can be accessed, changed, or used}
}

\newglossaryentry{segmentation fault}
{
  name=segmentation fault,
  description={a fault or error that arises when a program attempts to access a segment of memory that it is not allowed access to, usually resulting in the program being terminated by the operating system}
}

\newglossaryentry{shallow copy}
{
  name=shallow copy,
  description={in contrast to a deep copy, a shallow copy is merely a reference that refers to the original array or piece of data.  The two references point to the same data, so if the data is modified, both references will realize it.}
}

\newglossaryentry{short circuiting}
{
  name=short circuiting,
  description={the process by which the second operand in a logical statement is not evaluated if the
  	value of the expression is determined by the first operand}
}

\newglossaryentry{signature}
{
  name=signature,
  description={a function signature is how a function is uniquely identified.  A signature includes the name (identifier) of the function, its parameter list (and maybe types) and the return type}
}

\newglossaryentry{software}
{
  name=software,
  description={any set of machine-readable instructions that can be executed in a computer processor}
}

\newglossaryentry{software engineering}
{
  name=software engineering,
  description={the study and application of engineering principles to the design, development, and maintenance of complex software systems}
}

\newglossaryentry{spaghetti code}
{
  name=spaghetti code,
  description={a negative term used for code that is overly complex, disorganized or unstructured code}
}

\newglossaryentry{stack}
{
  name=stack,
  description={a data structure that stores elements in a LIFO (last-in first-out) manner; elements can be added to a stack via a \emph{push} operation which places the element on the ``top'' of the stack; elements can be removed from the top of the stack via a \emph{pop} operation}
}

\newglossaryentry{stack overflow}
{
  name=stack overflow,
  description={when a program runs out of stack space, it may result in a stack overflow and the termination of the program}
}

\newglossaryentry{static analysis}
{
  name=static analysis,
  description={the analysis of software that is performed on source (or object) code without actually running or compiling a program usually by using an automated tool that can detect actual or potential problems with the source code (other than syntactic problems that could easily be found by a compiler)}
}

\newglossaryentry{static dispatch}
{
  name=static dispatch,
  description={when function overloading is supported in a language, this is the mechanism by which the compiler determines \emph{which} function should be called based on the number and type of arguments passed to the function when it is called}
}

\newglossaryentry{static typing}
{
  name=static typing,
  description={a variable whose type is specified when it is created (declared) and does not change while the
  	variable remains in scope}
}

\newglossaryentry{string}
{
  name=string,
  description={a data type that consists of a sequence of characters which are encoded under some encoding standard such as ASCII or Unicode}
}

\newglossaryentry{string concatenation}
{
  name=string concatenation,
  description={an operation by which a string and another data type are combined to form a new string}
}

\newglossaryentry{syntactic sugar}
{
  name=syntactic sugar,
  description={syntax in a language or program that is not absolutely necessary (that is, the
  	same thing can be achieved using other syntax), but may be shorter, more convenient, or
	easier to read/write.  In general, such syntax makes the language ``sweeter'' for the
	humans reading and writing it}
}

\newglossaryentry{tautology}
{
  name=tautology,
  description={a logical statement that is always \True regardless of the truth values of the statement's variables}
}

\newglossaryentry{top-down design}
{
  name=top-down design,
  description={an approach to problem solving where a problem is broken down into smaller parts}
}

\newglossaryentry{token}
{
  name=token,
  description={when something (usually a string) is parsed, the individual components or elements are referred to as tokens}
}

\newglossaryentry{transpile}
{
  name=transpile,
  description={to (automatically) translate code in one programming language into code in another programming language, usually between two high-level programming languages}
}

\newglossaryentry{truncation}
{
  name=truncation,
  description={removing the fractional part of a floating-point number to make it an integer.  Truncation is \emph{not} a 
  	rounding operation}
}

\newglossaryentry{two's complement}
{
  name=two's complement,
  description={A way of representing signed (positive and negative) integers using the first bit as a sign bit (0 for positive, 1 for negative) and where negative numbers are represented as the complement with respect to $2^n$ (the result of subtracting the number from $2^n$) }
}

\newglossaryentry{type}
{
  name=type,
  description={a variable's type is the classification of the data it represents which could be numeric, string, boolean, or
  	a user defined type}
}

\newglossaryentry{type casting}
{
  name=type casting,
  description={converting or variable's type into another type, for example, converting an integer into a more general floating-point number, or
  	converting a floating-point number into an integer, truncating and losing the fractional part}
}

\newglossaryentry{underflow}
{
  name=underflow,
  description={when an arithmetic operation involving floating-point numbers results in a number that is smaller than the smallest representable
  number underflow occurs resulting in an invalid result}
}

\newglossaryentry{unwinding}
{
  name=unwinding,
  description={the process of removing a stack frame when returning from a function}
}

\newglossaryentry{Unicode}
{
  name=Unicode,
  description={an international character encoding standard used in programming languages and data formats}
}

\newglossaryentry{validation}
{
  name=validation,
  description={the process of verifying that data is correct or conforms to certain expectations including formatting, type, range of values,
  	represents a valid value, etc.}
}

\newglossaryentry{variable}
{
  name=variable,
  description={a memory location which stores a value that may be set using an assignment operator.  Typically a variable
  	is referred to using a name or \emph{identifier}}
}

\newglossaryentry{widget}
{
  name=widget,
  description={a generic term for a graphical user interface component such as a button or text box}
}

\newacronym{acmLabel}{ACM}{Association for Computing Machinery}

\newacronym{aluLabel}{ALU}{Arithmetic and Logic Unit}

\newacronym{apiLabel}{API}{Application Programmer Interface}

\newacronym[longplural={American National Standards Institute}]{ansiLabel}{ANSI}{American National Standards Institute}

\newacronym{asciiLabel}{ASCII}{American Standard Code for Information Interchange}

\newacronym{ceLabel}{CE}{Computer Engineering}

\newacronym{claLabel}{CLA}{Command Line Arguments}

\newacronym{cliLabel}{CLI}{Command Line Interface}

\newacronym{cmsLabel}{CMS}{Content Management System}

\newacronym{cmykLabel}{CMYK}{Cyan-Magenta-Yellow-Key}

\newacronym{cpuLabel}{CPU}{Central Processing Unit}

\newacronym{csLabel}{CS}{Computer Science}

\newacronym{cssLabel}{CSS}{Cascading Style Sheets}

\newacronym{csvLabel}{CSV}{Comma Separated Values}

\newacronym{cyaLabel}{CYA}{Cover Your Ass}

\newacronym{cwdLabel}{CWD}{Current Working Directory}

\newacronym{dryLabel}{DRY}{Don't Repeat Yourself}

\newacronym{ebLabel}{EB}{Exabyte}

\newacronym{ecmaLabel}{ECMA}{European Computer Manufacturers Association}

\newacronym{ediLabel}{EDI}{Electronic Data Interchange}

\newacronym{eofLabel}{EOF}{End Of File}

\newacronym{fifoLabel}{FIFO}{First-In First-Out}

\newacronym{fossLabel}{FOSS}{Free and Open Source Software}

\newacronym{gbLabel}{GB}{Gigabyte}

\newacronym{gccLabel}{GCC}{GNU Compiler Collection}

\newacronym{gdbLabel}{GDB}{GNU Debugger}

\newacronym{gimpLabel}{GIMP}{GNU Image Manipulation Program}

\newacronym{gifLabel}{GIF}{Graphics Interchange Format}

\newacronym{gisLabel}{GIS}{Geographic Information System}

\newacronym{gnuLabel}{GNU}{GNU's Not Unix!}

\newacronym{guiLabel}{GUI}{Graphical User Interface}

\newacronym{htmlLabel}{HTML}{HyperText Markup Language}

\newacronym{ideLabel}{IDE}{Integrated Development Environment}

\newacronym{jdbcLabel}{JDBC}{Java Database Connectivity}

\newacronym{jdkLabel}{JDK}{Java Development Kit}

\newacronym{jeeLabel}{JEE}{Java Enterprise Edition}

\newacronym{jitLabel}{JIT}{Just In Time}

\newacronym{jpegLabel}{JPEG}{Joint Photographic Experts Group}

\newacronym{jreLabel}{JRE}{Java Runtime Environment}

\newacronym{jsonLabel}{JSON}{JavaScript Object Notation}

\newacronym{jvmLabel}{JVM}{Java Virtual Machine}

\newacronym{iecLabel}{IEC}{International Electrotechnical Commission}

\newacronym{ieeeLabel}{IEEE}{Institute of Electrical and Electronics Engineers}

\newacronym{ipLabel}{IP}{Internet Protocol}

\newacronym{isoLabel}{ISO}{International Organization for Standardization}

\newacronym{kbLabel}{KB}{Kilobyte}

\newacronym{lifoLabel}{LIFO}{Last-In First-Out}

\newacronym{macLabel}{MAC}{Media Access Control}

\newacronym{mbLabel}{MB}{Megabyte}

\newacronym{mpegLabel}{MPEG}{Moving Picture Experts Group}

\newacronym{mp3Label}{MP3}{MPEG-2 Audio Layer III}

\newacronym{nistLabel}{NIST}{National Institute of Standards and Technology}

\newacronym{oopLabel}{OOP}{Object-Oriented Programming}

\newacronym{odbcLabel}{ODBC}{Open Database Connectivity}

\newacronym{oemLabel}{OEM}{Original Equipment Manufacturer}

\newacronym{pbLabel}{PB}{Petabyte}

\newacronym{pdfLabel}{PDF}{Portable Document Format}

\newacronym{phpLabel}{PHP}{PHP: Hypertext Preprocessor (a recursive backronym; used to stand for Personal Home Page)}

\newacronym{pngLabel}{PNG}{Portable Network Graphics}

\newacronym{pojoLabel}{POJO}{Plain Old Java Object}

\newacronym{posixLabel}{POSIX}{Portable Operating System Interface}

\newacronym{ramLabel}{RAM}{Random Access Memory}

\newacronym{replLabel}{REPL}{Read-Eval-Print Loop}

\newacronym{rgbLabel}{RGB}{Red-Green-Blue}

\newacronym{romLabel}{ROM}{Read-Only Memory}

\newacronym{rtfmLabel}{RTFM}{Read The ``Freaking'' Manual}

\newacronym{rtmLabel}{RTM}{Read The Manual}

\newacronym{sdkLabel}{SDK}{Software Development Kit}

\newacronym{seLabel}{SE}{Software Engineering}

\newacronym{seoLabel}{SEO}{Search Engine Optimization}

\newacronym{sqlLabel}{SQL}{Structured Query Language}

\newacronym{sslLabel}{SSL}{Secure Sockets Layer}

\newacronym{steamLabel}{STEAM}{Science, Technology, Engineering, Art, and Math}

\newacronym{stemLabel}{STEM}{Science, Technology, Engineering, and Math}

\newacronym{tbLabel}{TB}{Terabyte}

\newacronym{tcpLabel}{TCP}{Transmission Control Protocol}

\newacronym{tddLabel}{TDD}{Test-Driven Development}

\newacronym{tsvLabel}{TSV}{Tab Separated Values}

\newacronym{uiLabel}{UI}{User Interface}

\newacronym{urlLabel}{URL}{Uniform Resource Locator}

\newacronym{utfLabel}{UTF-8}{Universal (Character Set) Transformation Format--8-bit}

\newacronym{uxLabel}{UX}{User Experience}

\newacronym{vlsiLabel}{VLSI}{Very Large Scale Integration}

\newacronym{wwwLabel}{WWW}{World Wide Web}

\newacronym{w3cLabel}{W3C}{World Wide Web Consortium}

\newacronym{xmlLabel}{XML}{Extensible Markup Language}


%\glsaddall

%http://en.wikibooks.org/wiki/LaTeX/Indexing
\usepackage{makeidx}
\makeindex

\setlength{\parindent}{0pt}
\setlength{\parskip}{.35cm}

\usepackage{graphicx}
\usepackage{framed}
\usepackage{amssymb,amsmath,amsthm}

\usepackage{subfigure}

\theoremstyle{remark}
\newtheorem{ExInternal}{Exercise}[section]



\makeatletter
\let\@exercises\@empty%
\newcommand\exercise[2][]{%
    \g@addto@macro\@exercises{%
        \begin{ExInternal}[#1]%
            #2%
        \end{ExInternal}%
    }%
}

\newcommand\exerciseshere{%
    \subsection*{Exercises}
    \@exercises%
    \global\let\@exercises\@empty%
}
\makeatother

\usepackage{tikz}
\usetikzlibrary{backgrounds}
\usetikzlibrary{decorations.pathreplacing}
\usetikzlibrary{patterns}
\usetikzlibrary{fadings}
\usetikzlibrary{shapes,arrows,arrows.meta}
\usetikzlibrary{positioning}
% Define block styles
\tikzstyle{decision} = [diamond, draw, fill=yellow!20, 
    text width=6em, text badly centered, node distance=5cm, inner sep=0pt]
\tikzstyle{block} = [rectangle, draw, fill=blue!20, 
    text width=5em, text centered, node distance=5cm, minimum height=4em]
\tikzstyle{action} = [rectangle, draw, fill=green!20, 
    text width=5em, text centered, rounded corners, node distance=5cm, minimum height=4em]
\tikzstyle{line} = [draw, -latex']

\definecolor{mintedBackground}{rgb}{0.95,0.95,0.95}
\definecolor{mintedInlineBackground}{rgb}{.90,.90,1}

%\usepackage{newfloat}
\usepackage[newfloat=true,chapter]{minted}
\setminted{mathescape,
               linenos,
               autogobble,
               frame=none,
               framesep=2mm,
               framerule=0.4pt,
               %label=foo,
               xleftmargin=2em,
               xrightmargin=0em,
               startinline=true,  %PHP only, allow it to omit the PHP Tags *** with this option, variables using dollar sign in comments are treated as latex math
               numbersep=8pt, %gap between line numbers and start of line
               style=default, %syntax highlighting style, default is "default"
               			    %gallery: http://help.farbox.com/pygments.html
			    	    %list available: pygmentize -L styles
               bgcolor=mintedBackground} %prevents breaking across pages
               
\setmintedinline{bgcolor={mintedInlineBackground}}
\setminted[text]{bgcolor={mintedBackground},linenos=false,autogobble,xleftmargin=1em}
%\setminted[php]{bgcolor=mintedBackgroundPHP} %startinline=True}
\SetupFloatingEnvironment{listing}{name=Code Sample}
\SetupFloatingEnvironment{listing}{listname=List of Code Samples}

\BeforeBeginEnvironment{minted}{\vspace{.25cm}}
\AfterEndEnvironment{minted}{\vspace{.25cm}}

\usepackage[boxed,slide,linesnumbered,algochapter]{algorithm2e}
\SetKwProg{Fn}{Function}{}{end}
\SetKwComment{Comment}{//}{}
\DontPrintSemicolon
\SetKwSty{textsc} %
\IncMargin{5em}
%\SetAlFnt{\scriptsize} %
\SetKwInOut{Input}{Input} %
\SetKwInOut{Output}{Output} %
%\setalcapskip{1em} % changed to
\SetAlCapSkip{1em}
\setlength{\algomargin}{2em} %
%\Setvlineskip{0em} % changed to:
\SetVlineSkip{0em}
\SetKwRepeat{Do}{do}{while}

\usepackage[yyyymmdd,hhmmss]{datetime}

\theoremstyle{definition}
\newtheorem{theorem}{Theorem}
\newtheorem{exer}{Exercise}
\numberwithin{exer}{chapter}

\newtheorem{problem}{Problem}


%\usepackage{xspace}
\newcommand{\Neg}{\ensuremath{\neg}}
\renewcommand{\And}{\ensuremath{\mathbin{\textsc{And}}}}
%\renewcommand{\And}{
%\ifmmode\, \textsc{And} \,
%\else\textsc{And}\xspace\fi
%}
\newcommand{\Or}{\ensuremath{\mathbin{\textsc{Or}}}}
\newcommand{\True}{\emph{true}\xspace}
\newcommand{\False}{\emph{false}\xspace}
\newcommand{\Null}{\textsc{Null}}

\title{Computer Science I}
\subtitle{Programming in C, Java, and Beyond}
\author{Dr.\ Chris Bourke\\
        \href{mailto:cbourke@cse.unl.edu}{cbourke@cse.unl.edu} \\
        Department of Computer Science \& Engineering\\
        University of Nebraska--Lincoln\\
        Lincoln, NE 68588, USA
}

\date{\today\  \currenttime \\ Version 1.3.7}
