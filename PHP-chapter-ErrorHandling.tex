%!TEX root = ComputerScienceOne.tex

%%Chapter: Error Handling in PHP
\index{error handling!in PHP}
\index{PHP!error handling}

Modern versions of PHP support error handling through the 
use of exceptions.  PHP has several different predefined
types of exceptions and also allows you to define your own 
exception types by creating new classes that \emph{inherit} 
from the generic \mintinline{php}{Exception} 
class.  PHP uses the standard \mintinline{php}{try-catch-finally} 
control structure to handle exceptions and allows you to 
\mintinline{php}{throw} your own exceptions.


\section{Throwing Exceptions}

Though PHP defines several different types of exceptions, 
we'll only cover the generic \mintinline{php}{Exception} class.
We can \mintinline{php}{throw} an exception in PHP by using the
keyword \mintinline{php}{throw} and creating a new \mintinline{php}{Exception}
with an error message.  

\mintinline{php}{throw new Exception("Something went wrong");}

By using a generic \mintinline{php}{Exception}, we
can only attach a message to the exception (which can be
printed by code that catches the exception).  If we want
more fine-grained control over the type of exceptions, we
need to define our own exceptions.

\section{Catching Exceptions}

To catch an exception in PHP you can use the standard
\mintinline{php}{try-catch} control block.  Optionally (and as
of PHP version 5.5.0) you can use the \mintinline{php}{finally} 
block to clean up any resources or execute code regardless of
whether or not an exception was raised.  Consider
the simple task of reading input from a user 
and manually parsing its value into an integer.  If the
user enters a non-numeric value, parsing will fail
and we should instead throw an exception. Consider
the following function.

\begin{minted}{php}
function readNumber() {
  $input = readline("Please enter a number: ");
  if( is_numeric($input) ) {
    $value = floatval($input);
  } else {
    throw new Exception("Invalid input!");
  }
}
\end{minted}

Elsewhere in the code, we can surround a call to \mintinline{php}{readNumber()}
in a \mintinline{php}{try-catch} statement.

\begin{minted}{php}
try {
  readNumber();
} catch(Exception $e) {
  printf("Error: exception encountered: " . $e->getMessage());
  exit(1);
}
\end{minted}

In this example, we've simply displayed an error message to
the standard output and exited the program.  
We've made the design decision that this error should be fatal.
We could have chosen to handle this error differently in the
\mintinline{php}{catch} block.  The \mintinline{php}{$e->getMessage()}
prints the message that the exception was created with.  In
this case, \mintinline{php}{"Invalid input!"}.

\section{Creating Custom Exceptions}

As of PHP 5.3.0 it is possible to define custom exceptions 
by \emph{extending} the \mintinline{php}{Exception} class.
To do this, you need to declare a new class.  We cover classes
and objects in Chapter \ref{chapter:php:objects}.  
For now, we present a simple example.
Consider the example in the previous chapter of computing
the roots of a quadratic polynomial.  One possible error 
situation is when the roots are complex numbers.  We could
define a new PHP exception class as follows.

\begin{minted}{php}
/**
 * Defines a ComplexRoot exception class
 */
class ComplexRootException extends Exception
{
    public function __construct($message = null, 
    			    $code = 0, 
    			    Exception $previous = null) {
        // call the parent constructor
        parent::__construct($message, $code, $previous);
    }

    // custom string representation of object
    public function __toString() {
        return __CLASS__ . ": [{$this->code}]: {$this->message}\n";
    }
}
\end{minted}

Now in our code we can catch and even throw this new type
of exception.

\begin{minted}{php}
if( $b*$b - 4*$a*$c < 0) {
  throw new ComplexRootException("Cannot Handle complex roots");
}
\end{minted}

\begin{minted}{php}
try {
  $r1 = getRoot($a, $b, $c);
} catch(ComplexRootException $e) {
  //handle here
} catch(Exception $e) {
  //handle all other types of exceptions here
}
\end{minted}

In the code above we had \emph{two} \mintinline{php}{catch}
blocks.  Since we can have multiple types of exceptions, we
can catch each different type and handle them differently
if we choose. Each \mintinline{php}{catch} block catches a different 
type of exception.  The last \mintinline{php}{catch} block 
was written to catch a generic \mintinline{php}{Exception}.   
Much like an \mintinline{php}{if-else-if} statement, the
first type of exception that is caught is the block that will
be executed and they are all mutually exclusive.  Thus, a generic
``catch all'' block like this should always be the last 
\mintinline{php}{catch} block.  The most specific types
of exceptions should be caught first and the most general
types should be caught last.


