%!TEX root = ComputerScienceOne.tex

%%Chapter: Arrays in C
\index{arrays!in C}
\index{C!arrays}

C allows you to declare and use arrays.  Since C is statically
typed, arrays must also be typed when they are declared and
may only hold that particular type of element.  C supports the
use of both static arrays and dynamic arrays through standard
library calls.

\section{Basic Usage}

To declare a static array, you use syntax similar to declaring
a regular variable, but you use square brackets to designate
the variable as an array.  Within the square brackets you
indicate the size (number of elements) in the array.  For
example, 

\begin{minted}{c}
int arr[5];
double values[10];
\end{minted}

The two declarations above create arrays of size 5 and 10 
respectively.  The array types are also defined using the usual
keywords.  In this case, \mintinline{c}{arr} can only hold
\mintinline{c}{int} values and \mintinline{c}{values} can only
hold \mintinline{c}{double} values.  Arrays follow the same
naming rules and conventions as regular variables.  Many 
times, identifiers are made plural as an array naturally holds
more than one value.  

Another way to declare an array is to use the compound declaration
and assignment syntax whereby you can initialize an array to hold
a certain list of values.  

\begin{minted}{c}
int a[] = { 2, 3, 5, 7, 11 };
\end{minted}

Using this syntax we do not need to specify the size of the array 
as the compiler is smart enough to count the number of elements
we've provided.  The elements themselves are denoted inside curly
brackets and delimited with commas.

\subsubsection{Variable Length Arrays}

C99 introduced Variable Length Arrays (VLAs, which are also 
supported in GNU C89) which allow you to declare a static
array whose size is determined by a variable.  For example, 

\begin{minted}{c}
int n = 5;
int arr[n];
\end{minted}

or within a function, 

\begin{minted}{c}
void foo(int n) {
  int arr[n];
  ...
}
\end{minted}

In either case, care must be taken as static arrays are allocated
on the stack.  The stack is generally small and allocating even a
moderately large array on the stack may lead to a stack overflow.
In addition, VLAs are not supported in any C++ standard and
should be avoided if portable code is desired.


\subsubsection{Indexing}

Once an array has been created, its elements can be accessed
using indexing.  C uses the standard 0-indexing scheme so the
first element is at index 0, the second at index 1, etc.  Indexing
an element involves using the square bracket notation and
providing an index.  Once indexed, an array element can be
treated as a normal variable and can be used with other operators
such as the assignment operator or comparison operators.

\begin{minted}{c}
arr[0] = 42;
if(arr[4] < 0) {
  printf("negative!\n");
}
printf("arr[1] = %d\n", arr[1]);
\end{minted}

Recall that an index is actually an offset.  The compiler and
system know exactly how many bytes each \mintinline{c}{int}
element takes and so an index \mintinline{c}{i} calculates
exactly how many bytes from the first element the $i$-th
element is located at.  Consequently it is possible to index
elements that are beyond the range of the array.  For example, 
\mintinline{c}{arr[-1]} or \mintinline{c}{arr[5]} would attempt
to access an element immediately \emph{before} the first element
and immediately \emph{after} the last element.  Obviously, 
these elements are not part of the array.  

If these out-of-bound elements represent a memory space
that does not belong to our program, then it is likely that
a \gls{segmentation fault} will occur and terminate our program.
However, it is also likely that the memory 
space surrounding our array \emph{does} belongs to our program.  Still,
accessing those blocks of memory may not give us meaningful
values and modifying them could corrupt other variable values
or generally lead to \emph{undefined behavior}.  It is our
responsibility as programmers to write code that does not
go beyond the bounds of an array.

How can we keep track of the size
of the array so that we do not access elements after the last
element?  With C the responsibility again falls to us.  We must
always have secondary variables that store the size of an 
array.  If we pass an array to a function (see
below), we must also pass a ``size'' parameter so that the
function knows how big the array is.

\subsubsection{Iteration}

C provides no foreach loop to iterate over an array.  Instead,
the most natural way to iterate over the elements is a normal
for loop that increments an index variable.

\begin{minted}{c}
int i, n = 10;
int arr[n];
for(i=0; i<n; i++) {
  arr[i] = 5 * i;
}
\end{minted}

The for loop above initializes the variable \mintinline{c}{i} to zero,
corresponding to the first element in the array.  The continuation
condition is specifies that the loop continues while \mintinline{c}{i}
is strictly less than the size of the array.  This iteration for loop
is \glslink{idiom}{idiomatic} when dealing with arrays.  An alternative
was presented in Section \ref{section:c:loops:otherissues}.

\section{Dynamic Memory}

Recall that static arrays have many shortcomings (see Section 
\ref{section:arraysInDepth}).  In general they should be avoided
since stack space is limited and they cannot be returned from 
functions.  Fortunately, C provides several standard library 
functions that facilitate the creation and management of 
dynamically allocated arrays.

The primary function to allocate memory is the \textbf{m}emory
\textbf{alloc}ation function, \mintinline{c}{malloc()}: 

\mintinline{c}{void * malloc(size_t size);}

which takes a single parameter, the number of bytes that you 
wish to allocate.  The type, \mintinline{c}{size_t} is an alias for
an unsigned integer type, so you can think of it as an integer.
Thus, \mintinline{c}{malloc(200)} would allocate 200 bytes and
return a pointer to the memory space.  In some instances, the
allocation may fail.  For example, if the program or system has
run out of available memory or you simply request too much.
In the event of a failure, \mintinline{c}{malloc()} returns a 
\mintinline{c}{NULL} pointer.  The returned value can thus
be checked to see if the allocation was successful or not.

We don't have to manually calculate how many bytes we need
for an array of a particular size.  The macro  
\mintinline{c}{sizeof()} can be used to determine the number
of bytes any type of variable requires on a system.  For example, 
\mintinline{c}{sizeof(int)} gives the number of bytes an \mintinline{c}{int}
takes while \mintinline{c}{sizeof(double)} gives the number of
bytes for a \mintinline{c}{double}, etc.  Thus, if we want to
allocate an array of 100 integers, we could call \mintinline{c}{malloc()}
as 

\mintinline{c}{malloc(100 * sizeof(int));}

Using \mintinline{c}{sizeof()} is actually preferable as some systems
may use a different number of bytes for various types.

Finally, note the return type of \mintinline{c}{malloc()}: it is a 
\emph{void pointer}.  The \mintinline{c}{malloc()} function 
simply allocates chunks of memory.  It doesn't care that you
intend to use the memory to store integers or floating-point
numbers.  Thus, \mintinline{c}{malloc()} returns a ``generic''
pointer, simply an address in memory.  Once we have that
pointer we can treat it as an integer pointer, \mintinline{c}{int *}
or a floating-point pointer, \mintinline{c}{double *} depending
on what we want to store.  

One way of doing this is to explicitly \emph{cast} the void
pointer as the pointer that we want.  Some examples:

\begin{minted}{c}
int *arr = NULL;
double *values = NULL;

arr = (int *) malloc(sizeof(int) * 10);
values = (double *) malloc(sizeof(double) * 100);
\end{minted}

The pointer cast is just like when we casted \mintinline{c}{int}
types as \mintinline{c}{double} types so that we could perform
division without truncation.  In this case, we convert the
returned generic void pointer into a \mintinline{c}{int} pointer
and \mintinline{c}{double} pointer respectively.\footnote{Performing
an explicit pointer cast is actually not necessary in C as
the type system will do an implicit cast for us.  Whether
explicit or implicit types casts are ``better'' can evoke
debates akin to nerd holy wars.  Though there are advantages
and disadvantages to both \cite{stackOverflowCasts}, we perform explicit 
pointer casts in this book as clarity and intent is
more important than brevity.  Even more important is
that explicit pointer casts are \emph{necessary} in 
C++ so doing so in our C code makes our code
more portable.}

Once created, a dynamic array can be used just like a static
array.  You use the array's identifier as well as an index to
access or modify each element.  The same rules and pitfalls
apply, so care must be taken to not access elements outside
the bounds of the array.  Finally, when using \mintinline{c}{malloc()}
it is important to understand that the memory that is allocated
is uninitialized.  Just as with variables you cannot make any
assumptions as to the contents of the memory space that is
allocated.  It may contain garbage values, it may contain the
contents that occupied the memory last time it was used, 
etc.  A full example:

\begin{minted}{c}
int n = 100;
int *arr = NULL;
arr = (int *) malloc(sizeof(int) * n);
if(arr == NULL) {
  fprintf(stderr, "unable to allocate memory!\n");
  exit(1);
}
for(i=0; i<n; i++) {
  arr[i] = (i+1) * 10;
  printf("a[%d] = %d\n", i, arr[i]);
}
\end{minted}

There are other functions that can be used to allocate memory
in C.  For example, \mintinline{c}{calloc()} is similar to 
\mintinline{c}{malloc()} but initializes the contents
to zero (null bytes).  \mintinline{c}{realloc()} can be used to resize
an existing memory space (though it may fail if it cannot be 
expanded).  

\subsubsection{Deallocation}

Once dynamically allocated memory is no longer needed, we 
should release it so that it can be reused by the program or the
operating system.  The \mintinline{c}{free()} function in the
standard library does this for us.  All we need to do is provide
the pointer to \mintinline{c}{free()} and it deallocates the
memory block.

\begin{minted}{c}
free(arr);
\end{minted}

Once freed, accessing old memory pointed to by the \mintinline{c}{arr}
pointer is undefined behavior and may lead to unexpected or fatal
results.  

\section{Using Arrays with Functions}

Since arrays are represented using pointers, we can pass and
return arrays to and from functions simply by passing in a pointer.
When passing an array to a function, we also need to make sure
that we communicate to the function the size of the array as there
is no reliable way for the function to determine this.  By passing
both a pointer and an integer representing the size, we are implicitly
indicating that we are passing an array and not just a single value.

As an example, the following function takes an array of integers
and computes their sum.

\begin{minted}{c}
/**
 * This function computes the sum of elements in the
 * given array which contains n elements
 */
int computeSum(int *arr, int n) {
  int i;
  int sum = 0;
  for(i=0; i<size; i++) {
    sum += arr[i];
  }
  return sum;
}
\end{minted}

In this example we had no need to make changes to any of the elements
in the array.  However, the array was still passed by reference, meaning
we could have.  When passing arrays, we can use the keyword 
\mintinline{c}{const} (short for constant) to explicitly indicate that no changes will be made to the array elements.  For example, 

\mintinline{c}{int computeSum(const int *arr, int n)}

This is enforced by the compiler: if we do attempt
to make changes to a \mintinline{c}{const} array, it will be a compiler
error.

We can also create an array in a function and return it as 
a value.  As previously discussed, we will need to do so by 
creating a dynamically allocated array.  For example, the 
following function creates a \gls{deep copy} of the given 
integer array.  That is, a completely new array that is a distinct 
copy of the old array.  In contrast, a \gls{shallow copy}
would be if we simply made one reference point to another 
reference.


\begin{minted}{c}
/**
 * This function creates a new copy of the given integer
 * array which contains n elements and returns a pointer
 * to the new copy.
 */
int * makeCopy(const int *a, int n) {
  int *copy = (int *)malloc(sizeof(int) * n);
  int i;
  for(i=0; i<n; i++) {
    copy[i] = a[i];
  }
  return copy;
}
\end{minted}

The function returns an integer pointer.  Here we have a similar problem
with respect to the size of the array.  We only have one return value, which
must be the pointer.  In this particular example the calling function knows
how big the array is because it specified the size by passing in \mintinline{c}{n}. In general we could have communicated the size of the returned array 
through another pass by reference integer variable.

\section{Multidimensional Arrays}

C supports multidimensional arrays both static and dynamically
allocated; though we'll only focus on dynamically allocated 2-dimensional
arrays.  Conceptually, a 2-dimensional array of, say, integers can
be modeled as an array of pointers that point to an array of integers.
That is, a pointer to pointers, for example, \mintinline{c}{int **}.  
The initial pointer points to an array of integer pointers, \mintinline{c}{int *}
and each integer pointer points to an array of \mintinline{c}{int} variables.

To understand this better, let's look at the details of allocation.  Consider
the following code.

\begin{minted}{c}
int i;
int **myMatrix = NULL; 
myMatrix = (int **)malloc(n * sizeof(int*)); 
for(i=0; i<n; i++) {
  myMatrix[i] = (int *)malloc(n * sizeof(int)); 
}
\end{minted}

Line 3 invokes \mintinline{c}{malloc()} to create an array of $n$ integer
pointers, that is \mintinline{c}{int *} types.  We then go into a loop to
iterate over each of these pointers and invoke \mintinline{c}{malloc()} 
again to setup each array.  This process is visualized in Figure 
\ref{figure:mallocIllustration2D}.  Note the syntax: when we invoke \mintinline{c}{malloc()} to 
create an array of pointers, we use \mintinline{c}{sizeof(int*)} to
determine how many bytes each integer \emph{pointer} takes.  We
also do an explicit type cast of \mintinline{c}{(int **)} to match our
pointer-to-pointer(s) variable \mintinline{c}{myMatrix}.  

%\documentclass[12pt]{scrbook}
%
%\usepackage{tikz}
%\usepackage{minted}
%\usetikzlibrary{decorations.pathreplacing,arrows}
%
%\usepackage{fullpage}
%\usepackage{subfigure}
%\begin{document}
%
%
%Lorem Ipsum is simply dummy text of the printing and typesetting industry. Lorem Ipsum has been the industry's standard dummy text ever since the 1500s, when an unknown printer took a galley of type and scrambled it to make a type specimen book. It has survived not only five centuries, but also the leap into electronic typesetting, remaining essentially unchanged. It was popularised in the 1960s with the release of Letraset sheets containing Lorem Ipsum passages, and more recently with desktop publishing software like Aldus PageMaker including versions of Lorem Ipsum.

\begin{figure}
\centering

\subfigure[Initialization of the pointer-to-pointers.  The first invocation of \mintinline{c}{malloc()} sets up
an array of integer pointers, \mintinline{c}{int *} that are uninitialized (where they point to is undefined).]{
\begin{tikzpicture}[scale=0.65,transform shape]

% Define block styles
\tikzstyle{box} = [rectangle,
                   draw,
                   fill=white,
                   text width=3.1cm,
                   text centered,
                   inner sep=5pt,
                   node distance=.75cm]

\tikzstyle{line} = [draw, -latex']; %

\draw[white] (-1, 0) rectangle (19, -1);
    \node [box] (init) at (0,0) {\mintinline{c}{**myMatrix}};

    \node [box, below of=init,node distance=2cm] (p0) {\mintinline{c}{*myMatrix[0]}};
    \node[node distance = 3cm,right of=p0] (d0) {?};
    \draw[line] (p0) -- (d0);
    \node [box, below of=p0] (p1) {\mintinline{c}{*myMatrix[1]}};
    \node[node distance = 3cm,right of=p1] (d1) {?};
    \draw[line] (p1) -- (d1);
    \node [box, below of=p1] (p2) {\mintinline{c}{*myMatrix[2]}};
    \node[node distance = 3cm,right of=p2] (d2) {?};
    \draw[line] (p2) -- (d2);
    \node [below of=p2,node distance=1cm] (pd) {$\vdots$};
    \node [box, below of=pd,node distance=1cm] (pn) {\mintinline{c}{*myMatrix[n-1]}};
    \node[node distance = 3cm,right of=pn] (dn) {?};
    \draw[line] (pn) -- (dn);
    \path [line] (init) -- (p0);

%\onslide<3->{
%    \node [box, right of=p0,node distance=5cm] (m00) {\mintinline{c}{myMatrix[0][0]}};
%    \node [box, right of=m00,node distance=3.4cm] (m01) {\mintinline{c}{myMatrix[0][1]}};
%    \node [box, right of=m01,node distance=3.4cm] (m02) {\mintinline{c}{myMatrix[0][2]}};
%    \node [right of=m02,node distance=3cm] (m0d) {$\cdots$};
%    \node [box, text width=3.2cm, right of=m0d,node distance=3cm] (m0n) {\mintinline{c}{myMatrix[0][n-1]}};
%    \path [line] (p0) -- (m00);
%}
%\onslide<4->{
%    \node [box, right of=p1,node distance=5cm] (m10) {\mintinline{c}{myMatrix[1][0]}};
%    \node [box, right of=m10,node distance=3.4cm] (m11) {\mintinline{c}{myMatrix[1][1]}};
%    \node [box, right of=m11,node distance=3.4cm] (m12) {\mintinline{c}{myMatrix[1][2]}};
%    \node [right of=m12,node distance=3cm] (m1d) {$\cdots$};
%    \node [box, text width=3.2cm, right of=m1d,node distance=3cm] (m1n) {\mintinline{c}{myMatrix[1][n-1]}};
%    \path [line] (p1) -- (m10);
%}
%\onslide<5->{
%    \node [box, right of=p2,node distance=5cm]    (m20) {\mintinline{c}{myMatrix[2][0]}};
%    \node [box, right of=m20,node distance=3.4cm] (m21) {\mintinline{c}{myMatrix[2][1]}};
%    \node [box, right of=m21,node distance=3.4cm] (m22) {\mintinline{c}{myMatrix[2][2]}};
%    \node [right of=m22,node distance=3cm]        (m2d) {$\cdots$};
%    \node [box, text width=3.2cm, right of=m2d,node distance=3cm]   (m2n) {\mintinline{c}{myMatrix[2][n-1]}};
%    \path [line] (p2) -- (m20);
%}
%\onslide<6->{
%    \node [box, right of=pn,node distance=5cm]    (mn0) {\mintinline{c}{myMatrix[n-1][0]}};
%    \node [box, right of=mn0,node distance=3.4cm] (mn1) {\mintinline{c}{myMatrix[n-1][1]}};
%    \node [box, right of=mn1,node distance=3.4cm] (mn2) {\mintinline{c}{myMatrix[n-1][2]}};
%    \node [right of=mn2,node distance=3cm]        (mnd) {$\cdots$};
%    \node [box, text width=3.6cm, right of=mnd,node distance=3cm]   (mnn) {\mintinline{c}{myMatrix[n-1][n-1]}};
%    \path [line] (pn) -- (mn0);
%}

\end{tikzpicture}
}%end (a)

\subfigure[Initialization of the first pointer.  On the first iteration of the \mintinline{c}{for} loop when
\mintinline{c}{i = 0}, the first ``row'' is initialized when \mintinline{c}{malloc()} is invoked.]{
\begin{tikzpicture}[scale=0.65,transform shape]

% Define block styles
\tikzstyle{box} = [rectangle,
                   draw,
                   fill=white,
                   text width=3.1cm,
                   text centered,
                   inner sep=5pt,
                   node distance=.75cm]

\tikzstyle{line} = [draw, -latex']; %

    \node [box] (init) at (0,0) {\mintinline{c}{**myMatrix}};

    \node [box, below of=init,node distance=2cm] (p0) {\mintinline{c}{*myMatrix[0]}};

    \node [box, right of=p0,node distance=4.5cm] (m00) {\mintinline{c}{myMatrix[0][0]}};
    \node [box, right of=m00,node distance=3.4cm] (m01) {\mintinline{c}{myMatrix[0][1]}};
    \node [box, right of=m01,node distance=3.4cm] (m02) {\mintinline{c}{myMatrix[0][2]}};
    \node [right of=m02,node distance=3cm] (m0d) {$\cdots$};
    \node [box, text width=3.5cm, right of=m0d,node distance=3cm] (m0n) {\mintinline{c}{myMatrix[0][n-1]}};
    \path [line] (p0) -- (m00);

    \node [box, below of=p0] (p1) {\mintinline{c}{*myMatrix[1]}};
    \node[node distance = 3cm,right of=p1] (d1) {?};
    \draw[line] (p1) -- (d1);
    \node [box, below of=p1] (p2) {\mintinline{c}{*myMatrix[2]}};
    \node[node distance = 3cm,right of=p2] (d2) {?};
    \draw[line] (p2) -- (d2);
    \node [below of=p2,node distance=1cm] (pd) {$\vdots$};
    \node [box, below of=pd,node distance=1cm] (pn) {\mintinline{c}{*myMatrix[n-1]}};
    \node[node distance = 3cm,right of=pn] (dn) {?};
    \draw[line] (pn) -- (dn);
    \path [line] (init) -- (p0);

%}
%\onslide<4->{
%    \node [box, right of=p1,node distance=5cm] (m10) {\mintinline{c}{myMatrix[1][0]}};
%    \node [box, right of=m10,node distance=3.4cm] (m11) {\mintinline{c}{myMatrix[1][1]}};
%    \node [box, right of=m11,node distance=3.4cm] (m12) {\mintinline{c}{myMatrix[1][2]}};
%    \node [right of=m12,node distance=3cm] (m1d) {$\cdots$};
%    \node [box, text width=3.2cm, right of=m1d,node distance=3cm] (m1n) {\mintinline{c}{myMatrix[1][n-1]}};
%    \path [line] (p1) -- (m10);
%}
%\onslide<5->{
%    \node [box, right of=p2,node distance=5cm]    (m20) {\mintinline{c}{myMatrix[2][0]}};
%    \node [box, right of=m20,node distance=3.4cm] (m21) {\mintinline{c}{myMatrix[2][1]}};
%    \node [box, right of=m21,node distance=3.4cm] (m22) {\mintinline{c}{myMatrix[2][2]}};
%    \node [right of=m22,node distance=3cm]        (m2d) {$\cdots$};
%    \node [box, text width=3.2cm, right of=m2d,node distance=3cm]   (m2n) {\mintinline{c}{myMatrix[2][n-1]}};
%    \path [line] (p2) -- (m20);
%}
%\onslide<6->{
%    \node [box, right of=pn,node distance=5cm]    (mn0) {\mintinline{c}{myMatrix[n-1][0]}};
%    \node [box, right of=mn0,node distance=3.4cm] (mn1) {\mintinline{c}{myMatrix[n-1][1]}};
%    \node [box, right of=mn1,node distance=3.4cm] (mn2) {\mintinline{c}{myMatrix[n-1][2]}};
%    \node [right of=mn2,node distance=3cm]        (mnd) {$\cdots$};
%    \node [box, text width=3.6cm, right of=mnd,node distance=3cm]   (mnn) {\mintinline{c}{myMatrix[n-1][n-1]}};
%    \path [line] (pn) -- (mn0);
%}

\end{tikzpicture}
}%end (b)

\subfigure[Initialization of the second pointer.  On the second iteration of the \mintinline{c}{for} loop when
\mintinline{c}{i = 1}, the second ``row'' is initialized.]{
\begin{tikzpicture}[scale=0.65,transform shape]

% Define block styles
\tikzstyle{box} = [rectangle,
                   draw,
                   fill=white,
                   text width=3.1cm,
                   text centered,
                   inner sep=5pt,
                   node distance=.75cm]

\tikzstyle{line} = [draw, -latex']; %

    \node [box] (init) at (0,0) {\mintinline{c}{**myMatrix}};

    \node [box, below of=init,node distance=2cm] (p0) {\mintinline{c}{*myMatrix[0]}};

    \node [box, right of=p0,node distance=4.5cm] (m00) {\mintinline{c}{myMatrix[0][0]}};
    \node [box, right of=m00,node distance=3.4cm] (m01) {\mintinline{c}{myMatrix[0][1]}};
    \node [box, right of=m01,node distance=3.4cm] (m02) {\mintinline{c}{myMatrix[0][2]}};
    \node [right of=m02,node distance=3cm] (m0d) {$\cdots$};
    \node [box, text width=3.5cm, right of=m0d,node distance=3cm] (m0n) {\mintinline{c}{myMatrix[0][n-1]}};
    \path [line] (p0) -- (m00);

    \node [box, below of=p0] (p1) {\mintinline{c}{*myMatrix[1]}};

    \node [box, right of=p1,node distance=4.5cm] (m10) {\mintinline{c}{myMatrix[1][0]}};
    \node [box, right of=m10,node distance=3.4cm] (m11) {\mintinline{c}{myMatrix[1][1]}};
    \node [box, right of=m11,node distance=3.4cm] (m12) {\mintinline{c}{myMatrix[1][2]}};
    \node [right of=m12,node distance=3cm] (m1d) {$\cdots$};
    \node [box, text width=3.5cm, right of=m1d,node distance=3cm] (m1n) {\mintinline{c}{myMatrix[1][n-1]}};
    \path [line] (p1) -- (m10);

    \node [box, below of=p1] (p2) {\mintinline{c}{*myMatrix[2]}};
    \node[node distance = 3cm,right of=p2] (d2) {?};
    \draw[line] (p2) -- (d2);
    \node [below of=p2,node distance=1cm] (pd) {$\vdots$};
    \node [box, below of=pd,node distance=1cm] (pn) {\mintinline{c}{*myMatrix[n-1]}};
    \node[node distance = 3cm,right of=pn] (dn) {?};
    \draw[line] (pn) -- (dn);
    \path [line] (init) -- (p0);

%}
%\onslide<4->{
%    \node [box, right of=p1,node distance=5cm] (m10) {\mintinline{c}{myMatrix[1][0]}};
%    \node [box, right of=m10,node distance=3.4cm] (m11) {\mintinline{c}{myMatrix[1][1]}};
%    \node [box, right of=m11,node distance=3.4cm] (m12) {\mintinline{c}{myMatrix[1][2]}};
%    \node [right of=m12,node distance=3cm] (m1d) {$\cdots$};
%    \node [box, text width=3.2cm, right of=m1d,node distance=3cm] (m1n) {\mintinline{c}{myMatrix[1][n-1]}};
%    \path [line] (p1) -- (m10);
%}
%\onslide<5->{
%    \node [box, right of=p2,node distance=5cm]    (m20) {\mintinline{c}{myMatrix[2][0]}};
%    \node [box, right of=m20,node distance=3.4cm] (m21) {\mintinline{c}{myMatrix[2][1]}};
%    \node [box, right of=m21,node distance=3.4cm] (m22) {\mintinline{c}{myMatrix[2][2]}};
%    \node [right of=m22,node distance=3cm]        (m2d) {$\cdots$};
%    \node [box, text width=3.2cm, right of=m2d,node distance=3cm]   (m2n) {\mintinline{c}{myMatrix[2][n-1]}};
%    \path [line] (p2) -- (m20);
%}
%\onslide<6->{
%    \node [box, right of=pn,node distance=5cm]    (mn0) {\mintinline{c}{myMatrix[n-1][0]}};
%    \node [box, right of=mn0,node distance=3.4cm] (mn1) {\mintinline{c}{myMatrix[n-1][1]}};
%    \node [box, right of=mn1,node distance=3.4cm] (mn2) {\mintinline{c}{myMatrix[n-1][2]}};
%    \node [right of=mn2,node distance=3cm]        (mnd) {$\cdots$};
%    \node [box, text width=3.6cm, right of=mnd,node distance=3cm]   (mnn) {\mintinline{c}{myMatrix[n-1][n-1]}};
%    \path [line] (pn) -- (mn0);
%}

\end{tikzpicture}
}%end (c)

\subfigure[After the termination of the \mintinline{c}{for} loop, each row has been initialized and
the pointer-to-pointers can be treated like a 2-dimensional array.]{
\begin{tikzpicture}[scale=0.65,transform shape]

% Define block styles
\tikzstyle{box} = [rectangle,
                   draw,
                   fill=white,
                   text width=3.1cm,
                   text centered,
                   inner sep=5pt,
                   node distance=.75cm]

\tikzstyle{line} = [draw, -latex']; %

    \node [box] (init) at (0,0) {\mintinline{c}{**myMatrix}};

    \node [box, below of=init,node distance=2cm] (p0) {\mintinline{c}{*myMatrix[0]}};

    \node [box, right of=p0,node distance=4.5cm] (m00) {\mintinline{c}{myMatrix[0][0]}};
    \node [box, right of=m00,node distance=3.4cm] (m01) {\mintinline{c}{myMatrix[0][1]}};
    \node [box, right of=m01,node distance=3.4cm] (m02) {\mintinline{c}{myMatrix[0][2]}};
    \node [right of=m02,node distance=3cm] (m0d) {$\cdots$};
    \node [box, text width=3.5cm, right of=m0d,node distance=3cm] (m0n) {\mintinline{c}{myMatrix[0][n-1]}};
    \path [line] (p0) -- (m00);

    \node [box, below of=p0] (p1) {\mintinline{c}{*myMatrix[1]}};

    \node [box, right of=p1,node distance=4.5cm] (m10) {\mintinline{c}{myMatrix[1][0]}};
    \node [box, right of=m10,node distance=3.4cm] (m11) {\mintinline{c}{myMatrix[1][1]}};
    \node [box, right of=m11,node distance=3.4cm] (m12) {\mintinline{c}{myMatrix[1][2]}};
    \node [right of=m12,node distance=3cm] (m1d) {$\cdots$};
    \node [box, text width=3.5cm, right of=m1d,node distance=3cm] (m1n) {\mintinline{c}{myMatrix[1][n-1]}};
    \path [line] (p1) -- (m10);

    \node [box, below of=p1] (p2) {\mintinline{c}{*myMatrix[2]}};

    \node [box, right of=p2,node distance=4.5cm]    (m20) {\mintinline{c}{myMatrix[2][0]}};
    \node [box, right of=m20,node distance=3.4cm] (m21) {\mintinline{c}{myMatrix[2][1]}};
    \node [box, right of=m21,node distance=3.4cm] (m22) {\mintinline{c}{myMatrix[2][2]}};
    \node [right of=m22,node distance=3cm]        (m2d) {$\cdots$};
    \node [box, text width=3.5cm, right of=m2d,node distance=3cm]   (m2n) {\mintinline{c}{myMatrix[2][n-1]}};
    \path [line] (p2) -- (m20);

    \node [below of=p2,node distance=1cm] (pd) {$\vdots$};
    \node [box, below of=pd,node distance=1cm] (pn) {\mintinline{c}{*myMatrix[n-1]}};

    \node [box, text width=3.6cm,right of=pn,node distance=4.8cm]  (mn0) {\mintinline{c}{myMatrix[n-1][0]}};
    \node [box, text width=3.6cm,right of=mn0,node distance=3.8cm] (mn1) {\mintinline{c}{myMatrix[n-1][1]}};
    \node [box, text width=3.6cm,right of=mn1,node distance=3.8cm] (mn2) {\mintinline{c}{myMatrix[n-1][2]}};
    \node [right of=mn2,node distance=2.5cm]        (mnd) {$\cdots$};
    \node [box, text width=3.75cm, right of=mnd,node distance=2.5cm]   (mnn) {\mintinline{c}{myMatrix[n-1][n-1]}};
    \path [line] (pn) -- (mn0);

    \path [line] (init) -- (p0);


\end{tikzpicture}
}%end (d)

\caption[Dynamically Allocating Multidimensional Arrays]{The process of dynamically allocating
a 2-dimensional array of integers using \mintinline{c}{malloc()} in a for-loop.}
\label{figure:mallocIllustration2D}
\end{figure}

%\end{document}



Once the allocation has completed, we can treat \mintinline{c}{myMatrix}
as a normal 2-dimensional array and index each element with two
index variables.

\begin{minted}{c}
int i, j;
for(i=0; i<n; i++) {
  for(j=0; j<n; j++) {
    myMatrix[i][j] = 0;
  }
}
\end{minted}

To deallocate and free multidimensional arrays, we need to work backwards.
If we immediately freed the pointer-to-pointers, \mintinline{c}{free(myMatrix);}, 
we would lose all references to the individual array ``rows'' and would not
be able to free them, resulting in a \gls{memory leak}.  We need to free up 
each row first, \emph{then} we can free the pointer-to-pointers.

\begin{minted}{c}
for(i=0; i<n; i++) {
  free(myMatrix[i]);
}
free(myMatrix);
\end{minted}

\subsection{Contiguous 2-D Arrays}

The example depicted in Figure \ref{figure:mallocIllustration2D} 
constructs a two dimensional array.  However, each ``row'' of the
array was created using an independent call to \mintinline{c}{malloc()}
which may result in non-contiguous memory blocks (each row may not
be located in the memory address immediately following the previous
row).  In general, this is not a problem for most situations
(we can expect most implementations of \mintinline{c}{malloc()}
to be efficient).  However, it might be better in some scenarios if 
the two dimensional array were all one big block of contiguous memory.  

We can achieve this by a single call to \mintinline{c}{malloc()} and
some clever pointer manipulation.  We do this very similar to the
previous example.  Our first call to \mintinline{c}{malloc()} to set
up the pointer-to-pointers is the same.  However, instead of calling
\mintinline{c}{malloc()} over and over in a for loop, we make one
single call asking for a number of bytes to accommodate the entire
$n \times m$ array.  We then store the result at the ``beginning''
of the array (at index zero).

\begin{minted}{c}
int n = 5, m = 3, i, j;

int **arr  = (int **)malloc(sizeof(int *) * n);
arr[0] = (int *)malloc(sizeof(int) * (n * m));
\end{minted}

We're not done yet, however.  We still need to initialize all of the
other pointers.  To do so, we dereference the array (once) to get
the address of the start of the memory block and then compute
an \emph{offset} from this beginning on where the next ``row'' should
be.  Since each row has \mintinline{c}{m} elements, this arithmetic
is simple.

\begin{minted}{c}
for(i=1; i<n; i++) {
  arr[i] = (*arr + (m * i));
}
\end{minted}

Now we can treat the array like we would any other two dimensional array
by specifying two indices.

\begin{minted}{c}
for(i=0; i<n; i++) {
  for(j=0; j<m; j++) {
    arr[i][j] = 10 * i + j;
  }
}
\end{minted}

Which would result in an array that, conceptually, looks something like
the following.

\begin{minted}{text}
[   0   1   2  ]
[  10  11  12  ]
[  20  21  22  ]
[  30  31  32  ]
[  40  41  42  ]
\end{minted}

Which are all stored in one large chunk of memory.  To see this, 
we can print out the memory address of each ``row'' during a
particular run of the program:

\begin{minted}{text}
arr[0] = 0x7fe51b403270
arr[1] = 0x7fe51b40327c
arr[2] = 0x7fe51b403288
arr[3] = 0x7fe51b403294
arr[4] = 0x7fe51b4032a0
\end{minted}

Note that each hexadecimal value differs by \mintinline{text}{0x0c}
(that is, 12 bytes): \mintinline{text}{0x...70 + 0x0c = 0x...7c} and
\mintinline{text}{0x...7c + 0x0c = 0x...88} etc.  This is exactly how
many bytes the 3 integers (4 bytes each) takes in memory in each ``row.''

\section{Dynamic Data Structures}

The C standard library does not provide a variety of advanced, dynamic
data structures such as linked lists or sets.  As of C89 and POSIX.1 the
C standard library does provide hash tables and binary search trees, but
does not require implementations of other dynamic data structures.
Instead, you must either implement your own or utilize a third-party 
library such as glibc, the \gls{gnuLabel} C library 
(\url{http://www.gnu.org/software/libc/}).





