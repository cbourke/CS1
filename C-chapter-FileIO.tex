%!TEX root = ComputerScienceOne.tex

%%Chapter: FileIO in C

C provides several functions to manipulate and
process files.  Like other I/O functions, these
are all defined in the standard input/output
library, \mintinline{text}{stdio.h}.  Writing
binary or plaintext data is determined by which
functions you use.

In general whether or not a file input/output
stream is buffered or unbuffered is determined by
the system configuration.  There are some ways
in which this can be changed, but we will not 
cover them in detail.

\section{Opening Files}

Files are represented in C by a \mintinline{c}{FILE}
pointer type defined in the standard input/output
library.  As a pointer, it essentially points to the
file stored in memory.

To open a file, you use the \mintinline{c}{fopen()} 
function (short for \textbf{f}ile \textbf{open}) which
takes two arguments and returns a \mintinline{c}{FILE}
pointer:

\mintinline{c}{FILE *fopen(const char *path, const char *mode);}

The first argument is the file path/name that you want
to open for processing.  The second argument is a
string representing the ``mode'' that you want to open the
file in.  There are several supported modes, but the two 
we will be interested in are reading, in which case you
pass it \mintinline{c}{"r"} and writing in which case you
pass it \mintinline{c}{"w"}.  The path can be an absolute 
path, relative path, or may be omitted if the file is in the
current working directory.

\begin{minted}{c}
//open a file for reading (input):
FILE *input = fopen("/user/apps/data.txt", "r");
//open a file for reading (output):
FILE *output = fopen("./results.txt", "w");

if(input == NULL) {
  fprintf(stderr, "Unable to open input file");
  exit(1);
}

if(output == NULL) {
  fprintf(stderr, "Unable to open output file");
  exit(1);
}
\end{minted}

The two checks above check that the file opened successfully.
If the file opening failed, \mintinline{c}{fopen()} returns
\mintinline{c}{NULL}.  Opening a file can fail for a number
of reasons.  On \gls{posixLabel} systems for example, additional
information can be obtained by accessing the standard error
number, \mintinline{c}{errno} (see Section \ref{section:errno}).  
Some errors that can result:
\begin{itemize}
  \item \mintinline{c}{ENOENT} -- No such file or directory
  \item \mintinline{c}{EACCES} -- Permission denied
  \item \mintinline{c}{ENOMEM} -- Insufficient storage space is available
\end{itemize}
among many other possibilities.  These error codes can be
used to implement more specific error handling code if desired.

\section{Reading \& Writing}

When a file is opened, the file pointer returned by
\mintinline{c}{fopen()} initially points to the beginning
of the file.  As you read or write from it, the pointer
advances through the file content.

\subsection{Plaintext Files}

To process plaintext we use two functions, \mintinline{c}{fprintf()}
for output and \mintinline{c}{fscanf()} for input.  These
two functions should look familiar.  They work almost 
exactly the same as \mintinline{c}{printf()} and
\mintinline{c}{scanf()} that work with the standard output
and standard input respectively.  In fact, the standard in/out
are both \emph{files} so it makes sense that they can be
read from/written to.  The \mintinline{c}{f} prefixed versions
are simply generalizations that can be used for any file.
The only difference is that the first argument for these
two functions is the file you want to output to or read from.

\begin{minted}{c}
int x = 10;
double y = 3.14;

//write to a plaintext file
fprintf(output, "Hello World!\n");
fprintf(output, "x = %d, y = %f\n", x, y);

//read from a plaintext file
fscanf(input, "%d", &x);
fscanf(input, "%lf", &y);

//these are equivalent to printf, scanf:
fprintf(stdout, "Please enter an integer:");
fscanf(stdin, "%d", &x);
\end{minted}

Using \mintinline{c}{fscanf()} for arbitrary string 
input is potentially dangerous as there is limited
bounds checking.  We must store the input value from
the file into a string (character array), but if the
file or line contains more characters than the 
array can accommodate we may have a buffer overflow.

A better way to read input is to use \mintinline{c}{fgets()}
which allows us to limit the number of bytes that are
read.

\mintinline{c}{char *fgets(char *s, int size, FILE *stream);}

The first argument is the string that the input
data will be read into.  The second parameter is how
we limit the number of characters that will be read.
It actually reads in one fewer character, \mintinline{c}{size-1}
to account for the null-terminating character which
\mintinline{c}{fgets()} automatically inserts for us.
The last argument is the file pointer that we wish to
read from.

The behavior of \mintinline{c}{fgets()} is that it reads 
\emph{up to} \mintinline{c}{size-1} characters from the
input file and places the results into \mintinline{c}{s}.
If \mintinline{c}{fgets()} encounters either an \mintinline{c}{EOF}
symbol or an endline character, \mintinline{c}{'\n'} it 
stops reading.  In the case of an endline character, it is
included in the result and may need to be \glslink{chomp}{chomped} out
(that is, removed).

The \mintinline{c}{fgets()} function can be used to process
a file line by line until the end of the file is reached.
Each line can be processed (perhaps tokenized) individually
to extract particular pieces of data.  This is typically how
a \gls{csvLabel} or similar file may be processed.  To determine
if the end of the file has been reached, you can use the 
return value of the function: it returns \mintinline{c}{NULL}
when no more characters have been read.

\begin{minted}{c}
FILE *input = fopen("data.txt", "r");
//this assumes that no line is more than 999 characters
char line[1000];
//read the first line
char *s = fgets(line, 1000, input);
while(s != NULL) {

  //chomp the endline character from line:
  line[strlen(line)-1] = '\0';

  //process the current buffer
  //for demonstration, we simply print it:
  printf("line = %s\n", line);

  //read the next line 
  s = fgets(line, 1000, input);
}
\end{minted}

A similar function, \mintinline{c}{int fgetc(FILE *stream);}
allows us to get a single character from the input file 
(returned as an \mintinline{c}{int}) if we prefer to read
character by character.

\subsection{Binary Files}

To read and write binary data to files we use two different
functions:

\mintinline{c}{?size_t fread(void *ptr, size_t size, size_t nmemb, FILE *stream);} 

for reading from a file and 

\mintinline{c}{size_t fwrite(const void *ptr, size_t size, size_t nmemb, FILE *stream);}

to write to a file.  Both functions have the exact same argument:
\begin{enumerate}
  \item \mintinline{c}{*ptr} is a pointer to the item that 
  	is being read/written (possibly an array of elements)
  \item \mintinline{c}{size} is the number of bytes that each 
	element stored at \mintinline{c}{ptr} take, usually resolved
	by using \mintinline{c}{sizeof()}
  \item \mintinline{c}{nmemb} is the number of elements to be 
  	written/read
  \item \mintinline{c}{stream} is the file stream to be read 
	from/written to
\end{enumerate}

These functions allow you to read/write an arbitrary amount of
data.  When reading, it is your responsibility, as usual, to 
ensure that the \mintinline{c}{ptr} is large enough to accommodate
the data you are reading into it.

\begin{minted}{c}
int x = 10;

FILE *binaryOutputFile = fopen("demo.bin", "w");

//write a single int to a file:
fwrite(&x, sizeof(int), 1, binaryOutputFile);

FILE *binaryInputFile = fopen("input.bin", "r");

//read a single int from the file:
fread(&x, sizeof(int), 1, binaryInputFile);

//read an entire array from the file:
int *a = (int *) malloc(10 * sizeof(int));
fread(a, sizeof(int), 10, binaryInputFile);
\end{minted}

\section{Closing Files}

Once you are done processing a file, you should close it
using the \mintinline{c}{fclose()} function:

\mintinline{c}{int fclose(FILE *fp);}

which takes a single argument, the \mintinline{c}{FILE}
pointer of the file you wish to close.  Closing an invalid
file may result in an error or segmentation fault.  In any
case, after a file has been closed, it cannot be read from
or written to.  Doing so on a closed file is undefined
behavior. Failure to close a file \emph{may} result in a 
corrupted file. 






