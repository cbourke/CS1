%!TEX root = ComputerScienceOne.tex

%%Chapter: Arrays in Java

Java allows you to declare and use arrays.  Since Java is statically
typed, arrays must also be typed when they are declared and
may only hold that particular type of element.  Since Java has
automated garbage collection, memory management is greatly
simplified.  Finally, in Java, only locally scoped primitives and 
references are allocated on the program call stack.  As such,
there are no static arrays; all arrays are allocated in the heap
space.

\section{Basic Usage}

To declare an array reference, you use syntax similar to declaring
a regular variable, but you use square brackets to designate
the variable as an array.  Arrays for any type of variable can
be created including primitives and objects.

\begin{minted}{java}
int arr[] = null;
double values[] = null;
String names[] = null;
\end{minted}

This example\footnote{You may see code examples that use the
alternative notation \mintinline{java}{int[] arr = null;}.  Both are
acceptable.  Some would argue that this way is more correct because
it keeps the brackets with the type, avoiding the type-variable name-type
separation in our example.  Those preferring the \mintinline{java}{int arr[] = null;}
notation would argue that it is ``more natural'' because that is how we
ultimately index the array.  Six of One, Half-Dozen of the Other.} 
only declares 3 references that can refer to an array, 
it doesn't actually create them as the references are all 
\mintinline{java}{null}.  To create arrays of a particular size, we
need to initialize them using the keyword \mintinline{java}{new}.  

\begin{minted}{java}
arr = new int[10];
values = new double[20];
names = new String[5];
\end{minted}

Each of these initializations creates a new array (allocated on the 
heap) of the specified size (10, 20, and 5 respectively).  These 
arrays can only hold values of the specified type (\mintinline{java}{int},
\mintinline{java}{double}, and \mintinline{java}{String} respectively).

As with regular variables, the default value for each element in
these new arrays will be zero for the numeric types and \mintinline{java}{null}
for object types.  Alternatively, you could use a compound declaration/assignment
syntax to initialize specific values:

\begin{minted}{java}
int arr[] = { 2, 3, 5, 7, 11 };
\end{minted}

Using this syntax we do not need to specify the size of the array 
as the compiler is smart enough to count the number of elements
we've provided.  The elements themselves are denoted inside curly
brackets and delimited with commas.  

For both types of syntax, the actual array is always allocated on the heap
while the reference variables, \mintinline{java}{arr} \mintinline{java}{values}, 
and \mintinline{java}{names} are stored in the program call stack.

\subsubsection{Indexing}

Once an array has been created, its elements can be accessed
by indexing.  Java uses the standard 0-indexing scheme so the
first element is at index 0, the second at index 1, etc.  Indexing
an element involves using the square bracket notation and
providing an index.  Once indexed, an array element can be
treated as a normal variable and can be used with other operators
such as the assignment operator or comparison operators.

\begin{minted}{java}
arr[0] = 42;
if(arr[4] < 0) {
  System.out.println("negative!");
}
System.out.println("arr[1] = "+ arr[1]);
\end{minted}

Recall that an index is actually an offset.  The compiler and
system know exactly how many bytes each 
element takes and so an index \mintinline{java}{i} calculates
exactly how many bytes from the first element the $i$-th
element is located at.  Consequently it is possible to index
elements that are beyond the range of the array.  For example, 
\mintinline{java}{arr[-1]} or \mintinline{java}{arr[5]} would attempt
to access an element immediately \emph{before} the first element
and immediately \emph{after} the last element.  Obviously, 
these elements are not part of the array.  

If you attempt to access an element outside the bounds of
an array, the \gls{jvmLabel} will raise an \mintinline{java}{IndexOutOfBoundsException}
which is a \mintinline{java}{RuntimeException} that you can
catch and handle if you choose.  To prevent such an exception
you can write code that does not exceed the bounds of the
array.  Java arrays have a special \emph{length} property that
gives you the size of the array.  You can access the property
using the dot operator, so \mintinline{java}{arr.length} would
give the value 5.  

\subsubsection{Iteration}

Using the length property you can design a for-loop that
increments an index variable to iterate over the elements in
an array.

\begin{minted}{java}
int arr[] = new int[5];
for(i=0; i<arr.length; i++) {
  arr[i] = 5 * i;
}
\end{minted}

The for loop above initializes the variable \mintinline{java}{i} to zero,
corresponding to the first element in the array.  The continuation
condition is specifies that the loop continues while \mintinline{java}{i}
is strictly less than the size of the array denoted using the
\mintinline{java}{arr.length} property.  This iteration for-loop
is idiomatic when dealing with arrays.

In addition, Java (as of version 5) supports foreach loops that 
allow you to iterate over the elements of an array without using
an index variable.  Java refers to these loops as ``enhanced
for-loops'', but they are essentially foreach loops.  For example:

\begin{minted}{java}
for(int a : arr) {
  System.out.println(a);
}
\end{minted}

The syntax still uses the keyword \mintinline{java}{for}, but instead
of an index variable, it specifies the type, the loop-scoped variable
identifier followed by a colon and the array that you want to iterate
over.  Each iteration of the loop updates the variable \mintinline{java}{a}
to the next value in the array.  

\section{Dynamic Memory}

The use of the keyword \mintinline{java}{new} dynamically allocates 
memory space (on the heap) for the array.  Because Java has
automated garbage collection, if the reference to the array goes
out of scope, it is automatically cleaned up by the \gls{jvmLabel}.
This process is automated and essentially transparent to us.
There is no reliable way to force garbage collection and in general, 
we shouldn't.  There are sophisticated algorithms and heuristics
at work that determine the ``optimal'' time to perform garbage
collection.

\begin{minted}{java}
int arr[];
//create a new array of size 10:
arr = new arr[10];
//lose the reference by explicitly setting it to null:
arr = null;
//all references to the old memory are now lost and it is 
//eligible for garbage collection

//we can also reallocate new memory: 
arr = new arr[20];
\end{minted}

\section{Using Arrays with Methods}

We can pass and return arrays to and from methods in Java
To do so we use the same syntax as when we define them
by specifying a type and using square brackets.  As an example, 
the following method takes an array of integers and computes 
its sum.

\begin{minted}{java}
/**
 * This method computes the sum of elements in the
 * given array which contains n elements
 */
public static int computeSum(int arr[]) {
  int sum = 0;
  for(int i=0; i<size; i++) {
    sum += arr[i];
  }
  return sum;
}
\end{minted}

In Java, arrays are always passed by reference.  Though we
did not make any changes to the contents of the passed array
in the particular example, in general we could have.  Any
such changes would be realized in the calling method.  Unfortunately, 
there is no mechanism by which we can prevent changes to
arrays when passed to methods.\footnote{The use of the keyword
\mintinline{java}{final} only prevents us from changing the
array reference, not modifying the contents.}

We can also create an array in a method and return it as a value.  
For example, the following method creates a \gls{deep copy} of
the given integer array.  That is, a completely new array that
is a distinct copy of the old array.  In contrast, a \gls{shallow copy}
would be if we simply made one reference point to another reference.

\begin{minted}{java}
/**
 * This method creates a new copy of the given array
 * and returns it.
 */
public static int [] makeCopy(int a[]) {
  int copy[] = new int[a.length];
  for(int i=0; i<n; i++) {
    copy[i] = a[i];
  }
  return copy;
}
\end{minted}

The method returns an integer array.  In fact, this method seems
so useful, that it is already provided as part of the Java \gls{sdkLabel}.
The class \mintinline{java}{Arrays} contains dozens of utility
methods that process arrays.  In particular there is a \mintinline{java}{copyOf()}
method that allows you to create deep copies of arrays and
even allows you to expand or shrink their size.

\section{Multidimensional Arrays}

Java supports multidimensional arrays, though we'll only focus 
on 2-dimensional arrays.  To declare and allocate 2-dimensional
arrays, we simply use two square brackets with two specified
dimensions.

\begin{minted}{java}
int matrix[][] = new int[10][20];
\end{minted}

This creates a 2-dimensional array of integers with 10 rows and
20 columns.  Once created, we can index individual elements by
specifying the row and column.

\begin{minted}{java}
for(int i=0; i<matrix.length; i++) {
  for(int j =0; j<matrix[i].length; j++) {
    matrix[i][j] = 10;
  }
}
\end{minted}

\section{Dynamic Data Structures}

The Java \gls{sdkLabel} provides a rich assortment of dynamic
data structures as alternatives to arrays including lists, sets and
maps.  The classes that support and implement these data 
structures are defined in the Collections Library under the 
\mintinline{java}{java.util} package.  We will cover how to use
some of these data structures, but we will not go into the
details of how they are implemented nor the \gls{oopLabel}
concepts that underly them.

The Java \mintinline{java}{List} is an \emph{interface} that
defines a dynamic list data structure.  This data structure
provides a dynamic collection that can grow and shrink 
automatically as you add and remove elements from it.  It
is an interface, so it doesn't actually provide an implementation, 
just a specification for the publicly available methods that
you can use on it.  To common implementations are the 
\mintinline{java}{ArrayList}, which uses an array to hold 
elements, and \mintinline{java}{LinkedList} which stores
elements in linked nodes.

To create an instance of either of these lists, you use the
\mintinline{java}{new} keyword and the following syntax.

\begin{minted}{java}
List<Integer> a = new ArrayList<Integer>();
List<String> b = new LinkedList<String>();
\end{minted}

The first line creates a new instance of an \mintinline{java}{ArrayList}
that is \emph{parameterized} to hold \mintinline{java}{Integer} 
types.  The second creates a new instance of a \mintinline{java}{LinkedList}
that has been parameterized to only hold \mintinline{java}{String}
types.  Because of this parameterization, it would be a 
compiler error to attempt to add anything other than \mintinline{java}{Integer}s
to the first list or anything other than \mintinline{java}{String}s to the
second.

Once these lists have been created, you can add and remove
elements as follows.

\begin{minted}{java}
a.add(42);
a.add(81);
a.add(17);

b.add("Hello");
b.add("World");
b.add("Computers!");
\end{minted}

The order that you add elements is preserved, so in the first list, the
first element would be 42, the second 81, and the last 17.  You can
remove elements by specifying an index of the element to remove.
Like arrays, lists are 0-indexed.

\begin{minted}{java}
a.remove(0);

b.remove(2);
b.remove(0);
\end{minted}

As you remove elements, the indices are ``shifted'' down, so that after
removing the first element in the list \mintinline{java}{a}, 81 becomes
the new first element.  Removing the last then the first element in the
list \mintinline{java}{b} leaves it with only one element, \mintinline{java}{"World"}
as the first element (at index 0).

You can also retrieve elements from a list using 0-indexing and
the \mintinline{java}{get()} method. 

\begin{minted}{java}
List<Double> values = new ArrayList<Double>();
values.add(3.14);
values.add(2.71);
values.add(42.0);

double x = values.get(1); //get the second value, 2.71
double y = values.get(2); //get the third value, 42.0
\end{minted}

Any attempt to access an element that lies outside the bounds of
the \mintinline{java}{List}, will
result in an \mintinline{java}{IndexOutOfBoundsException} just as 
with arrays.  To stay within bounds you can use the \mintinline{java}{size()}
method to determine how many elements are in the collection.
In this example, \mintinline{java}{values.size()} would return an 
integer value of 3.

Finally, most collections implement the \mintinline{java}{Iterable}
interface which allows you iterate over the elements using an 
enhanced for-loop just as with arrays.

\begin{minted}{java}
for(Double x : values) {
  System.out.println(x);
}
\end{minted}

There are dozens of other methods that allow you to insert, remove, and
retrieve elements from a Java \mintinline{java}{List}; refer to the documentation
for details.

Another type of collection supported by the Collections library is a 
\mintinline{java}{Set}.  A set differs from a list in that elements are not
stored in any particular order; there is no concept of the first element
or last element.  Moreover, a set does not allow duplicate elements.\footnote{Duplicates
are determined by how the \mintinline{java}{equals()} and possibly the
\mintinline{java}{hashCode()} methods are implemented in your particular
objects.}  A commonly used implementation of the \mintinline{java}{Set}
interface is the \mintinline{java}{HashSet}.  

\begin{minted}{java}
Set<String> names = new HashSet<String>();

names.add("Robin");
names.add("Little John");
names.add("Marian");

//this has no effect since Robin is already part of the set:
names.add("Robin");
\end{minted}

Since the elements in a \mintinline{java}{Set} are unordered, 
we cannot use an index-based \mintinline{java}{get()} method
as we did with a set.  Fortunately, we can still use an enhanced
for-loop to iterate over the elements.

\begin{minted}{java}
for(String name : names) {
  System.out.println(name);
}
\end{minted}

When this code executes we cannot expect any particular order of
the three names.  Any permutation of the three may be printed.  If
we executed the loop more than once we may even observe a different
enumeration of the names!

Finally, Java also supports a \mintinline{java}{Map} data structure
which allows you to store key-value pairs.  The keys and values
can be any object type and are specified by two parameters when
you create the map.  A common implementation is the \mintinline{java}{HashMap}.

\begin{minted}{java}
//define a map that maps integers (keys) to strings (values):
Map<Integer, String> numbers = new HashMap<Integer, String>();

//add key-value pairs:
numbers.add(10, "ten");
numbers.add(20, "twenty");

//retrieve values given a key:
String name = numbers.get(20); //name equals "twenty"

//invalid mappings result in null:
name = numbers.get(30); //name is null
\end{minted}

The Collections library is much more extensive than what we've 
presented here.  It includes stacks, queues, hash tables, balanced 
binary search trees, and many other dynamic data structure 
implementations.
