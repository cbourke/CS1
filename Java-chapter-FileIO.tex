%!TEX root = ComputerScienceOne.tex

%%Chapter: FileIO in Java
\index{file I/O!in Java}
\index{Java!file I/O}

Java provides several different classes and 
utilities that support manipulating and processing
files.  In general, most file operations may
result in an \mintinline{java}{IOException}, 
a checked exception that \emph{must} be caught
and handled.

\section{File Input}

Though there are several ways that you can do file
input, the easiest is to use the familiar 
\mintinline{java}{Scanner} class.  We've previously
used this class to read from the standard input, 
\mintinline{java}{System.in}, but it can also be
used to read from a file using the \mintinline{java}{File} 
class (in the \mintinline{java}{java.io} package).  
The \mintinline{java}{File} class is just
a representation of a file resource and does not
represent the actual file (it cannot be opened and
closed).  To initialize a \mintinline{java}{Scanner} to read
from a file you can use the following.

\begin{minted}{java}
Scanner s = null;
try {
  s = new Scanner(new File("/user/apps/data.txt"));
} catch (FileNotFoundException e) {
  //handle the exception here
}
\end{minted}

When initializing a \mintinline{java}{Scanner} to
read from a file, the exception, 
\mintinline{java}{FileNotFoundException} \emph{must} be caught
and handled as it is a \emph{checked} exception.  Otherwise, once created, the \mintinline{java}{Scanner}
class does not throw any checked exceptions.

Once created, you can use the usual methods and functionality
provided by the \mintinline{java}{Scanner} class including
reading the \mintinline{java}{nextInt()}, \mintinline{java}{nextDouble()}, 
next string using \mintinline{java}{next()} or even the
entire \mintinline{java}{nextLine()}.  This last one
can be used in conjunction with \mintinline{java}{hasNext()}
to read an entire file, line-by-line.

\begin{minted}{java}
String line;
while(s.hasNext()) {
  line = s.nextLine();
  //process the line
}
\end{minted}

Once we are done reading the file, we can close the 
\mintinline{java}{Scanner} to free up resources:
\mintinline{java}{s.close();}.  We could have placed
all this code within one large \mintinline{java}{try-catch}
block with perhaps a \mintinline{java}{finally} block
to close the \mintinline{java}{Scanner} once were were
done to ensure that it would be closed regardless of
any exceptions.  However, Java 7 introduced a new 
construct, the \emph{try-with-resources} statement.

The \index{try-with-resources} try-with-resources statement 
allows us to place
the initialization of a ``closeable'' resource (defined
by the \mintinline{java}{AutoCloseable} interface) into
the \mintinline{java}{try} statement.  The JVM will then
automatically close this resource upon the conclusion of
the try-catch block.  We can still provide
a \mintinline{java}{finally} block if we wish, but this 
relieves us of the need to explicitly close the resource
in a \mintinline{java}{finally} block.  A full example:

\begin{minted}{java}
File f = new File("/user/apps/data.txt");
//initialize a Scanner in the try statement
try (Scanner s = new Scanner(f)) {
  String line;
  while(s.hasNext()) {
    line = s.nextLine();
	//process the line
  }
} catch (Exception e) {
  //handle the exception here
}
\end{minted}

Using the \mintinline{java}{Scanner} class to do file 
input offers a more abstract interaction with a file.  
It also uses a buffered input stream for performance.
Binary files can still be read using \mintinline{java}{nextByte()}.
However, when dealing with binary files, the better 
solution is to use a class that
models and abstracts the underlying file.  For example, 
if you are reading or writing an image file, you should
use the \mintinline{java}{java.awt.Image} class to
read and write to files.  The \gls{jdkLabel} and other
libraries offer a wide variety of classes to model all
kinds of data.

\section{File Output}

There are several ways to achieve file output, 
but we'll look at the two most useful ways.  First,
we consider how to do plaintext output 
using a buffered stream for performance.  Unfortunately,
to do this requires the nesting of several classes. 
We create a new \mintinline{java}{FileWriter} specifying
the path and file name, we then ``wrap'' that in a
\mintinline{java}{BufferedWriter} for better performance.
Finally, for convenience, we wrap that into a 
\mintinline{java}{PrintWriter} which offers many convenient
methods for writing primitive and \mintinline{java}{String} 
types.  It also offers a \mintinline{java}{printf()} style
method for formatting.

\begin{minted}{java}
int x = 10;
double pi = 3.14;
FileWriter fw = null;
try {
  fw = new FileWriter("data.txt");
} catch(IOException ioe) {
  throw new RuntimeException(ioe);
}
PrintWriter pw = new PrintWriter(new BufferedWriter(fw));

pw.println("Hello World!");
pw.println(x);
pw.printf("x = %d, pi = %f\n", x, pi);

pw.close();
\end{minted}

The \mintinline{java}{close()} method will conveniently
close all the underlying resources (in this case, the 
\mintinline{java}{BufferedWriter} and \mintinline{java}{FileWriter}) 
for us.  In addition, \mintinline{java}{PrintWriter}
implements \mintinline{java}{AutoCloseable} and so it can
be used in a try-catch-with resources statement.

Another ``convenience'' of \mintinline{java}{PrintWriter} is
that is ``swallows'' exceptions (just as the \mintinline{java}{Scanner}
class did).  That means we don't have to deal explicitly with
the checked \mintinline{java}{IOException}s that the
underlying classes throw as the \mintinline{java}{PrinteWriter}
silently catches them (though doesn't handle them).  However, 
this can also be viewed as a disadvantage in that if we want
to do error handling, we need to manually check if there was 
an error (using \mintinline{java}{checkError()}).
The \mintinline{java}{PrintWriter} class is intended mostly
for formatted output.  It does not provide a way to write
binary data to an output file.  Just as with binary input, 
it is best to use a class that abstracts the file type and
data so that we don't have to deal with the low-level details
of the binary data.

However, if necessary, binary output can be done using a
\mintinline{java}{FileOutputStream}.  Typically, you can
load all your data into a byte array and dump it all at
once.

\begin{minted}{java}
byte data[] = ...;
try (FileOutputStream fos = 
         new FileOutputStream(new File("outfile.bin")) ){
  fos.write(data);
} catch(IOException ioe) {
  throw new RuntimeException(ioe);
}
\end{minted}

This example uses a try-with-resources statement
so the \mintinline{java}{FileOutputStream} will automatically
be closed for us. You \emph{could} wrap a \mintinline{java}{FileOutputStream} 
in a \mintinline{java}{BufferedWriter}, but it
will likely not gain you anything in terms of performance.  A
buffered output stream is better if your writes are 
frequent and ``small.'' Here small means smaller than
the size of the buffer (\mintinline{java}{BufferedWriter}
is typically 8\gls{kbLabel}).  Writing several individual
integer values for example would be better done by buffering
them and writing them all at once.  
In our example, we simply wanted to dump a bunch of data, 
likely more than 8KB in practice, all at once.  Moreover, 
using a \mintinline{java}{BufferedWriter} would lose us the
ability to write raw byte data.


