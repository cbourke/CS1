%!TEX root = ComputerScienceOne.tex

%Strings - exercises

\section{Exercises}

\begin{exer}
Write a function to return the \emph{index} of the maximum 
element in an array of numbers.
\end{exer}

\begin{exer}
Write a function to return the \emph{index} of the minimum 
element in an array of numbers.
\end{exer}

\begin{exer}
Write a function to compute the mean (average) of an array of numbers.
\end{exer}

\begin{exer}
Write a function to compute the standard deviation of an array of numbers, 
  $$\sigma = \sqrt{\frac{1}{N} \sum_{i=1}^N (x_i - \mu)^2}$$
where $\mu$ is the mean of the array of numbers.
\end{exer}

\begin{exer}
Write a function that takes two arrays of numbers that are \emph{sorted}
and \emph{merges} them into one array (retuning a new array as a 
result).  
\end{exer}

\begin{exer}
Write a function that takes an integer $n$ and produces a new
array of size $n$ filled with 1s.
\end{exer}

\begin{exer}
Write a function that takes an array of numbers are computes 
returns the \emph{median} element.  The median is defined as 
follows:
 \begin{itemize}
   \item If $n$ is odd, the median is the $\frac{n+1}{2}$-th
       largest element
   \item If $n$ is even, the median is the average of the
       $\frac{n}{2}$ and the $(\frac{n}{2} + 1)$-th largest
       elements
 \end{itemize}
\end{exer}

\begin{exer}
The dot product of two arrays (or vectors) of the same 
dimension is defined as the sum of the product of each of 
their entries.  That is, 
  $$\sum_{i=1}^n a_i \times b_i$$
Write a function to compute the dot product of two arrays 
(you may assume that they are of the same dimension)
\end{exer}

\begin{exer}
The \emph{norm} of an $n$-dimensional vector, 
$\vec{x} = (x_1, x_2, \ldots, x_n)$ captures the notion of 
``distance'' in a higher dimensional space and is defined as
  $$\|\vec{x}\| = \sqrt{x_1^2 + \cdots + x_n^2}$$
Write a function that takes an array of numbers that 
represents an $n$-dimensional vector and computes
its norm.
\end{exer}

\begin{exer}
Write a function that takes two arrays $A, B$ and creates
and returns a new array that is the \emph{concatenation}
of the two.  That is, the new array will contain all elements
$a$ followed by all elements in $b$.
\end{exer}

\begin{exer}
Write a function that takes an array of numbers $A$ 
and an element $x$ and returns true/false if $A$ contains
$x$
\end{exer}

\begin{exer}
Write a function that takes an array of numbers $A$, an 
element $x$ and two indices $i, j$ and returns true/false 
if $A$ contains $x$ somewhere between index $i$ and
$j$.
\end{exer}

\begin{exer}
Write a function that takes an array of numbers $A$ 
and an element $x$ and returns the \emph{multiplicity} of
$x$; that is the number of times $x$ appears in $A$.
\end{exer}

\begin{exer}
Write a function to compute a \emph{sliding window} 
mean.  That is, it computes the average of the first $m$ 
numbers in the array.  The next value is the average of 
the values index from 1 to $m$, then $2$ to $m+1$, etc.
The last window is the average of the last $m$ elements.  
Obviously, $m \leq n$ (for $m = n$, this is the usual mean). 
Since there is more than one value, your function will return 
a (new) array of means of size $n-m+1$.  
\end{exer}

\begin{exer}
Write a function to compute the \emph{mode} of an array of 
numbers.  The mode is the most \emph{common} value that 
appears in the array.  For example, if the array contained the 
elements $2, 9, 3, 4, 2, 1, 8, 9, 2$, the mode would be
2 as it appears more than any other element.  The mode may 
not be unique; multiple elements could appear the same, 
maximal number of times.  Your function should simply return 
\emph{a} mode.
\end{exer}

\begin{exer}
Write a function to find \emph{all} modes of an array.  That 
is, it should find all modes in an array and return a new array 
containing all the mode values.
\end{exer}

\begin{exer}
Write a function to filter out certain elements from an array.  
Specifically, the function will create a new array containing 
only elements that are greater than or equal to a certain 
threshold $\delta$.  
\end{exer}

\begin{exer}
Write a function that takes an array of numbers and 
creates a new ``deep'' copy of the array.  In addition, the
function should take a new ``size'' parameter which will
be the size of the copy.  If the new size is less than
the original, then the new array will be a \emph{truncated} 
copy.  If the new size is greater then the copy will be 
\emph{padded} with zero values at the end.
\end{exer}

\begin{exer}
Write a function that takes an array $A$ and two indices
$i, j$ and returns a new array that is a subarray of 
$A$ consisting of elements $i$ through $j$.
\end{exer}

\begin{exer}
Write a function that takes two arrays $A, B$ and creates
and returns a new array that represents the \emph{unique
intersection} of $A$ and $B$.  That is, an array that contains
elements that are in both $A$ \emph{and} $B$.  However, 
elements should not be included more than once.
\end{exer}

\begin{exer}
Write a function that takes two arrays $A, B$ and creates
and returns a new array that represents the \emph{unique
union} of $A$ and $B$.  That is, an array that contains
elements that are \emph{either} in $A$ \emph{or} $B$
(or both).  However, elements should not be included more than once.
\end{exer}

\begin{exer}
Write a function that takes an array of numbers and returns the
sum of its elements.
\end{exer}

\begin{exer}
Write a function that takes an array of numbers and two indices
$i, j$ and computes the sum of its elements between $i$ and
$j$ inclusive.  
\end{exer}

\begin{exer}
Write a function that takes \emph{two} arrays of numbers and
determines if they are \emph{equal}: that is, each index contains 
the same element.
\end{exer}

\begin{exer}
Write a function that takes \emph{two} arrays of numbers and
determines if they both contain the same elements (though 
are not necessarily equal) regardless of their multiplicity.  That is, 
the function should return true even if the arrays' elements 
appear a different number of times or in a different order.
For example $[2, 2, 3]$ would be equal to an array containing 
$[3, 2, 3, 3, 3, 2, 2, 2]$.
\end{exer}

\begin{exer}
A \emph{suffix array} is a lexicographically sorted array of 
all suffixes of a string.  Suffix arrays have many applications 
in text indexing, data compression algorithms and in 
bioinformatics.  Write a program that takes a string 
and produces its suffix array.  

For example, the suffixes of \mintinline{text}{science} are
\mintinline{text}{science}, \mintinline{text}{cience}, \mintinline{text}{ience}, \mintinline{text}{ence}, \mintinline{text}{nce}, \mintinline{text}{ce}, and \mintinline{text}{e}. The suffix array (sorted) would look something 
like the following.

\begin{minted}{text}
ce
cience
e
ence
ience
nce 
science
\end{minted}
\end{exer}

\begin{exer} 
An array of size $n$ represents a \emph{permutation} if it 
contains all integers $0, 1, 2, \ldots, (n-1)$ exactly once.  
Write a function to determine if an array is a permutation or 
not.
\end{exer}

\begin{exer}
The $k$-th order statistic of an array is the $k$-th largest 
element.  For our purposes, $k$ starts at 0, thus the minimum 
element is the $0$-th order statistic and the largest element is 
the $n-1$-th order statistic.

Another way to view it is: suppose we were to \emph{sort} the 
array, then the $k$-th order statistic would be the element at index 
$k$ in the sorted array.  Write a function to find the $k$-th order statistic.
\end{exer}


%%%% MATRICES %%%%%

\begin{exer}
Write a function that takes two $n \times n$ square matrices 
$A, B$ and returns a new $n \times n$ matrix $C$ which is 
the product, $A \times B$.  The product of two matrices of 
dimension $n \times n$ is defined as follows:
$$c_{ij} = \sum_{k=1}^n a_{ik} b_{kj}$$
Where $1 \leq i,j \leq n$ and $c_{ij}$ is the $(i,j)$-th entry of the
matrix $C$.  
\end{exer}

\begin{exer}
We can multiply a matrix by a single scalar value $x$ by simply
multiplying each entry in the matrix by $x$.  Write a function that
takes a matrix of numbers and an element $x$ and performs
scalar multiplication.
\end{exer}

\begin{exer}
Write a function that takes two matrices and determines if 
they are equal (all of their elements are the same).
\end{exer}

\begin{exer}
Write a function that takes a matrix and an index $i$ and 
returns a new \emph{array} that contains the elements in the
$i$-th row of the matrix.
\end{exer}

\begin{exer}
Write a function that takes a matrix and an index $j$ and 
returns a new \emph{array} that contains the elements in the
$j$-th column of the matrix.
\end{exer}

\begin{exer}
Iterated Matrix Multiplication is where you take a square 
matrix, $A$ and multiply it by itself $k$ times, 
  $$A^k = \underbrace{A \times A \times \cdots \times A}_{k \, \mathrm{ times}}$$
Write a function to compute the $k$-th power of a matrix $A$.  
\end{exer}

\begin{exer}
The \emph{transpose} of a square matrix is an operation
that ``flips'' the matrix along the diagonal from the upper left
to the lower right.  In particular the values $m_{i,j}$ and $m_{j, i}$
are swapped.  Write a function to transpose a given matrix
\end{exer}

\begin{exer}
Write a function that takes a matrix, a row index $i$, and a number
$x$ and adds $x$ to each value in the $i$-th row.
\end{exer}

\begin{exer}
Write a function that takes a matrix, a row index $i$, and a number
$x$ and multiples each value in the $i$-th row with $x$.
\end{exer}

\begin{exer}
Write a function that takes a matrix, and two row indices $i, j$
and \emph{swaps} each value in row $i$ with the value in row $j$
\end{exer}

\begin{exer}
A special matrix that is often used is the \emph{identity} matrix.  
The identity matrix is an $n \times n$ matrix with 1s on its diagonal 
and zeros everywhere else.  Write a function that, given $n$, 
creates a new $n\times n$ identity matrix.
\end{exer}

\begin{exer}
Write a function to convert a matrix of integers to 
floating point numbers.
\end{exer}

\begin{exer}
Write a function to determine if a given matrix is a 
\emph{permutation matrix}.  A permutation matrix is a 
matrix that represents a permutation of the rows of an 
identity matrix.  That is, $A$ is a permutation matrix if 
every row and every column has exactly one 1 and the 
rest are zeros.
\end{exer}

\begin{exer}
Write a function to determine if a given square matrix 
is \emph{symmetric}.  A matrix is symmetric if $m_{i,j} = m_{j,i}$ 
for all $i, j$. 
\end{exer}

\begin{exer}
Write a function to compute the \emph{trace} of a matrix.  
The trace of a matrix is the sum of its elements along its 
diagonal.
\end{exer}

\begin{exer}
Write a function that returns a ``resized'' copy of a matrix.  
The function takes a matrix of size $n \times m$ (not necessarily 
square) and creates a copy that is $p \times q$ ($p, q$ are part
of the input to the function).  If $p < n$ and/or $q < m$, the 
values are ``cut off''.  If $p > n$ and/or $q > m$, the matrix is 
padded (to the right and to the bottom) with zeros.  
\end{exer}

\begin{exer}
A \emph{submatrix} is a matrix formed by selecting certain rows 
and columns from a larger matrix.  Write a function that constructs 
a submatrix from a larger matrix.  To do so, the function will take
a matrix as well as two row indices $i, j$ and two column indices
$k, \ell$ and it will return a new matrix which consists of entries
from the $i$-th row through the $j$-th row and $k$-th column 
through the $\ell$-th column.

For example, if $A$ is
$$\mathbf{A}=\begin{bmatrix}
    8 & 2 & 4 & 1 \\
   10 & 4 & 2 & 3 \\
   12 & 42 & 1 & 0 \\
  \end{bmatrix}.$$
then a call to this function with $i=1, j=2, k=2, \ell=3)$ should
result in
$$\mathbf{A}=\begin{bmatrix}
   2 & 3 \\
   1 & 0 \\
  \end{bmatrix}.$$
\end{exer}

\begin{exer}
The \emph{Kronecker} product (\url{http://en.wikipedia.org/wiki/Kronecker_product}) is a matrix operation on two matrices
that produces a larger block matrix.  Specifically, if $A$ is an 
$m \times n$ matrix and $B$ is a $p \times q$ matrix, then the
Kronecker product $A \otimes B$ is the $mp \times nq$ block matrix:

$$\mathbf{A}\otimes\mathbf{B} = \begin{bmatrix} a_{11} \mathbf{B} & \cdots & a_{1n}\mathbf{B} \\ \vdots & \ddots & \vdots \\ a_{m1} \mathbf{B} & \cdots & a_{mn} \mathbf{B} \end{bmatrix}$$

more explicitly:

$${\mathbf{A}\otimes\mathbf{B}} = \begin{bmatrix}
   a_{11} b_{11} & a_{11} b_{12} & \cdots & a_{11} b_{1q} &
                   \cdots & \cdots & a_{1n} b_{11} & a_{1n} b_{12} & \cdots & a_{1n} b_{1q} \\
   a_{11} b_{21} & a_{11} b_{22} & \cdots & a_{11} b_{2q} &
                   \cdots & \cdots & a_{1n} b_{21} & a_{1n} b_{22} & \cdots & a_{1n} b_{2q} \\
   \vdots & \vdots & \ddots & \vdots & & & \vdots & \vdots & \ddots & \vdots \\
   a_{11} b_{p1} & a_{11} b_{p2} & \cdots & a_{11} b_{pq} &
                   \cdots & \cdots & a_{1n} b_{p1} & a_{1n} b_{p2} & \cdots & a_{1n} b_{pq} \\
   \vdots & \vdots & & \vdots & \ddots & & \vdots & \vdots & & \vdots \\
   \vdots & \vdots & & \vdots & & \ddots & \vdots & \vdots & & \vdots \\
   a_{m1} b_{11} & a_{m1} b_{12} & \cdots & a_{m1} b_{1q} &
                   \cdots & \cdots & a_{mn} b_{11} & a_{mn} b_{12} & \cdots & a_{mn} b_{1q} \\
   a_{m1} b_{21} & a_{m1} b_{22} & \cdots & a_{m1} b_{2q} &
                   \cdots & \cdots & a_{mn} b_{21} & a_{mn} b_{22} & \cdots & a_{mn} b_{2q} \\
   \vdots & \vdots & \ddots & \vdots & & & \vdots & \vdots & \ddots & \vdots \\
   a_{m1} b_{p1} & a_{m1} b_{p2} & \cdots & a_{m1} b_{pq} &
                   \cdots & \cdots & a_{mn} b_{p1} & a_{mn} b_{p2} & \cdots & a_{mn} b_{pq}
\end{bmatrix}. $$
Write a function that computes the Kronecker product.
\end{exer}

\begin{exer}
The Hadamard product is an entry-wise product of two matrices 
of equal size.  Let $\mathbf{A}, \mathbf{B}$ be two $n \times m$ 
matrices, then the Hadamard product is defined as follows.
$$\mathbf{A} \circ \mathbf{B} = \begin{pmatrix} a_{11} & a_{12} & \cdots & a_{1m} \\
 a_{21} & a_{22} & \cdots & a_{2m} \\
\vdots & \vdots & \ddots & \vdots \\
 a_{n1} & a_{n2} & \cdots & a_{nm} \\
\end{pmatrix}\circ\begin{pmatrix}
 b_{11} & b_{12} & \cdots & b_{1m} \\
 b_{21} & b_{22} & \cdots & b_{2m} \\
\vdots & \vdots & \ddots & \vdots \\
 b_{n1} & b_{n2} & \cdots & b_{nm} \\
\end{pmatrix} =\begin{pmatrix}
 a_{11}b_{11} & a_{12}b_{12} & \cdots & a_{1m}b_{1m} \\
 a_{21}b_{21} & a_{22}b_{22} & \cdots & a_{2m}b_{2m} \\
\vdots & \vdots & \ddots & \vdots \\
 a_{n1}b_{n1} & a_{n2}b_{n2} & \cdots & a_{nm}b_{nm} \\
\end{pmatrix}$$
Write a function to compute the Hadamard product of two 
$n \times m$ matrices.
\end{exer}

