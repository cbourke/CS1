%!TEX root = ComputerScienceOne.tex

%Functions  - exercises

\section{Exercises}

\begin{exer}
\label{exercise:functions:gcd}
Recall that the \emph{greatest common divisor} (gcd) of two positive integers, 
$a$ and $b$ is the largest positive integer that divides both $a$ and
$b$.  Adapt the solution from Exercise \ref{exercise:loops:gcd} into
a function.  If the language you use supports it, return the gcd via
a pass by reference variable.
\end{exer}

\begin{exer}
Write a function that \emph{scales} an input $x$ to to its scientific
notation scale so that $1 \leq x < 10$.  If you language supports 
pass by reference, the amount that $x$ is shifted should be stored 
in a pass-by-reference parameter.  For example, a call to this function 
with $x = 314.15$ should return $3.1415$ and the amount it is scaled
by is $n=-2$.
\end{exer}

\begin{exer}
Write a function that returns the most significant digit of a floating
point number.  The function should only return an integer in the range 
1--9 (it should return zero only if $x = 0$).
\end{exer}

\begin{exer}
Write a function that, given an integer $x$, sums the values of
its digits.  That is, for $x = 29423$ the sum $2 + 9 + 4 + 2 + 3 = 20$.
\end{exer}

\begin{exer}
Write a function to convert radians to degrees using the formula,
  $$deg = \frac{180\cdot rad}{\pi}$$
Write another function to covert degrees to radians.
\end{exer}

\begin{exer}
\label{exercise:functions:circleStats}
Write functions to compute the diameter, circumference and
area of a circle given its radius.  If your language supports 
pass by reference, compute all three of these with one function.
\end{exer}

\begin{exer}
The \emph{arithmetic-geometric} mean of two numbers $x, y$, denoted 
$M(x, y)$ (or $\mathrm{agm}(x, y)$) can be computed iteratively as 
follows.  Initially, $a_1 = \frac{1}{2}(x + y)$ and $g_1 = \sqrt{xy}$ (i.e. the 
normal arithmetic and geometric means).  Then, compute
  $$\begin{array}{rcl}
    a_{n+1} & = & \frac{1}{2}(a_n+g_n) \\
    g_{n+1} & = & \sqrt{a_n g_n}
    \end{array}$$
The two sequences will converge to the same number which is the arithmetic-geometric 
mean of $x, y$.  Obviously we cannot compute an infinite sequence, so we compute until
$|a_n - g_n| < \epsilon$ for some small number $\epsilon$.
\end{exer}

\begin{exer}
\label{exercise:functions:apr}
Write a function to compute the annual percentage yield (APY) given
an annual percentage rate (APR) using the formula 
  $$APY = e^{APR} - 1$$
\end{exer}

\begin{exer}
\label{exercise:functions:airDistance}
Write a function that will compute the air distance between two locations given their
latitudes and longitudes.  Use the formula as in Exercise \ref{exercise:basics:airDistance}.
\end{exer}

\begin{exer}
\label{exercise:functions:rgbToCMYK}
Write a function to convert a color represented in the RGB (red-green-blue) 
color model (used in digital monitors) to a CMYK (cyan-magenta-yellow-key) 
used in printing.  RGB values are integers in the range $[0, 255]$ while CMYK 
are fractional numbers in the range $[0, 1]$.  To convert to CMYK, you first need 
to scale each integer value to the range $[0, 1]$ by simply computing
  $$r' = \frac{r}{255}, \quad g' = \frac{g}{255}, \quad b' = \frac{b}{255}$$
and then using the following formulas:
\begin{align*}
K & = 1-\max\{r', g', b'\} \\
C & = (1-r'-k) / (1-k) \\
M & = (1-g'-k) / (1-k) \\
Y & = (1-b'-k) / (1-k) \\
\end{align*}
\end{exer}

\begin{exer}
\label{exercise:functions:cmykToRGB}
Write a function to convert from CMYK to RGB using the following formulas.
\begin{align*}
r & = 255 \cdot (1 - C) \cdot (1-K) \\
g & = 255 \cdot (1 - M) \cdot (1-K) \\
b & = 255 \cdot (1 - Y) \cdot (1-K) \\
\end{align*}
\end{exer}

\begin{exer}
Write some functions to convert an RGB color to a gray scale, ``removing'' the color 
values.  An RGB color value is grayscale if all three components have the same 
value.  To transform a color value to grayscale, there there are several possible 
techniques.  The average method simply sets all three values to the average:
  $$\frac{r + g + b}{3}$$
The lightness method averages the most prominent and least prominent colors:
  $$\frac{\max\{r, g, b\} + \min\{r, g, b\}}{2}$$
The luminosity technique uses a weighted average to account for a human perceptual
preference toward green:
  $$0.21 r + 0.72 g + 0.07 b$$
\end{exer}

\begin{exer}
\label{exercise:functions:squareRoots}
Adapt the methods to compute a square root in Exercise 
\ref{exercise:loops:squareRoots} into functions.
\end{exer}

\begin{exer}
\label{exercise:functions:naturalLog}
Adapt the methods to compute the natural logarithm in Exercise 
\ref{exercise:loops:naturalLog} into functions.
\end{exer}

\begin{exer}
\label{exercise:functions:weight}
Weight (mass in the presence of gravity) can be measured in 
several scales: kilogram force (kgf), pounds (lbs), ounces (oz), or 
Newtons (N).  To convert between these scales, you can use the following facts:

\begin{itemize}
  \item 1 kgf is equal to 2.20462 pounds
  \item There are 16 ounces in a pound
  \item 1 kgf is equal to 9.80665 Newtons
\end{itemize}

Write a collection of functions to convert between these scales.
\end{exer}

\begin{exer}
\label{exercise:functions:length}
Length can be measured by several different units.  We will concern 
ourselves with the following scales: kilometer, mile, nautical mile, and 
furlong.  A measure in each one of these scales can be converted to 
another using the following facts.
\begin{itemize}
  \item One mile is equivalent to 1.609347219 kilometers
  \item One nautical mile is equivalent to 1.15078 miles
  \item A furlong is $\frac{1}{8}$-th of a mile
\end{itemize}
Write a collection of functions to convert between these scales.
\end{exer}

\begin{exer}
\label{exercise:functions:temperature}
Temperature can be measured in several scales: Celsius, Kelvin, 
Fahrenheit, and Newton.  To convert between these scales, you 
can use the following conversion table.

\begin{table}[h]
\centering
\begin{tabular}{|c||c|c|c|c|}
\hline
  From/To      & Celsius & Kelvin & Fahrenheit & Newton \\
\hline
\hline
  Celsius        & -- & $c + 273.15$ & $c \frac{9}{5} + 32$ & $c \frac{33}{100}$ \\
\hline
  Kelvin          & $k - 273.15$  & -- & $\frac{9}{5}k - 459.67$ & $.33k - 90.1395$ \\
\hline
  Fahrenheit  & $(f - 32) \frac{5}{9}$ & $\frac{5}{9}f + 255.372$ & -- & $\frac{11}{60}f - \frac{88}{15}$\\
\hline
  Newton      & $n \frac{100}{33}$ & $\frac{100}{33}n + 273.15$ & $\frac{60}{11}n + 32$ & -- \\
\hline
\end{tabular}
\caption{Conversion Chart}
\end{table}
Write a collection of functions to convert between these scales.
\end{exer}

\begin{exer}
\label{exercise:functions:energy}
Energy can be measured in several different scales: calories ($c$), 
joules ($J$), ergs ($erg$) and foot-pound force (ft-lbf) among others.  
To convert between these scales, you can use the following facts:

\begin{itemize}
  \item 1 erg equals $1.0 \times 10^{-7} J$
  \item 1 ft-lbs equals $1.3558$ joules
  \item 1 calorie is equal to $4.184$ joules
\end{itemize}
Write a collection of functions to convert between these scales.
\end{exer}

\begin{exer}
\label{exercise:functions:pressure}
Pressure is a measure of force applied to the surface of an object 
per unit area.  There are several units that can be used to measure 
pressure:
\begin{itemize}
  \item Pascal (Pa) which is one Newton per square meter
  \item Pound-force Per Square Inch (psi) 
  \item Atmosphere (atm) or standard atmospheric pressure
  \item The torr, an absolute scale for pressure
\end{itemize}

To convert between these units, you can use the following formulas.
\begin{itemize}
  \item 1 psi is equal to 6,894.75729 Pascals, 1 psi is equal to $0.06804596$ atmospheres
  \item 1 atmosphere is equal to 101,325 Pascals
  \item 1 torr is equal to $\frac{1}{760}$ atmosphere and $\frac{101,325}{760}$ Pascals
\end{itemize}
Write a collection of functions to convert between these scales.
\end{exer}



