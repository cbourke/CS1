%!TEX root = ComputerScienceOne.tex

%Recursion - exercises

\section{Exercises}

\begin{exer}
The binomial coefficients, $C(n,k)$ or ${n \choose k}$ (``$n$ choose
$k$''), are defined as the number of ways you can select $k$ distinct
items from a collection of $n$ items.  A direct combinatorial
definition is
$${n \choose k} = \frac{n!}{k!(n-k)!}$$
An alternative is Pascal's identity, which gives a recurrence to
compute this value:
$${n \choose k} = {n-1 \choose k} + {n-1 \choose k-1}$$
Where ${n \choose 0} = 1$ for any $n$ and for all $k > n$, ${n\choose
k} = 0$. Finally, ${n \choose 1} = n$.
\begin{enumerate}
  \item Write a recursive function using Pascal's identity to compute
  	${n \choose k}$.  Benchmark its performance.
  \item Write a recursive version that uses memoization to avoid 
  	recomputing values
  \item Modify your functions to utilize an arbitrary precision 
  	numeric type so that you can compute arbitrarily large values.
\end{enumerate}
\end{exer}

\begin{exer}
The Jacobsthal sequence is very similar to the Fibonacci sequence 
in that it is defined by its two previous terms.  The difference is that 
the second term is multiplied by two.  
$$J_n = \left\{
\begin{array}{ll}
0 & \textrm{if } n = 0 \\
1 & \textrm{if } n = 1 \\
J_{n-1} + 2J_{n-2} & \textrm{otherwise}
\end{array}
\right.$$

\begin{enumerate}
  \item Write a recursive function that computes the $n$-th Jacobsthal
    number.  Benchmark its performance.
  \item Write a recursive version that uses memoization to avoid 
  	recomputing values
  \item Modify your functions to utilize an arbitrary precision 
  	numeric type so that you can compute arbitrarily large values.
\end{enumerate}
\end{exer}

