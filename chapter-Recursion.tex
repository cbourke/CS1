%!TEX root = ComputerScienceOne.tex

%%Chapter: Recursion

Suppose we wanted to write a simple program that counted down, printing
10, 9, 8, \ldots, 2, 1 and when it reached zero it printed a ``Happy
New Year'' message.  Likely our first instinct would be to write a 
very simple for-loop using an increment variable.  But what if we lived
in a world \emph{without} the usual loop control structures that we are now
familiar with.  How might we write such a program?

After thinking about it for a while, we might think: well, we don't have
loops, but we still have functions.  In particular what if we had a function
that took the ``current'' value of our counter variable and decremented it, 
passing it to another function, which did the same thing.  For example, we 
could pass 10 to such a function, which would then subtract 1, passing 9
to another function and so on.  A check could be made to see if the value was 
zero, in which case we print our special message and no longer call any more 
functions.

In fact, we would not need to define 10 different functions to do so.  Instead, 
we could define one function that \emph{called itself}.  It might look something
like Algorithm \ref{algo:recursiveCountdown}.

\begin{algorithm}[H]
  \Input{An integer $n \geq 0$}
  \Output{A countdown of integers $n, \ldots 0$}
 
  \eIf{$n = 0$}{
    output ``Happy New Year!!!'' \;
  }{
    output $n$ \;
    $\textsc{CountDown}(n-1)$ \label{algo:line:recursiveCall} \;
  }
\caption{Recursive $\textsc{CountDown}(n)$ Function}
\label{algo:recursiveCountdown}
\end{algorithm}

The function in this case is called $\textsc{CountDown}()$.  In Line 
\ref{algo:line:recursiveCall} the function calls itself on a decremented
value.  When a function calls itself, it is a \emph{recursive} function and
many languages support \emph{recursion}.

This is not that odd of a concept.  We've seen many examples where functions
invoke other functions.  As we've seen, each function call simply creates
a new stack frame on the program stack.  There is nothing particularly special
about which functions call which other functions, so there is very little
difference when a function calls itself. 

This was not just a toy example.  There are many programming languages in which
recursion is used as a matter of course.  Functional programming languages
tend to avoid control structures like loops and even (mutable) variables.  
Instead, control flow is defined by evaluating a series of functions, making
recursion a fundamental operation.  

Recursion is extensively used in mathematics.  Recurrence relations or
recursive functions are common.  The Fibonacci sequence is
a common, if not overused\footnote{The Fibonacci sequence is nothing
special; its simply a second order linear homogenous recurrence relation
with coefficients of 1.  You can define many such relations.  The 
near reverence that so many people attribute to it borders on mysticism.}
example.  It has a simple definition: the next value in the sequence is
simply the sum of the two previous values.  The sequence starts with the 
\emph{initial values} of 1.  The first few terms in the sequence:
  $$1, 1, 2, 3, 5, 8, 13, 21, 34, 55, 89, \ldots$$
The more formal mathematical definition can be stated as follows.
\[
F_n = \left\{
\begin{array}{ll}
1 & \text{if } n = 0 \\
1 & \text{if } n = 1 \\
F_{n-1} + F_{n-2} & \text{otherwise}
\end{array}
\right.
\]

The Fibonacci sequence is the cliche example for recursion.  We can 
define an algorithmic function to compute the $n$-th Fibonacci number
as follows.

\begin{algorithm}[H]
  \Input{An integer $n \geq 0$}
  \Output{The $n$-th Fibonacci number, $F_n$}
 
  \eIf{$n \leq 1$}{
    output 1 \;
  }{
    output $\textsc{Fibonacci}(n-1) + \textsc{Fibonacci}(n-2)$ \;
  }
\caption{Recursive $\textsc{Fibonacci}(n)$ Function}
\label{algo:recursiveFibonacci}
\end{algorithm}

Though hackneyed, it does provide a good example for how recursive functions
work.  We'll also utilize it as an example of \emph{why you should avoid
recursion in practice}.  We'll also use it to illustrate how the problems
with recursion can be mitigated or avoided altogether.

\section{Writing Recursive Functions}

When writing a recursive function, there are several key elements that
we need to take care of to ensure that it functions correctly.  In particular, 
every recursive function requires at least one \emph{base case} or
base condition which serves as a terminating condition for the recursion.
A base case is a condition which, instead of making a recursive call, 
it processes and returns a value.  Without a base case, the recursion would 
continue unbounded: the  function would call itself over and over again, 
creating new stack frame after stack frame until we run out of stack space.  
If a program makes too many function calls and runs out of stack memory, 
it may lead to a \gls{stack overflow}\gls{stack overflow} and the 
termination of the program.  As we'll see below, even if we don't have
unbounded recursion, it is still possible to run out of stack space
even with simple recursion.

The other key element that we need is to ensure that every recursive
call \emph{makes progress} toward one of the terminating conditions.  
If no progress is made, then again we may have an unbounded recursion.
In the Fibonacci example in Algorithm \ref{algo:recursiveFibonacci}, the
base case can be found in the first if-statement: when $n$ reaches $1$
or less, no recursive calls are made.  In the else-statement, we make
two recursive calls, but both of them make progress toward this base 
case.  The first decrements $n$ by 1 and the second by 2, eventually
reaching $n = 1$.

\subsection{Tail Recursion}

Making many function calls can be costly in terms of stack space.  One
optimization that can be made is to use \emph{tail recursion}.  The last
operation that a function executes is referred to as the \emph{tail}
operation.  If a function invokes another function as its tail operation, 
its a \emph{tail call}.  For example, consider the following snippet of
code:

\begin{minted}{c}
int foo(int x) {
  ...
  return bar(x-1) + 1;
}
\end{minted}

Here, \mintinline{c}{foo()} calls \mintinline{c}{bar()} but it is 
\emph{not} the last operation before it returns.  Instead, it invokes
\mintinline{c}{bar()}, takes the result and adds one \emph{then} 
returns to the calling function.  Note that decrementing \mintinline{c}{x}
is performed \emph{before} the invocation of \mintinline{c}{bar()}.
In contrast, consider the following modified code:

\begin{minted}{c}
int foo(int x) {
  ...
  return bar(x-1);
}
\end{minted}

Here, the invocation of \mintinline{c}{bar()} is the last operation
performed by \mintinline{c}{foo()}.  Thus, this is a tail call.

Tail calls have the advantage that a language or compiler can generally
optimize the function call with respect to the stack frame.  Since the
function \mintinline{c}{foo()} is essentially done with its computation, 
its stack frame is no longer needed.  The system, therefore, can reuse
the stack frame.  Tail recursion is such an important optimization, some
languages require it or ``guarantee'' it in other ways.

\section{Avoiding Recursion}

Recursion is not essential; some languages do not even support recursion.
In fact, any recursive function can be rewritten to not use recursion.
Usually, you can write an equivalent loop structure or use a stack
data structure to replace the recursion.
So why use it?  Proponents would argue that recursion allows you to 
write simple code that more closely matches mathematical functions and
expressions.  Recursion is also a natural way to think about certain
problem solving techniques such as divide-and-conquer (see Chapter 
\ref{chapter:searchingSorting}.  It is also a natural way to code
in functional programming languages.

These arguments, however, are \emph{subjective}.  One person's 
``cleaner'' or ``more understandable'' code is another person's 
spaghetti code hack.  What is ``natural'' for one person may be 
``weird'' and ``odd'' for another.  However, there are many other
arguments against recursion, many of which are \emph{objective} 
reasons: that recursion is more expensive, less efficient, and can
easily lead to inefficient, exponential algorithms.

In general, recursion requires lots of function calls which requires 
creating and removing lots of stack frames.  This usually results in 
a lot of overhead and resources being used to perform the computation.  
Unless you are using a language in which recursion is optimized and
made to be more efficient (such as functional programming
languages), this is a lot more expensive than using simple loops and
iteration.  

Another reason to avoid recursion is that it can lead to a lot of
extraneous re-computations.  The cliched example of the Fibonacci
recursion is a prime example of this.\footnote{Its overuse as
an example of recursion is even less explicable as it solves a problem
that no one cares about.}  Consider the computation of 
$\textsc{Fibonacci}(5)$.  This results in two recursive calls, each of
those calls results in 2 recursive calls and so on as depicted in 
Figure \ref{figure:fibonacciTree}.

\begin{figure}
\centering

\begin{tikzpicture}
  [scale=.50,transform shape,
   level distance=4cm,
   every node/.style={rectangle,
                      thin,
                      draw=black,
                      %top color = white,
                      %bottom color = black!50,
                      inner sep=5pt,
                      %drop shadow={opacity=0.35,shadow xshift=.2ex, shadow yshift=-.2ex}
                      },
   edge from parent/.style={->,draw,>=latex'},
   level 1/.style={sibling distance=16cm},
   level 2/.style={sibling distance=8cm},
   level 3/.style={sibling distance=4cm},
   level 4/.style={sibling distance=4cm}]
  \node {\mintinline{text}{fibonacci(5)}}
     child {node {\mintinline{text}{fibonacci(4)}}
       child {node {\mintinline{text}{fibonacci(3)}}
         child {node {\mintinline{text}{fibonacci(2)}}
          child {node {\mintinline{text}{fibonacci(1)}}}
          child {node {\mintinline{text}{fibonacci(0)}}}
         }
         child {node {\mintinline{text}{fibonacci(1)}}}
       }
       child {node {\mintinline{text}{fibonacci(2)}}
          child {node {\mintinline{text}{fibonacci(1)}}}
          child {node {\mintinline{text}{fibonacci(0)}}}
       }
     }
     child {node {\mintinline{text}{fibonacci(3)}}
        child {node {\mintinline{text}{fibonacci(2)}}
          child {node {\mintinline{text}{fibonacci(1)}}}
          child {node {\mintinline{text}{fibonacci(0)}}}
        }
        child {node {\mintinline{text}{fibonacci(1)}}}
     };
\end{tikzpicture}

\caption{Recursive Fibonacci Computation Tree}
\label{figure:fibonacciTree}
\end{figure}

As depicted in the figure, several function calls are repeated: 
$\textsc{Fibonacci}(3)$ is called twice, $\textsc{Fibonacci}(2)$
is called 3 times, etc.  15 total function calls are made to
compute $\textsc{Fibonacci}(5)$.  In general, the computation
of $\textsc{Fibonacci}(n)$ will result in an \emph{exponential}
number of function calls.  In fact, the number of function calls
to compute $F_n$ with this recursive solution will be equal to 
  $$F_{n-2} + \sum_{i=0}^{n-1} F_{i}$$ 
That is, more than the first $n-1$ Fibonacci numbers combined!
It should come as no surprise that the Fibonacci sequence
grows \emph{exponentially} and thus sow would the number of 
operations with this recursive solution.

To put this in perspective, consider computing $F_{45} = 1,836,311,903$ 
($n = 45)$, the maximum representable value for a 32-bit signed 2s
complement integer.  Executing a C implementation of this recursive
algorithm took about 8 seconds\footnote{On a 2.7GHz Intel Core i7.} 
and required 3,672,623,805 function calls!

What if we wanted to compute $F_{100} = 573,147,844,013,817,084,101$ 
(573 quintillion) it would result in 1,146,295,688,027,634,168,201 
(1.146 sextillion) function calls.  Using the same hardware, at 
$4.59\times 10^8$ (459 million) function calls per second, it would
take $2.497 \times 10^{12}$ seconds to compute.  That would be more than
79,191 \emph{years}!  Even if we performed these (useless) calculations
on hardware that was 1 million times faster than my laptop, it would
still take over 4 \emph{weeks}!

\subsection{Memoization}
\index{memoization}

The inefficiency in the example above comes from the fact that we make 
the same function calls on the same values over and over.  One way to
avoid recomputing the same values is to \emph{store} them into a table
(or \emph{tableau} if you prefer being fancy).  Then, when you need
to compute a value, you look at the table to see if it has already
been computed.  If it has, we reuse the value stored in the table, otherwise
we compute it by making the appropriate recursive calls.  Once computed,
we place the value into the table so that it can be looked up on subsequent
function calls.  This approach is usually referred to as \gls{memoization}.

The ``table'' in this case is very general: it can be achieved using 
a number of different data structures including simple arrays, or even
maps (mapping input value(s) to output values).  The table is essentially
serving as a \gls{cache} for the previously computed values. An 
illustration of how this might work can be found in Algorithm 
\ref{algo:recursiveFibonacciMemoization}.  Here, the recursion only
occurs if the value $F_n$ is not yet defined.

\begin{algorithm}[H]
  \Input{An integer $n \geq 0$, a global map $M$ that maps $n$ values to $F_n$}
  \Output{The $n$-th Fibonacci number, $F_n$}
 
  \eIf{$F_n$ is defined in $M$}{
    output $M(n)$ \;
  }{
    $a \leftarrow \textsc{Fibonacci}(n-1)$ \;
    $b \leftarrow \textsc{Fibonacci}(n-2)$ \;
    Define $M(n) = a + b$ \;
    output $(a + b)$ \;
  }
\caption{Recursive $\textsc{Fibonacci}(n)$ Function With Memoization}
\label{algo:recursiveFibonacciMemoization}
\end{algorithm}

In many functional programming languages, memoization is implicit and
provided by the language itself to ensure that we do not have the same
problems with recomputing values as we observed above.  In languages
such as C and Java, memoization becomes our responsibility and is not
an optimization provided by the language itself.  

Note, however, that if we are filling in a table of values anyway, we
really don't need to make recursive calls at all.  We simply need to 
figure out in what order to fill out the values in the table.  This is
actually the basis of a powerful programming technique known as
\gls{dynamic programming} which is a ``bottom-up'' approach to solving
problems by combining solutions to smaller ones.


