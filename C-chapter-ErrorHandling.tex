%!TEX root = ComputerScienceOne.tex

%%Chapter: Error Handling in C

The C language does not support exceptions or exception handling.
Instead, the usual method of error handling is done through 
defensive programming.  As a user, it is your responsibility to
write code that checks for invalid or unsafe operations before
executing them and handle the error appropriately.

One way that we've already seen to handle errors in C is to use
the \mintinline{c}{exit()} function in the standard library.  This function
immediately terminates the execution of our program and ``returns''
an integer valued ``exit code.''  This exit code can be used to
communicate errors between different processes.  The process that
executed the program can use it to determine if the program exited
with or without an error.  By convention, 0 indicates an error while
a non-zero value indicates an error.  Using the \mintinline{c}{exit()} 
function should only be done when an error should be considered
\emph{fatal} or severe enough that the program should immediately
terminate.

\section{Language Supported Error Codes}
\label{section:errno}

In the standard libraries, some functions are designed to indicate a
n error by returning a special ``flag'' value such as $-1$ or 
\mintinline{c}{NULL}.  In general, though, the standard library functions
communicate an error code represented as an integer value.
As previously mentioned, C uses the convention that 0 
indicates ``no error'' while a non-zero value indicates an error.

Error codes are communicated through a special global variable called
\mintinline{c}{errno} which is defined in the \mintinline{c}{errno.h} 
standard header file.  This header file also defines several macros 
that are identified with various error codes.  The C standard actually
only requires the following three error codes to be supported.  

\begin{itemize}
  \item \mintinline{c}{EDOM} indicates an error in the \emph{domain} 
  	of a function; that is, an error with respect to the function's input value(s).
	For example, calling the math library's \mintinline{c}{sqrt(-1)} on a negative
	value results in an \mintinline{c}{EDOM} error as C does not support
	complex numbers.
  \item \mintinline{c}{ERANGE} indicates an error in the \emph{range}
  	of a function; that is, an error with respect to the function's output value(s).
	For example, calling the math library's \mintinline{c}{log(0)} with a zero
	value results in an \mintinline{c}{ERANGE} error as C does not 
	support $-\infty$ as an actual number.
  \item \mintinline{c}{EILSEQ} indicates an illegal byte sequences in 
  	characters on systems that use UTF-8.
\end{itemize}

All three of these are defined in the \mintinline{c}{errno.h} header file.
Depending on the system, additional error codes may also be defined
and supported (see POSIX Error Codes below).  

When an error occurs, a function will set the global variable 
\mintinline{c}{errno} to one of these error code values.  Upon returning 
from a function, you can check for these error codes.  Since these 
error codes are represented as integers, you simply use the numerical 
comparison operator, \mintinline{c}{==}.  You can check for no error
by making a comparison to zero.

In addition, the standard string library, defined in the header file, 
\mintinline{c}{string.h} provides a function, 
\mintinline{c}{char * strerror(errno)} that can be used to map 
the value in \mintinline{c}{errno} to a human-readable error message.  
We discuss strings in detail later on, but we can see how to use this 
function in the following demonstration (see below).  The output of this 
program is as follows.

\begin{minted}{text}
result: 1.4142, error: 0
result: -nan, error: 33
it was an EDOM error
Error Message: Numerical argument out of domain
result: -inf, error: 34
it was an ERANGE error
Error Message: Numerical result out of range
\end{minted}

For this particular system, the \mintinline{c}{EDOM} and 
\mintinline{c}{ERANGE} error codes were associated with the
integer values 33 and 34 respectively.  These numbers are 
not necessarily the same on all systems so comparisons must 
be made against the macro names for portability.

\inputminted{c}{code/errorDemo.c}

\subsection{POSIX Error Codes}

\gls{posixLabel} is an \gls{ieeeLabel} set of standards for maintaining compatibility between
various operating systems.  It defines several rules and expectations
that operating systems must adhere to in order to be POSIX 
\emph{compliant}.  Such standards allow developers to develop 
code that should be interoperable among a collection of different
operating systems, reducing the need for rewrites and reimplementations
for different systems.

In particular, \gls{posixLabel} compliant systems define many other error 
codes (see \cite{posixErrno}) that can be used beyond the three
mentioned above.  For example, the \mintinline{c}{ENOENT} error
code corresponds to ``No such file or directory'' and \mintinline{c}{EACCES}
corresponds to a ``Permission denied'' error.

\section{Error Handling By Design}

In our own code we \emph{could} communicate errors to calling
functions by setting \mintinline{c}{errno}, however, we may run 
into compatibility issues with the standard error codes or POSIX 
error codes.  Instead, it may be more appropriate to do error 
handling by utilizing the return value of a function to communicate
an error code.  

That is, to do error handling we could design our functions to always
return an \mintinline{c}{int} value to indicate an error: 0 for no error
and some non-zero value to indicate various different types of errors.
Of course, as a consequence any value that needs to be ``returned''
to the calling function would need to be done so via a pass by reference
variable.  

As an example, let's revisit the quadratic roots example in Section
\ref{subsection:c:quadraticRoots}.  By returning the two output values
via pass by reference variables, we freed up the return value.  We
can now modify our function to return an integer indicating an
error code instead.

Previously we had identified several different types of errors: division
by zero (if $a = 0$), complex roots (if $b^2 - 4ac < 0$) and a 
\mintinline{c}{NULL} pointer error if the variables passed by reference
were \mintinline{c}{NULL}.  We can now modify our function to 
check for these errors and return an appropriate error code.
We return zero in the event that no error was encountered.

\begin{minted}{c}
int quadraticRoots(double a, double b, double c, 
                   double *root1, double *root2) {
  if(a == 0) {
    return 1;
  } else if( b*b - 4*a*c < 0 ) {
    return 2;
  } else if(root1 == NULL || root2 == NULL) {
    return 3;
  }
  double discriminant = sqrt(b*b - 4*a*c);
  *root1 = (-b + discriminant) / (2*a);
  *root2 = (-b - discriminant) / (2*a);
  return 0;
}
\end{minted}

Now when a function invokes our \mintinline{c}{quadraticRoots()}
it can check to see what kind of error code it returned and
handle the error in whatever way it wants.  

There is still an issue, however.  The usage of the integers 1, 2, 3
to indicate the various errors was arbitrary.  These are essentially
\glspl{magic number} that the calling function would have to deal
with by making comparisons with various integers.  The numbers
themselves are meaningless and someone using them would 
need to constantly refer to documentation to understand which
integer corresponded to which error condition.

It would be much better if we could follow the strategy of the
\mintinline{c}{errno} and define human-readable identifiers 
for each error code.  We could accomplish this by defining
macros, but another solution is to use an \emph{enumerated
type}.

\section{Enumerated Types}

C allows you to define \glspl{enumerated type} which allow you to
define a fixed list of possible values.  For example, the days of the
week or months of the year are possible values used in a date.
However, they are more-or-less fixed (no one will be adding a new
day of the week or month any time soon).  An enumerated type
allows us to define these values using human-readable keywords
(much like the \mintinline{c}{#define} macro).  

To declare an enumerated type we use the keywords 
\mintinline{c}{typedef enum} (short for type definition
and enumeration).  We then provide a comma delimited
list of keywords inside a code block.  At the end of the
code block we provide the type's identifier.  That is, the
name of the type itself.  For example, the names of 
integer and floating-point types in C are \mintinline{c}{int} and 
\mintinline{c}{double} respectively.  This identifier gives
our type a name that we can later use to declare variables
of that type.  Consider the following example.

\begin{minted}{c}
typedef enum {
  SUNDAY,
  MONDAY,
  TUESDAY,
  WEDNESDAY,
  THURSDAY,
  FRIDAY,
  SATURDAY
} DayOfWeek;
\end{minted}

In this example we've defined an enumeration of the days 
of the week.  The name of the type itself is \mintinline{c}{DayOfWeek}
and we can now declare variables of this type.  The possible 
\emph{values} it can take are \mintinline{c}{SUNDAY}, 
\mintinline{c}{MONDAY}, etc. and we can use these keywords 
in our program.  For example, 

\begin{minted}{c}
DayOfWeek today = MONDAY;

if(today == SUNDAY || today == SATURDAY) {
  printf("Its the weekend!\n");
}
\end{minted}

Note the modern naming conventions: the type identifier uses 
upper camel casing while the enumerated values follow an 
upper case underscore convention.  Though our example does
not contain a value with multiple words, if it had, we would have
used an underscore to separate them.  Furthermore, enumerated
type declarations are usually placed in separate header files
along with function prototype declarations.  

Care must be taken when using enumerated types, however.  
Internally, C simply associates integers with the values.  Thus, 
in our example, \mintinline{c}{SUNDAY} is actually \mintinline{c}{0}, 
\mintinline{c}{MONDAY} is \mintinline{c}{1}, and 
\mintinline{c}{SATURDAY} is \mintinline{c}{6}.  When we do
assignments or equality comparisons, we're actually just
comparing integers.  

Consequently, a \mintinline{c}{DayOfWeek} variable \emph{may}
be assigned values that do not correspond to our enumeration. 
For example, 

\mintinline{c}{DayOfWeek today = 1000;}

is valid code and will not (in general) result in any compiler errors
or warnings, even though it is assigning an invalid value to the 
variable.  Care must be taken to only assign valid values to an
enumerated type variable.  Proper error checking should also
be done.

Despite this limitation, using enumerated types in C provides an
obvious advantage.  Without an enumerated type we'd be forced
to use a collection of \glspl{magic number} to indicate values.  
Even for something as simple as the days of the week we'd be
constantly trying to remember: which day is Wednesday again?
I forget, do our week start with Monday or Sunday?  Etc.  By
using an enumerated type these questions are mostly moot as
we can use the more human-readable keywords and eliminate
the guess work.

\section{Using Enumerated Types for Error Codes}

Let's apply enumerated types to our \mintinline{c}{quadraticRoots()}
function example from before.  First, we define our enumerated
type which includes all types of errors including a \mintinline{c}{NO_ERROR}
type.  Note that we start with \mintinline{c}{NO_ERROR} as its value
will be zero following our convention.

\begin{minted}{c}
typedef enum {
  NO_ERROR,
  DIV_BY_ZERO_ERROR,
  COMPLEX_ROOT_ERROR,
  NULL_POINTER_ERROR
} ErrorCode;
\end{minted}

Now in the \mintinline{c}{quadraticRoots()} function, we can
return the appropriate error code.

\begin{minted}{c}
ErrorCode quadraticRoots(double a, double b, double c, 
                   double *root1, double *root2) {
  if(a == 0) {
    return DIV_BY_ZERO_ERROR;
  } else if( b*b - 4*a*c < 0 ) {
    return COMPLEX_ROOT_ERROR;
  } else if(root1 == NULL || root2 == NULL) {
    return NULL_POINTER_ERROR;
  }
  double discriminant = sqrt(b*b - 4*a*c);
  *root1 = (-b + discriminant) / (2*a);
  *root2 = (-b - discriminant) / (2*a);
  return NO_ERROR;
}
\end{minted}
