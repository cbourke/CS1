%!TEX root = ComputerScienceOne.tex

%%Chapter: Error Handling
\index{error handling}

Writing perfect code is difficult.  The more complex a system or
code base, the more likely it is to have \index{bug} \glspl{bug}.  That is, flaws
or mistakes in a program that result in incorrect behavior or 
unintended consequences.  The term ``bug'' has been used in
engineering for quite a while.  The term was popularized in the
context of computer systems by Grace Hopper who, when working
on the Naval Mark II computer in 1946, tracked a malfunction to
a literal bug, a moth, trapped in a relay \cite{hopperBug11}.

Some of the biggest modern engineering failures can be
tracked to simple software bugs.  For example, on September 
26th, 1983 a newly installed Soviet early warning 
system gave indication that nuclear missiles had been launched 
on the Soviet Union by the United States.  Stanislav Petrov, a 
lieutenant colonel in the Soviet Air Defense Forces and duty officer 
at the time, did not trust the new system and did not report the
incident to superiors who may have ordered a counter strike.  
Petrov was correct as the false alarm was caused by sunlight
reflections off of high altitude clouds as well as other bugs in 
the newly deployed system \cite{stanislav83}.  

In September 1999 the Mars Climate Orbiter, a project intended to
study the Martian climate and atmosphere was lost after it entered
into the upper atmosphere of Mars and disintegrated.
The error was due to a subsystem that 
measured the craft's momentum in a non-standard pound force per 
second unit when all other systems expected the standard newton
second unit \cite{mars1999}.  The loss of the project was
calculated at over \$125 million.

There are numerous other examples, some that have caused
inconvenience to users (such as the Zune bug mentioned in 
Section \ref{subsection:infiniteLoops}) to bugs in medical devices
that have cost dozens of lives to those resulting in the loss of
millions of dollars \cite{wikipediaListofBugs}.

In some sense, Software Engineering and programming 
is unique.  If you build a bridge and forget one bolt its likely 
not going to cause the bridge to collapse.  If you draw up 
plans for a development and the land survey is a few inches 
off overall, its not a catastrophic problem.  However, if you 
forget one character or are off by one number in a program, 
it can cause a complete system failure.

There are a variety of reasons for why bugs make it into systems.
Bugs could be the result of a fundamental misunderstanding of
the problem or requirements.  Poor management and
the pressure of time constraints to deliver a project may make
developers more careless.  A lack of proper testing may mean
many more bugs survive the development process than otherwise
should have.  Even expert programmers can overlook a simple
mistake when writing thousands of lines of code.  

Given the potential for error, it is important to have good software development
methodologies that emphasize testing a system at all levels.
Working in teams where regular code reviews are held so that
colleagues can examine, critique, and catch potential bugs
are essential for writing robust code.

Modern coding tools and techniques can also help to improve
the robustness of code.  For example, \index{debugger}
\glspl{debugger} are 
tools that help a developer \gls{debug} (that is, find and fix
the cause of an error) a program.  Debuggers 
allow you to simulate the execution of a program statement-by-statement 
and view the current state of the program such
as variable values.  You can ``step through'' the execution
line by line to find where an error occurs in order to localize
and identify a bug.

Other tools allow you to perform \gls{static analysis} on 
source code to search for \emph{potential} problems.  
That is, problems that are not syntax errors and are not 
necessarily bugs that are causing problems, but instead 
are \glspl{anti-pattern} or \glspl{code smell}.  Anti-patterns
are essentially common bad-habits that can be found in
code.  They are an \emph{attempted} solution to a commonly
encountered problem but which don't actually solve the
problem or introduces new problems.  Code smells are 
``symptoms'' in a source code that indicate a possible
deeper design or implementation flaw.  Failure to adhere
to good programming principles such as properly initializing
variables or failure to check for null values are examples of
smells.  Static analysis tools automatically examine the
code base for potential issues like these.  For example, a
\index{linter} \gls{lint} (or linter) is a tool that can examine source code 
suspicious or non-portable code or code that does not
comply with generally accepted standards or ways of 
doing things.

Even if code contains no bugs, it is still susceptible to
errors.  For example, a program could connect to a 
remote database to \gls{query} and process data.  However, if
the network connection is temporarily unavailable, the
program will not be able to execute properly.  Because
of the potential of such errors, it is important to write
cod that is not only bug free but is also robust and resilient.  We must anticipate possible 
error conditions and write code to detect, prevent, or 
recover from such errors.  Generally, this is referred to 
as \emph{error handling}.

Much of what we now consider Software Engineering was
pioneered by people like Margaret Hamilton who was the
lead Apollo flight software designer at NASA.  During the Apollo 
11 Moon landing (1969), an error in one system 
caused the lander's computer to become overworked with
data.  However, because the system was designed with a
robust architecture, it could detect and handle such situations
by prioritizing more important tasks (those related to landing)
over lower priority tasks.  The resilience that was built into
the system is credited with its success \cite{hamiltonApollo11}.


\section{Error Handling}

In general, errors are potential conditions or situations 
that can reasonably be anticipated by a developer.  For
example, if we write code to open and process a file, 
there are several things that could go wrong.  The file
may not exist, or we may not have permission on the system to read
it, or the formatting in the file may be corrupted or not
as expected.  Still yet, everything could be fine with the
file, but it may contain erroneous or invalid values.

If an error can be anticipated, we may be able to write
code that detects the particular error condition and
\emph{handles} it by executing code that may be
able to recover from the error condition.  In the case of
a missing file for example, we may be able to prompt the 
user for an alternate file.  

We may be able to detect but not necessarily recover from
certain errors.  For example, if the file has been corrupted
in example above, there may not be a way to properly 
``fix'' it.  If it contains invalid data, we may not even want
the program to fix it as it may indicate a bug or other issue
that needs to be addressed.  Still yet, there may be some
error conditions that we cannot recover from at all as they
are completely unexpected.  In such instances, we may
want the error to result in the termination of the program
in which case the error is considered \index{fatal error}
\emph{fatal}.  

\section{Error Handling Strategies}

There are several general strategies for performing error
handling.  We'll look at two general methods here: defensive
programming and exceptions.

\subsection{Defensive Programming}
\index{defensive programming}

Defensive programming is a ``look before you leap'' strategy.
Suppose we have a potentially ``dangerous'' section of code; 
that is, a line or block of code whose execution could encounter
or result in an error condition.  Before we execute the code, we perform
a check to see if the error condition is present (usually using
a conditional statement).  If the error condition does not hold, 
then we proceed with the code as normal.  However, if the
error condition does hold, instead of executing the code, we
execute alternative code that \emph{handles} the error.

Suppose we are about to divide by a number.
To prevent a division by zero error, we can check if our
denominator is zero or not.  If it is, then we raise or handle
the error instead of performing the division.  What should be 
done in such a case?  We could, as an 
alternative, use a predefined value as a result instead.  Or 
we could notify the user and ask for an alternative.  Or we
could log the error and proceed as normal.  Or we could 
decide that the error is so egregious that it should be fatal
and terminate the execution of the program.

Which is the right way to handle this error?  It really depends on 
your design requirements really.  This raises the question, 
though: ``who'' is responsible for making these decisions?
Suppose we're designing a function for a library that is 
not just for our project but others as well (as is the
case with the standard library functions).  Further, the
function we're designing could have multiple different 
error conditions that it checks for.  In this scenario there
are two entities that could handle the errors: the function
itself and the code that invokes the function.

Suppose that we decide to handle the errors \emph{inside} the 
function.  As designers of the function, we've made
the decision to handle the errors \emph{for} the user (the
code that invokes our function).  Regardless of how we
decide to handle the errors, this design decision has
essentially taken any decision making ability away from 
users.  This is not very flexible for someone using our 
code.  If they have different design considerations or
requirements, they may need or want to handle the errors
in a different way than we did.  

Now suppose that we decide \emph{not} to handle the
errors inside our function.  Defensive programming may
still be used to prevent the execution of code that results
in an error.  However, we now need a way to \emph{communicate}
the error condition to the calling function so that it can 
know what type of error happened and handle it appropriately.

\subsubsection{Error Codes}
\index{error code}

One common pattern to communicate errors to a calling
function is to use the return type as an \emph{error code}.
Usually this is an integer type.  By convention 0 is used to
indicate ``no error'' and various other non-zero values are
used to indicate various types of errors.  Depending on 
the system and standard used, error codes may have a 
predefined value or may be specific to an application or library.

One problem with using the return type to indicate errors is 
that functions are no longer able to use the return type to 
return an actual computed value.  If a language supports
pass by reference, then this is not generally a problem.  However,
even with such languages there are situations where the
return type \emph{must} be used to return a value.  In
such cases, the function can still communicate a general
error message by returning some flag value such as null.

Alternatively, a language may support error codes by using
a shared global variable that can be set by a function to
indicate an error.  The calling function can then examine
the variable to see if an error occurred during the invocation
of the function.

\subsubsection{Limitations}

Defensive programming has its limitations.  Let's return
to the example of processing a file.  To check for all four 
of the error conditions we identified, we would need a 
series of checks similar to the following.

\begin{algorithm}[H]
\If{file does not exists}{
  return an error code \;
}
\If{we do not have permissions}{
  return an error code \;
}
\If{the file is corrupted}{
  return an error code \;
}
\If{the file contains invalid values}{
  return an error code \;
}
process file data \;
\end{algorithm}

A problem arises when an error condition is checked and
does not hold.  Then, later in the execution, circumstances change
and the error condition does hold.  However, since it was already
checked for, the program remains under the assumption that the
error condition does not hold.  For example, suppose that another
process or program deletes the file that we wish to process after 
its existence has been checked but before we start processing it.
Because of the sequential nature of our program, this type of 
error checking is susceptible to these issues.

\subsection{Exceptions}
\index{exception}
\index{error handling!exception}

An \gls{exception} is an event or occurrence of an anomalous, 
erroneous or ``exceptional'' condition that requires special 
handling.  Exceptions interrupt the normal flow of control in a
program by handing the flow of control over to \emph{exception
handlers}.

Languages usually support exception handling using a try-catch
control structure such as the following.

\begin{minted}{text}
try {
  //potentially dangerous code here  
} catch(Exception e) {
  //exception handling code here
}
\end{minted}

The \mintinline{text}{try} is used to encapsulate potentially dangerous
code, or simply code that would fail if an error condition occurs.  If an
error occurs at some point within the \mintinline{text}{try} block, control
flow is immediately transferred to the \mintinline{text}{catch} block.
The \mintinline{text}{catch} block is where you specify \emph{how}
to handle the exception.  If the code in the \mintinline{text}{try} block 
does not result in an exception, then control flow will skip over the 
\mintinline{text}{catch} statement and resume normally after.  

It is important to understand how exceptions interrupt the normal
control flow.  For example, consider the following pseudocode.

\begin{minted}{text}
try {
  statement1;
  statement2;
  statement3;
} catch(Exception e) {
  //exception handling code here
}
\end{minted}

Suppose \mintinline{text}{statement1} executes with no error but
that when \mintinline{text}{statement2} executes, it results an
exception.  Control flow is then transferred to the \mintinline{text}{catch}
block, skipping \mintinline{text}{statement3} entirely.  In 
general, there may not be a mechanism for your \mintinline{text}{catch}
block to recover and execute \mintinline{text}{statement3}.
Therefore, it may be necessary to make your \mintinline{text}{try-catch}
blocks fine-grained, perhaps having only a single statement within
the \mintinline{text}{try} statement.

Some languages only support a generic \mintinline{text}{Exception} and
the type of error may need to be communicated through other means 
such as a string error message.  Still other languages may support
many different types of exceptions and you may be able to provide
\emph{multiple} \mintinline{text}{catch} statements to handle each
one differently.  In such languages, the order in which you place
your \mintinline{text}{catch} statements may be important.  Similar
to an if-else-if statement, the first one that matches the exception
will be the one that executes.  Thus, it is best practice to order
your \mintinline{text}{catch} blocks from the most specific to the
most general. 

Some languages also support a third \mintinline{text}{finally} control
statement as in the following example.

\begin{minted}{text}
try {
  //potentially dangerous code here  
} catch(Exception e) {
  //exception handling code here
} finally {
  //unconditionally executed code here
}
\end{minted}

The \mintinline{text}{try-catch} block operates as previously 
described.  However, the \mintinline{text}{finally} block will
execute \emph{regardless} of whether or not an exception
was raised.  If no exception was raised, then the \mintinline{text}{try}
block will fully execute and the \mintinline{text}{finally} block
will execute immediately after.  If an exception was raised, 
control flow will be transferred to the \mintinline{text}{catch}
block.  After the \mintinline{text}{catch} block has executed, the
\mintinline{text}{finally} block will execute.

\mintinline{text}{finally} blocks are generally used to handle
resources that need to be ``cleaned up'' whether or not an
exception occurs.  For example, opening a connection to a
database to retrieve and process data.  Whether or not
an exception occurs during this process the connection will
need to be properly closed as it represents a substantial 
amount of resources (a network connection, memory and
processing time on both the server and client machines, 
etc.).  Failure to properly close the connection may result
in wasted resources.  By placing the clean up code inside
a \mintinline{text}{finally} statement, we can be assured that
it will execute regardless of an error or exception.

In addition to handling exceptions, a language may allow you
to ``throw'' your own exceptions by using the keyword 
\mintinline{text}{throw}.
In this way you can also practice defensive programming.
You could write a conditional statement to check for an error
condition and then \mintinline{text}{throw} an appropriate exception.



