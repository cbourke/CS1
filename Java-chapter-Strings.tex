%!TEX root = ComputerScienceOne.tex

%%Chapter: Strings in Java

As we've previously seen, Java has a \mintinline{java}{String} 
class in the standard \gls{jdkLabel}.  Internally, Java 
strings are stored as arrays of characters.  However, because
of the \mintinline{java}{String} class, we never directly 
interact with this representation, making using strings
much easier than in other languages.  Java strings have
also supported \gls{Unicode} since version 1.

Moreover, Java provides many methods as part of the 
\mintinline{java}{String} class that can be used to process
and manipulate strings.  These methods \emph{do not} change
the strings since strings in Java are \gls{immutable}.  
Instead, these methods operate by returning a \emph{new}
modified string that can then be stored in a variable.

\section{Basics}

As we've previously seen, we can declare \mintinline{java}{String}
variables and assign them values using the regular
assignment operator.

\begin{minted}{java}
String firstName = "Thomas";
String lastName = "Waits";

//we can also reassign values
firstName = "Tom";
\end{minted}

Note that the reassignment in the last line in the example
does \emph{not} change the original string.  It just makes
the variable \mintinline{java}{firstName} point to a new
string.  If there are no other references to the old string,
it becomes eligible for garbage collection and the \gls{jvmLabel}
takes care of it.

Strings can be \emph{copied} using the \mintinline{java}{String}
class's copy constructor: 

\mintinline{java}{String copy = new String(firstName);}

However, since strings are immutable, there is rarely reason
to create such a \gls{deep copy}.

Though you can't change individual characters in a string, 
you can access them using the \mintinline{java}{charAt()}
method and providing an index.  Characters in a string are
0-indexed just as with elements in arrays. 

\begin{minted}{java}
String fullName = "Tom Waits";
//access individual characters:
char firstInitial = fullName.charAt(0); //'T'
char lastInitial = fullName.charAt(4); //'W'

if(fullName.charAt(8) == 's') {
  ...
}
\end{minted}

\section{String Methods}

The \mintinline{java}{String} class provides dozens of 
convenient methods that allow you to process and modify strings.
We highlight a few of the more common ones here.  A full
list of supported functions can be found in standard 
documentation.

\subsubsection{Length}

When accessing individual characters in a string, it
is necessary that we know the length of the string so
that we do not access invalid characters.  The 
\mintinline{java}{length()} method returns an \mintinline{java}{int}
that represents the number of characters in the
string.  

\begin{minted}{java}
String s = "Hello World!";
char last = s.charAt(s.length()-1);
//last is '!'
\end{minted}

Using this method we can easily iterate over each 
character in a string.

\begin{minted}{java}
for(int i=0; i<fullName.length(); i++) {
  System.out.printf("fullName[%d] = %c\n", i, fullName.charAt(i));
}

//or we can use toCharArray and an enhanced for-loop:
for(char c : fullName.toCharArray()) {
  System.out.println(c);
}
\end{minted}

\subsubsection{Concatenation}

Java has a concatenation operator built into the language.
The familiar plus sign, \mintinline{java}{+} can be used
to combine one or more strings by appending them to each
other.  Concatenation results in a new string.

\begin{minted}{java}
String firstName = "Tom";
String lastName = "Waits";

String formattedName = lastName + ", " + firstName;
//formattedName now contains "Waits, Tom"
\end{minted}

Concatenation also works with numeric types and even
objects.

\begin{minted}{java}
int x = 10;
double y = 3.14;

String s = "Hello, x is " + x + " and y = " + y;
//s contains "Hello, x is 10 and y = 3.14"
\end{minted}

When used with objects, the concatenation operator 
ends up invoking the object's \mintinline{java}{toString()}
method.  The plus operator as a concatenation operator 
is actually \gls{syntactic sugar}.  When the code is compiled, 
it is actually replaced with equivalent code that uses 
a series of the \mintinline{java}{StringBuilder} class's 
(a mutable version of the \mintinline{java}{String} class) 
\mintinline{java}{append()} method.  The first example
above may actually be replaced with the following code 
that does not use the concatenation operator.

\begin{minted}{java}
String firstName = "Tom";
String lastName = "Waits";

String formattedName = new StringBuilder(lastName)
                       .append(", ")
                       .append(firstName)
                       .toString();
\end{minted}

You can also use the \mintinline{java}{StringBuilder} class
yourself directly.

\subsubsection{Computing a Substring}

There are two methods that allow you to compute
a substring of a string.  The first allows you to
specify a beginning index with the entire remainder
of the string being included in the returned string.
The second allows you to specify a beginning index
as well as an ending index.  In both cases, the
beginning index is \emph{inclusive} (that is the resulting
string includes the character at that index), but
in the second, the ending index is \emph{exclusive}
(it is not included).

\begin{minted}{java}
String name = "Thomas Alan Waits";

String firstName  = name.substring(0, 6);  //"Thomas"
String middleName = name.substring(7, 11); //"Alan"
String lastName   = name.substring(12);    //"Waits"
\end{minted}

The result of the two argument \mintinline{java}{substring()}
method will always have length equal to \mintinline{java}{endIndex - beginIndex}.

\section{Arrays of Strings}

We often need to deal with collections of strings.  In 
Java, we can define arrays of strings.  Indeed, we've seen 
arrays of strings before.  In a Java class's 
\mintinline{c}{main()} method, 
command line arguments are passed as an array of strings:

\mintinline{java}{public static void main(String args[])}

We can create our own arrays of strings similar to
how we created arrays of \mintinline{java}{int}
and \mintinline{java}{double} types.

\begin{minted}{java}
//create an array that can hold 5 strings
String names[] = new String[5];

names[0] = "Margaret Hamilton";
names[1] = "Ada Lovelace";
names[2] = "Grace Hopper";
names[3] = "Marie Curie";
names[4] = "Hedy Lamarr";
\end{minted}

Better yet, we can use dynamic collections such as
a \mintinline{java}{List} or a \mintinline{java}{Set} 
of strings.  

\begin{minted}{java}
List<String> names = new ArrayList<String>();

names.add("Margaret Hamilton");
names.add("Ada Lovelace");
names.add("Grace Hopper");
names.add("Marie Curie");
names.add("Hedy Lamarr");

System.out.println(names.get(2));  //prints "Grace Hopper"
\end{minted}

\section{Comparisons}

When comparing strings in Java, we cannot use the 
numerical comparison operators such as \mintinline{java}{==}, or
\mintinline{java}{<}.  Because strings are represented as
arrays, using these operators actually compares the
variables' \emph{memory addresses}.

\begin{minted}{java}
String a = new String("Hello World!");
String b = new String("Hello World!");

if(a == b) {
  System.out.println("strings match!");
}
\end{minted}

The code above will not print anything even though
the strings \mintinline{java}{a} and \mintinline{java}{b}
have the same content.  This is because \mintinline{java}{a == b}
is comparing the memory address of the two variables.
Since they point to different memory addresses 
(created by two separate calls to the constructors)
they are not equal.

Instead, there are several \gls{comparator} methods that 
Java provides to compare strings.  Each string has a
\mintinline{java}{compareTo()} method\footnote{This method
is part of the \mintinline{java}{String} class due to the
fact that strings implement the \mintinline{java}{Comparable}
interface which defines a lexicographic ordering.} that takes another
string and returns something negative, zero, or something
positive depending on the relative lexicographic ordering 
of the strings.

\begin{minted}{java}
int x;
String a = "apple";
String b = "zelda";
String c = "Hello";
x = a.compareTo("banana"); //x is negative
x = b.compareTo("mario");  //x is positive
x = c.compareTo("Hello");  //x is zero

//shorter strings precede longer strings:
x = a.compareTo("apples"); //x is negative

String d = "Apple";
x = d.compareTo("apple");  //x is negative
\end{minted}

In the last example, \mintinline{java}{"Apple"} precedes
\mintinline{java}{"apple"} since uppercase letters are
ordered before lowercase letters according to the
\gls{asciiLabel} table.  We can also make comparisons
ignoring case if we need to using \mintinline{java}{compareToIgnoreCase()} method which works the same way.  This is a 
case-insensitive version of the method.  Here, 
\mintinline{java}{d.compareToIgnoreCase("apple")} will return 
zero as the two strings are the same ignoring the cases.

Note that the commonly used \mintinline{java}{equals()} method
only returns \mintinline{java}{true} or \mintinline{java}{false}
depending on whether or not two strings are the same or 
different.  They cannot be used to provide a relative
ordering of two strings.

\begin{minted}{java}
String a = "apple";
String b = "apple";
String c = "Hello";

boolean result;
result = a.equals(b); //true
result = a.equals(c); //false
\end{minted}

\section{Tokenizing}

Recall that \emph{tokenizing} is the process of splitting
up a string along some \emph{delimiter}.  For example, 
the comma delimited string, \mintinline{java}{"Smith,Joe,12345678,1985-09-08"}
contains four pieces of data delimited by a comma.  
Our aim is to split this string up into four separate 
strings so that we can process each one.

Java provides a very simple method to do this called
\mintinline{java}{split()} that takes a string delimiter
and returns an array of strings containing the tokens.
For example, 

\begin{minted}{java}
String data = "Smith,Joe,12345678,1985-09-08";

String tokens[] = data.split(",");
//tokens is { "Smith", "Joe", "12345678", "1985-09-08" }

String dateTokens[] = tokens[3].split("-");
//dateTokens is now { "1985", "09", "08" }
\end{minted}

The delimiter this method uses is actually a 
\gls{regular expression}; a sequence of characters
that define a search \emph{pattern} in which
special characters can be used to define complex
patterns.  For example, the complex expression
\mintinline{text}{^[+-]?(\d+(\.\d+)?|\.\d+)([eE][+-]?\d+)?$}
will match any valid numerical value including
scientific notation.  We will not cover regular 
expressions in depth, but to demonstrate their 
usefulness, here's an example by which you can 
split a string along any and all whitespace:

\begin{minted}{java}
String s = "Alpha Beta \t Gamma \n Delta    \t\nEpsilon";
String tokens[] = s.split("[\\s]+");
//tokens is now { "Alpha", "Beta", "Gamma", "Delta", "Epsilon" }
\end{minted}

