%!TEX root = ComputerScienceOne.tex

%%Chapter: Strings in PHP
\index{strings!in PHP}
\index{PHP!strings}

As we've previously seen, PHP has a built-in string type.
Internally, PHP strings are simply a sequence of bytes, 
but for our purposes we can treat it as a 0-indexed 
character array.  PHP strings are mutable and can be
changed, but it is considered best practice to treat
them as mutable and rely on the many functions PHP 
provides to manipulate strings.

\section{Basics}

As we've previously seen, we can create strings by
simply assigning a string literal value to a variable
as PHP is a dynamically typed language.  Strings
can be specified by either single quotes or double
quotes (there are no individual characters in PHP, 
only single character strings), but we will mostly
use the double quote syntax.

\begin{minted}{php}
$firstName = "Thomas";
$lastName = "Waits";

//we can also reassign values
$firstName = "Tom";
\end{minted}

The reassignment in the last line in the example
effectively destroys the old string.  The assignment
operator can also be used to make copies of strings

\begin{minted}{php}
$firstName = "Thomas";
$alias = $firstName;
\end{minted}

It is important to understand that this assignment 
essentially makes a \gls{deep copy} of the string.
Changes to the first do not affect the second one.

You can make changes to individual characters in a
string by treating it like a zero-indexed array.

\begin{minted}{php}
$a = "hello";
$a[0] = "H";
$a[5] = "!";
//a is now "Hello!"
\end{minted}

Note that the last line extends the string by adding
an additional character.  You can even remove characters
by setting them to the empty string.

\begin{minted}{php}
$a = "Apples!";
$a[5] = "";
//a is now "Apple!"
\end{minted}

\section{String Functions}

PHP provides dozens of convenient functions that allow 
you to process and modify strings.  We highlight a few 
of the more common ones here.  A full list of supported 
functions can be found in standard documentation.  Because
of history of PHP, many of the same functions defined in
the C string library can also be used in PHP.

\subsubsection{Length}

When accessing individual characters in a string, it
is necessary that we know the length of the string so
that we do not access invalid characters (though doing
so is not an error, it just results in \mintinline{php}{NULL}).
The \mintinline{php}{strlen()} function returns an
integer that represents the number of characters in the
string.  

\begin{minted}{php}
$s = "Hello World!";
$x = strlen($s); //x is 12
$s = "";
$x = strlen($s); //x is 0

//careful:
$s = NULL
$x = strlen($s); //x is 0
\end{minted}

As demonstrated in the last example, \mintinline{php}{strlen()}
will return 0 even for \mintinline{php}{NULL} strings.
Recall that we can distinguish between these two situations
we can use \mintinline{php}{is_null()}.  Using this function
we can easily iterate over each individual character in a string.

\begin{minted}{php}
$fullName = "Tom Waits";
for($i=0; $i<strlen($fullName); $i++) {
  printf("fullName[%d] = %s\n", $i, $fullName[$i]);
}
\end{minted}

This would print the following

\begin{minted}[fontsize=\scriptsize]{text}
fullName[0] = T
fullName[1] = o
fullName[2] = m
fullName[3] =  
fullName[4] = W
fullName[5] = a
fullName[6] = i
fullName[7] = t
fullName[8] = s
\end{minted}

\subsubsection{Concatenation}

PHP has a concatenation operator built into the language.
To concatenate one or more strings together, you can use a
simple period between them as the concatenation operator.
Concatenation results in a new string.

\begin{minted}{php}
$firstName = "Tom";
$lastName = "Waits";

$formattedName = lastName . ", " . firstName;
//formattedName now contains "Waits, Tom"
\end{minted}

Concatenation also works with other variable types.

\begin{minted}{php}
$x = 10;
$y = 3.14;

$s = "Hello, x is " . $x . " and y = " . $y;
//s contains "Hello, x is 10 and y = 3.14"
\end{minted}

\subsubsection{Computing a Substring}

PHP provides a simple function, \mintinline{php}{substr()} 
to compute a substring of a string.  It takes at at 
least 2 arguments: the string to operate on
and the starting index.  There is a third, 
optional parameter that allows you to 
specify the length of the resulting substring.

\begin{minted}{php}
$name = "Thomas Alan Waits";

$firstName  = substr($name, 0, 6); //"Thomas"
$middleName = substr($name, 7, 4); //"Alan"
$lastName   = substr($name, 12);   //"Waits"
\end{minted}

As in the final example, omitting the optional
length parameter results in the entire remainder
of the string being returned as the substring.

\section{Arrays of Strings}

We often need to deal with collections of strings.  
In PHP we can define arrays of strings.  Indeed, we've seen 
arrays of strings before.  When processing command line
arguments, PHP defines an array of strings, 
\mintinline{php}{$argv}.  Each string can be accessed
using an index, \mintinline{php}{$argv[0]} for example
is always the name of the script.

We can create our own arrays of strings using the
same syntax as with other arrays.

\begin{minted}{php}
$names = array(
  "Margaret Hamilton",
  "Ada Lovelace",
  "Grace Hopper",
  "Marie Curie",
  "Hedy Lamarr");
\end{minted}

\section{Comparisons}

When comparing strings in PHP, we can use the usual
numerical operators such as \mintinline{php}{===}, 
\mintinline{php}{<}, or \mintinline{php}{<=} which
will compare the strings lexicographically.  However,
this is generally discouraged because of type juggling
issues and strict vs loose equality/inequality comparisons.

Instead, there are several \gls{comparator} methods that 
PHP provides to compare strings based on their content.
\mintinline{php}{strcmp($a, $b)} takes two strings and returns 
an integer based on the lexicographic ordering. of 
\mintinline{php}{$a} and \mintinline{php}{$b}.  If 
\mintinline{php}{$a} precedes \mintinline{php}{$b}, 
\mintinline{php}{strcmp()} returns something negative.  It
returns zero if \mintinline{php}{$a} and \mintinline{php}{$b} 
have the same content.  Otherwise it returns something
positive if \mintinline{php}{$b} precedes \mintinline{php}{$a}.

Some examples:

\begin{minted}{php}
$x = strcmp("apple", "banana"); //x is negative
$x = strcmp("zelda", "mario");  //x is positive
$x = strcmp("Hello", "Hello");  //x is zero

//shorter strings precede longer strings:
$x = strcmp("apple", "apples"); //x is negative

$x = strcmp("Apple", "apple");  //x is negative
\end{minted}

In the last example, \mintinline{php}{"Apple"} precedes
\mintinline{php}{"apple"} since uppercase letters are
ordered before lowercase letters according to the
\gls{asciiLabel} table.  We can also make comparisons
ignoring case if we need to using the alternative, 
\mintinline{php}{strcasecmp($a, $b)}, a case-insensitive 
version.  Here, \mintinline{php}{strcasecmp("Apple", "apple")} 
will return zero as the two strings are the same 
ignoring the cases.

The comparison functions also have length-limited versions, 
\mintinline{php}{strncmp($a, $b, $n)} and
\mintinline{php}{strncasecmp($a, $b, $n)}.  Both will only 
make comparisons in the first \mintinline{php}{$n}
characters of the strings.  Thus, \mintinline{php}{strncmp("apple", "apples", 5)} will result in zero as the two strings are equal in the first
5 characters.

\section{Tokenizing}

Recall that \emph{tokenizing} is the process of splitting
up a string along some \emph{delimiter}.  For example, 
the comma delimited string, \mintinline{php}{"Smith,Joe,12345678,1985-09-08"}
contains four pieces of data delimited by a comma.  
Our aim is to split this string up into four separate 
strings so that we can process each one.

PHP provides several functions to to this, 
\mintinline{php}{explode()} and \mintinline{php}{preg_split()}.

The simpler one, \mintinline{php}{explode()} takes
two arguments: the first one is a string delimiter and
the second is the string to be processed.  It then
returns an array of strings.

\begin{minted}{php}
$data = "Smith,Joe,12345678,1985-09-08";

$tokens = explode(",", $data);
//tokens is [ "Smith", "Joe", "12345678", "1985-09-08" ]

$dateTokens = explode("-", $tokens[3]);
//dateTokens is now [ "1985", "09", "08" ]
\end{minted}

The more sophisticated one, \mintinline{php}{preg_split()}
also takes two arguments\footnote{The ``preg'' stands
for \textbf{P}erl Compatible \textbf{Reg}ular Expression.}
, but instead of a simple
delimiter, it actually uses a \gls{regular expression}; 
a sequence of characters that define a search 
\emph{pattern} in which special characters can 
be used to define complex patterns.  For example, 
the complex expression
\mintinline{text}{^[+-]?(\d+(\.\d+)?|\.\d+)([eE][+-]?\d+)?$}
will match any valid numerical value including
scientific notation.  We will not cover regular 
expressions in depth, but to demonstrate their 
usefulness, here's an example by which you can 
split a string along any and all whitespace:

\begin{minted}{php}
$s = "Alpha Beta \t Gamma \n Delta    \t\nEpsilon";
$tokens = preg_split("/[\s]+/", $s);
//tokens is now [ "Alpha", "Beta", "Gamma", "Delta", "Epsilon" ]
\end{minted}
