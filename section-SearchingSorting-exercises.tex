%!TEX root = ComputerScienceOne.tex

%Searching & Sorting - exercises

\section{Exercises}

\begin{exer}
Give an input example input demonstrating that the Quick Sort algorithm, as 
presented, is unstable.  Run through the algorithm to demonstrate how it 
results in an unstable sort.
\end{exer}

%\begin{solution}
%There are many possibilities, for example:
%  $$5, 7, 10, 2_a, 2_b$$
%\end{solution}

\begin{exer}
Implement the sorting algorithms described in this chapter in the programming
language of your choice.  Then setup experiments: run each on randomly 
ordered collections many times to get an average run time.  Graph these
empirical results and see if they match the theoretical results of
our analysis.  Be sure to also include any built-in sorting functionality
your language provides as a benchmark.
\end{exer}

\begin{exer}
Ternary search is similar to binary search in that it searches a sorted
array.  However, instead of splitting the array in two, it splits it into
three partitions and makes two comparisons to determine which third the
element lies in (if it is in the collection).  Implement ternary search and 
benchmark it against a binary search implementation.
\end{exer}

\begin{exer}
Data has a ``natural'' ordering: numbers are ordered in nondecreasing 
order, strings are ordered lexicographically (according to the ASCII text 
table).  However, in many situations, the natural ordering isn't the 
\emph{expected} ordering.  For example, class years are usually ordered Freshman, Sophomore, Junior, Senior whereas the natural ordering 
would order them Freshman, Junior, Senior, Sophomore.  

Write a program to sort a collection of strings according to 
an arbitrary artificial ordering.  That is, instead of the A--Z 
alphabetic ordering, we will order them according to an alternative 
ordering of the English alphabet.  As an example, one possible
ordering would be:

\mintinline{text}{n e d c r h a l g k m z f w j o b v x q y i p u s t}

Your job will be to \emph{sort} a list of English language words 
using this artificial ordering of the English alphabet.  Specifically, 
you will read in an input file in the following format.  The first line 
is the new ordering of English letters, each separated by a space.  
Each line after that contains a single string.  
You will read in this file, process it and produce a reordered list 
of the words sorted according to the new ordering.  

An example input:

\begin{minted}{text}
n e d c r h a l g k m z f w j o b v x q y i p u s t
vcawufotrb
laencfuesw
gvtkwekfom
vrsfqictqc
wmcvmjmtet
qetegyqelu
newaxdtjlt
nfrfrwkknj
fzqrvgblov
gkkmgwwwpa
\end{minted}

The result ordering: 

\begin{minted}{text}
Artificial Ordering: 
newaxdtjlt
nfrfrwkknj
laencfuesw
gkkmgwwwpa
gvtkwekfom
fzqrvgblov
wmcvmjmtet
vcawufotrb
vrsfqictqc
qetegyqelu
\end{minted}

For simplicity, you can assume that all words will be lower 
case and no non-alphabetic characters are used.  However, 
you may not assume that all words will be the same length. 
Words of a shorter length that are a prefix of another word 
should be ordered first.  For example, ``newax'' should come 
before ``newaxn'' in the ordering above.
\end{exer}
