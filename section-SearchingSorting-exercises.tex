%!TEX root = ComputerScienceOne.tex

%Searching & Sorting - exercises

\section{Exercises}

\begin{exer}
Give an input example input demonstrating that the Quick Sort algorithm, as 
presented, is unstable.  Run through the algorithm to demonstrate how it 
results in an unstable sort.
\end{exer}

%\begin{solution}
%There are many possibilities, for example:
%  $$5, 7, 10, 2_a, 2_b$$
%\end{solution}

\begin{exer}
Implement the sorting algorithms described in this chapter in the programming
language of your choice.  Then setup experiments: run each on randomly 
ordered collections many times to get an average run time.  Graph these
empirical results and see if they match the theoretical results of
our analysis.  Be sure to also include any built-in sorting functionality
your language provides as a benchmark.
\end{exer}

\begin{exer}
Ternary search is similar to binary search in that it searches a sorted
array.  However, instead of splitting the array in two, it splits it into
three partitions and makes two comparisons to determine which third the
element lies in (if it is in the collection).  Implement ternary search and 
benchmark it against a binary search implementation.
\end{exer}

\begin{exer}
TODO
\end{exer}
