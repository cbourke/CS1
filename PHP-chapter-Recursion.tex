%!TEX root = ComputerScienceOne.tex

%%Chapter: Recursion in PHP

PHP supports recursion with no special syntax.  However,
recursion is generally expensive and iterative or
other non-recursive solutions are generally preferred.  We present
a few examples to demonstrate how to write recursive functions in PHP.

The first example of a recursive function we gave was the toy count down
example.  In PHP it could be implemented as follows.

\begin{minted}{php}
function countDown($n) {
  if($n===0) {
    printf("Happy New Year!\n");
  } else {
    printf("%d\n", $n);
    countDown($n-1);
  }
}
\end{minted}

As another example that actually does something useful, consider the
following recursive summation function that takes an array, its size
and an index variable.  The recursion works as follows: if the index
variable has reached the size of the array, it stops and returns zero
(the base case).  Otherwise, it makes a recursive call to 
\mintinline{php}{recSum()}, incrementing the index variable by 1.  When
the function returns, it adds its result to the $i$-th element
in the array.  To invoke this function we would call it with an initial
value of 0 for the index variable: \mintinline{php}{recSum($arr, 0)}.

\begin{minted}{php}
function recSum($arr, $i) {
  if($i === count($arr)) {
    return 0;
  } else {
    return recSum($arr, $i+1) + $arr[$i];
  }
}
\end{minted}

This example was not tail-recursive as the recursive call was not the
final operation (the sum was the final operation).  To make this function
tail recursive, we can carry the summation through to each function call
ensuring that the summation is done prior to the recursive function call.

\begin{minted}{php}
function recSumTail($arr, $i, $sum) {
  if($i === count($arr)) {
    return $sum;
  } else {
    return recSumTail($arr, $i+1, $sum + $arr[$i]);
  }
}
\end{minted}

As a final example, consider the following PHP implementation of the 
naive recursive Fibonacci sequence.  An additional condition has been
included to check for ``invalid'' negative values of $n$ for which
we throw an exception.

\begin{minted}{php}
function fibonacci($n) {
  if($n < 0) {
    throw new Exception("Undefined for n<0.");
  } else if($n <= 1) {
    return 1;
  } else {
    return fibonacci($n-1) + fibonacci($n-2);
  }
}
\end{minted}

PHP is not a language that provides implicit memoization.  Instead, 
we need to explicitly keep track of values using a table.  In the
following example, the table is passed through as an argument.

\begin{minted}{php}
function fibonacciMemoization($n, $table) {
  if($n < 0) {
    return 0;
  } else if($n <= 1) {
    return 1;
  } else if(isset($table[$n])) {
    return $table[$n];
  } else {
    $a = fibonacciMemoization($n-1, $table);
    $b = fibonacciMemoization($n-2, $table);
    $result = ($a + $b);
    $table[$n] = $result;
    return $result;
  }
}
\end{minted}


