%!TEX root = ComputerScienceOne.tex

%%Chapter: Searching & Sorting in PHP

PHP provides over a dozen different sorting functions each with 
different properties and behavior.  Recall that PHP has associative arrays
which store elements as key-value pairs.  Some functions sort by keys,
others sort by the value (with some in ascending order, others in 
descending order).  Some functions preserve the key-value mapping
and others do not.  For simplicity, we will focus on two of the most 
common sorting functions, 
\mintinline{php}{sort()}, which sorts the elements in ascending order
and \mintinline{php}{usort()} which sorts according to a comparator
function.

\section{Comparator Functions}

Consider a ``generic'' Quick Sort algorithm as was presented in
Algorithm \ref{algo:quickSort}.  The algorithm itself specifies how to 
sort elements, but it doesn't specify how they are \emph{ordered}.  
The difference is subtle but important.  Quick Sort needs 
to know when two elements, $a, b$ are in order, out of order, or equivalent 
in order to decide which partition each element goes in.  However, it 
doesn't ``know'' anything about the elements $a$ and $b$ themselves.  
They could be numbers, they could be strings, they could be user-defined 
objects.  

A sorting algorithm still needs to be able to determine the proper ordering 
in order to sort.  In PHP this is achieved through a \emph{comparator function}, 
which is a function that is responsible for comparing two elements and
determining their proper order.  A comparator function has the following 
signature and behavior.

\begin{itemize}
  \item The function takes two elements \mintinline{php}{$a} and 
    \mintinline{php}{$b} to be compared.  
  \item The function returns an integer indicating the relative ordering of
  	the two elements:
	\begin{itemize}
	  \item It returns something negative if \mintinline{php}{$a}
	  comes before \mintinline{php}{$b} (that is, $a < b$)
	  \item It returns zero if \mintinline{php}{$a} and \mintinline{php}{$b} are
	  equal ($a = b$)
	  \item It returns something positive if \mintinline{php}{$a}
	  comes after \mintinline{php}{$b} (that is, $a > b$)
	\end{itemize}
\end{itemize}

There is no guarantee on the value's magnitude, it does \emph{not}
necessarily return $-1$ or $+1$; it just returns \emph{something} negative or
positive.  We've previously seen this pattern when comparing strings.  The
standard PHP library provides a function, \mintinline{php}{strcmp($a, $b)} that
has the same contract: it takes two strings and returns something
negative, zero or something positive depending on the lexicographic ordering
of the two strings.  

The PHP language ``knows'' how to compare built-in types like numbers
and strings.  However, to generalize the comparison operation, 
we can define comparator functions that encapsulate more complex logic.
As a simple first example, let's write a comparator function that orders 
numbers in ascending order.  

\begin{minted}{php}
function cmpInt($a, $b) {
  if($a < $b) {
    return -1;
  } else if($a === $b) {
    return 0;
  } else {
    return -1;
  }
}
\end{minted}

What if we wanted to order integers in the opposite order?  We could write
another comparator in which the comparisons or values are reversed.  Even
simpler, we could reuse the comparator above and ``flip'' the sign by 
multiplying by $-1$ (this is one of the purposes of writing functions:
code reuse).  Even simpler still, we could just flip the arguments we pass to 
\mintinline{php}{cmpInt()} to reverse the order.

\begin{minted}{php}
function cmpIntDesc($a, $b) {
  return cmpInt($b, $a);
}
\end{minted}

To illustrate some more examples, consider the \mintinline{php}{Student} class
we defined in Code Sample \ref{code:php:studentClass}.  The following code
samples demonstrate various ways of ordering \mintinline{php}{Student} 
instances based on one or more of their components.

\begin{minted}{php}
/**
 * A comparator function to order Student instances by 
 * last name/first name in alphabetic order
 */
function studentByNameCmp($a, $b) {
  int result = strcmp($a->getLastName(), $b->getLastName());
  if(result == 0) {
    return strcmp($a->getFirstName(), $b->getFirstName());
  } else {
    return result;
  }
}
\end{minted}

\begin{minted}{php}
/**
 * A comparator function to order Student instances by 
 * last name/first name in reverse alphabetic order
 */
function studentByNameCmpDesc($a, $b) {
  return studentByNameCmp($b, $a);
}
\end{minted}

\begin{minted}{php}
/**
 * A comparator function to order Student instances by 
 * id in ascending numerical order
 */
function studentIdCmp($a, $b) {
  if($a->getId() < $b->getId()) {
    return -1;
  } else if($a->getId() === $b->getId()) {
    return 0;
  } else {
    return 1;
  }
}
\end{minted}

\begin{minted}{php}
/**
 * A comparator function to order Student instances by 
 * GPA in descending order
 */
function studentGpaCmp($a, $b) {
  if($a->getGpa() > $b->getGpa()) {
    return -1;
  } else if($a->getGpa() == $b->getGpa()) {
    return 0;
  } else {
    return 1;
  }
}
\end{minted}

\subsection{Searching}

PHP provides a linear search function, \mintinline{php}{array_search()} that
can be used to search for an element in an array.  The array can be specified
to use loose comparisons (default) or strict comparisons.  It returns the
key (i.e.\ index) of the first matching element it finds and 
\mintinline{php}{false} if the search was unsuccessful.  For example:

\begin{minted}{php}
$arr = array(10, 8, 3, 12, 4, 42, 7, 108);
$index = array_search(12, $arr); //index is now 3

$arr = array("hello", 10, "mixed", 12, "20");
//a loose search:
$index = array_search(20, $arr); //index is now 4
//a strict search:
$index = array_search(20, $arr, false); //index is now false.
\end{minted}

PHP does not provide a standard binary search function.  Though you can
write your own binary search implementation, likely the reason that that
PHP does not provide one is because one is not needed.  The purpose of
binary search is to search a sorted array efficiently.  However, PHP
arrays are not usual arrays: they are associative arrays, essentially
key-value maps.  Retrieving an element via its key is essentially a 
constant-time operation, even more efficient that binary search.  A better
solution may be to simply store the elements using a proper key which
can be used to retrieve the element later on.  

\subsection{Sorting}

PHP's \mintinline{php}{sort()} function can be used to sort elements
in ascending order.  This is useful if you have arrays of numbers or
strings, but doesn't work very well if you have an array of mixed types
or objects.  

\begin{minted}{php}
$arr = array("banana", "Apple", "zebra", "apple", "bananas");
sort($arr);
//arr is now ("Apple", "apple", "banana", "bananas", "zebra")

$arr = array(10, 8, 3, 12, 4, 42, 7, 108);
sort($arr);
//arr is now (3, 4, 7, 8, 10, 12, 42, 108)
\end{minted}

PHP provides a more versatile sorting function, \mintinline{php}{usort()}
(\textbf{u}ser defined \textbf{sort}) that accepts a comparator function that
it uses to define the ordering of elements.  To pass a comparator function
to the \mintinline{php}{usort()} function, we pass a string value containing
the name of the comparator function we wish to use.  Recall that function
names in PHP are case insensitive, though it is still best practice to
match the naming. Several examples of the usage 
of this function are presented in Code Sample \ref{code:php:sortExamples}.

\begin{listing}[H]
\begin{minted}{php}
$arr = array(10, 8, 3, 12, 4, 42, 7, 108);
usort($arr, "cmpIntDesc");
//arr is now 108, 42, 12, 10, 8, 7, 4, 3

//roster is an array of Student instances
$roster = ...

//sort by name:
usort($roster, "studentByNameCmp");

//sort by ID:
usort($roster, "studentIdCmp");

//sort by GPA:
usort($roster, "studentGpaCmp");
\end{minted}
\caption{Using PHP's \mintinline{php}{usort()} Function}
\label{code:php:sortExamples}
\end{listing}


