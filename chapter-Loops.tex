%!TEX root = ComputerScienceOne.tex

%%Chapter: Loops

Computers are really good at automation.  A key aspect of automation is 
the ability to repeat a process over and over on different pieces of data until some condition
is met.  For example, if we have a collection of numbers and we want to find their
sum we would \emph{iterate} over each number, adding it to a total, until we have
examined every number.  Another example may include sending an email message
to each student in a course.  To automate the process, we could iterate over each
student record and \emph{for each} student we would generate and send the email.

Automated repetition is where \emph{loops} come in handy.  Computers are perfectly suited 
for performing such repetitive tasks.  We can write a single block of code that performs
some action or processes a single piece of data, then we can write a loop around that
block of code to execute it a number of times.

Loops provide a much better alternative than repeating (cut-paste-cut-paste) the same
code over and over with different variables.  Indeed, we wouldn't even do this in real life.
Suppose that you took a 100 mile trip.  How would you describe it?  Likely,
you wouldn't say, ``I drove a mile, then I drove a mile, then I drove a mile, $\ldots$'' repeated
100 times.  Instead, you would simply state ``I drove 100 miles'' or maybe even, ``I drove
until I reached my destination.''  

Loops allow us to write concise, repeatable code that can be applied to each element in 
a collection or perform a task over and over again until some condition is met.  When 
writing a loop, there are three essential components:

\begin{itemize}
  \item An \emph{initialization} statement that specifies how the loop begins
  \item A \emph{continuation} (or \emph{termination}) condition that specifies whether the loop 
  	should continue to execute or terminate
  \item An \emph{iteration} statement that makes progress toward the termination condition
\end{itemize}

The initialization statement is executed before the loop begins and serves as a way
to set the loop up.  Typically, the initialization statement involves setting the initial value 
of some variable.

\begin{figure}
\centering
\begin{tikzpicture}[node distance=3cm,auto,scale=1.0,transform shape]
    % Place nodes
    
%    \tikzstyle{decision} = [diamond, draw, fill=blue!20, 
%    text width=4.5em, text badly centered, node distance=3cm, inner sep=0pt]
%\tikzstyle{block} = [rectangle, draw, fill=blue!20, 
%    text width=5em, text centered, rounded corners, minimum height=4em]
%\tikzstyle{line} = [draw, -latex']

    \node [block,text width=8em] (a) {Initialization: $i \leftarrow 1$};
    \node [decision,node distance=3.5cm,text width=6em,text centered, below of=a] (decide) {Continuation: $i \leq 10$?};
    \node [block,text width=8em, fill=green!20, below of=decide] (body) {loop body};
    \node [block,text width=8em, below of=body] (c) {Iteration: $i \leftarrow (i + 1)$};
    \node[block,node distance=3.5cm,text width=8em,below of=c] (d) {remaining program};
    
    %position dummy nodes
    %\node[right of=b] (null1) {};
    %\node[right of=decide] (null2) {};
    
    % Draw edges
    \path [line] (a) -- (decide);
    \path [line] (body) -- (c);
    %\path [line] (c) -- (d);
    \path [line] (decide) -- node {true} (body);
    \path [line] (c)   --++  (-3,0) node[below] {repeat} |- (decide);
    \path [line] (decide)   --++  (3,0) node [above] {false} |- (d);

% \path [line] (decide)   --++  (3,0) node [below] {true} |- (b);
%    \path [line] (decide) |- (null1) -- (null2) -| (b);
    %\path [line] (decide) |- node{true} (b);

%    \path [line] (decide) -- node {true}(a);
  %  \path [line] (decide) -- node {false}(b);
    %\path [line] (a) |- (c);
    %\path [line] (b) -- (c);
\end{tikzpicture}
\caption{A Typical Loop Flow Chart}
\label{figure:loopFlowChart}
\end{figure}

The continuation statement is a logical statement (that evaluates to \emph{true} or \emph{false})
that specifies if the loop should continue (if the value is \emph{true}) or should terminate
(if the value is \emph{false}).  Upon termination, code returns to a sequential control flow and
the program continues.  

The iteration statement is intended to update the state of a program to make progress
toward the termination condition.  If we didn't make such progress, the loop would continue
on forever as the termination condition would never be satisfied.  This is known as an \index{infinite loop} \emph{infinite loop} \index{infinite loop}, and results in a program that never terminates.

As a simple example, consider the following outline.

\begin{itemize}
  \item Initialize the value of a variable $i$ to 1
  \item While the value of $i$ is less than or equal to $10\ldots$ (continuation condition)
  \item Perform some action (this is sometimes referred to as the \emph{loop body}
  \item Iterate the variable $i$ by adding one to its value
\end{itemize}

The code outline above specifies that some action is to be performed once for
each value: $i=1, i=2, \ldots, i=10$, after which the loop terminates.  Overall, the
loop executes a total of 10 times.  Prior to each of the 10 executions, the value of $i$
is checked; as it is less than or equal to 10, the action is performed.  At the end of
each of the 10 iterations, the variable $i$ is incremented by 1 and the termination
condition is checked again, repeating the process.  There are several different types of loops that vary in syntax and style but they 
all have the same three basic components. 

\section{While Loops}
\index{while loop}
\index{loops!while loop}

A \emph{while loop} is a type of loop that places the three components in their 
logical order.  The initialization statement is written before the loop code.  
Typically the keyword \textsc{while} is used to specify the continuation/termination
condition.  Finally, the iteration statement is usually performed at the end of the loop
\emph{inside} the code block associated with the loop.  A small, counter-controlled 
while loop is presented in Algorithm \ref{algo:counterControlledWhileLoop} which 
illustrates the previous example of iterating a variable $i$ from 1 to 10.

\begin{algorithm}[H]
\caption{Counter-Controlled While Loop}
\label{algo:counterControlledWhileLoop}
$i \leftarrow 1$ \Comment{Initialization statement}
\While{$\left( i \leq 10 \right)$}{ 
  Perform some action \;
  $i \leftarrow (i + 1)$ \Comment{Iteration statement}
}
\end{algorithm}

Prior to the \textsc{while} statement, the variable $i$ is initialized to $1$.  This action
is only performed once and it is done so before the loop code.  Then, before the loop
code is executed, the continuation condition is checked.  Since $i = 1 \leq 10$, the 
condition evaluates to \emph{true} and the loop code block is executed.  The last
line of the code block is the iteration statement, where $i$ is incremented by 1 and
now has a value of $2$.  The code returns to the \emph{top} of the loop and again
evaluates the continuation condition (which is still true as $i = 2 \leq 10$).  

On the 10th iteration of the loop when $i = 10$, the loop will execute for the last
time.  At the end of the loop, $i$ is incremented to 11.  The loop \emph{still} returns
to the top and the continuation condition is \emph{still} checked one last time.  However,
since $i = 11 \not\leq 10$, the condition is now \emph{false} and the loop terminates.
Regular sequential control flow returns and the program continues executing whatever
code is specified after the loop.

\subsection{Example}
\label{subsection:whileLoopExample}

In the previous example we knew that we wanted the loop to execute ten
times.  Though you can use a while loop in counter-controlled situations, while loops are 
typically used in scenarios when you may not know how many iterations you want 
the loop to execute for.  Instead of a straightforward iteration, the loop itself may
update a variable in a less-than-predictable manner.  

As an example, consider the problem of \emph{normalizing} a number as is typically done in scientific notation.  Given a number $x$ (for simplicity, we'll consider $x \geq 1$),
we divide it by 10 until its value is in the interval $[1, 10)$, keeping track of how many times we've 
divided by 10.  For example, if we have the number $x = 32,145.234$, we would divide
by 10 four times, resulting in $3.2145234$ so that we could express it as
  $$3.2145234 \times 10^{4}$$
A simple realization of this process is presented in Algorithm 
\ref{algo:normalizingNumberWhileLoop}.  Rather than some fixed number of
iterations, the number of times the loop executes depends on how large $x$ is.  For the example
mentioned, it executes 4 times; for an input of $x = 10,000,000$ it would execute 7 times.
A while loop allows us to specify the repetition process without having to know up front
how many times it will execute.

\begin{algorithm}[H]
\caption{Normalizing a Number With a While Loop}
\label{algo:normalizingNumberWhileLoop}
\Input{A number $x$, $x \geq 0$}
\Output{$x$ normalized, $k$ its exponent}
$k \leftarrow 0$ \;
\While{$x > 10$}{
  $x \leftarrow (x / 10)$ \;
  $k \leftarrow (k + 1)$ \; 
}
output $x, k$ \;
\end{algorithm}

\section{For Loops}
\index{for loop}
\index{loops!for loop}

A \emph{for loop} is similar to a while loop but allows you to specify 
the three components on the same line.  In many cases, this results in a loop that
is more readable; if the code block in a while loop is long it may be difficult to see
the initialization, continuation, and iteration statements clearly.  For loops are 
typically used to iterate over elements stored in a collection such as an
array (see Chapter \ref{chapter:arrays}).  Usually the keyword \textsc{for} is used to identify all three components.  
A general example is given in Algorithm \ref{algo:generalForLoop}.

\begin{algorithm}[]
\caption{A General For Loop}
\label{algo:generalForLoop}
\For{$\left(\right. \langle initialization \rangle$; $\langle continuation \rangle$; $\langle iteration \rangle \left.\right)$}{
  Perform some action \;
}
\end{algorithm}

Note the additional syntax: in many programming languages, semicolons are used
at the end of executable statements.  Semicolons are also used to delimit each of
the three loop components in a for-loop (otherwise there may be some ambiguity as
to where each of the components begins and ends).  However, the semicolons are
typically only placed after the initialization statement and continuation condition and 
are \emph{omitted} after the iteration statement.  A more concrete example is given in 
Algorithm \ref{algo:counterControlledForLoop} which is equivalent
to the counter-controlled while loop we examined earlier.

\begin{algorithm}[h]
\caption{Counter-Controlled For Loop}
\label{algo:counterControlledForLoop}
\For{$\left(\right.$ $i \leftarrow 1$; $i \leq 10$; $i \leftarrow (i + 1)$ $\left.\right)$}{
  Perform some action \;
}
\end{algorithm}

Though all three components are written on the same line, the initialization statement
is only ever executed once; at the beginning of the loop.  The continuation condition is
checked prior to each and every execution of the loop.  Only if it evaluates to \emph{true}
does the loop body execute.  The iteration condition is performed \emph{at the end} of
each loop iteration.  

\subsection{Example}
\label{subsection:summationExample}

As a more concrete example, consider Algorithm \ref{algo:summationForLoop} in which 
we do the same iteration ($i$ will take on the values $1, 2, 3, \ldots, 10$), but in each 
iteration we add the value of $i$ \emph{for that iteration} to a running total, $sum$.

\begin{algorithm}[h]
\caption{Summation of Numbers in a For Loop}
\label{algo:summationForLoop}
$sum \leftarrow 0$ \;
\For{$\left(\right.$ $i \leftarrow 1$; $i \leq 10$; $i \leftarrow (i + 1)$ $\left.\right)$}{
  $sum \leftarrow (sum + i)$ \;
}
\end{algorithm}

Again, the initialization of $i = 1$ is only performed once.  On the first iteration of the loop, 
$i = 1$ and so $sum$ will be given the value $sum + i = 0 + 1 = 1$  At the end of the loop, 
$i$ will be incremented and will have a value of $2$.  The continuation condition is still satisfied, 
so once again the loop body executes and $sum$ will be given the value $sum + i = 1 + 2 = 3$.
On the 10th (last) iteration, $sum$ will have a value $1 + 2 + 3 + \cdots + 9 = 45$ and $i = 10$.
Thus $sum + i = 45 + 10 = 55$ after which $i$ will be incremented to 11.  The continuation condition
is \emph{still} checked, but since $11 \not\leq 10$, the loop body will not be executed and
the loop will terminate, resulting in a final $sum$ value of $55$.

\section{Do-While Loops}
\index{do-while loop}
\index{loops!do-while loop}

Yet another type of loop is the \emph{do-while} loop.  One major difference between this
type of loop and the others is that it is always executed \emph{at least once}.  The way that
this is achieved is that the continuation condition is checked \emph{at the end} of the loop
rather than prior to is execution.  The same counter-controlled example can be found in
Algorithm \ref{algo:counterControlledDoWhileLoop}.

\begin{algorithm}[h]
\caption{Counter-Controlled Do-While Loop}
\label{algo:counterControlledDoWhileLoop}
$i \leftarrow 1$ \;
\Do{$i \leq 10$}{
  Perform some action \;
  $i \leftarrow (i + 1)$ \;
}
\end{algorithm}

In contrast to the previous examples, the loop body is executed on the first iteration without
checking the continuation condition.  Only after the loop body, including the incrementing of 
the iteration variable $i$ is the continuation condition checked.  If \emph{true}, the loop repeats
at the beginning of the loop body.

\begin{figure}[!h]
%\begin{wrapfigure}[35]{R}{0.50\textwidth}
\centering
\begin{tikzpicture}[node distance=3cm,auto,scale=1.0,transform shape]
    % Place nodes
    
%    \tikzstyle{decision} = [diamond, draw, fill=blue!20, 
%    text width=4.5em, text badly centered, node distance=3cm, inner sep=0pt]
%\tikzstyle{block} = [rectangle, draw, fill=blue!20, 
%    text width=5em, text centered, rounded corners, minimum height=4em]
%\tikzstyle{line} = [draw, -latex']

    \node [block,text width=8em] (a) {Initialization: $i \leftarrow 1$};
    \node [block,text width=8em, fill=green!20, below of=a] (b) {loop body};
    \node [block,text width=8em, below of=b] (c) {Iteration: $i \leftarrow (i + 1)$};
    \node [decision,node distance=3.5cm,text width=6em,text centered, below of=c] (decide) {Continuation: $i \leq 10$?};
    \node[block,node distance=3.5cm,text width=8em,below of=decide] (d) {remaining program};
    
    %position dummy nodes
    %\node[right of=b] (null1) {};
    %\node[right of=decide] (null2) {};
    
    % Draw edges
    \path [line] (a) -- (b);
    \path [line] (b) -- (c);
    \path [line] (c) -- (decide);
    \path [line] (decide) -- node {false} (d);
    \path [line] (decide)   --++  (3,0) node [below] {true} |- (b);
%    \path [line] (decide) |- (null1) -- (null2) -| (b);
    %\path [line] (decide) |- node{true} (b);

%    \path [line] (decide) -- node {true}(a);
  %  \path [line] (decide) -- node {false}(b);
    %\path [line] (a) |- (c);
    %\path [line] (b) -- (c);
\end{tikzpicture}
\caption{A Do-While Loop Flow Chart.  The continuation condition is checked \emph{after} the loop body.}
\label{figure:doWhileLoopFlowChart}
%\end{wrapfigure}
\end{figure}

Do-while loops are typically used in scenarios in which the first iteration is used to ``setup'' 
the continuation condition (thus, it needs to be executed at least once).  A common example
is if the loop body performs an operation that may result in an error code (or flag) that is
either \emph{true} (an error occurred) or \emph{false} (no error occurred).  

\begin{algorithm}[h]
\caption{Flag-Controlled Do-While Loop}
\label{algo:flagControlledDoWhileLoop}
\Do{isError}{
  Read some data \;
  isError $\leftarrow$ result of reading \;
}
\end{algorithm}

From this perspective, a do-while loop can also be seen as a do-until loop: perform a task
\emph{until} some condition is no longer satisfied.  The subtle wording difference implies that
we'll perform the action before checking to see if it should be performed again.

\section{Foreach Loops}
\label{section:foreachLoops}
\index{foreach loop}
\index{loops!foreach loop}

Many languages support a special type of loop for iterating over individual elements in 
a collection (such as a set, list, or an array).  In general, such loops are referred to as \emph{foreach}
loops.  These types of loops are essentially \gls{syntactic sugar}: iterating over a collection
could be achieved with a for loop or a while loop, but foreach loops provide a more
convenient way to iterate over a collections.  We will revisit these loops
when we examine arrays in Chapter \ref{chapter:arrays}.  For now, we look at a simple example
in Algorithm \ref{algo:foreachLoop}.

\begin{algorithm}[h]
\caption{Example Foreach Loop}
\label{algo:foreachLoop}
\ForEach{element $a$ in the collection $A$}{
  process the element $a$ \;
}
\end{algorithm}

How the elements are stored in the collection and how they are iterated over is not our 
(primary) concern.  We simply want to apply the same block of code to each element, 
the foreach loop handles the details on how each element is iterated over.  The syntax
also provides a way to refer to each element (the $a$ variable in the algorithm).  On
each iteration of the loop, the foreach loop updates the reference $a$ to the \emph{next}
element in the array.  The loop terminates after it has iterated through each and every
one of the elements.  In this way, a foreach loop simplifies the syntax: we don't have to 
specify any of the three components ourselves.  As a more
concrete example, consider iterating over each student in a course roster.  For each
student, we wish to compute their grade and then email them the results.  The foreach
loop allows us to do this without worrying about the iteration details (see Algorithm 
\ref{algo:foreachGradeLoop}).

\begin{algorithm}[h]
\caption{Foreach Loop Computing Grades}
\label{algo:foreachGradeLoop}
\ForEach{(student $s$ in the class $C$)}{
  $g \leftarrow $ compute $a$'s grade \;
  send $a$ an email informing them of their grade $g$ \;
}
\end{algorithm}

\section{Other Issues}

\subsection{Nested Loops}
\index{nested loops}

Just as with conditional statements, we can nest loops within loops to perform
more complex processes.  Though you can do this with any type of loop, we 
present a simple example using for loops in Algorithm \ref{algo:nestedForLoops}.

\begin{algorithm}[h]
\caption{Nested For Loops}
\label{algo:nestedForLoops}
$n \leftarrow 10$ \;
$m \leftarrow 20$ \;
\For{($i \leftarrow 1$; $i \leq m$; $i \leftarrow (i + 1)$)}{
  \For{($j \leftarrow 1$; $j \leq n$; $j \leftarrow (j + 1)$)}{
    output $(i, j)$ \;
  }
}
\end{algorithm}

The outer for loop executes a total of 20 times while the inner for loop 
executes 10 times.  Since the inner for loop is \emph{nested} inside the 
outer loop, the entire inner loop executes all 10 iterations \emph{for each}
of the 20 iterations of the outer loop.  Thus, in total the inner most output
operation executes $10 \times 20 = 200$ times.  Specifically, it outputs 
$(1, 1), (1, 2), \ldots, (1, 10), (2, 1), (2, 2), \ldots, (2, 10), (3, 1), \ldots, (20, 10)$.  Nested loops are comment when iterating over elements 
in two-dimensional 
arrays such as tabular data or matrices.  Nested loops can also be used 
to process all \emph{pairs} in a collection of elements.

\subsection{Infinite Loops}
\label{subsection:infiniteLoops}
\index{infinite loop}
\index{loops!infinite loop}

Sometimes a simple mistake in the design of a loop can make it execute forever.  
For example, if we accidentally iterate a variable in the wrong direction or write the
\emph{opposite} termination/continuation condition.  Such a loop is referred to as 
an \emph{infinite loop}.  As an example, suppose we forgot the increment operation
from a previous example.

\begin{algorithm}[h]
\caption{Infinite Loop}
\label{algo:infiniteLoop}
$sum \leftarrow 0$ \;
$i \leftarrow 1$ \;
\While{$i \leq 10$}{
  $sum \leftarrow (sum + i)$ \;
}
\end{algorithm}

In Algorithm \ref{algo:infiniteLoop} we never make progress toward the terminating
condition!  Thus, the loop will execute forever, $i$ will continue to have the value 0
and since $0 \leq 10$, the loop body will continue to execute.  Care is needed in the
design of your loops to ensure that they make progress toward the termination condition.

Most of the time an infinite loop is not something you want and usually you must 
terminate your buggy program externally (sometimes referred to as ``killing'' it).  
However, infinite loops do have their uses.  A \emph{poll} loop is a loop that is
\emph{intended} to not terminate.  At a system level, for example, a computer
may \emph{poll} devices (such as input/output devices) one-by-one to see if
there is any active input/output request.  Instead of terminating, the poll loop
simply repeats itself, returning back to the first device.  As long as the computer
is in operation, we don't want this process to stop.  This can be viewed as an 
infinite loop as it doesn't have any termination condition.

\subsubsection{The Zune Bug}
\label{subsubsection:zuneBug}

Though proper testing and debugging should reduce the likelihood of such bugs,
there are several notable instances in which an infinite loop impacted real 
software.  One such instance was the Microsoft \emph{Zune} bug.  The Zune
was a portable music player, a competitor to the iPod.  At about midnight on 
the night of December 31st, 2008, Zunes everywhere failed to startup properly.
A firmware clock driver designed by a 3rd party company contained the following
code. 

\begin{listing}[!h]
\begin{minted}{c}
  while(days > 365) {
    if(IsLeapYear(year)) {
      if(days > 366) {
        days -= 366;
        year += 1;
     }
    } else {
      days -= 365;
      year += 1;
    }
  }
\end{minted}
  \caption{Zune Bug}
  \label{code:c:zuneBug}
\end{listing}

2008 was a leap year, so the check on line 2 evaluated to \emph{true}.
However, though December 31st, 2008 was the 366th day of the year (\mintinline{c}{days = 366})
the third line evaluated to \emph{false} and the loop was repeated without any of
the program state being updated.  The problem was ``fixed'' 24 hours later when 
it was the 367th day and line 3 worked.  The problem was that line 3 \emph{should} have been
\mintinline{c}{days >= 366)}.  

The failure was that this code was never tested on the 
``corner cases'' that it was designed for.  No one thought to test the driver to see
if it worked on the last day of a leap year.  The code worked the vast majority of the time, 
but this illustrates the need for rigorous testing.

\subsection{Common Errors}

When writing loops its important to use the proper syntax in the
language in which you are coding.  Many languages use semicolons
to terminate executable statements.  However, the \mintinline{c}{while}
statements are not executable: they are part of the control structure
of the language and do not have semicolons at the end.  A misplaced
semicolon could be a syntax error, or it could be syntactically correct
but lead to incorrect results.  A common error is to place a semicolon
at the end of a \mintinline{c}{while} statement as in 

\mintinline{c}{while(count <= 10); //WRONG!!!}

In this example, the \mintinline{c}{while} loop binds to an \emph{empty}
executable statement and results in an infinite loop!

Other common errors are the result of misidentifying either the
initialization statement or the continuation condition.  Starting a
counter at 1 instead of zero, or using a $\leq$ comparison instead
of a $<$ , etc.  These can lead to a loop being \emph{off-by-one} 
resulting in a logic error.

Other errors are the result of using improper variable types.
Recall that operations involving floating-point numbers can have 
round off and precision errors, $\frac{1}{3} + \frac{1}{3} + \frac{1}{3}$
may not be equal to one for example.  It is best to avoid using 
floating-point numbers or comparisons in the control of your loops.
Boolean and integer types are much less error prone.  

Finally, you must always ensure that your loops are making progress
\emph{toward} the termination condition.  A failure to properly 
increment a counter can lead to incorrect results or even an infinite
loop.

\subsection{Equivalency of Loops}

It might not seem obvious at first, but in fact, \emph{any} type of loop can be re-written 
as another type of loop and perform \emph{equivalent} operations.  That is, any while loop
can be rewritten as an equivalent for loop.  Any do-while loop can be rewritten as an 
equivalent while loop!  

So why do we have different types of loops?  The short answer is that we want our programming
languages to be flexible.  We could design a language in which every loop
\emph{had} to be a while loop for example, but there are some situations in which it would be 
more ``natural'' to write code with a for loop.  By providing several options, programmers have
the choice of which type of loop to write.  

In general, there are no ``rules'' as to which loop to apply to which situation.  There are general
trends, best practices, and situations where it is more common to use one loop rather than another,
but in the end it does come down to personal choice and style.  Some software projects or
organizations may have established guidelines or style guide that establishes such guidelines in the interest of consistency and uniformity.  

\section{Problem Solving With Loops}

Loops can be applied to any problem that requires repetition of some sort 
or to simplify
repeated code.  When designing loops, it is important to identify the three components by asking
the questions:
\begin{itemize}
  \item Where does the loop start?  What variables or other state may need to be initialized or setup
  	prior to the beginning of the loop?	
  \item What code needs to be repeated?  How can it be generalized to depend on loop control variables?
  	This helps you to identify and write the loop body.
  \item When should the loop end?  How many times do we want it to execute?  This helps you to 
  	identify the continuation and/or termination condition.
  \item How do we make progress toward the termination condition?  What variable(s) need to be incremented
  	and how?
\end{itemize}

\section{Examples}

\subsection{For vs While Loop}

Let's consider how to write a loop to compute the classic geometric series, 
  $$\frac{1}{1-x} = \sum_{k=0}^\infty x^k = 1 + x + x^2 + x^3 + \cdots$$
Obviously a computer cannot compute an infinite series as it is required to 
terminate in a finite number of steps.  Thus, we can approach this problem in
a number of different ways.  

One way we could approximate the series is to compute it out to a fixed
number of terms.  To do so, we could initialize a $sum$ variable to zero, then 
iteratively compute and add terms to the $sum$ until we have computed $n$ 
terms.  To keep track of the terms, we can define a counter variable, $k$ as in 
the summation.  

Following our strategy, we can identify the initialization: $k$ should start at $0$.
The iteration is also easy: $k$ should be incremented by 1 each time.  The 
continuation condition should continue the loop until we have computed $n$ terms.
However, since $k$ starts at 0, we would want to continue while $k < n$.  We
would not want to continue the iteration when $k = n$ as that would make $n + 1$
iterations (again since $k$ starts at 0).  Further, since we know the number of
iterations we want to execute, a for loop is arguably the most appropriate loop
for this problem.  Our solution is presented in Algorithm \ref{algo:geometricSeriesForLoop}.

\begin{algorithm}[h]
\caption{Computing the Geometric Series Using a For Loop}
\label{algo:geometricSeriesForLoop}
\Input{$x$, $n \geq 0$}
\Output{An approximated value of $\frac{1}{1-x}$ using a geometric series}
$sum \leftarrow 0$ \;
\For{($k = 0$; $k < n$; $k \leftarrow (k + 1)$)}{
  $sum \leftarrow (sum + x^k)$ \;
}
output $sum$ \;
\end{algorithm}

As an alternative, consider the following approach: instead of computing a 
predefined number of terms, what if we computed terms until the difference
between the value in the previous iteration and the value in the current 
iteration is negligible, say less than some small $\epsilon$ amount.  We could
stop our computation because any further iterations would only affect the 
summation less and less.  That is, the current value represents a ``good
enough'' approximation.  That way, if someone wanted an even better
approximation, they could specify a smaller $\epsilon$.

This approach will be more straightforward with a while loop since the 
continuation condition will be more along the lines of ``while the estimation
is not yet good enough, continue the summation.''  This approach will
also be easier if we keep track of both a \emph{current} and a \emph{previous}
value of the summation, then computing and checking the difference will
be easier.

\begin{algorithm}[h]
\caption{Computing the Geometric Series Using a While Loop}
\label{algo:geometricSeriesWhileLoop}
\Input{$x$, $\epsilon > 0$}
$sum_{prev} \leftarrow 0$ \;
$sum_{curr} \leftarrow 1$ \;
$k \leftarrow 1$ \;
\While{$|sum_{prev} - sum_{curr}| \geq \epsilon$}{
  $sum_{prev} \leftarrow sum_{curr}$ \;
  $sum_{curr} \leftarrow (sum_{curr} + x^k)$ \;
  $k \leftarrow (k + 1)$ \;
}
output $sum$ \;
\end{algorithm}

On lines 1--2 we initialize our values to ensure that the while loop will
execute at least once.  In the continuation condition, we use the absolute
value of the difference as the series can \emph{oscillate} between negative
and positive values.

\subsection{Primality Testing}

An integer $n > 1$ is called \emph{prime} if the only integers that 
divide it evently are 1 
and itself.  Otherwise it is called \emph{composite}.  For example, 30 is composite
as it is divisible by 2, 3, and 5 among others.  However, 31 is prime as it is only 
divisible by 1 and 31.

Consider the problem of determining whether or not a given integer $n$ is prime or 
composite, referred to as \emph{primality testing}.  A straightforward way of determining
this is to simply try dividing by every integer $2$ up to $\sqrt{n}$: if any of these integers
divides $n$, then $n$ is composite.  Otherwise, if none of them do, $n$ is prime.  
Observe that we only need to go up to $\sqrt{n}$ since any prime divisor greater
than that will correspond to \emph{some} prime divisor less than $\sqrt{n}$.

A simple for loop can be constructed to capture this idea. Our initialization clearly starts at $i = 2$, 
incrementing by 1 each time until $i$ has exceeded $\sqrt{n}$.  This solution
is presented in Algorithm \ref{algo:primalityTesting}.  Of course this is certainly
not the most efficient way to solve this problem, but we will not go into 
more advanced algorithms here.

\begin{algorithm}[h]
\caption{Determining if a Number is Prime or Composite}
\label{algo:primalityTesting}
\Input{$n > 1$}
\For{($i\leftarrow 2$; $i \leq \sqrt{n}$; $i \leftarrow (i + 1)$)}{
  \If{$i$ divides $n$}{
    output \emph{composite} \;
  }
}
output $prime$ \;
\end{algorithm}

Now consider this more general problem: given an integer $m > 1$, determine
\emph{how many} prime numbers $\leq m$ there are.  A key observation is that 
we've \emph{already solved} part of the problem: determining if a given number
is prime in the previous exercise.  To solve this more general problem, we could \emph{reuse}
or adapt our previous solution.  In particular, we could surround the previous
solution in an \emph{outer} loop and iterate over integers from 2 up to $m$.  
The inner loop would then determine if the integer is prime and instead of 
outputting a result, could increment a counter of the number of primes
it has found so far.
This solution is presented in Algorithm \ref{algo:primalityCounting}.

\begin{algorithm}[H]
\caption{Counting the number of primes.}
\label{algo:primalityCounting}
\Input{$m > 1$}
$numberOfPrimes \leftarrow 0$ \;
\For{($j=2$; $j\leq m$; $j \leftarrow (j + 1)$)}{
  $isPrime \leftarrow true$ \;
  \For{($i\leftarrow 2$; $i \leq \sqrt{j}$; $i \leftarrow (i + 1)$)}{
    \If{($i$ divides $j$)}{
      $isPrime \leftarrow false$ \;
    }
  }
  \If{($isPrime$)}{
    $numberOfPrimes \leftarrow (numberOfPrimes + 1)$ \;
  }
}
output $numberOfPrimes$ \;
\end{algorithm}

\subsection{Paying the Piper}
\label{subsection:loanAmortization}

Banks issue loans to customers as one lump sum called a \emph{principle} $P$ that
the borrower must pay back over a number of \emph{terms}.  Usually payments
are made on a monthly basis.  Further, banks charge an amount of \emph{interest} on a
loan measured as an Annual Percentage Rate (APR).  Given these conditions, 
the borrower makes monthly payments determined by the following formula.
$$ {monthlyPayment} = \frac{iP}{1 - (1 + i)^{-n}} $$
Where $i = \frac{apr}{12}$ is the monthly interest rate, and $n$ is the number of terms
(in months).

For simplicity, suppose we borrow $P = \$1,000$ at 5\% interest ($apr = 0.05$) to 
be paid back over a term of 2 years ($n = 24$).  Our monthly payment would 
(rounded) be 
  $$ {monthlyPayment} = \frac{\frac{.05}{12}\cdot 1000}{1 - (1 + \frac{.05}{12})^{-24}} = \$43.87$$
When the borrower makes the first month's payment, some of it goes to interest, 
some of it goes to paying down the balance.  Specifically, one month's interest on \$1,000 is
  $$\$1,000 \cdot \frac{0.05}{12} = \$4.17$$
and so $\$43.87 - \$4.17 = \$39.70$ goes to the balance, making the new balance
\$960.30.  The next month, this new balance is used to compute the new interest payment, 
  $$\$960.30 \cdot \frac{0.05}{12} = \$4.00$$
And so on until the balance is fully paid.  This process is known as \emph{loan amortization}.

Let's write a program that will calculate a loan amortization schedule given the 
inputs as described above.  To start, we'll need to compute the monthly payment
using the formula above and for that we'll need a monthly interest rate.  The
balance will be updated month-to-month, so we'll use another variable to represent the balance.  Finally, we'll want to track the current month in the loan 
schedule process.  

Once we have these variables setup, we can start a loop that will repeat once for
each month in the loan schedule.  We could do this using either type of loop, but
for this exercise, let's use a while loop.  Using our $month$ variable, we'll start by
initializing it to 1 and run the loop through the last month, $n$.  

On each iteration we compute that month's interest and principle payments as above, 
update the balance, and also be sure to update our $month$ counter variable to
ensure we're making progress toward the termination condition.  On each 
iteration we'll also output each of these variables to the user.  The full 
program can be found in Algorithm \ref{algo:loanAmortization}.

If we were to actually implement this we'd need to be more careful.  This outlines
the basic process, but keep in mind that US dollars are only accurate to cents.
A monthly payment can't be \$43.871 cents.  We'll need to take care to round
properly.  This introduces another issue: by rounding the final month's payment
may not match the expected monthly payment (we may over or under pay in the 
final month).  An actual implementation may need to handle the final month's 
payment separately with different logic and operations than are outside 
the loop.

\begin{algorithm}[H]
\Input{A principle, $P$, a number of terms, $n$, an APR, $apr$}
\Output{A loan amortization schedule}
$balance \leftarrow P$ \Comment{The initial balance is the principle}
$i \leftarrow \frac{apr}{12}$ \Comment{monthly interest rate}
$monthlyPayment \leftarrow \frac{iP}{1 - (1 + i)^{-n}} $ \;
$month \leftarrow 1$ \Comment{A month counter}
\While{($month \leq n$)}{
  $monthInterest \leftarrow i \cdot balance$ \;
  $monthPrinciple \leftarrow monthlyPayment - monthInterest$ \;
  $balance \leftarrow balance - monthPrinciple$ \;
  $month = (month + 1)$ \;
  output $month, monthInterest, monthPrinciple, balance$ \;
}
\caption{Computing a loan amortization schedule}
\label{algo:loanAmortization}
\end{algorithm}



