%!TEX root = ComputerScienceOne.tex

%%Chapter: Strings
\index{strings}

A \gls{string} is an ordered sequence of characters.
We've previously seen string data types as literals.  Most 
languages allow you to define and use static string literals
using the double quote syntax.  We used strings to specify
output formatting using \mintinline{c}{printf()}-style functions
for example.  When reading input from a user, we read it
as a string and converted the input to numbers.  We also
described some basic operations on strings including 
concatenation.  We now examine strings in more depth.  

Programming languages vary greatly in how they represent
string data types.  Some languages have string types
built-in to the language and others require that you use
arrays and yet others treat strings as a special type of
object.  One issue with string representations is determining where
and how the string ends.  Some languages use a \emph{length
prefix} string representation.  The length (the number
characters in the string) is stored in a special location at
the beginning of a string.  Then the string characters are
stored as an array.  Other \gls{oopLabel} languages use a special
character, the \emph{null-terminating character} to denote
the end of a string.  Still other languages store strings
as arrays or dynamic arrays and the ``bookkeeping''
is done internally as part of an object representation.

Other details vary as well.  Most languages support the
basic ASCII characters, others have full Unicode support
or support Unicode through a library.  Most languages
also provide large libraries of functions and operations
that make working with strings easier.

\section{Basic Operations}

Depending on how a language supports strings, it may
support various basic operations to create strings and
assign them to variables.  Usually languages allow you
to create and use string literals using the double quote 
syntax.  Modifying a string or copying one string into 
another may be supported by the built-in assignment
operator or it may require the use of a \emph{copy}
function.  When copying strings, similar issues come 
into play as with arrays.  The ``copy'' could be a shallow
copy or a deep copy (see Section \ref{subsection:shallowVsDeep}).

Other basic operations may include accessing and/or
modifying individual characters in a string.  In order
to do so, we may also need to know a string's length
(so that we do not access invalid characters).  A
language could provide this as part of the string itself
(usually called a \emph{property} of the string) or 
through a function call.  We can further use such functionality
to iterate over the individual characters in a string using
an index-controlled for-loop.

More advanced operations on strings include \gls{concatenation}
which is the operation of combining one or more strings
to create a new string.  Concatenation simply \emph{appends}
one string to the end of another string.  Another common 
operation is to extract a \emph{substring}
from a string, that is create a new string from a portion of
another string.  Commonly, this is done via some standard
function that may operate by specifying indices and/or the
length of the desired substring.  Finally, it is also common 
to deal with collections of strings.  
Some languages allow you to create arrays of strings or 
dynamic collections (lists or sets) of strings.  for languages
in which strings are arrays of characters, an array of strings
might be implemented with a 2-dimensional array of characters.

\section{Comparisons}
\index{strings!comparison}

When processing strings there are several other standard operations.
In particular, we often have need to make comparisons between
two string variables or between a string variable and a literal.
Some languages allow you to use the same operators such as
\mintinline{c}{==} or even \mintinline{c}{<} to make comparisons
between strings.  The implied behavior would compare strings for
equality (case sensitive) or for \gls{lexicographic} order.  For example
``Apple'' $<$ ``Banana'' might evaluate to \True because 
``Apple'' precedes ``Banana'' in alphabetic order.

Many languages, however, require that you make string comparisons
using a function.  Using the equality operator \mintinline{c}{==} may
be correct syntactically, but is usually making a pointer or 
reference comparison which evaluates to true if and only if the
two variables represent the same memory address.  Even if two
string variables have the same \emph{content}, the equality operator
may evaluate to \False if they are distinct (deep) copies of the
same string.  Likewise, the inequality operators $<, \leq$, etc. may
only be comparing memory addresses which is meaningless
for comparing strings.

The solution that many languages provide is the use of a \index{comparator}
\gls{comparator}, which is either a function or an object that
facilitates the comparison of strings (and more generally, any object).
Generally, a comparator function takes two arguments, $a, b$
and compares them, not just for equality, but for their relative
order: does $a$ ``come before'' $b$ or does $b$ ``come before''
$a$, or are they equal.  To distinguish between these three
cases, a comparator returns an integer value with the following
general contract: it returns
\begin{itemize}
  \item Something negative if $a < b$
  \item Zero if $a = b$
  \item Something positive if $a > b$
\end{itemize}
Using this contract we can determine the relative ordering of
any two strings.  In general we cannot make any assumptions
about the actual value that a comparator returns, only that it
returns \emph{something} negative or positive.  The actual
magnitude of the returned value need not be $-1$ or $+1$, 
and it may not even have any predefined meaning.

\section{Tokenizing}
\index{strings!tokenizing}

It is common to store different pieces of data as a string such
that each individual piece of data is demarcated by some 
\emph{delimiter}.  For example, \gls{csvLabel} or \gls{tsvLabel}
data use commas and tabs to delimit data.  For example, the
string 

\mintinline{text}{Smith,Joe,12345678,1985-09-08}

is a CSV string holding data about a particular person (last name, 
first name, ID, date of birth).  Often we need to process such
strings to extract each individual piece of data.

Processing such strings is usually referred to as \index{parse} 
\glslink{parse}{parsing}.
In particular, a string is ``split'' into a collection of individual
strings called \glspl{token} (thus the process is also sometimes
referred to as tokenizing).  In the example above, the
string would be processed into 4 individual strings, 
\mintinline{text}{Smith}, \mintinline{text}{Joe}, \mintinline{text}{12345678},
and \mintinline{text}{1985-09-08}.  Each string could further
be tokenized if needed, such as parsing the date of birth
to extract the year, month, and date.

Most languages provide a function to facilitate tokenization.  Some
do so by directly returning an array or collection of the resulting
tokens (usually with the delimiter removed).  Others have a
more manual process that requires a loop structure to iterate
over each token.
