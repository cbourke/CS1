%!TEX root = ComputerScienceOne.tex

%%Chapter: FileIO in PHP

Because of the history of PHP, file functions, 
just like string functions, were mostly influenced
by the C standard library functions and have
very similar naming and usage.  Writing
binary or plaintext data is determined by which
functions you use.

In general whether or not a file input/output
stream is buffered or unbuffered is determined by
the system configuration.  There are some ways
in which this can be changed, but we will not 
cover them in detail.

\section{Opening Files}

Files are represented in PHP as a ``resource'' that
can be passed around to other functions to read and
write to the file.  For our purposes, a resource is
simply a variable that can be stored and passed
to other functions.  

To open a file, you use the \mintinline{php}{fopen()} 
function (short for \textbf{f}ile \textbf{open}) which
requires two arguments and returns a file resource.
The first argument is the file path/name that you want
to open for processing.  The second argument is a
string representing the ``mode'' that you want to open the
file in.  There are several supported modes, but the two 
we will be interested in are reading, in which case you
pass it \mintinline{php}{"r"} and writing in which case you
pass it \mintinline{php}{"w"}.  The path can be an absolute 
path, relative path, or may be omitted if the file is in the
current working directory.

\begin{minted}{php}
//open a file for reading (input):
$input = fopen("/user/apps/data.txt", "r");
//open a file for reading (output):
$output = fopen("./results.txt", "w");

if(!$input) {
  printf("Unable to open input file\n");
  exit(1);
}

if(!$output) {
  printf("Unable to open output file\n");
  exit(1);
}
\end{minted}

The two checks above check that the file opened successfully.
If opening the file failed, \mintinline{php}{fopen()} returns
\mintinline{php}{false} (and the interpreter issues  warning).  

\section{Reading \& Writing}

When a file is opened, the file resource returned by
\mintinline{php}{fopen()} initially points to the beginning
of the file.  As you read or write from it, the resource
advances through the file content.

\subsubsection{Reading}

There are a couple of ways to read input from a file.
To read a file line by line, we could use \mintinline{php}{fgets()}
to get each line.  It takes a single argument: the file
``handle'' that was returned by \mintinline{php}{fopen()}
and returns a string representing the entire line 
\emph{including} the endline character.  If necessary, 
leading and trailing whitespace can be removed using 
\mintinline{php}{trim()}.  To determine the end of a
file you can use \mintinline{php}{feof()} which returns
true when the resource has reached the end of the file.
Here is an example processing an entire file line-by-line:

\begin{minted}{php}
$h = fopen("input.data", "r"); 
while(!feof($h)) {
  //read the next line:
  $line = fgets($h);
  //trim it:
  $line = trim($line);
  //process it, we'll just print it
  print $line; 
}
\end{minted}

Alternatively there is a convenient function, 
\mintinline{php}{file_get_contents()} that will retrieve
the entire file as a string.

\mintinline{php}{$fileContents = file_get_contents("./inputFile.txt");}

\subsubsection{Writing}

There are several ways that we can output to 
files, but the easiest is to simply use \mintinline{php}{fwrite()}
(short for \textbf{f}ile \textbf{write}).  This is a
``binary-safe'' file output function that takes
two arguments: the file ``handle'' to write to and
a string of data to be written.  It also returns
an integer representing the number of bytes written
to the file (or \mintinline{php}{FALSE} on error).

\begin{minted}{php}
$x = 10;
$y = 3.14;

//write to a plaintext file
fwrite($output, "Hello World!\n");
fwrite($output, "x = $x, y = $y\n");
\end{minted}

\subsection{Using URLs}

A nice feature of PHP is that you can use URLs as file
names to read and write to a URL.  ``Reading'' from a
URL would simply mean connecting to a remote resource, 
such as a webservice and downloading its contents 
(which may be \gls{htmlLabel}, \gls{xmlLabel} or \gls{jsonLabel}
data).  Writing to a URL may be used to \emph{post}
data to a web page in order to receive a response.  

As an example, we can download a webpage in PHP as follows.

\begin{minted}{php}
$h = fopen("http://cse.unl.edu", "r");
$contents = "";
while(!feof($h)) {
  $contents .= fgets($h);
}

//or just:
$contents = file_get_contents("http://cse.unl.edu");
\end{minted}

\subsection{Closing Files}

Once you are done processing a file, you should close it
using the \mintinline{php}{fclose()} function.

\mintinline{php}{fclose($h);}

which takes a single argument, the file handle that you
wish to close.  



