%!TEX root = ComputerScienceOne.tex

%%Chapter: Recursion in C
\index{recursion!C}
\index{C!recursion}

C supports recursion with no special syntax necessary.  However, as a structured, 
procedural language, recursion is generally expensive and iterative or
other non-recursive solutions are generally preferred.  We present
a few examples to demonstrate how to write recursive functions in C.

The first example of a recursive function we gave was the toy count down
example.  In C it could be implemented as follows.

\begin{minted}{c}
void countDown(int n) {
  if(n==0) {
    printf("Happy New Year!\n");
  } else {
    printf("%d\n", n);
    countDown(n-1);
  }
}
\end{minted}

As another example that actually does something useful, consider the
following recursive summation function that takes an array, its size
and an index variable.  The recursion works as follows: if the index
variable has reached the size of the array, it stops and returns zero
(the base case).  Otherwise, it makes a recursive call to 
\mintinline{c}{recSum()}, incrementing the index variable by 1.  When
the function returns, it adds its result to the $i$-th element
in the array.  To invoke this function we would call it with an initial
value of 0 for the index variable: \mintinline{c}{recSum(arr, n, 0)}.

\begin{minted}{c}
int recSum(const int *arr, int size, int i) {
  if(i == size) {
    return 0;
  } else {
    return recSum(arr, size, i+1) + arr[i];
  }
}
\end{minted}

This example was not tail-recursive as the recursive call was not the
final operation (the sum was the final operation).  To make this function
tail recursive, we can carry the summation through to each function call
ensuring that the summation is done prior to the recursive function call.

\begin{minted}{c}
int recSumTail(const int *arr, int size, int i, int sum) {
  if(i == size) {
    return sum;
  } else {
    return recSumTail(arr, size, i+1, sum + arr[i]);
  }
}
\end{minted}

As a final example, consider the following C implementation of the 
naive recursive Fibonacci sequence.  An additional condition has been
included to check for ``invalid'' negative values of $n$ for which
zero is returned.

\begin{minted}{c}
int fibonacci(int n) {
  if(n < 0) {
    return 0;
  } else if(n <= 1) {
    return 1;
  } else {
    return fibonacci(n-1) + fibonacci(n-2);
  }
}
\end{minted}

C is not a language that provides implicit \index{memoization} 
memoization.  Instead, 
we need to explicitly keep track of values using a table.  In the
following example, the table is passed to the function as an argument.

\begin{minted}{c}
int fibonacciMemoization(int n, int *table) {
  if(n < 0) {
    return 0;
  } else if(n <= 1) {
    return 1;
  } else if(table[n] > 0) {
    return table[n];
  } else {
    int a = fibonacciMemoization(n-1, table);
    int b = fibonacciMemoization(n-2, table);
    int result = (a + b);
    table[n] = result;
    return result;
  }
}
\end{minted}

It is the responsibility of the calling function to ensure that the 
\mintinline{c}{table} array is large enough to accommodate all values.
In this case should be at least of size $(n+1)$ to compute the $n$-th Fibonacci
number.



