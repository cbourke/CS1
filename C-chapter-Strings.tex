%!TEX root = ComputerScienceOne.tex

%%Chapter: Strings in C
\index{strings!in C}
\index{C!strings}

C has no built-in string type.  Instead, strings are represented
as arrays of \mintinline{c}{char} elements.  They differ from, say
arrays of \mintinline{c}{int} or \mintinline{c}{double} types, however,
in that they are \emph{null terminated} arrays.  The end of the
string must \emph{always} be denoted with a null-terminating 
character, \mintinline{c}{'\0'} (the 0 valued character in the \gls{asciiLabel}
table).  Failure to properly null-terminate a string may lead
to undefined behavior or fatal errors.  

C provides a good variety of functions in its standard string library
(included in the \mintinline{text}{string.h} header).  All of these 
functions operate under the assumption that the strings passed
to it are null-terminated.  It is your responsibility to ensure that
they are.  Moreover, some of the functions that operate on strings
will inset the null-terminating character for us, but others will not.
It is important to understand the expectations and guarantees
of each function.  

%\documentclass[12pt]{scrbook}
%
%\usepackage{tikz}
%\usepackage{minted}
%\usetikzlibrary{decorations.pathreplacing,arrows}
%
%\usepackage{fullpage}
%\usepackage{subfigure}
%\begin{document}
%
%
%Lorem Ipsum is simply dummy text of the printing and typesetting industry. Lorem Ipsum has been the industry's standard dummy text ever since the 1500s, when an unknown printer took a galley of type and scrambled it to make a type specimen book. It has survived not only five centuries, but also the leap into electronic typesetting, remaining essentially unchanged. It was popularised in the 1960s with the release of Letraset sheets containing Lorem Ipsum passages, and more recently with desktop publishing software like Aldus PageMaker including versions of Lorem Ipsum.

\begin{figure}
\centering

\begin{tikzpicture}
% size of each node
\def\sz{10mm}
% node style definition
\tikzstyle{block} = [
	draw, fill=black!10, rectangle,
	minimum height=\sz, minimum width=\sz ];
\tikzstyle{plain} = [draw=none,fill=none];
% array element definition
\def\arr{\mintinline{text}{H}, 
         \mintinline{text}{e}, 
         \mintinline{text}{l}, 
         \mintinline{text}{l}, 
         \mintinline{text}{o}, 
         \mintinline{text}{\0},
         \mintinline{text}{?}, 
         \mintinline{text}{?}, 
         \mintinline{text}{?}, 
         \mintinline{text}{?}
         };
%\def\x{0}; % x pos of arr
%\def\y{0}; % y pos of arr
%\newcounter{ind};
\setcounter{ind}{0};
\node[plain] at (-1.75, 1) { index };
\node[plain] (a) at (-2.75, 0) { \mintinline{c}{char *s} };
\foreach \item in \arr
{
	\node[block] (a\theind) at (\theind*\sz,0) { \item };
	\node[plain] at (\theind*\sz,1.0) { \theind };
	\addtocounter{ind}{1};
}
\draw[->,>=stealth'] (a) -- (a0);
\end{tikzpicture}

\caption[Example of a character array (string) in C.]{A string in
C is achieved by using a \mintinline{c}{char} array.  However, the
string is terminated by a null-terminating character, \mintinline{text}{\0}.
Though an array may have space for additional characters, they are
irrelevant if the null terminator precedes them.}
\label{figure:cStringExample}
\end{figure}

%\end{document}



\section{Character Arrays}

We've previously used string literals when using the standard input
and output functions, \mintinline{c}{printf()} and \mintinline{c}{scanf()}.
You can also create static strings (as static arrays) using the standard
assignment operator.  Some examples:

\begin{minted}{c}
char firstName[] = "Tom";
char lastName[] = "Waits";
\end{minted}

This syntax can only be used when creating static strings (they are
allocated on the stack and locally scoped).  The compiler is able
to scan the string literal and determine how many characters are
needed and even inserts the null-terminating character for us.  Thus, 
the length of the two strings in the example are 3 and 5 respectively, 
but the array size created for them will be 4 and 6 respectively to 
accommodate the null-terminating character.

Static strings have the same limitations as static arrays.  Since they
are allocated on the stack, they cannot be returned from
a function.  Just as with \mintinline{c}{int} and \mintinline{c}{double} 
types, however, we can dynamically allocate memory to hold \mintinline{c}{char}
types using \mintinline{c}{malloc()}.  The important difference being
that we need to always allocate at least one more character to accommodate
the null-terminating character.

\begin{minted}{c}
char *fullName = NULL;
fullName = (char *) malloc(sizeof(char) * 10);
\end{minted}

This example creates a dynamically allocated \mintinline{c}{char} array
that is able to hold \emph{9} characters (since one will be needed for
the null-terminating character).  However, once we have created the array, 
we cannot simply assign an entire string to it using the usual assignment
operator.  

\begin{minted}{c}
//THIS IS *WRONG*:
fullName = "Tom Waits";
\end{minted}

This will compile, but doesn't give us what we want.  Recall
that \mintinline{c}{fullName} is a character pointer.  Using
the assignment operator simply makes it \emph{point} to the
static, literal string \mintinline{c}{"Tom Waits"}.  In fact, 
we lose our reference to the dynamically allocated array that
was created with \mintinline{c}{malloc()}, resulting in a 
\index{memory leak} \gls{memory leak}.

Instead, we need to use a function in the standard library to
\emph{copy} a string.  The function in the standard library 
that allows you to copy strings is as follows.

\mintinline{c}{char *strcpy(char *dest, const char *src);}

The name, \mintinline{c}{strcpy} is short for 
``string copy.''\footnote{The abbreviations are mostly historic: in the 70s and
80s when memory was measured in kilobytes, saving a few characters
made a significant difference.} Both arguments are character
pointers.  In keeping with the use of the assignment operator, 
the second argument (``source'') is copied into
the first, ``destination'' (just as with an assignment operator, 
the value on the right-hand-side is copied into the variable on 
the left-hand-side). Moreover, the second argument has been 
marked as \mintinline{c}{const} indicating that it will not be 
changed.  The contents of the first argument \emph{will} be 
changed since we are copying a string into it, erasing whatever 
contents it had prior.  Finally, the function returns a pointer 
to the \mintinline{c}{dest} argument, mostly so that it can be
used in nested function calls (though we'll avoid such confusingly
terse ``tricks'').  From our example:

\begin{minted}{c}
//"assign" (copy) "Tom Waits" into the string fullName:
strcpy(fullName, "Tom Waits");
\end{minted}

In addition, we can access and modify individual characters in
a string using the usual indexing and assignment operator with
\mintinline{c}{char} literals.  

\begin{minted}{c}
//access individual characters:
char firstInitial = fullName[0]; //'T'
char lastInitial = fullName[4]; //'W'

//printing:
printf("First Initial: %c\n", fullName[0]);

//modifying:
firstInitial[0] = 't';
firstInitial[4] = 'w';
//fullName is now "tom waits"
\end{minted}

\section{String Library}

The standard string library provides many other convenient
functions that allow you to process and modify strings.
We highlight a few of the more common ones here.  A full
list of supported functions can be found in standard 
documentation.

\subsubsection{Length}

As with regular arrays, we are responsible for ensuring
that we do not access characters outside the character array
of a string.  Since a string is null-terminated, 
there is a nice function provided by the string library to
determine its length, 

\mintinline{c}{size_t strlen(const char *s);}

Recall that \mintinline{c}{size_t} can essentially be
treated as an integer, indicating the number of bytes 
in the passed string.  Since a character is a single
byte, this function tells us how many character are in
the given string \emph{not including} the null-terminating
string.  This function is an abbreviation for
``string length.''  Using this function we can easily iterate over each 
character in a string.

\begin{minted}{c}
int i;
for(i=0; i<strlen(fullName); i++) {
  printf("fullName[%d] = %c\n", i, fullName[i]);
}
\end{minted}

\subsubsection{Concatenation}

A concatenation function is provided that allows you 
to append one string to the end of another.

\mintinline{c}{char *strcat(char *dest, const char *src);}

Similar to the \mintinline{c}{strcpy()} function, 
\mintinline{c}{strcat()} appends the ``source'' (\mintinline{c}{src})
string to the end of the ``destination'' (\mintinline{c}{dest})
string.  It is \emph{your} responsibility to ensure
that the destination string is large enough to accommodate the
source string.  If it is not, then it could lead to undefined
behavior as \mintinline{c}{strcat()} overwrites memory after the
end of the destination string.  Further, \mintinline{c}{strcat()}
\emph{will} copy the null-terminating character for you so that
the resulting string (now stored in \mintinline{c}{dest}) is a
valid null-terminated string.

\begin{minted}{c}
char *formattedName = (char *)malloc(11 * sizeof(char));
//copy "Waits" into formattedName
strcpy(formattedName, lastName);
//append a ", " to formattedName
strcat(formattedName, ", ");
//append the firstName to the formattedName
strcat(formattedName, firstName);
//formattedName now contains "Waits, Tom"
\end{minted}

\subsubsection{Byte-Limited Versions}

C also provides several \emph{byte-limited} versions of
the copy and concatenation functions:

\mintinline{c}{char *strncpy(char *dest, const char *src, size_t n);}

\mintinline{c}{char *strncat(char *dest, const char *src, size_t n);}

They work similarly in that they copy/concatenate the
source, \mintinline{c}{src} string into the destination,
\mintinline{c}{dest} string.  However, there is a third
parameter, \mintinline{c}{n} which specifies \emph{at
most} how many bytes to copy/concatenate.  The parameter, 
\mintinline{c}{n} allows you to limit the number of 
characters that the operation uses.  If either of
these functions encounters the null-terminating character
before \mintinline{c}{n} bytes have been copied/concatenated, 
they stop and copy the null-terminating character for us.

However, these functions do not always handle
the null-terminating character for us.  If the null-terminating
character appears in the first \mintinline{c}{n} bytes, 
it will be copied for us, but if it is not, we need to 
be sure to handle it ourselves.  For example, 

\begin{minted}{c}
char email[] = "twaits@cse.unl.edu";
char *login = (char *) malloc(7 * sizeof(char));

//only copy the first 6 characters:
strncpy(login, email, 6);
//at this point, login contains "twaits" but is
//not null-terminated, so we manually terminate it
login[6] = '\0';
\end{minted}

\subsubsection{Computing a Substring}

The byte-limited versions of the copy function can
be used to compute a substring of another string.
The parameter \mintinline{c}{n} can be used to limit
the length of the string, but how might we specify
where the substring should start?  

For example, in the string \mintinline{c}{"Thomas Alan Waits"}
we may want to get the substring representing his
middle name, \mintinline{c}{"Alan"}.  The length
is 4 and we can specify that the copying should start by 
using an index.  In this case, we want the copying to
start at index \mintinline{c}{7}, the 8th character.  
If the string is stored in an array named \mintinline{c}{name}, 
this would be \mintinline{c}{name[7]}.  However, indexing
an array like this results in a single character and
\mintinline{c}{strncpy()} expects a string (a 
character \emph{pointer}).  Fortunately, we know how
to do this: using the referencing operator, we can
turn the 8th character into a character pointer, 
\mintinline{c}{&name[7]}.  A full example:

\begin{minted}{c}
char name[] = "Thomas Alan Waits";
char *middleName = (char *) malloc(sizeof(char) * 5);
strncpy(middleName, &name[7], 5);
middleName[4] = '\0';
//middleName now holds "Alan"
\end{minted}

In the call to \mintinline{c}{malloc()}, we included space
for a null-terminating character.  Furthermore, \mintinline{c}{strncpy()}
does not insert the null-terminating character for us in this case.
In line 4 we manually insert it.

\section{Arrays of Strings}

We often need to deal with collections of strings.  An
array of strings can be viewed as a 2-dimensional 
character array.  Indeed, we've seen this before.  In
the typical C program's \mintinline{c}{main()} function, 
command line arguments are passed as the second
parameter:

\mintinline{c}{int main(int argc, char **argv)}

which is represented as a double character pointer, 
\mintinline{c}{char **}.  Conceptually, each row is
a string, \mintinline{c}{char *}.  The first parameter,
\mintinline{c}{argc} indicates how many rows there are.
The \mintinline{c}{main()} function doesn't need to
be told how big each ``row'' is since strings are 
null-terminated.

We can create our own arrays of strings similar to
how we created 2-dimensional arrays of \mintinline{c}{int}
and \mintinline{c}{double} types.

\begin{minted}{c}
//create an array that can hold 5 strings
char **names = (char **) malloc(5 * sizeof(char*));
int i;
for(i=0; i<5; i++) {
  //each string can hold at most 19 characters
  names[i] = (char *) malloc(sizeof(char) * 20);
}
strcpy(names[0], "Margaret Hamilton");
strcpy(names[1], "Ada Lovelace");
strcpy(names[2], "Grace Hopper");
strcpy(names[3], "Marie Curie");
strcpy(names[4], "Hedy Lamarr");
\end{minted}

\section{Comparisons}

When comparing strings in C, we cannot use the numerical
comparison operators such as \mintinline{c}{==}, or
\mintinline{c}{<}.  Because strings are represented as
arrays, using these operators actually compares the
variable's \emph{memory addresses}.

\begin{minted}{c}
char *a = (char *) malloc(sizeof(char) * 13);
char *b = (char *) malloc(sizeof(char) * 13);
strcpy(a, "Hello World!");
strcpy(b, "Hello World!");

if(a == b) {
  printf("strings match!\n");
}
\end{minted}

The code above will not print anything even though
the strings \mintinline{c}{a} and \mintinline{c}{b}
have the same content.  This is because \mintinline{c}{a == b}
is comparing the memory address of the two variables.
Since they point to different memory addresses 
(created by two separate calls to \mintinline{c}{malloc()})
they are not equal.

The C string library provides a standard \gls{comparator}
function to compare strings based on their content:

\mintinline{c}{int strcmp(const char *a, const char *b);}

The function takes two strings and returns an integer based
on the lexicographic ordering of \mintinline{c}{a} and
\mintinline{c}{b}.  If \mintinline{c}{a} precedes \mintinline{c}{b}, 
\mintinline{c}{strcmp()} returns something negative.  It
returns zero if \mintinline{c}{a} and \mintinline{c}{b} 
have the same content.  Otherwise it returns something
positive if \mintinline{c}{b} precedes \mintinline{c}{a}.
Some examples:

\begin{minted}{c}
int x;
x = strcmp("apple", "banana"); //x is negative
x = strcmp("zelda", "mario");  //x is positive
x = strcmp("Hello", "Hello");  //x is zero

//shorter strings precede longer strings:
x = strcmp("apple", "apples"); //x is negative
//uppercase precede lowercase:
x = strcmp("Apple", "apple");  //x is negative
\end{minted}

In the last example, \mintinline{c}{"Apple"} precedes
\mintinline{c}{"apple"} since uppercase letters are
ordered before lowercase letters according to the
\gls{asciiLabel} table.  We can also make comparisons
ignoring case if we need to using the alternative:

\mintinline{c}{int strcasecmp(const char *s1, const char *s2);}

which is a case-insensitive version.  Here, 
\mintinline{c}{strcasecmp("Apple", "apple")} will return 
zero as the two strings are the same ignoring the cases.

The comparison functions also have byte-limited versions, 

\mintinline{c}{int strncmp(const char *s1, const char *s2, size_t n);}

and

\mintinline{c}{int strncasecmp(const char *s1, const char *s2, size_t n);}

Both will only make comparisons in the first \mintinline{c}{n}
bytes of the strings.  Thus, the comparison, 
\mintinline{c}{strncmp("apple", "apples", 5)} will result in zero as 
the two strings are equal in the first 5 bytes.

\section{Conversions}

We've previously examined the functions \mintinline{c}{atoi()} 
and \mintinline{c}{atof()} that allow you to convert strings
that hold numeric values to \mintinline{c}{int} and 
\mintinline{c}{double} values respectively.  Another way to
convert strings to numbers is to use a function similar to
the familiar \mintinline{c}{scanf()} function which reads
its string from the standard input.  The \mintinline{c}{sscanf()}
function reads its ``input'' from a string instead of the
standard input.

\begin{minted}{c}
char s[] = "103212.3214";
double x;
sscanf(s, "%lf", &x);
//x now has the value 103212.3214
\end{minted}

The \mintinline{c}{sscanf()} function differs in its first 
argument: the string that contains the value you want to parse.
Otherwise, the second two arguments are as in \mintinline{c}{scanf()}:
the format (as a string) and the variable(s) that the results
should be stored in (passed by reference).

Likewise, there is a companion \mintinline{c}{sprintf()} which
is similar to the \mintinline{c}{printf()} function, but instead
of printing to the standard output, it ``prints'' to the string.
That is, the result is placed in a string.

\begin{minted}{c}
int x = 10;
double y = 3.14;
char *s = (char *) malloc(sizeof(char) * 50);
sprintf(s, "The value of x is %d, y = %f.", x, y);
//s now contains "The value of x is 10, y = 3.140000."
\end{minted}

\section{Tokenizing}

Recall that \emph{tokenizing} is the process of splitting
up a string along some \emph{delimiter}.  For example, 
the comma delimited string, \mintinline{c}{"Smith,Joe,12345678,1985-09-08"}
contains four pieces of data delimited by a comma.  
Our aim is to split this string up into four separate 
strings so that we can process each one. The C string library 
provides a tokenizing function:

\mintinline{c}{char *strtok(char *str, const char *delim);}

which works as follows.  The first argument is the string
that you want to tokenize and the second contains the
delimiter that you want to split along.  The second argument
is actually a string and allows you to specify more than one
delimiter, but we'll restrict our attention to single 
character delimiters.  The function returns a pointer to
the first token in the string.  To get the second and
all subsequent tokens, we call \mintinline{c}{strtok()} 
again, but we pass it \mintinline{c}{NULL} as the first
argument to continue parsing the same string. 

If we pass a new string as the first argument to 
\mintinline{c}{strtok()} the tokenization process will
start over on the new string.  Note that the first
argument does not have the \mintinline{c}{const} keyword.
This is because \mintinline{c}{strtok()} \emph{will make
changes} to the string during the tokenization process.
If the string needs to be preserved, tokenization should
be performed on a deep copy of the string.

When there are no more tokens in the string, 
\mintinline{c}{strtok()} returns \mintinline{c}{NULL} 
to indicate no more tokens are in the string.  This
logic can be used to write a while loop to iterate
over each token.  Consider the following example.

\begin{minted}{c}
char data[] = "Smith,Joe,12345678,1985-09-08";
char *token = NULL;
//make the initial call to strtok:
token = strtok(data, ",");
while(token != NULL) {
  printf("token: %s\n", token);
  //get the next token:
  token = strtok(NULL, ",");
}
\end{minted}

This outputs the following:

\begin{minted}{text}
token: Smith
token: Joe
token: 12345678
token: 1985-09-08
\end{minted}


It should be noted that \mintinline{c}{strtok()} is
\emph{not} \index{reentrant} \gls{reentrant}.  It can only work on 
one string at a time.  In a multithreaded application, 
two threads cannot both use \mintinline{c}{strtok()}
or they would end up processing each other's strings.
Even in a non-threaded application, we need to be
careful.  For example, we cannot process a string using
\mintinline{c}{strtok()} in one function and then call
another function that does the same.  C does provide
a reentrant version, \mintinline{c}{strtok_r()} that
can be used.


