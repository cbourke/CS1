%!TEX root = ComputerScienceOne.tex

%%Chapter: Methods in Java

As an object-oriented programming language, functions in
Java are usually referred to as \emph{methods} and are 
essential to writing programs.  The distinction is that a 
function is usually a standalone element while methods
are functions that are members of a class.  In Java, since
everything is a class or belongs to a class, standalone
functions cannot be defined.

In Java you can define your own methods, but they
need to be placed within a class.  Usually methods that
act on data in the class (or instances of the class, see
Chapter \ref{chapter:java:objects}) or have common
functionality are placed into one class.  For example, all
the basic math methods are all part of the 
\mintinline{java}{java.lang.Math} class.  It is not uncommon
to place similar methods together into one ``utility'' class.

Java supports method overloading, so within the same 
class you can define multiple methods with the same name
as long as they differ in either the number of type of 
parameters.  For example, in the \mintinline{java}{java.lang.Math}
class, there are 3 versions of the absolute value method,
\mintinline{java}{abs()}, one that takes/returns an 
\mintinline{java}{int}, one that takes/returns a \mintinline{java}{double}
and one for \mintinline{java}{float} types.  Naming conflicts
can easily be solved by ensuring that you place your 
methods in a class/package that is unique to your application.

In Java, the 8 primitive types (\mintinline{java}{int}, \mintinline{java}{double},
\mintinline{java}{char}, \mintinline{java}{boolean}, etc.) are 
passed by value.  All object types, however, such as the 
wrapper classes \mintinline{java}{Integer}, \mintinline{java}{Double}
as well as \mintinline{java}{String}, etc. are passed by 
reference.  That is, the memory address in the \gls{jvmLabel}
is passed to the method.  This is done for efficiency, for 
objects that are ``large'' it would be inefficient to copy the
entire object into the call stack in order to pass it to a
method.  

However, though object types are passed by reference, the
method cannot necessarily change them.  Recall that the wrapper
classes \mintinline{java}{Integer}, \mintinline{java}{Double} and
the \mintinline{java}{String} class are all \gls{immutable}, meaning
that once created they cannot be modified.  Thus, even though
they are passed by reference, the method that receives them
cannot change them.

There are many \emph{mutable} objects in Java.  The 
\mintinline{java}{StringBuilder} class for example is a mutable
object.  If you pass a \mintinline{java}{StringBuilder} instance
to a method, that method is free to invoke mutator methods 
(any methods that \emph{change} the object's state).  Since
it is the \emph{same} object as in the calling method, the
calling method can ``see'' those changes.

As of Java 5, you can write and use vararg methods.  The
\mintinline{java}{System.out.printf()} method is a prime
example of this.  However, we will not discuss in detail how
to do this.  Instead, refer to standard Java documentation.
Finally, parameters are not optional in Java.  This is because
Java supports method overloading.  You can write multiple
versions of the same method that each take a different
number of arguments.  You can even design them so that
the more specific versions (with fewer arguments) invoke
the more general versions (with more arguments), passing
in sensible ``defaults.'' when doing so.

\section{Defining Methods}

Defining methods is fairly straightforward.  First you create 
class to place them in.  Then you provide the method signature
along with the body of the method.  In addition, there are 
several \emph{modifiers} that you can place in the method 
signature to specify its \emph{visibility} and whether or not
the method ``belongs'' to the class or to instances of the 
class.  This is a concept we'll explore in Chapter \ref{chapter:java:objects}.
For now, we'll only focus on what is needed to get started.

Typically, the documentation for methods is included with 
the method definition using ``Javadoc'' style comments.
Consider the following examples.

\begin{minted}{java}
/**
 * Computes the sum of the two arguments.
 * @param a
 * @param b
 * @return the sum, <code>a + b</code>
 */
public static int sum(int a, int b) {
  return (a + b);
}

/**
 * Computes the Euclidean distance between the 2-D points, 
 * (x1,y1) and (x2,y2).
 * @param x1
 * @param y1
 * @param x2
 * @param y2
 * @return
 */
public static double getDistance(double x1, double y1, 
                        double x2, double y2) {
  double xDiff = (x1-x2);
  double yDiff = (y1-y2);
  return Math.sqrt( xDiff * xDiff + yDiff * yDiff);
}

/**
 * Computes a monthly payment for a loan with the given
 * principle at the given APR (annual percentage rate) which
 * is to be repaid over the given number of terms.
 * @param principle - the amount borrowed
 * @param apr - the annual percentage rate
 * @param terms - number of terms (usually months)
 * @return
 */
public static double getMonthlyPayment(double principle, 
                           double apr, int terms) {
  double rate = (apr / 12.0);
  double payment = (principle * rate) / (1-Math.pow(1+rate, -terms));
  return payment;
}
\end{minted}

In each of the examples above, the first modifier keyword we
used was \mintinline{java}{public}.  This makes the method visible
to all other parts of the code base.  Any other piece of code can
invoke the method and take advantage of the functionality it
provides.  Alternatively, we could have used the keywords 
\mintinline{java}{private}, to make it only visible to other methods
in the same class, \mintinline{java}{protected} or ``package
protected'' by omitting the modifier altogether.  We'll discuss
these in detail later on.  We'll mostly want our methods to
be available, so we'll make most of them \mintinline{java}{public}.

The second modifier is \mintinline{java}{static} which makes it
so that the method belongs to the class itself rather than instances
of the class.  We'll discuss objects and instances in detail later
on.  For now, we'll simply make all of our methods \mintinline{java}{static}.

After the modifiers, we provide the method signature including 
the return type, its identifier (name), and its parameter list.
Method names must follow the same naming rules as variables: 
they must begin with an alphabetic character and may contain 
alphanumeric characters as well as underscores.  However, 
using modern coding conventions  we usually name methods 
using lower camel casing.

Immediately after the signature we provide a method body which
contains the code that will be run upon invocation of the method.
The method body is enclosed using opening/closing curly brackets.

\subsection{Void Methods}

The keyword \mintinline{java}{void} can be used in Java to indicate
a method does \emph{not} return a value, in which case it is
called a ``void method.''  Though it is not necessary, it is still
good practice to include a \mintinline{java}{return} statement.

\begin{minted}{java}
public static void printCopyright()
  System.out.println("(c) Bourke 2015");
}
\end{minted}

In the example above, we've also illustrated how to define a 
method that has no inputs.  

\subsection{Using Methods}

Once a method has been defined in a class, you can make
use of the method as follows.  First, you may need to import
the class itself depending on where it is.  For example, suppose
that the examples we've presented so far are contained in a 
class named \mintinline{java}{Utils} (short for ``utilities'') which 
is in a package named \mintinline{java}{unl.cse}.  Then in the 
class in which we want to call some of these functions we would
import it using 

\mintinline{java}{import unl.cse.Utils;}

prior to the class declaration.  Once the class has been imported, 
we can invoke a method in the class by first referencing the class
and using the \emph{dot operator} to access one of its methods.
For example, 

\begin{minted}{java}
int a = 10, b = 20;
int c = Utils.sum(a, b); //c contains the value 30

//invoke a method with literal values:
double dist = Utils.getDistance(0.0, 0.0, 10.0, 20.0);

//invoke a method with a combination:
double p = 1500.0;
double r = 0.05;
double monthlyPayment = Utils.getMonthlyPayment(p, r, 60);
\end{minted}

The \mintinline{java}{Utils.methodName()} syntax is used because
the methods are \mintinline{java}{static}--they belong to the class
and so must be invoked \emph{through} the class using the
class's name.  We've previously seen this syntax when using
\mintinline{java}{System.} or \mintinline{java}{Math.} with the
standard \gls{jdkLabel} library functions.

\subsection{Passing By Reference}

Java does not allow you the ability to specify if a variable
is passed by reference or by value.  Instead, all primitive types
are passed by value while all object types are passed by 
reference.  Moreover, most of the built-in types such as 
\mintinline{java}{Integer} and \mintinline{java}{String} are
immutable, even though they are passed by reference, any
method that receives them cannot change them.  Only
if the passed object is \emph{mutable} can the method
make changes to it (by invoking its methods).  

As an example, consider the following piece of code.  The
\mintinline{java}{StringBuilder} class is a mutable string
object.  You can \emph{change} the string contents stored
in a \mintinline{java}{StringBuilder} by calling one of its
many methods such as \mintinline{java}{append()}, which
will add whatever string you give it to the end.

In the main method, we create two objects, a 
\mintinline{java}{String} and a \mintinline{java}{StringBuilder}
and pass it to a method that makes changes to both by
appending \mintinline{java}{" world!"} to them.  Understand
what happens here though.  The first line in \mintinline{java}{change()}
actually creates a \emph{new} string and then changes
what the parameter variable \mintinline{java}{s} references.
The reference to the original string, \mintinline{java}{"Hello"} is lost and
replaced with the new string.  In contrast, the \mintinline{java}{StringBuilder}
instance is actually changed via its \mintinline{java}{append()}
method but is \emph{still the same} object.

\begin{minted}{java}
public class Mutability {
  
  public static void change(String s, StringBuilder sb) {    
    s = s + " world!";
    sb.append(" world!");
    
    System.out.println("change: s  = " + s);
    System.out.println("change: sb = " + sb);
  }

  public static void main(String args[]) {
    String a = "Hello";
    StringBuilder b = new StringBuilder("Hello");

    System.out.println("main: s = " + a);
    System.out.println("main: b = " + b);
    
    change(a, b);

    System.out.println("main after: s = " + a);
    System.out.println("main after: b = " + b);

  }
}
\end{minted}

To see this, observe the following output.  When we
return to the main method, the original string \mintinline{java}{s}
is unchanged (since it was immutable).  However, the
\mintinline{java}{StringBuilder} has been changed by
the method.

TODO: figure?

\begin{minted}{text}
main: s = Hello
main: b = Hello
change: s  = Hello world!
change: sb = Hello world!
main after: s = Hello
main after: b = Hello world!
\end{minted}

\section{Examples}

\subsection{Generalized Rounding}

Recall that the standard math library provides a 
\mintinline{java}{Math.round()} method that rounds a number to 
the nearest whole number.  Often, we've had need to round to 
cents as well.  We now have the ability to write a method to do 
this for us.  Before we do, however, let's think more generally.  
What if we wanted to round to the nearest tenth?  Or
what if we wanted to round to the nearest 10s or 100s place?  Let's
write a general purpose rounding method that allows us to specify
\emph{which} decimal place to round to.  

The most natural input values would be to specify the place using
an integer exponent.  That is, if we wanted to round to the nearest
tenth, then we would pass it $-1$ as $0.1 = 10^{-1}$, $-2$ if we wanted
to round to the nearest 100th, etc.  On the positive end passing in 0
would correspond to the usual round function, 1 to the nearest 10s spot, 
and so on.  

Moreover, we could demonstrate good code reuse (as well as procedural
abstraction) by \emph{scaling} the input value and reusing the functionality
already provided in the math library's \mintinline{java}{Math.round()} 
method.  We could further define a \mintinline{java}{roundToCents()} 
method that used our generalized round method.  Finally, we could place
all of these methods into \mintinline{java}{RoundUtils} Java class for
good organization.

\begin{minted}{java}
package unl.cse;

/**
 * A collection of rounding utilities
 *
 */
public class RoundUtils {
  
  /**
   * Rounds to the nearest digit specified by the place
   * argument.  In particular to the (10^place)-th digit
   *
   * @param x the number to be rounded
   * @param place the place to be rounded to
   * @return
   */
  public static double roundToPlace(double x, int place) {
    double scale = Math.pow(10, -place);
    double rounded = Math.round(x * scale) / scale;
    return rounded;
  }

  /**
   * Rounds to the nearest cent (100th place)
   * 
   * @param x
   * @return
   */
  public static double roundToCents(double x) {
    return RoundUtils.roundToPlace(x, -2);
  }

}
\end{minted}

Observe that this class does not contain a \mintinline{java}{main()}
method.  That means that this class is \emph{not} executable 
itself.  It only provides functionality to other classes in the code
base.  







