%!TEX root = ComputerScienceOne.tex

%%Chapter: Objects in Java

Java is a class-based object oriented programming language, meaning
that it facilitates the creation of objects through the use of classes.
Classes are essentially ``blueprints'' for creating instances of 
objects.  We've been implicitly using classes all along since everything
in Java must be a class or belong to a class.  Now, however, we will
start using classes in more depth rather than simply using static 
methods.

An \emph{object} is an entity that is characterized by \emph{identity}, 
\emph{state} and \emph{behavior}.  The identity of an object is an
aspect that distinguishes it from other objects.  The variables and
values that a variable takes on within an object is its state.  Typically
the variables that belong to an object are referred to as \emph{member} 
variables.  Finally, an object may also have functions that operate
on the data of an object.  In the context of object oriented programming, 
a function that belongs to an object is referred to as a (member)
\emph{method}.

As a class-based object oriented language, Java implements objects
using \emph{classes}.  A class is essentially a blue print for creating
\emph{instances} of the class.  A class simply specifies the
member variables and member methods that belong to instances of the
class.  We discuss how to create and use instances of a class below.
However, to begin, let's define a class that models a student by
defining member variables to support a first name, last name, a
unique identifier, and GPA.

To declare a class, we use the \mintinline{java}{class} keyword.
Inside the class (denoted by curly brackets), we place any code that
\emph{belongs} to the class.  To declare member variables within
a class, we use the normal variable declaration syntax, but we
do so outside any methods.

\begin{minted}{java}
package unl.cse;

public class Student {

  //member variables:
  String firstName;
  String lastName;
  int id;
  double gpa;

}
\end{minted}

Recall that a package declaration allows you to organize classes and
code within a package (directory) hierarchy.  Moreover, source code
for a class \emph{must} be in a source file with the same name 
(and is case sensitive) with the \mintinline{text}{.java} extension.
Our \mintinline{java}{Student} class would need to be in a file named
\mintinline{text}{Student.java} and would be compiled to a class named
\mintinline{text}{Student.class}.  

\section{Data Visibility}

Recall that encapsulation involves not only the grouping of data, but
the \emph{protection} of data.  The class declaration above achieves
the grouping of data.  To provide for the protection of data, Java
defines several \emph{visibility} keywords that specify what segments
of code can ``see'' the variables.  Visibility in this context determines
whether or not a segment of code can \emph{access} and/or \emph{modify}
the variable's value.  Java defines four levels of visibility using
they keywords \mintinline{java}{public}, \mintinline{java}{protected}
and \mintinline{java}{private} (a fourth level is defined by the
absence of any of these keywords).  Each of these keywords can
be applied to both member variables and member methods.

\begin{itemize}
  \item \mintinline{java}{public} -- This is the least restrictive 
    visibility level and makes the member variable visible to every
    class.
  \item \mintinline{java}{protected} -- This is a bit more restrictive
    and makes it so that the member variable is only visible to the
    code in the same class, same package,or any \emph{subclass} of the 
    class.\footnote{Subclasses are involved with \emph{inheritance}, 
    another object oriented programming concept that we will not
    discuss here).}
  \item No modifier -- the absence of any modifier means that the member
    variable is visible to any class in the same package.  This is also
  referred to as the \emph{default} or \emph{package protected} 
  visibility level.
  \item \mintinline{java}{private} -- this is the most restrictive 
    visibility level, \mintinline{java}{private} member variables are
    only visible to instances of the class itself.  As we'll see later
    this also means that the member variables are visible \emph{between}
    instances.  That is, one instance can see another instance's variables.
\end{itemize}

Table \ref{table:javaVisibilityKeywords} summarizes these four keywords
with respect to their access levels.  It is important to understand that
\emph{protection} is in the context of encapsulation and does not involve
protection in the sense of ``security.''  The protection is this context
is a design principle.  Limiting the access of variables only affects 
how the rest of the code base interacts with our class and its data.  
Encapsulation can easily be ``broken'' by other code (through reflection
or other means) and the values of variables can be accessed or modified.

\begin{table}[h]
\centering
\begin{tabular}{|l|c|c|c|c|}
\hline
Modifier & Class & Package & Subclass & World \\
\hline\hline
\mintinline{java}{public} & Y & Y & Y & Y \\
\hline
\mintinline{java}{protected} & Y & Y & Y & N \\
\hline
none (default)        & Y & Y & N & N \\
\hline
\mintinline{java}{private} & Y & N & N & N \\
\hline
\end{tabular}
\caption{Java Visibility Keywords \& Access Levels}
\label{table:javaVisibilityKeywords}
\end{table}

We will now update our class declaration to incorporate these visibility
keywords.  In general, it is best practice to make member variables 
\mintinline{java}{private} and control access to them via accessor and
mutator methods (see below) unless there is a compelling design
reason to increase their visibility.  

\begin{minted}{java}
package unl.cse;

public class Student {

  //member variables:
  private String firstName;
  private String lastName;
  private int id;
  private double gpa;

}
\end{minted}

\section{Methods}

The third aspect of encapsulation involves the grouping of methods that
act on an object's data.  Within a class, we can declare member methods
using the syntax we're already familiar with.  We declare a member
method by providing a signature and body.  We can use the same visibility
keywords as with member variables in order allow or restrict access
to the methods.  With methods, visibility and access determine whether 
or not the method may be invoked.

In contrast to the methods we defined in Chapter 
\ref{chapter:java:functions}, when defining a member method, we
do \emph{not} use the \mintinline{java}{static} keyword.  Making
a variable or a method \mintinline{java}{static} means that the 
method belongs to the class and not to instances of the class.  
Thus, a \mintinline{java}{static} method would not be able to
access the member variables or methods of an instance unless 
it also had a reference to that instance.  

Again, we add to our example by providing two \mintinline{java}{public}
methods that compute and return a result on the member variables.  
We also use javadoc style comments to document each member method.

\begin{minted}{java}
package unl.cse;

public class Student {

  //member variables:
  private String firstName;
  private String lastName;
  private int id;
  private double gpa;
  
  /**
   * Returns a formatted String of the Student's
   * name as Last, First.
   */
  public String getFormattedName() {
    return lastName + ", " + firstName;
  }
  
  /**
   * Scales the GPA, which is assumed to be on a
   * 4.0 scale to a percentage.
   */
  public double getGpaAsPercentage() {
    return gpa / 4.0;
  }
  
}
\end{minted}

\subsection{Accessor \& Mutator Methods}

Since we have made all the member variables \mintinline{java}{private},
no code outside the class may access or modify their values.  It is
generally good practice to make member variables private to restrict
this access.  However, if we still want code outside the object to
access or mutate (that is, change), we can define accessor and mutator
methods (or just simply getter and setter methods) to facilitate this.

Each getter method returns the value of the instance's variable while
each setter method takes a value and sets the instance's variable to
the new value.  It is common to name each getter/setter by prefixing
a \mintinline{java}{get} and \mintinline{java}{set} to the variable's
name using lower camel casing.  For example:

\begin{minted}{java}
public String getFirstName() {
  return firstName;
}

public void setFirstName(String firstName) {
  firstName = firstName;
} 
\end{minted}

In the setter example, there is a problem: the code has no effect.
There are two variables named \mintinline{java}{firstName}: the
instance's member variable and the variable in the method parameter.
The scoping rules of Java mean that the parameter variable name(s)
take precedent.  This code has no effect because its essentially setting
the parameter variable to itself.  It is essentially doing the following.

\begin{minted}{java}
int a = 10;
a = a;
\end{minted}

Setting a variable to itself has no effect.  To solve this, we use
something called \gls{open recursion}.  When an instance of a class
is created, for example, 

\mintinline{java}{Student s = ...;}

the reference variable \mintinline{java}{s} is how we can refer
to it.  This variable, however, exists \emph{outside} the class.  
Inside the class, we need a way to refer to the instance itself.
In Java we use the keyword \mintinline{java}{this} to refer to the
instance \emph{inside} the class.  For example, the member variables
of an instance can be accessed using the \mintinline{java}{this} 
keyword along with the dot operator (more below).  In our example, 
\mintinline{java}{this.firstName} would refer to the instance's
\mintinline{java}{firstName} and not to the parameter variable.
Even when it is not necessary to use the \mintinline{java}{this}
keyword (as in the getter example above) it is still best practice
to do so.  Our updated getters and setter methods would thus look like
the following.

\begin{minted}{java}
public String getFirstName() {
  return this.firstName;
}

public void setFirstName(String firstName) {
  this.firstName = firstName;
} 
\end{minted}

One advantage to using getters and setters (as opposed to naively
making everything public) is that you can have greater control over
the values that your variables can take.  For example, we may want
to do some data validation by rejecting \mintinline{java}{null}
values or invalid values.  For example:

\begin{minted}{java}
public void setFirstName(String firstName) {
  if(firstName == null) {
     throw new IllegalArgumentException("names cannot be null");
  } else {
    this.firstName = firstName;
  }
} 

public void setGpa(double gpa) {
  if(gpa < 0.0 || gpa > 4.0) {
     throw new IllegalArgumentException("GPAs must be in [0, 4.0]");
  } else {
    this.gpa = gpa;
  }
} 
\end{minted}

Controlling access of member variables through getters and setters
is good encapsulation.  Doing so makes your code more predictable and
more testable.  Making your member variables \mintinline{java}{public}
means that any piece of code can change their values.  There is no
way to do validation or prevent bad values.  

In fact, it is good practice to not even have setter methods.  If 
the value of member variables cannot be changed, it makes the object
\gls{immutable}.  We've seen this before with the built-in wrapper
classes (\mintinline{java}{Integer}, \mintinline{java}{String}, etc.).
Immutability is a nice property because it makes instances of the
class thread-safe.  That is, we can use instances of the class in a 
multithreaded program without having to worry about threads changing
the values of the instance on one another.  Immutable classes are also
safer to use in certain collections such as \mintinline{java}{Set}s.
Elements in a \mintinline{java}{Set} are unique; attempting to 
add a duplicate element will have no effect on the \mintinline{java}{Set}.
However, if the elements we add are mutable, we could end up
with duplicates.  This is because uniqueness is tested only when the
element is added to the set.  We could add an element that is unique,
then end up changing it so that it matches another element in the
\mintinline{java}{Set}, violating the assumption of the collection.

\section{Constructors}

If we make the (good) design decision to make our class immutable,
we still need a way to initialize the values.  This is where 
\emph{constructors} come in.  A constructor is a special method
that specifies how an object is constructed.  With built-in primitive
variables such as an \mintinline{java}{int}, the Java language
(compiler and \gls{jvmLabel}) ``know'' how to interpret and
assign a value to such a variable.  However, with user-defined
objects such as our \mintinline{text}{Student} class, we need
to specify how the object is created.

A constructor method has special syntax.  Though it may still 
one of the visibility keywords, it has no return type and its
name is the same as the class.  A constructor may take any
number of parameters.  For example, the following constructor
allows someone to construct a \mintinline{java}{Student} instance
and specify all four member variables.

\begin{minted}{java}
public Student(String firstName, String lastName, 
               int id, double gpa) {
  this.firstName = firstName;
  this.lastName = lastName;
  this.id = id;
  this.gpa = gpa;
}
\end{minted}

Java allows us to define multiple constructors, or even no constructor
at all.  If we do not specify a constructor, Java provides every 
class with a \emph{default}, no argument constructor.  This 
constructor uses default values for all member variables (zero
for numerical types, \mintinline{java}{false} for \mintinline{java}{boolean}
types, and \mintinline{java}{null} for objects).  If we do specify
a constructor, this default constructor becomes unavailable.  However,
we can always ``restore'' it by explicitly defining it.

\begin{minted}{java}
public Student() {
  this.firstName = null;
  this.lastName = null;
  this.id = 0;
  this.gpa = 0.0;
}
\end{minted}

Alternatively, we can define constructors that accept a subset of 
variable values.

\begin{minted}{java}
public Student(String firstName, String lastName) {
  this.firstName = firstName;
  this.lastName = lastName;
  this.id = 0;
  this.gpa = 0.0;
}
\end{minted}

In both of these examples, we repeated a lot of code.  One shortcut
is to make all your constructors call the most general constructor.
To invoke another constructor, we use the \mintinline{java}{this}
keyword as a method call.  For example:

\begin{minted}{java}
public Student() {
  this(null, null, 0, 0.0);
}

public Student(String firstName, String lastName) {
  this(firstName, lastName, 0, 0.0);
}
\end{minted}

Another, very useful type of constructor is the \emph{copy constructor}
that allows you to make a \emph{copy} of an instance by passing
it to a constructor.  The following example copies each of the
member variables of \mintinline{java}{s} into \mintinline{java}{this}.

\begin{minted}{java}
public Student(Student s) {
  this(s.firstName, s.lastName, s.id, s.gpa);
}
\end{minted}

\section{Usage}

Once we have defined our class and its constructors, we can 
create and use instances of it.  Just as with regular variables, 
we need to declare instances of a class by providing the type
and a variable name.  For example:

\begin{minted}{java}
Student s = null;
Student t = null;
\end{minted}

Both of these declarations are simply just \emph{reference} variables.
They may refer to an instance of the class \mintinline{java}{Student}, but
we have initialized them to \mintinline{java}{null}.  To make
these variables refer to valid instances, we invoke a constructor
by using the \mintinline{java}{new} keyword and providing arguments
to the constructor.

\begin{minted}{java}
Student s = new Student("Alan", "Turing", 1234, 3.4);
Student t = new Student("Margaret", "Hamilton", 4321, 3.9);
\end{minted}

The process of creating a new instance by invoking a constructor is
referred to as \emph{instantiation}.  Once instances have been instantiated,
they can be used by invoking their methods via the \emph{dot operator}.
The dot operator is used to select a member variable (if 
\mintinline{java}{public}) or member method.

\begin{minted}{java}
System.out.println(s.getFormattedName());

if(s.getGpa() < t.getGpa()) {
  System.out.println(t.getFirstName() + " has a better GPA");
}
\end{minted}

\section{Common Methods}

Every class in Java has several methods that they \emph{inherit} 
from the \mintinline{java}{java.lang.Object} class.  Here, we will 
highlight three of these methods.

The \mintinline{java}{toString()} method returns a \mintinline{java}{String}
representation of the object.  However, the default behavior that all
classes inherit from the \mintinline{java}{Object} class is that it
returns a string containing the fully qualified class name 
(that is package, and class name) along with a \gls{hexadecimal}
representation of the \gls{jvmLabel} memory address at which the
instance is stored.  An example with our \mintinline{java}{Student}
class may produce something like: \mintinline{text}{unl.cse.Student@272d7a10}.

We can \emph{override} this functionality and change the behavior
of the \mintinline{java}{toString()} method to return whatever we want.
To do so, we simply define the function (with a signature that matches
the \mintinline{java}{toString()} method in the \mintinline{java}{Object}
class) in our class.  For example, we can change the method to
return the values of all the class's variables in whatever format we 
want.

\begin{minted}{java}
public String toString() {
  return String.format("%s, %s (ID = %d); %.2f", 
         this.lastName,
         this.firstName,
         this.id,
         this.gpa);
}
\end{minted}

This example would return a \mintinline{java}{String} containing 
something similar to

\mintinline{text}{"Hamilton, Margaret (ID = 4321); 3.90"}

The \mintinline{java}{toString()} method is a very convenient way to
print instances of your class.  

Two other methods, the \mintinline{java}{equals()} and the
\mintinline{java}{hashCode()} method are used extensively by
other classes such as the Collections library.  The 
\mintinline{java}{equals()} method:

\mintinline{java}{public boolean equals(Object obj)}

takes another object \mintinline{java}{obj} and returns 
\mintinline{java}{true} or \mintinline{java}{false} depending on 
whether or not the instance is ``equal'' to \mintinline{java}{obj}.
Recall that identity is one of the defining characteristics of
objects.  This method is how Java achieves identity; it defines
exactly what ``equality'' means.  By default, the behavior
inherited from the \mintinline{java}{Object} class simply checks
if the object, \mintinline{java}{this} and the passed object, 
\mintinline{java}{obj} are located at the same memory address, 
essentially \mintinline{java}{(this == obj)}.  

However, \emph{conceptually} we may want different behavior.
For example, two \mintinline{java}{Student} objects may be the
same if they have the same \mintinline{java}{id} value (it is, 
after all supposed to be a \emph{unique} identifier).  Alternatively, 
we may consider two objects to be the same if \emph{every} 
member variable holds the same value.  Even with only a four
member variables, the logic can get quite complicated, especially
if we have to account for \mintinline{java}{null} values.
For this reason, many \glspl{ideLabel} provide functionality to
automatically generate such methods.  The following example was
generated by an \gls{ideLabel}.

\begin{minted}[fontsize=\scriptsize]{java}
public boolean equals(Object obj) {
  if (this == obj) {
    return true;
  }
  if (obj == null) {
    return false;
  }
  if (!(obj instanceof Student)) {
    return false;
  }
  Student other = (Student) obj;
  if (firstName == null) {
    if (other.firstName != null) {
      return false;
    }
  } else if (!firstName.equals(other.firstName)) {
    return false;
  }
  if (Double.doubleToLongBits(gpa) != 
      Double.doubleToLongBits(other.gpa)) {
    return false;
  }
  if (lastName == null) {
    if (other.lastName != null) {
      return false;
    }
  } else if (!lastName.equals(other.lastName)) {
    return false;
  }
  if (nuid != other.nuid) {
    return false;
  }
  return true;
}
\end{minted}

Related, the \mintinline{java}{hashCode()} method, 

\mintinline{java}{public int hashCode()}

returns an \emph{integer} representation of the object.  As with
the \mintinline{java}{equals()} method, the default behavior is
to return a value associated with the memory address of the instance.
In general, however, the behavior should be overridden to be based
on the \emph{entire} state of the object.  A \emph{hash} is simply
a function that maps data to a ``small'' set of values.  In this
context, we are mapping object instances to integers so that they
can be used in hash table-based data structures such as 
\mintinline{java}{HashSet} or \mintinline{java}{HashMap}.  The
\mintinline{java}{hashCode()} method is used to map an instance
to an integer so that it can be used as an \emph{index} in these
data structures.  Again, most \glspl{ideLabel} will provide
functionality to generate a good implementation for the 
\mintinline{java}{hashCode()} method (as the example below is).

How ever you design the \mintinline{java}{equals()} and 
\mintinline{java}{hashCode()} method, there is a requirement:
if two instances are equal (that is, \mintinline{java}{equals()}
returns \mintinline{java}{true}) then they \emph{must} have
the same \mintinline{java}{hashCode()} value.  This requirement
is necessary to ensure that hash table-based data structures
operate properly.  Note that it is \emph{okay} if unequal objects
have equal or unequal hash values.  This rule only applies when
the objects are equal.  

\begin{minted}{java}
public int hashCode() {
  final int prime = 31;
  int result = 1;
  result = prime * result + ((firstName == null) ? 0 : 
                              firstName.hashCode());
  long temp;
  temp = Double.doubleToLongBits(gpa);
  result = prime * result + (int) (temp ^ (temp >>> 32));
  result = prime * result + ((lastName == null) ? 0 : 
                              lastName.hashCode());
  result = prime * result + nuid;
  return result;
}
\end{minted}

\section{Composition}

Another important concept when designing classes is \emph{composition}.
Composition is a mechanism by which an object is made up of other objects.
One object is said to ``own'' an instance of another object.  We've
already seen this with our \mintinline{java}{Student} example: the
\mintinline{java}{Student} class owns two instances of the
\mintinline{java}{String} class.  

To illustrate the importance of composition, we could extend the 
design of our \mintinline{java}{Student} class to include a date of 
birth.  However, a date of birth is also made up of multiple pieces
of data (a year, a month, a date, and maybe even a time and/or locale).
We could design our own date/time class to model this, but its generally
best to use what the language already provides.  Java 8 introduced the
\mintinline{java}{java.time} package in which there are many updated
and improved classes for dealing with dates and times.  The class
\mintinline{java}{LocalDate} for example, could be used to model a
date of birth:

\begin{minted}{java}
private LocalDate dateOfBirth;
\end{minted}

We can take this concept further and have our own user-defined classes
own instances of each other.  For example, we could define a 
\mintinline{java}{Course} class and then update our \mintinline{java}{Student} class to own a collection of \mintinline{java}{Course}
objects representing a student's class schedule (this type of
collection ownership is sometimes referred to as \emph{aggregation}
rather than strict composition).

\begin{minted}{java}
private Set<Course> schedule;
\end{minted}

Both of these updates beg the question: who is responsible for instantiating
the instances of \mintinline{java}{dateOfBirth} and the 
\mintinline{java}{schedule}?  Should we force the ``outside'' user
of our \mintinline{java}{Student} class to build the 
\mintinline{java}{LocalDate} instance and pass it to a constructor?
Should we allow the outside code to simply provide us a date of birth
as a string?  Both of these are design choices that have advantages
and disadvantages that have to be considered.

What about the course schedule?  We could require that a user
provide the constructor with a pre-computed \mintinline{java}{Set}
of courses, but care must be taken.  Consider the following
(partial) example.

\begin{minted}{java}
public Student(..., Set<Course> schedule) {
  ...
  this.schedule = schedule;
}
\end{minted}

This is an example of a \emph{shallow} copy: the instance's
\mintinline{java}{schedule} variable is referencing the same collection
as the parameter variable.  If a change is made, say a course is
added, via one of these references, then the change is effectively
made for both.  If we are going to design it this way, it would
be better to make a \emph{deep} copy of the set of courses:

\begin{minted}{java}
public Student(..., Set<Course> schedule) {
  ...
  this.schedule = new HashSet<Course>(schedule);
}
\end{minted}

This is the same pattern we described above: almost every data
structure in the Java Collections library has a (deep) copy
constructor.  

Alternatively, we could make our design a bit more flexible by
allowing the construction of a \mintinline{java}{Student} instance
without having to provide a course schedule.  Instead, we could
add a method that allowed the outside code to add a course to
the \mintinline{java}{schedule}.  Something like the following.

\begin{minted}{java}
public void addCourse(Course c) {
  this.schedule.add(c);
}
\end{minted}

This adds some flexibility to our object, but removes the immutability
property.  Design is always a balance and compromise between competing
considerations.

\section{Example}

We present the full and completed \mintinline{java}{Student} class in
Code Sample \ref{code:java:studentClass}.

%\begin{listing}[H]
\newpage
%bgcolor={} resets the bgcolor to none so that the long minted
% example will break correctly across pages.
\inputminted[fontsize=\small,bgcolor={}]{java}{code/Student.java}
%use the captionof command in the caption package to give it a 
%caption and label:
\captionof{listing}{The completed Java \mintinline{java}{Student} class.
\label{code:java:studentClass}
}
%\end{listing}


