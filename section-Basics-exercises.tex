%!TEX root = ComputerScienceOne.tex

%Basic I/O - exercises

\section{Exercises}

%\begin{exer}
%Write a program to convert a temperature in degrees Fahrenheit 
%($f$) to degrees Celsius ($c$), using the equation
%  $$c = \frac{5}{9}\Big(f - 32\Big)$$
%\end{exer}

\begin{exer}
Write a program that calculates mileage deduction for income tax using the
standard rate of \$0.575 per mile.  Your program will read in a beginning
and ending odometer reading and calculate the difference and total 
deduction.  Take care that your output is in whole cents.  An example run
of the program may look like the following.

\begin{minted}{text}
INCOME TAX MILEAGE CALCULATOR
Enter beginning odometer reading--> 13505.2
Enter ending odometer reading--> 13810.6
You traveled 305.4 miles.  At $.575 per mile,
your reimbursement is $175.61
\end{minted}
\end{exer}

\begin{exer}
Write a program to compute the total ``cost'' $C$ of a loan.  That is, the total amount of interest paid over
the life of a loan.  To compute this value, use the following formula.
$$C = \frac{p\cdot i \cdot \left(1 + i\right)^{12n}}{(1+i)^{12n} - 1} * 12n - p$$
where
\begin{itemize}
  \item $p$ is the starting principle amount
  \item $i = \frac{r}{12}$ where $r$ is the APR on the interval $[0, 1]$
  \item $n$ is the number of years the loan is to be paid back
\end{itemize}
\end{exer}

\begin{exer}
Write a program to compute the annualized appreciation of an asset (say a house).  The program
should read in a purchase price $p$, a sale price $s$ and compute their difference $d = s - p$ 
(it should support a loss or gain).  Then, it should compute an appreciation rate: $r = \frac{d}{p}$ along
with an (average) \emph{annualized appreciation rate} (that is, what was the appreciation rate in each
year that the asset was held that compounded ):
$$(1 + r)^{\frac{1}{y}}-1$$
Where $y$ is the number of years (possibly fractional) the asset was held (and $r$ is on the 
scale $[0, 1]$).
\end{exer}

\begin{exer}
The annual percentage yield (APY) is a much more accurate measure of the true cost of a 
loan or savings account that compounds interest on a monthly or daily basis.  For a large enough 
number of compounding periods, it can be calculated as:
 $$APY = e^{i} - 1$$
where $i$ is the nominal interest rate ($6\% = 0.06$).  Write a program that prompts the user 
for the nominal interest rate and outputs the APY.
\end{exer}

\begin{exer}
Write a program that calculates the speed of sound ($v$, feet-per-second) in the 
air of a given temperature $T$ (in Fahrenheit). Use the formula,
 $$v = 1086 \sqrt{\frac{5T + 297}{247}}$$
Be sure your program does not lose the fractional part of the quotient
in the formula shown and format the output to three decimal places.
\end{exer}

\begin{exer}
\label{exercise:radiansToDegree}
Write a program to convert from radians to degrees using the formula
  $$deg = \frac{180\cdot rad}{\pi}$$
However, radians are on the scale $[0, 2\pi)$.  After reading input
from the user be sure to do some error checking and give an error
message if their input is invalid.
\end{exer}

\begin{exer}
Write a program to compute the Euclidean Distance between two points, 
$(x_1, y_2)$ and $(x_2, y_2)$ using the formulate:
$$\sqrt{(x_1-x_2)^2+(y_1-y_2)^2}$$
\end{exer}

\begin{exer}
Write a program that will compute the value of $\sin(x)$ using the first 4 terms of 
the Taylor series:
  $$\sin(x) \approx x - \frac{x^3}{3!} + \frac{x^5}{5!} - \frac{x^7}{7!}$$
In addition, your program will compute the \emph{absolute} difference between this 
calculation and a standard implementation of the sine function supported in
your language.  Your program should prompt the user for an
input value $x$ and display the appropriate output.  Your output should looks \emph{something}
like the following.

\begin{minted}{text}
Sine Approximation
===================
Enter x: 1.15
Sine approximation: 0.912754
Sine value:         0.912764
Difference:         0.000010
\end{minted}
\end{exer}

\begin{exer}
Write a program to compute the roots of a quadratic equation:
  $$ax^2 + bx + c = 0$$
using the well-known quadratic formula:
  $$\frac{-b \pm \sqrt{b^2 - 4ac}}{2a}$$
Your program will prompt the user for the values, $a, b, c$ and output each real root.
However, for ``invalid'' input ($a = 0$ or values that would result in complex roots), the
program will instead output a message that informs the user why that the inputs are invalid
(with a specific reason).
\end{exer}

\begin{exer}
One of Ohm's laws can be used to calculate the amount of \emph{power} 
in Watts (the rate of energy conversion; 1 joule per second) in terms of Amps (a measure of current, 1 amp = $6.241 \times 10^{18}$ 
electrons per second) and Ohms (a measure of electrical resistance).  Specifically:
  $$W = A^2 \cdot O$$
Develop a simple program to read in two of the terms from the user and output the third.
\end{exer}

\begin{exer}
Ohm's Law models the current through a conductor as follows:
 $$I = \frac{V}{R}$$
where $V$ is the voltage (in volts), $R$ is the resistance (in Ohms) and $I$ is the
current (in amps).  Write a program that, given two of these values computes the
third using Ohm's Law.

The program should work as follows: it prompts the user for units of the first value: the user
should be prompted to enter \texttt{V}, \texttt{R}, or \texttt{I} and should then be
prompted for the value.  It should then prompt for the second unit (same options) and then
the value.  The program will then output the third value depending on the input.  An example
run of the program:

\begin{minted}{text}
Current Calculator
==============
Enter the first unit type (V, R, I): V
Enter the voltage: 25.75
Enter the second unit type (V, R, I): I
Enter the current: 72
The corresponding resistance is 0.358 Ohms
\end{minted}
\end{exer}

\begin{exer}
Consider the following linear system of equations in two unknowns:
$$\begin{array}{rcl}
  ax + by & = & c \\
  dx + ey & = & f 
\end{array}$$
Write a program that prompts the user for the coefficients in such a system (prompt for $a, b, c, d, e, f$).  Then 
output a solution to the system (the values for $x, y$).  Take care to handle situations in which
the system is \emph{inconsistent}.
\end{exer}

\begin{exer}
The surface area of a sphere of radius $r$ is
  $$4\pi r^2$$
and the volume of a sphere with radius $r$ is 
  $$\frac{4}{3}\pi r^3$$
Write a program that prompts the user for a radius $r$ and outputs the surface area and volume of 
the corresponding sphere.  If the radius entered is invalid, print an error message and exit.  Your
output should look something like the following.

\begin{minted}{text}
Sphere Statistics
=================
Enter radius r: 2.5
area: 78.539816
volume: 65.449847
\end{minted}
\end{exer}

\begin{exer}
Write a program that prompts for the latitude and longitude of two locations (an origin and a 
destination) on the globe.  These numbers are in the range $[-180, 180]$ (negative values 
correspond to the western and southern hemispheres).  Your program should then compute 
the air distance between the two points using the Spherical Law of Cosines.  In particular, 
the distance $d$ is
 $$d = \arccos{(\sin(\varphi_1) \sin(\varphi_2) + \cos(\varphi_1) \cos(\varphi_2) \cos(\Delta) )} \cdot R$$
\begin{itemize}
  \item $\varphi_1$ is the latitude of location $A$, $\varphi_2$ is the latitude of location $B$
  \item $\Delta$ is the difference between location $B$'s longitude and location $A$'s longitude
  \item $R$ is the (average) radius of the earth, 6,371 kilometers
\end{itemize}
Note: the formula above assumes that latitude and longitude are measured in radians $r$, $-\pi \leq r \leq \pi$.  
See Exercise \ref{exercise:radiansToDegree} for how to convert between them.  Your program output should 
look something like the following.  

\begin{minted}{text}
City Distance
========================
Enter latitude of origin: 41.9483
Enter longitude of origin: -87.6556
Enter latitude of destination: 40.8206
Enter longitude of destination: -96.7056
Air distance is 764.990931
\end{minted}

\end{exer}

\begin{exer}
Write a program that prompts the user to enter in a number of days.  Your program should
then compute the number of years, weeks, and days that number represents.  For this 
exercise, ignore leap years (thus all years are 365 days). Your output should look something 
like the following.

\begin{minted}{text}
Day Converter
=============
Enter number of days: 1000
That is 
  2 years
  38 weeks
  4 days
\end{minted}
\end{exer}

\begin{exer}
The derivative of a function $f(x)$ can be estimated using the difference function:
  $$f'(x) \approx \frac{f(x+\Delta x) - f(x)}{\Delta x}$$
That is, this gives us an estimate of the slope of the tangent line at the point $x$.  
Write a program that prompts the user for an $x$ value and a $\Delta x$ value and 
outputs the value of the difference function for all three of the following functions:
  $$\begin{array}{rcl}
  	f(x) & = & x^2 \\
	f(x) & = & \sin(x) \\
	f(x) & = & \ln(x)
    \end{array}$$
Your output should looks something like the following.

\begin{minted}{text}
Derivative Approximation
===================
Enter x: 2
Enter delta-x: 0.1
(x^2)'  ~= 4.100000
sin'(x) ~= -0.460881
ln'x(x) ~= 0.487902
\end{minted}

In addition, your program should check for invalid inputs: $\Delta x$ cannot be zero, and $\ln(x)$ is undefined
for $x \leq 0$.  If given invalid inputs, appropriate error message(s) should be output instead.
\end{exer}

\begin{exer}
Write a program that prompts the user to enter two points in the plane, $(x_1, y_1)$ 
and $(x_2, y_2)$ which define a line segment $\ell$.  Your program should then compute 
and output an equation for the perpendicular line intersecting the \emph{midpoint} of $\ell$.  
You should take care that invalid inputs (horizontal or vertical lines) are handled appropriately.
An example run of your program would look something like the following.

\begin{minted}{text}
Perpendicular Line
====================
Enter x1: 2.5
Enter y1: 10
Enter x2: 3.5
Enter y2: 11
Original Line: 
  y = 1.0000 x + 7.5000
Perpendicular Line: 
  y = -1.0000 x + 13.5000
\end{minted}
\end{exer}

\begin{exer}
Write a program that computes the total for a bill.  The program should prompt the user for a 
sub-total.  It should then prompt whether or not the customer is entitled to an employee discount (of 15\%) by 
having them enter 1 for yes, 2 for no.  It should then compute the new sub-total and apply a 7.35\% sales tax, and
print the receipt details along with the grand total.  Take care that you properly round each operation.

An example run of the program should look something like the following.

\begin{minted}{text}
Please enter a sub-total: 100
Apply employee discount (1=yes, 2=no)? 1

Receipt
========================
Sub-Total   $  100.00
Discount    $   15.00
Taxes       $    6.25
Total       $   91.25
\end{minted}
\end{exer}



