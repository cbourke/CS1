%!TEX root = ComputerScienceOne.tex

%%Chapter: Functions in PHP

Functions are essential in PHP programming.  As we've 
already seen, PHP provides a large library of standard 
functions to perform basic input/output, math, and many 
other functions.  PHP also provides the ability to define 
and use your own functions.

PHP does not support function overloading, so when you
define a function and give it a name, that name cannot be
in conflict with any other function name in the standard library
or any other code that you might use.  Therefore, careful 
though should go into the design of your functions.  

PHP supports both call by value and call by reference.
As of PHP 5.6, vararg functions are also supported (though
earlier versions supported some vararg-like functions 
such as \mintinline{php}{printf()}).  However, we will not
go into detail here.  Finally, another feature of PHP is that
function parameters are all optional.  You may invoke
a function with a subset of the parameters; depending on
your PHP setup, it may be issue a warning that a parameter
was omitted.  However, PHP allows you to define default 
values for optional parameters.

 \section{Defining \& Using Functions}

In general, you can define functions anywhere in your PHP script
or codebase.  They can even appear \emph{after} code that
invokes them because PHP essentially \emph{hoists} the function
definitions by doing two passes of the script.  However, it is good
style to include function definitions at the top of your script or
in a separate PHP file for organization.

\subsection{Declaring Functions}

In PHP, to declare a function you use the keyword 
\mintinline{php}{function}.  Because PHP is dynamically typed, 
a function can return \emph{any} type.  Therefore, you do
not declare the return type (just as you do not declare a variable's
type).  After the \mintinline{php}{function} keyword you do
provide an identifier and parameters as the function signature.
Immediately following, you provide the function body enclosed
with opening/closing curly brackets.  

Typically, the documentation for functions is included with its
declaration.  Consider the following examples.  In these examples 
we use a commenting style known as ``doc comments.'' This 
style was originally developed for Java but has since been 
adopted by many other languages.

\begin{minted}{php}
/**
 * Computes the sum of the two arguments.
 */
function sum($a, $b) {
  return ($a + $b);
}

/**
 * Computes the Euclidean distance between the 2-D points, 
 * (x1,y1) and (x2,y2).
 */
function getDistance($x1, $y1, $x2, $y2) {
  $xDiff = ($x1-$x2);
  $yDiff = ($y1-$y2);
  return sqrt( $xDiff * $xDiff + $yDiff * $yDiff);
}

/**
 * Computes a monthly payment for a loan with the given
 * principle at the given APR (annual percentage rate) which
 * is to be repaid over the given number of terms (usually
 * months).
 */
function getMonthlyPayment($principle, $apr, $terms) {
  $rate = ($apr / 12.0);
  $payment = ($principle * $rate) / (1-pow(1+$rate, -$terms));
  return $payment;
}
\end{minted}

Function identifiers (names) follow similar naming rules as 
variables, however they do not begin with a dollar sign.  
Function names must begin with an alphabetic character 
and may contain alphanumeric characters as well as 
underscores.  However, using modern coding conventions 
we usually name functions using lower camel casing.  Another
quirk of PHP is that function names are \emph{case insensitive}.
Though we declared a function, \mintinline{php}{getDistance()}
above, it could be invoked with either \mintinline{php}{getdistance()}, 
\mintinline{php}{GETDISTANCE} or any other combination of
capital/lower case letters.  However, good code will use consistent
naming and your function calls \emph{should} match their
declaration.

The keyword \mintinline{php}{return} is used to specify the value
that is returned to the calling function.  Whatever value you end
up returning is the return type of the function.  Since you do not
specify variable or return types, functions are usually referred
to as returning a ``mixed'' type.  You could design a function that,
given one set of inputs, returns a number while another set of
inputs ends up returning a string.  

You can use the syntax \mintinline{php}{return;} to return no
value (you do not use the keyword \mintinline{text}{void}).  
In practice, however, the function ends up returning 
\mintinline{php}{null} when doing this.  

\subsection{Organizing Functions}

There are many coding standards that guide how PHP code
should be organized.  We'll only discuss a simple mechanism 
here.  One way to organize functions is to collect functions
with similar functionality into separate PHP source files.  

Suppose the functions above are in a PHP source file named 
\mintinline{text}{utils.php}.  We could include them in another
source file (our ``main'' source file) using an \mintinline{php}{include_once}
function invocation.  An example:

\begin{minted}{php}
<?php

include_once("utils.php");

//we can now use the functions in utils.php:
$p = getMonthlyPayment(1000, 0.05, 12);
\end{minted}

The \mintinline{php}{include_once} function essentially loads and
evaluates the given PHP source file at the point in the code in which
it is invoked.  The ``once'' in the function refers to the fact that if
the source file was already included in the script/code before, it will
not be included a second time.  This allows you to include the same
source file in multiple source files without a conflict.

\subsection{Calling Functions}

The syntax for calling a function is to simply provide the function 
name followed by parentheses containing values or variables 
to pass to the function.  Some examples:

\begin{minted}{php}
$a = 10, $b = 20;
$c = sum($a, $b); //c contains the value 30

//invoke a function with literal values:
$dist = getDistance(0.0, 0.0, 10.0, 20.0);

//invoke a function with a combination:
$p = 1500.0;
$r = 0.05;
$monthlyPayment = getMonthlyPayment($p, $r, 60);
\end{minted}

\subsection{Passing By Reference}

By default, all types (including numbers, strings, etc.) are passed
by value.  To be able to pass arguments by reference, we need to 
use slightly different syntax when defining our functions.

To specify that a parameter is to be passed by reference, we place
an ampersand, \mintinline{php}{&} in front of it in the function signature.
\footnote{Those familiar with pointers in C will note that this is the
exact \emph{opposite} of the C operator.}
No other syntax is necessary and when you call the function, PHP
automatically takes care of the referencing/dereferencing for you.
Consider the following examples.

\begin{minted}{php}
<?php

function swap($a, $b) {
  $t = $a;
  $a = $b;
  $b = $t;
}

function swapByRef(&$a, &$b) {
  $t = $a;
  $a = $b;
  $b = $t;
}

$x = 10;
$y = 20;

printf("x = %d, y = %d\n", $x, $y);
swap($x, $y);
printf("x = %d, y = %d\n", $x, $y);
swapByRef($x, $y);
printf("x = %d, y = %d\n", $x, $y);

?>
\end{minted}

The first function, \mintinline{php}{swap()} passes both variables
by value.  Swapping the values only affects the \emph{copies} of
the parameters.  The original variables \mintinline{php}{$x} and
\mintinline{php}{$y} will be unaffected.  In the second function, 
\mintinline{php}{swap2()}, both variables are passed by reference
as there are ampersands in front of them.  Swapping them inside
the function, swaps the original variables.  The output to this 
code is as follows.

\begin{minted}{text}
x = 10, y = 20
x = 10, y = 20
x = 20, y = 10
\end{minted}

Observe that when we invoked the function, 
\mintinline{php}{swapByRef($x, $y);} we used the same syntax
as the pass by value version.  The only syntax needed to 
pass by reference is in the function signature itself.

\subsection{Function Pointers}

Functions are just pieces of code that reside somewhere in 
memory just as variables do.  Since we can pass variables by
reference, it also makes sense that we would do the same with
functions.  

In PHP, functions are first-class citizens\footnote{Some would use
a much more restrictive definition of first-class and would \emph{not}
consider them first-class citizens in this sense} meaning that you
can assign a function to a variable just as you would a numeric value.
For example, you can do the following.

\begin{minted}{php}
$func = swapByRef;

$func($x, $y);
\end{minted}

In the example above, we assigned the function \mintinline{php}{swapByRef()}
to the variable \mintinline{php}{$func} by using its identifier.  The
variable essentially holds a reference to the \mintinline{php}{swapByRef()} 
function.  Since it refers to a function, we can also invoke the function
using the variable as in the last line.  This allows you to treat functions
as callbacks to other functions.  We will revisit this concept in Chapter
\ref{chapter:php:searchingSorting}.

\section{Examples}

\subsection{Generalized Rounding}

Recall that the standard math library provides a \mintinline{php}{round()}
function that rounds a number to the nearest whole number.  
Often, we've had need to round to cents as well.  We now have 
the ability to write a function to do this for us.  Before we do, 
however, let's think more generally.  What if we wanted to round 
to the nearest tenth?  Or what if we wanted to round to the nearest 
10s or 100s place?  Let's write a general purpose rounding function 
that allows us to specify \emph{which} decimal place to round.  

The most natural input values would be to specify the place using
an integer exponent.  That is, if we wanted to round to the nearest
tenth, then we would pass it $-1$ as $0.1 = 10^{-1}$, $-2$ if we wanted
to round to the nearest 100th, etc.  On the positive end passing in 0
would correspond to the usual round function, 1 to the nearest 10s 
spot, and so on.  

Moreover, we could demonstrate good code reuse (as well 
as procedural abstraction) by \emph{scaling} the input value 
and reusing the functionality already provided in the math 
library's \mintinline{php}{round()} function.  We could further 
define a \mintinline{php}{roundToCents()} function that used 
our generalized round function.  Consider the following.

\begin{minted}{php}
<?php

/**
 * Rounds to the nearest digit specified by the place
 * argument.  In particular to the (10^place)-th digit
 */
function roundToPlace($x, $place) {
  $scale = pow(10, -$place);
  $rounded = round(x * $scale) / $scale;
  return $rounded;
}

/**
 * Rounds to the nearest cent
 */
function roundToCents($x) {
  return roundToPlace($x, -2);
}

?>
\end{minted}

We could place these functions into a file named \mintinline{text}{round.php}
and include them in another PHP source file.

\subsection{Quadratic Roots}

Another advantage of passing variables by reference is that we
can ``return'' multiple values with one function call.  Functions
are limited in that they can only return at most one value.  But
if we pass multiple parameters by reference, the function can
manipulate the contents of them, thereby communicating (though
not strictly returning) multiple values.  

Consider again the problem of computing the roots of a quadratic
equation, 
  $$ax^2 + bx + c = 0$$
using the quadratic formula,
 $$\frac{-b \pm \sqrt{b^2 - 4ac}}{2a}$$
Since there are two roots, we may have to write two functions, 
one for the ``plus'' root and one for the ``minus'' root both of 
which take the coefficients, $a, b, c$ as arguments.  However,
if we wrote a single function that took the coefficients as parameters
by value as well as two other parameters by reference, we could
place \emph{both} root values, one in each of the by-reference
variables.

\begin{minted}{php}
function quadraticRoots($a, $b, $c, &$root1, &$root2) {
  $discriminant = sqrt($b*$b - 4*$a*$c);
  $root1 = (-$b + $discriminant) / (2*$a);
  $root2 = (-$b - $discriminant) / (2*$a);
  return;
}
\end{minted}

By using pass by reference variables, we avoid multiple functions.
Recall that there could be several ``bad'' inputs to this function.  The 
roots could be complex values, the coefficient $a$ could be zero, 
etc.  In the next chapter, we examine how we can \emph{handle}
these errors.






