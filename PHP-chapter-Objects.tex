%!TEX root = ComputerScienceOne.tex

%%Chapter: Objects in PHP

Object-oriented features have been continually added to PHP with
each new version.  Starting with version 5, PHP has had a full,
class-based object-oriented programming support, meaning
that it facilitates the creation of objects through the use of classes
and class declarations.  Classes are essentially ``blueprints'' for 
creating instances of objects.  

An \emph{object} is an entity that is characterized by \emph{identity}, 
\emph{state} and \emph{behavior}.  The identity of an object is an
aspect that distinguishes it from other objects.  The variables and
values that a variable takes on within an object is its state.  Typically
the variables that belong to an object are referred to as \emph{member 
variables}.  Finally, an object may also have functions that operate
on the data of an object.  In the context of object-oriented programming, 
a function that belongs to an object is referred to as a \emph{member method}.

A class declaration specifies the member variables and member 
methods that belong to instances of the
class.  We discuss how to create and use instances of a class below.
However, to begin, let's define a class that models a student by
defining member variables to support a first name, last name, a
unique identifier, and GPA.

To declare a class, we use the \mintinline{php}{class} keyword.
Inside the class (denoted by curly brackets), we place any code that
\emph{belongs} to the class.  To declare member variables in
a class, we specify the variable names and their \emph{visibility}
inside the class, but outside any methods in the class.

\begin{minted}{php}
class Student {

  //member variables:
  private $firstName;
  private $lastName;
  private $id;
  private $gpa;

}
\end{minted}

To organize code, it is common practice to place class declarations in
separate files with the same name as the class.  For example, this
\mintinline{php}{Student} class declaration would be placed in a 
file named \mintinline{text}{Student.php} and included in any other
script files that utilized the class.

\section{Data Visibility}

Recall that encapsulation involves not only the grouping of data, but
the \emph{protection} of data.  The class declaration above achieves
the grouping of data.  To provide for the protection of data, PHP
defines several \emph{visibility} keywords that specify what segments
of code can ``see'' the variables.  Visibility in this context determines
whether or not a segment of code can \emph{access} and/or \emph{modify}
the variable's value.  PHP defines three levels of visibility using
they keywords \mintinline{php}{public}, \mintinline{php}{protected}
and \mintinline{php}{private}.  Each of these keywords can
be applied to both member variables and member methods.

\begin{itemize}
  \item \mintinline{php}{public} -- This is the least restrictive 
    visibility level and makes the member variable visible to any
    code segment.
  \item \mintinline{php}{protected} -- This is a bit more restrictive
    and makes it so that the member variable is only visible to the
    code in the same class, or any \emph{subclass} of the 
    class.\footnote{Subclasses are involved with \emph{inheritance}, 
    another object-oriented programming concept that we will not
    discuss here.}
  \item \mintinline{php}{private} -- this is the most restrictive 
    visibility level, \mintinline{php}{private} member variables are
    only visible to instances of the class itself.  
\end{itemize}

Table \ref{table:phpVisibilityKeywords} summarizes these four keywords
with respect to their access levels.  It is important to understand that
\emph{protection} is in the context of encapsulation and does not involve
protection in the sense of ``security.''  Protection is this context
is a design principle.  Limiting the access of variables only affects 
how the rest of the code base interacts with our class and its data.  

\begin{table}[h]
\centering
\begin{tabular}{|l|c|c|c|}
\hline
Modifier & Class  & Subclass & World \\
\hline\hline
\mintinline{php}{public} & Y  & Y & Y \\
\hline
\mintinline{php}{protected} & Y  & Y & N \\
\hline
\mintinline{php}{private} & Y & N & N \\
\hline
\end{tabular}
\caption{PHP Visibility Keywords \& Access Levels}
\label{table:phpVisibilityKeywords}
\end{table}

In general, it is best practice to make member variables 
\mintinline{php}{private} and control access to them via accessor and
mutator methods (see below) unless there is a compelling design
reason to increase their visibility.

\section{Methods}

The third aspect of encapsulation involves the grouping of methods that
act on an object's data.  Within a class, we can declare member methods
using the syntax we're already familiar with.  We declare a member
method by using the keyword \mintinline{php}{function} and providing a signature and body.  We can use the same visibility
keywords as with member variables in order allow or restrict access
to the methods.  With methods, visibility and access determine whether 
or not the method may be invoked.

%TODO: discuss static keyword
%In contrast to the methods we defined in Chapter 
%\ref{chapter:php:functions}, when defining a member method, we
%do \emph{not} use the \mintinline{java}{static} keyword.  Making
%a variable or a method \mintinline{java}{static} means that the 
%method belongs to the class and not to instances of the class.  
%Thus, a \mintinline{java}{static} method would not be able to
%access the member variables or methods of an instance unless 
%it also had a reference to that instance.  

We add to our example by providing two \mintinline{php}{public}
methods that compute and return a result on the member variables.  
We also use javadoc-style comments to document each member method.

\begin{minted}{php}
class Student {

  //member variables:
  private $firstName;
  private $lastName;
  private $id;
  private $gpa;

  /**
   * Returns a formatted String of the Student's
   * name as Last, First.
   */
  public function getFormattedName() {
    return $this->lastName . ", " . $this->firstName;
  }
  
  /**
   * Scales the GPA, which is assumed to be on a
   * 4.0 scale to a percentage.
   */
  public function getGpaAsPercentage() {
    return $this->gpa / 4.0;
  }
  
}
\end{minted}

There is some new syntax in the example above.  In the member methods, 
we need a way to refer to the instance's member variables.  The
keyword \mintinline{php}{$this} is used to refer to the instance, 
this is known as \index{open recursion!in PHP} \gls{open recursion}.  

When an instance of a class is created, for example, 

\mintinline{php}{$s = new Student();}

the reference variable \mintinline{php}{$s} is how we can refer
to it.  This variable, however, exists \emph{outside} the class.  
Inside the class, we need a way to refer to the instance itself.
Since we don't have a variable \emph{inside} the class to reference 
the instance itself, PHP provides the keyword \mintinline{php}{$this}
in order to do so.  Then, to access the member variables we use
the \emph{arrow operator} (more below) and reference the member
variable via its identifier 
\emph{but with no dollar sign}.\footnote{You \emph{can} use the syntax \mintinline{php}{$this->$foo}
but it will assume that \mintinline{php}{$foo} is a string that
contains the \emph{name} of another variable, for example, 
if \mintinline{php}{$foo = "firstName";} then \mintinline{php}{$this->$foo}
would resolve to the instance's \mintinline{php}{$firstName}
variable.  This is useful if your object has been dynamically 
created by adding variables at runtime that were not part of
the original class declaration.}

\subsection{Accessor \& Mutator Methods}

Since we have made all the member variables \mintinline{php}{private},
no code outside the class may access or modify their values.  It is
generally good practice to make member variables \mintinline{php}{private} 
to restrict access.  However, if we still want code outside the object to
access or mutate (that is, change) the variable values, we can define accessor 
and mutator methods (or simply \emph{getter} and \emph{setter} 
methods) to facilitate this.

Each getter method returns the value of the instance's variable while
each setter method takes a value and sets the instance's variable to
the new value.  It is common to name each getter/setter by prefixing
a \mintinline{php}{get} and \mintinline{php}{set} to the variable's
name using lower camel casing.  For example:

\begin{minted}{php}
public function getFirstName() {
  return $this->firstName;
}

public function setFirstName($firstName) {
  $this->firstName = $firstName;
} 
\end{minted}

One advantage to using getters and setters (as opposed to naively
making everything \mintinline{php}{public}) is that you can have 
greater control over the values that your variables can take.  For 
example, we may want to do some data validation by rejecting
\mintinline{php}{null} values or invalid values.  For example:

\begin{minted}{php}
public function setFirstName($firstName) {
  if($firstName === null) {
     throw new Exception("names cannot be null");
  } else {
    $this->firstName = $firstName;
  }
} 

public function setGpa($gpa) {
  if($gpa < 0.0 || $gpa > 4.0) {
     throw new Exception("GPAs must be in [0, 4.0]");
  } else {
    $this->gpa = $gpa;
  }
} 
\end{minted}

Controlling access of member variables through getters and setters
is good encapsulation.  Doing so makes your code more predictable and
more testable.  Making your member variables \mintinline{php}{public}
means that any piece of code can change their values.  There is no
way to do validation or prevent bad values.  

In fact, it is good practice to not even have setter methods.  If 
the value of member variables cannot be changed, it makes the object
\gls{immutable}.  Immutability is a nice property because it makes 
instances of the class thread-safe.  That is, we can use instances of the class in a multithreaded program without having to worry about threads changing the values of the instance on one another.  

\section{Constructors}

If we make the (good) design decision to make our class immutable,
we still need a way to initialize the values.  This is where 
a \emph{constructor} comes in.  A constructor is a special method
that specifies how an object is constructed.  With built-in
variable types such as an numbers or strings, PHP ``knows'' how to 
interpret and assign a value.  However, with 
user-defined objects such as our \mintinline{php}{Student} class, 
we need to specify how the object is created.

Just as with functions outside of classes, PHP does not support 
function overloading inside classes.  That is, you can have
one and only one function with a given identifier (name).  Thus,
there is only \emph{one} possible constructor.  Moreover, PHP 
reserves the name \mintinline{php}{__construct} for the constructor
method.  The two underscores are a naming convention used by PHP
to denote \index{PHP!magic method} 
``magic methods'' that are reserved and have a special
purpose in the language.  Further, magic methods \emph{must} be
made \mintinline{php}{public}.  Some magic methods provide default
behavior while others do not.  For example, if you do not
define a constructor method, the default behavior will be to
create an object whose member variables all have \mintinline{php}{null} values.  

The following constructor allows a user to construct an instance of
our \mintinline{php}{Student} instance and specify all four member 
variables.

\begin{minted}{php}
public function __construct($firstName, $lastName, $id, $gpa) {
  $this->firstName = $firstName;
  $this->lastName = $lastName;
  $this->id = $id;
  $this->gpa = $gpa;
}
\end{minted}

Though we cannot define multiple constructors, we can use the 
default value feature of PHP functions to allow a user to call our
constructor with a different number of parameters.  For example, 

\begin{minted}{php}
public function __construct($firstName, $lastName, 
                            $id = 0, $gpa = 0.0) {
  $this->firstName = $firstName;
  $this->lastName = $lastName;
  $this->id = $id;
  $this->gpa = $gpa;
}
\end{minted}

\section{Usage}

Once we have defined our class and its constructors, we can 
create and use instances of it.  With regular variables, 
we simply need to assign them to an instance of an object and
their type will dynamically change to match.  To create new
instances, we invoke a constructor by using the \mintinline{php}{new} 
keyword and providing arguments to the constructor.

\begin{minted}{php}
$s = new Student("Alan", "Turing", 1234, 3.4);
$t = new Student("Margaret", "Hamilton", 4321, 3.9);
$u = new Student("John", "Smith");
\end{minted}

The process of creating a new instance by invoking a constructor is
referred to as \emph{instantiation}.  Once instances have been instantiated,
they can be used by invoking their methods via the same arrow
operator we used to access member variables.  Outside the class, however
this will only work if the member method is \mintinline{php}{public}.

\begin{minted}{php}
print $t->getFormattedName() . "\n";

if($s->getGpa() < $t->getGpa()) {
  print $t->getFirstName() . " has a better GPA\n";
}
\end{minted}

\section{Common Methods}

%TODO: move or rename this section to subsection?

Another useful magic method is the \mintinline{php}{__toString()} 
method which returns a string representation of the object.  Unlike
the constructor method, there is \emph{no} default behavior with the
\mintinline{php}{__toString()} method.  If you do not define this
function, it cannot be used (and any attempts to do so will result
in a fatal error).  We can define the method to return the values of 
all or some of the class's variables in whatever format we want.

\begin{minted}{php}
public function __toString() {
  return sprintf("%s, %s (ID = %d); %.2f", 
         $this->lastName,
         $this->firstName,
         $this->id,
         $this->gpa);
}
\end{minted}

This would return a string containing something similar to 

\mintinline{text}{"Hamilton, Margaret (ID = 4321); 3.90"}

\section{Composition}

Another important concept when designing classes is \emph{composition}.
Composition is a mechanism by which an object is made up of other objects.
One object is said to ``own'' an instance of another object.  

To illustrate the importance of composition, we could extend the 
design of our \mintinline{php}{Student} class to include a date of 
birth.  However, a date of birth is also made up of multiple pieces
of data (a year, a month, a date, and maybe even a time and/or locale).
We could design our own date/time class to model this, but its generally
best to use what the language already provides.  PHP 5.2 introduced the
\mintinline{php}{DateTime} object in which there is a lot of functionality
supporting the representation and comparison of dates and time.

We can take this concept further and have our own user-defined classes
own instances of each other.  For example, we could define a 
\mintinline{php}{Course} class and then update our 
\mintinline{php}{Student} class to own a collection of 
\mintinline{php}{Course} objects representing a student's class 
schedule (this type of collection ownership is sometimes referred to 
as \emph{aggregation} rather than composition).

Both of these design updates beg the question: who is responsible for 
instantiating the instances of \mintinline{php}{$dateOfBirth} and the 
\mintinline{php}{$schedule}?  Should we force the ``outside'' user
of our \mintinline{php}{Student} class to build a
\mintinline{php}{DateTime} instance and pass it to a constructor?
Should we allow the outside code to simply provide us a date of birth
as a string and make the constructor responsible for creating the
proper \mintinline{php}{DateTime} instance?  Do we require that a user 
create a complete array of \mintinline{php}{Course} instances and 
provide it to the constructor at instantiation?

A more flexible approach might be to allow the construction of a 
\mintinline{php}{Student} instance without having to provide a course 
schedule.  Instead, we could add a method that allowed the outside 
code to add a course to the student.  For example, 

\begin{minted}{php}
public function addCourse($c) {
  $this->schedule[] = $c;
}
\end{minted}

This adds some flexibility to our object, but removes the immutability
property.  Design is always a balance and compromise between competing
considerations.

\section{Example}

We present the full and completed \mintinline{php}{Student} class in
Code Sample \ref{code:php:studentClass}.

%\begin{listing}[H]
\newpage
%bgcolor={} resets the bgcolor to none so that the long minted
% example will break correctly across pages.
\inputminted[fontsize=\small,bgcolor={}]{php}{code/Student.php}
%use the captionof command in the caption package to give it a 
%caption and label:
\captionof{listing}{The completed PHP \mintinline{php}{Student} class.
\label{code:php:studentClass}
}
%\end{listing}


