%!TEX root = ComputerScienceOne.tex

%%Chapter: Searching & Sorting in Java

Java provides several methods to search and sort arrays as well as 
\mintinline{java}{List}s of elements of \emph{any} type.  These
methods are able to operate on collections of any type because there
are several \index{overloading}
\emph{overloaded} versions of these functions as well as
versions that take a \mintinline{java}{Comparator} object that specifies
\emph{how} the elements should be ordered.

\section{Comparators}
\index{comparator!in Java}
\index{Java!Comparators}

Let's consider a ``generic'' Quick Sort algorithm as was presented in
Algorithm \ref{algo:quickSort}.  The algorithm itself specifies how to 
sort elements, but it doesn't specify how they are \emph{ordered}.  The 
difference is subtle but important.  Quick Sort needs to 
know when two elements, $a, b$ are in order, out of order, or equivalent 
in order to decide which partition each element goes in.  However, it 
doesn't ``know'' anything about the elements $a$ and $b$ themselves.  
They could be numbers, they could be strings, they could be user-defined 
objects.  

A sorting algorithm still needs to be able to determine the proper ordering 
in order to sort.  In Java this is achieved by using a 
\mintinline{java}{Comparator} object, which is responsible for comparing 
two elements and determining their proper order.  A \mintinline{java}{Comparator<T>}
is an interface in Java that specifies a single method: 

\mintinline{java}{public int compare(T a, T b)}

\begin{itemize}
  \item A \mintinline{java}{Comparator<T>} is parameterized to operate on any
  	type \mintinline{java}{T}.  The parameterization of \mintinline{java}{T}
	is basically a variable for a type.  When you create a 
	\mintinline{java}{Comparator}, you specify an actual type just as
	we did with with \mintinline{java}{ArrayList<Integer>}.  However, the
	implementation itself is generic and can operate on any type.
  \item The method takes two instances, \mintinline{java}{a} and \mintinline{java}{b}
    whose type matches the parameterized type \mintinline{java}{T}
  \item The method returns an integer indicating the relative ordering of
  	the two elements:
	\begin{itemize}
	  \item It returns something negative, \mintinline{java}{< 0} if 
	  \mintinline{java}{a} comes before \mintinline{java}{b} (that is, $a < b$)
	  \item It returns zero if \mintinline{java}{a} and \mintinline{java}{b} are
	  equal ($a = b$)
	  \item It returns something positive, \mintinline{java}{> 0} if 
	  \mintinline{java}{a} comes after \mintinline{java}{b} (that is, $a > b$)
	\end{itemize}
\end{itemize}

There is no guarantee on the value's magnitude, it does \emph{not}
necessarily return $-1$ or $+1$; it just returns \emph{something} negative or
positive.  We've previously seen this pattern when comparing strings and
other wrapper classes.  Each of the standard types implements something similar, 
the \mintinline{java}{Comparable} interface, that specifies a \mintinline{java}{compareTo()} method with the same basic contract.  Strings for example, are
ordered in lexicographic ordering. Numeric types such as 
\mintinline{java}{Integer} and \mintinline{java}{Double} also have 
\mintinline{java}{compareTo()} methods that order elements in ascending order.
Java refers to these built-in orderings as ``natural'' orderings.  

For user-defined classes, however, we need to specify how to order them.  
We could use the \mintinline{java}{Comparable<T>} interface as the built-in
wrapper classes have, but using a \mintinline{java}{Comparator} is more 
flexible.  Making a class \mintinline{java}{Comparable} locks you into \emph{one}
``natural'' ordering.  If you wanted a different ordering, you would still have
to use a \mintinline{java}{Comparator}.  Even for the wrapper classes, if we
wanted a descending order, we would need to use a \mintinline{java}{Comparator}.

As an example, let's write a \mintinline{java}{Comparator} that orders 
\mintinline{java}{Integer} types in ascending order, matching the
``natural'' ordering already defined.

\begin{minted}{java}
Comparator<Integer> cmpInt = new Comparator<Integer>(){
  public int compare(Integer a, Integer b) {
    if(a < b) {
      return -1;
    } else if(a == b) {
      return 0;
    } else {
      return 1;
    }
  }
};
\end{minted}

This is new syntax.  When we create a \mintinline{java}{Comparator} in 
Java we are creating a new instance of a class.  However, we didn't define
a class.  We didn't use \mintinline{java}{public class...} nor did we place
it into a \mintinline{text}{.java} source file.  Instead, this is an
\index{anonymous class}
\gls{anonymous class} declaration.  The class we've created has no name; 
the \mintinline{java}{cmpInt} is the \emph{variable's} name, but the
class itself has no name, its ``anonymous.''  This syntax allows us to
define and instantiate a class \emph{inline} without having to create a
separate class source file.  This is an advantage because we generally
do not need multiple instances of a \mintinline{java}{Comparator}; they
would all have the same functionality anyway.  An anonymous class allows
us to create ad-hoc classes with a one-time (or a single purpose) use.

Another issue with this method is that it may not be able to handle 
\mintinline{java}{null} values.  When \mintinline{java}{a} and \mintinline{java}{b}
get unboxed for the comparisons, if they are \mintinline{java}{null}, 
a \mintinline{java}{NullPointerException} will be thrown.  We discuss
how to deal with this below in Section \ref{subsection:sortingObjectsWithNulls}.

Another issue with this \mintinline{java}{Comparator} is the second
case where we use the equality operator \mintinline{java}{==} to compare
values.  With the less-than operator, the two integers get unboxed and
their values are compared.  However, when we use the \mintinline{java}{==}
operator on object instances, it is comparing \emph{memory addresses} in
the \gls{jvmLabel}, not the actual values.  We could solve this issue by
rearranging our cases so that the equality is our final case, avoiding the
use of the equality operator.  Even better, however, we can exploit the
built-in natural ordering of the integers by using the \mintinline{java}{compareTo()}
method.

\begin{minted}{java}
Comparator<Integer> cmpInt = new Comparator<Integer>(){
  public int compare(Integer a, Integer b) {
    return a.compareTo(b);
  }
};
\end{minted}

What if we wanted to order integers in the opposite, descending order?  
We could simply write another \mintinline{java}{Comparator} that reversed
the ordering:

\begin{minted}{java}
Comparator<Integer> cmpIntDesc = new Comparator<Integer>(){
  public int compare(Integer a, Integer b) {
    return b.compareTo(a);
  }
};
\end{minted}

We might be tempted to instead \emph{reuse} the original comparator and
write line 3 above simply as 

\mintinline{java}{return cmpInt(b, a);}

However, Java will not allow you to reference another ``outside'' variable
like this inside a \mintinline{java}{Comparator} object.  An anonymous
class in Java is a sort-of \gls{closure}; a function that has a scope
in which variables are \emph{bound}.  In this case, the anonymous class
has a reference to the \mintinline{java}{cmpInt} variable, but it is
not necessarily ``bound'' as the variable could be reassigned outside
the \mintinline{java}{Comparator}.  If we want to do something like this, 
Java forces us to make the \mintinline{java}{cmpInt} variable 
\mintinline{java}{final} so that it cannot be changed.

To illustrate some more examples, consider the \mintinline{java}{Student} 
object we defined in Code Sample \ref{code:java:studentClass}.  The following 
Code Samples demonstrate various ways of ordering \mintinline{java}{Student} 
instances based on one or more of their member variable values.

\begin{minted}{java}
/**
 * This Comparator orders Student objects by 
 * last name/first name
 */
Comparator<Student> byName = new Comparator<Student>() {
  @Override
  public int compare(Student a, Student b) {
    if(a.getLastName().equals(b.getLastName())) {
      return a.getFirstName().compareTo(b.getFirstName());
    } else {
      return a.getLastName().compareTo(b.getLastName());
    }
  }
};
\end{minted}

\begin{minted}{java}
/**
 * This Comparator orders Student objects by 
 * last name/first name in descending (Z-to-A) order
 */
Comparator<Student> byNameDesc = new Comparator<Student>() {
  @Override
  public int compare(Student a, Student b) {
    if(b.getLastName().equals(a.getLastName())) {
      return b.getFirstName().compareTo(a.getFirstName());
    } else {
      return b.getLastName().compareTo(a.getLastName());
    }
  }
};
\end{minted}

\begin{minted}{java}
/**
 * This Comparator orders Student objects by 
 * id in ascending order
 */
Comparator<Student> byId = new Comparator<Student>() {
  @Override
  public int compare(Student a, Student b) {
  	return a.getId().compareTo(b.getId());
  }
};
\end{minted}

\begin{minted}{java}
/**
 * This Comparator orders Student objects by 
 * GPA in descending order
 */
Comparator<Student> byGpa = new Comparator<Student>() {
  @Override
  public int compare(Student a, Student b) {
  	return b.getGpa().compareTo(a.getGpa());
  }
};
\end{minted}

\section{Searching \& Sorting}

We now turn our attention to the search and sorting methods provided by
the \gls{jdkLabel}.  Most of these methods are generic implementations 
that either sort an array or collection using the natural ordering or 
take a take a parameterized \mintinline{java}{Comparator<T>} to determine
the ordering.

\subsection{Searching}

\subsubsection{Linear Search}

Java \mintinline{java}{Set} and \mintinline{java}{List} collections provide
several linear search methods.  Both have a 
\mintinline{java}{public boolean contains(Object o)} method that returns
\mintinline{java}{true} or \mintinline{java}{false} depending on whether or
not an object matching \mintinline{java}{o} is in the collection.  
The \mintinline{java}{List} collection
has an additional \mintinline{java}{public int indexOf(Object o)} method
that returns the index of the first matched element and $-1$ if no such 
element is found (\mintinline{java}{Set}s
are unordered and have no indices, so this method would not apply).  
All of these functions are equality-based and they do not take a 
\mintinline{java}{Comparator} object.  Instead, the elements are compared
using the object's \mintinline{java}{equals()} (and possibly 
\mintinline{java}{hashCode()}) method.  The first element \mintinline{java}{x}
such that  \mintinline{java}{x.equals(o)} returns \mintinline{java}{true} 
is the element that is determined to match.  For this reason, it is important
to override both of these methods when designing objects (see Section \ref{subsection:importanceOfEqualsHashCode}).

\subsubsection{Binary Search}

The \mintinline{java}{Arrays} and \mintinline{java}{Collections} classes
provide many variations on methods that implement binary search.  All
of these methods assume that the array or \mintinline{java}{List} being
searched are sorted appropriately (you cannot use binary search on a 
\mintinline{java}{Set} as it is an unordered collection).

The \mintinline{java}{Arrays} class provides several 
\mintinline{java}{binarySearch()} methods, one for each primitive type 
as well as variations that allow you to search within a specified range 
of elements.
The most generally useful version is as follows:

\mintinline{java}{public static <T> int binarySearch(T[] a, T key, Comparator<T> c)}

That is, it takes an array of elements as well as key and a 
\mintinline{java}{Comparator} all of the same type \mintinline{java}{T}.
It returns an integer representing the index at which it finds the first
matching element (there is no guarantee that the first element in the
sorted list is returned).  If no match is found, then the method returns
something negative.\footnote{It actually returns a negative value corresponding
to the \emph{insertion point} at which the element would be if it had
existed.}  Another version has the same behavior but can be called without
a \mintinline{java}{Comparator}, relying instead on the natural ordering of
elements.  For this variation, the type of elements \emph{must} implement
the \mintinline{java}{Comparable} interface.

The \mintinline{java}{Collections} class provides a similar method, 

\mintinline{java}{public static <T> int binarySearch(List<T> list, T key, Comparator<T> c)}

The only difference is that the method takes a \mintinline{java}{List} instead
of an array.  Otherwise, the behavior is the same.  We present several examples
in Code Sample \ref{code:java:searchExamples}.

\begin{listing}[H]
\begin{minted}{java}
ArrayList<Student> roster = ...
		
Student castroKey = null;
int castroIndex;
		
//create a "key" that will match according to the 
// Student.equals() method
castroKey = new Student("Starlin", "Castro", 131313, 3.95);
castroIndex = roster.indexOf(castroKey);
System.out.println("at index " + castroIndex + ": " + 
  roster.get(castroIndex));

//create a key with only the necessary fields to match 
/  the comparator
castroKey = new Student("Starlin", "Castro", 0, 0.0);
//sort the list according to the comparator
Collections.sort(roster, byName);
castroIndex = Collections.binarySearch(roster, castroKey, byName);
System.out.println("at index " + castroIndex + ": " + 
  roster.get(castroIndex));

//create a key with only the necessary fields to match 
//  the comparator
castroKey = new Student(null, null, 131313, 0.0);
//sort the list according to the comparator
Collections.sort(roster, byId);
castroIndex = Collections.binarySearch(roster, castroKey, byId);
System.out.println("at index " + castroIndex + ": " + 
  roster.get(castroIndex));
\end{minted}
\caption{Java Search Examples}
\label{code:java:searchExamples}
\end{listing}

\subsection{Sorting}

As with searching, the \mintinline{java}{Arrays} and 
\mintinline{java}{Collections} classes provide parameterized sorting 
methods to sort arrays and \mintinline{java}{List}s.  In Java 6 and prior, 
the implementation was a hybrid merge/insertion sort.  Java 7
switched the implementation to Tim Sort.  Here too, there are several 
variations that rely on the natural ordering or allow you to sort a 
subpart of the collection.

The \mintinline{java}{Arrays} provides the following method, which sorts 
arrays of a particular type with a \mintinline{java}{Comparator} for that type.

\mintinline{java}{public static <T> void sort(T[] a, Comparator<T> c)}

Likewise, \mintinline{java}{Collections} provides the following method.

\mintinline{java}{public static <T> void sort(List<T> list, Comparator<T> c)}

Several examples of the usage of these methods are presented in Code
Sample \ref{code:java:sortExamples}.

\begin{listing}[H]
\begin{minted}{java}
List<Student> roster = ...
Student rosterArr[] = ...
Comparator byName = ...
Comparator byGPA = ...

//sort by name:
Collections.sort(roster, byName);
Arrays.sort(rosterArr, byName);

//sort by GPA:
Collections.sort(roster, byGPA);
Arrays.sort(rosterArr, byGPA);
\end{minted}
\caption{Using Java Collection's Sort Method}
\label{code:java:sortExamples}
\end{listing}

\section{Other Considerations}

\subsection{Sorted Collections}

The Java Collections framework also provides several sorted data structures.
Though an ordered collection such as a \mintinline{java}{List} can be sorted
calling the \mintinline{java}{Collections.sort()} method, a sorted data
structure \emph{maintains} an ordering.  As you add elements to 
the data structure, they are inserted in the appropriate spot.  Such data 
structures also prevent you from arbitrarily inserting elements in any 
order.  

Java defines the \mintinline{java}{SortedSet<T>} interface for sorted 
collections with a \mintinline{java}{TreeSet<T>} implementation.\footnote{As 
the name implies, the implementation is a balanced binary
search tree, in particular a red-black tree.}  You can construct a
\mintinline{java}{TreeSet<T>} by providing it a \mintinline{java}{Comparator<T>}
to use to maintain the ordering (alternatively, you may omit the
\mintinline{java}{Comparator} and the \mintinline{java}{TreeSet} will
use the natural ordering of elements).

\begin{minted}{java}
Comparator<Integer> cmpIntDesc = new Comparator<Integer>() { ... };
SortedSet<Integer> sortedInts = new TreeSet<Integer>(cmpIntDesc);
sortedInts.add(10);
sortedInts.add(30);
sortedInts.add(20);
//sortedInts has elements 30, 20, 10 in descending order

SortedSet<Student> sortedStudents = new TreeSet<Student>(byName);
//add students, they will be sorted by name
for(Student s : sortedStudents) {
  System.out.println(s);
}
\end{minted}

When using a \mintinline{java}{SortedSet} it is important that the 
\mintinline{java}{Comparator} properly orders all elements.  A 
\mintinline{java}{SortedSet} is still a \mintinline{java}{Set}: it
does not allow duplicate elements.  If the \mintinline{java}{Comparator} 
you use returns zero for any element that you attempt to insert, it
will not be inserted.  To demonstrate how this might fail, consider
our \mintinline{java}{byName} \mintinline{java}{Comparator} for 
\mintinline{java}{Student} objects.  Suppose two students have
the same first name and last name, ``John Doe'' and ``John Doe,'' but
have different IDs (as they are different people).
The \mintinline{java}{Comparator} would return 0 for these
two objects because they have the same first/last name, even though
they are distinct people.  Only one of these objects could 
exist in the \mintinline{java}{SortedSet}.
To solve this problem, it is important to define \mintinline{java}{Comparator}s
that ``break ties'' appropriately.  In this case, we would want to 
modify the \mintinline{java}{Comparator} to return a comparison
of IDs if the first \emph{and} last name are the same.  

\subsection{Handling \mintinline{java}{null} values}
\label{subsection:sortingObjectsWithNulls}

When sorting collections or arrays of objects, we may need to consider 
the possibility of uninitialized \mintinline{java}{null} objects.  Some
collections allow \mintinline{java}{null} element(s) while others do not.
How we handle these is a design decision.  We could ignore it, in which 
case such elements would likely result in a \mintinline{java}{NullPointerException} 
and expect the user to prevent or handle such instances.  This may be 
the preferable choice in most instances, in fact.

Alternatively, we could handle \mintinline{java}{null} objects in the 
design of our \mintinline{java}{Comparator}.  Code Sample 
\ref{code:java:sortingObjectsWithNulls} presents a \mintinline{java}{Comparator} 
for our \mintinline{java}{Student} class that orders \mintinline{java}{null}
instances first.

\begin{listing}[H]
\begin{minted}{java}
Comparator<Student> byNameWithNulls = new Comparator<Student>() {
  @Override
  public int compare(Student a, Student b) {
    if(a == null && b == null) {
      return 0;
    } else if(a == null && b != null) {
      return -1;
    } else if(a != null && b == null) {
      return 1;
    } else {
      if(a.getLastName().equals(b.getLastName())) {
        return a.getFirstName().compareTo(b.getFirstName());
      } else {
        return a.getLastName().compareTo(b.getLastName());
      }
    }
  }
};
\end{minted}
\caption{Handling Null Values in Java Comparators}
\label{code:java:sortingObjectsWithNulls}
\end{listing}

This solution only handles \mintinline{java}{null} \mintinline{java}{Student}
values not \mintinline{java}{null} values within the \mintinline{java}{Student}
object.  If the getters used in the \mintinline{java}{Comparator} return
\mintinline{java}{null} values, then we could still have 
\mintinline{java}{NullPointerException}s.  


\subsection{Importance of \mintinline{java}{equals()} and \mintinline{java}{hashCode()} Methods}
\label{subsection:importanceOfEqualsHashCode}

Recall that in Chapter \ref{chapter:java:objects} we briefly discussed the
importance of overloading the \mintinline{java}{equals()} and 
\mintinline{java}{hashCode()} methods of our objects.  We reiterate this
in the context of searching.  Recall that the \mintinline{java}{Set} and
\mintinline{java}{List} collections provide linear search methods that do
not require the use of a \mintinline{java}{Comparator}.  This is because
the collections use the object's \mintinline{java}{equals()} to determine 
when a particular instance has been found.  In the case of hash-based 
data structures such as \mintinline{java}{HashSet}, the
\mintinline{java}{hashCode()} method is also important.
To illustrate the importance of these methods, consider the following 
code.

\begin{minted}{java}
Student a = new Student("Joe", "Smith", 1234, 3.5);		
Student b = new Student("Joe", "Smith", 1234, 3.5);
		
Set<Student> s = new HashSet<Student>();
s.add(a);
s.add(b);
\end{minted}

If we \emph{do not} override the \mintinline{java}{equals()} and
\mintinline{java}{hashCode()} methods, at the end of this code 
snippet, the set will contain \emph{both} of the \mintinline{java}{Student}
instances even though they are \emph{conceptually} the same object
(all of their member variables will have the same values).  

However, if we \emph{do} override the \mintinline{java}{equals()} and
\mintinline{java}{hashCode()} methods as described in Section 
\ref{section:javaCommonMethods}, then the code snippet above would
result in the \mintinline{java}{Set} only having one object (the
first one added).  This is because during the attempt to add the second
instance, the \mintinline{java}{equals()} method would have returned
\mintinline{java}{true} when compared to the first instance and the
duplicate element would not have been added to the \mintinline{java}{Set}
(as \mintinline{java}{Set}s do not allow duplicates).

Also recall that we saw some of the advantages of immutable objects. 
Again, we point out another advantage.  Suppose that we properly 
override the \mintinline{java}{equals()} and the \mintinline{java}{hashCode()} 
methods in our \mintinline{java}{Student} object.  Further, suppose
that we make our \mintinline{java}{Student} class mutable: allowing
a user to change the ID via a setter.  Consider the following code snippet.

\begin{minted}{java}
Student a = new Student("Joe", "Smith", 1234, 3.5);		
Student b = new Student("Joe", "Smith", 5679, 3.5);
		
Set<Student> s = new HashSet<Student>();
s.add(a);
s.add(b);
//s now contains both students
b.setId(1234);
\end{minted}

When we instantiate the two \mintinline{java}{Student} instances, they
are distinct as their IDs are different.  So when we add them to the set, 
their \mintinline{java}{equals()} method returns \mintinline{java}{false}
and they are both added to the \mintinline{java}{Set}.  However, when 
we change the ID using the setter on the last line, both objects are now
identical, violating the no-duplicates property of the \mintinline{java}{Set}.
This is not a failing of the \mintinline{java}{Set} class, just one of
the many consequences of designing mutable objects.  


\subsection{Java 8: Lambda Expressions}
\index{lambda expression}

Java 8 introduced a lot of functional-style syntax, including 
\glspl{lambda expression}.  Lambda expressions are essentially anonymous
functions that can be passed around to other methods or objects.
One use for lambda expressions is if we want to sort a \mintinline{java}{List}
with respect to one data field, we need not build a full 
\mintinline{java}{Comparator}.  Instead we can use the following syntax.

\begin{minted}{java}
List<Student> roster = ...

roster.sort((a, b) -> a.getLastName().compareTo(b.getLastName()));
\end{minted}

The new syntax is the use of the arrow operator as a lambda 
expression.  In this case, it maps a pair, \mintinline{java}{(a, b)}
to the value of the expression 
\mintinline{java}{a.getLastName().compareTo(b.getLastName())}
(which takes advantage of the fact that strings implement the 
\mintinline{java}{Comparable} interface).  With respect to 
\mintinline{java}{Comparator}s, we can use the following 
syntax to build a more complex ordering.

\begin{minted}{java}
Comparator<Student> c = 
  (a, b) -> a.getLastName().compareTo(b.getLastName());
c = c.thenComparing( (a, b) -> 
                     a.getFirstName().compareTo(b.getFirstName()));
c = c.thenComparing( (a, b) -> a.getGpa().compareTo(b.getGpa()));

//pass the comparator to the sort method:
roster.sort(c);
\end{minted}

We can make this even more terse using method references, another
new feature in Java 8.

\begin{minted}{java}
//using getters as key extractors 
myList.sort(
  Comparator.comparing(Student::getLastName)
            .thenComparing(Student::getFirstName)
            .thenComparing(Student::getGpa));
\end{minted}

There are several other convenient methods provided by the updated
\mintinline{java}{Comparator} interface.  For example, 
the \mintinline{java}{reversed()} member method returns a new
\mintinline{java}{Comparator} that defines the reversed order.
The static methods, \mintinline{java}{nullsFirst()} and 
\mintinline{java}{nullsLast()} can be used to modify a 
\mintinline{java}{Comparator} to order \mintinline{java}{null}
values.
