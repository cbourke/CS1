%!TEX root = ComputerScienceOne.tex

%%Chapter: Loops in PHP
\index{loops!in PHP}
\index{PHP!loops}

PHP supports while loops, for loops, and do-while loops using the keywords
\mintinline{php}{while}, \mintinline{php}{for}, and \mintinline{php}{do} (along with 
another \mintinline{php}{while}).  Continuation conditions for loops are 
enclosed in parentheses, \mintinline{php}{(...)} and blocks of code
associated with the loop are enclosed in curly brackets.  

\section{While Loops}

Code Sample \ref{code:php:whileLoop} contains an example of a basic
while loop in PHP.  As with conditional statements, our code style
places the opening curly bracket on the same line as the \mintinline{php}{while}
keyword and continuation condition.  The inner block of code is also
indented and all lines in the block are indented to the same level.

\begin{listing}
\begin{minted}{php}
$i = 1; //Initialization
while($i <= 10) {  //continuation condition
  //perform some action
  $i++; //iteration
}
\end{minted}
  \caption{While Loop in PHP}
  \label{code:php:whileLoop}
\end{listing}

In addition, the continuation condition does \emph{not} contain 
a semicolon since it is not an executable statement.  Just as with
an if-statement, if we \emph{had} placed a semicolon it would 
have led to unintended results.  Consider the following:

\begin{minted}{php}
while($i <= 10); {
  //perform some action
  $i++; //iteration
}
\end{minted}

A similar problem occurs.  The \mintinline{php}{while} keyword and
continuation condition bind to the next executable statement or
code block.  As a consequence of the semicolon, the executable
statement that gets bound to the while loop is \emph{empty}.  What
happens is even worse: the program will enter an infinite loop.  To
see this, the code is essentially equivalent to the following:

\begin{minted}{php}
while($i <= 10) {
}
{
  //perform some action
  $i++; //iteration
}
\end{minted}

In the while loop, we never increment the counter variable \mintinline{php}{$i}, 
the loop does nothing, and so the computation will continue on
forever!  It
is valid PHP and will run, but obviously won't work as intended.
Avoid this problem by using proper syntax.

Another common use for a while loop is a flag-controlled loop in which
we use a Boolean flag rather than an expression to determine if a loop
should continue or not.  Since PHP has built-in Boolean types, we can 
use a variable along with the keywords \mintinline{php}{true} and \mintinline{php}{false}.  An example
can be found in Code Sample \ref{code:php:flagControlledWhileLoop}.

\begin{listing}
\begin{minted}{php}
$i = 1;
$flag = true;
while($flag) {
  //perform some action
  $i++; //iteration  
  if($i>10) {
    $flag = false;
  }
}
\end{minted}
  \caption{Flag-controlled While Loop in PHP}
  \label{code:php:flagControlledWhileLoop}
\end{listing}

\section{For Loops}

For loops in PHP use the familiar syntax of placing the initialization, continuation
condition, and iteration on the same line as the \mintinline{php}{for} keyword.
An example can be found in Code Sample \ref{code:php:forLoop}.

\begin{listing}[H]
\begin{minted}{php}
$i;
for($i=1; $i<=10; $i++) {
  //perform some action
}
\end{minted}
  \caption{For Loop in PHP}
  \label{code:php:forLoop}
\end{listing}

Semicolons are placed at the end of the initialization and
continuation condition, but \emph{not} the iteration statement.  
With while
loops, the opening curly bracket is placed on the same line as the \mintinline{php}{for}
keyword.  Code within the loop body is indented, all at the same indentation level.

\section{Do-While Loops}

PHP also supports do-while loops.  Recall that the difference between a
while loop and a do-while loop is when the continuation condition is checked.
For a while loop it is \emph{prior} to the beginning of the loop body and in
a do-while loop it is at the \emph{end} of the loop.  This means that a do-while 
always executes \emph{at least once}.  An example can be found in Code
Sample \ref{code:php:doWhileLoop}.

\begin{listing}
\begin{minted}{php}
$i;
do {
  //perform some action
  $i++;
} while($i <= 10);
\end{minted}
  \caption{Do-While Loop in PHP}
  \label{code:php:doWhileLoop}
\end{listing}

The opening curly bracket is again on the same
line as the keyword \mintinline{php}{do}.  The \mintinline{php}{while} keyword and
continuation condition are on the same line as the closing curly bracket.
In a slight departure from previous syntax, a semicolon \emph{does} appear
at the end of the continuation condition even though it is not an
executable statement.

\section{Foreach Loops}

Finally, PHP supports foreach loops using the keyword \mintinline{php}{foreach}.
Some of this will be a preview of Section \ref{chapter:php:arrays} where we discuss 
arrays in PHP,\footnote{Actually, PHP supports \emph{associative} arrays, which are
not the same thing as traditional arrays.} In short you can iterate over the 
elements of an array as follows.

\begin{minted}{php}
$arr = array(1.41, 2.71, 3.14);
foreach($arr as $x) {
  //x now holds the "current" element in arr
}
\end{minted}

In the \mintinline{php}{foreach} syntax we specify the array we want to iterate
over, \mintinline{php}{$arr} and use the keyword \mintinline{php}{as}.  The
last element in the statement is the variable name that we want to use within
the loop.  This should be read as ``\mintinline{php}{foreach} element \mintinline{php}{$x}
in the array \mintinline{php}{$arr}...''.  Inside the loop, the variable \mintinline{php}{$x}
will be automatically updated on each iteration to the next element in \mintinline{php}{$arr}.

\section{Examples}

\subsection{Normalizing a Number}

Let's revisit the example from Section \ref{subsection:whileLoopExample} in which 
we \emph{normalize} a number by continually dividing it by 10 until it is less 
than 10.  The code in Code Sample \ref{code:php:normalizeWhileLoop} specifically
refers to the value $32145.234$ but would work equally well with any value of 
\mintinline{php}{$x}.

\begin{listing}[H]
\begin{minted}{php}
$x = 32145.234;
$k = 0;
while($x > 10) {
  $x = $x / 10;
  $k++;
}
\end{minted}
  \caption{Normalizing a Number with a While Loop in PHP}
  \label{code:php:normalizeWhileLoop}
\end{listing}

\subsection{Summation}

Let's revisit the example from Section \ref{subsection:summationExample} in which
we computed the sum of integers $1 + 2 + \cdots + 10$.  The code is presented in
Code Sample \ref{code:php:summationForLoop}

\begin{listing}[H]
\begin{minted}{php}
$sum = 0;
for($i=1; $i<=10; $i++) {
  $sum += $i;
}
\end{minted}
  \caption{Summation of Numbers using a For Loop in PHP}
  \label{code:php:summationForLoop}
\end{listing}

We could easily generalize this code.  Instead of computing
a sum up to a particular number, we could have written it to sum up to another
variable \mintinline{php}{$n}, in which case the for loop would instead look like the
following.

\begin{minted}{php}
for($i=1; $i<=$n; $i++) {
  $sum += $i;
}
\end{minted}

\subsection{Nested Loops}

Recall that you can write loops within loops.  The inner loop will execute fully 
\emph{for each} iteration of the outer loop.  An example of two nested of
loops in PHP can be found in Code Sample \ref{code:php:nestedForLoops}.

\begin{listing}[H]
\begin{minted}{php}
$n = 10;
$m = 20;
for($i=0; $i<$n; $i++) {
  for($j=0; $j<$m; $j++) {
    printf("(i, j) = (%d, %d)\n", $i, $j);
  }
}
\end{minted}
  \caption{Nested For Loops in PHP}
  \label{code:php:nestedForLoops}
\end{listing}

The inner loop executes for $j = 0, 1, 2, \ldots, 19 < m = 20$ for a total
of 20 iterations.  However, it executes 20 times \emph{for each} iteration of
the outer loop.  Since the outer loop executes for $i = 0, 1, 2, \ldots, 9 < n = 10$, 
the total number of times the \mintinline{php}{printf()} statement executes is
$10 \times 20 = 200$.  In this example, the sequence $(0, 0), (0, 1), (0, 2), \ldots, (0,19), (1, 0), \ldots, (9, 19)$
will be printed.

\subsection{Paying the Piper}

Let's adapt the solution for the loan amortization schedule we developed in 
Section \ref{subsection:loanAmortization}.  First, we'll read the principle, 
terms, and interest as command line inputs. Adapting the formula for 
the monthly payment and using the standard
math library's \mintinline{php}{pow()} function, we get

\begin{minted}{php}
  $monthlyPayment = ($monthlyInterestRate * $principle) / 
  	(1 - pow( (1 + $monthlyInterestRate), -$n));
\end{minted}

The monthly payment may come out to be a fraction of a cent, say \$43.871.  For 
accuracy, we need to ensure that all of the figures for currency are rounded
to the nearest cent.  The standard math library does have a \mintinline{php}{round()}
function, but it only rounds to the nearest whole number, not the nearest
100th.

However, we can \emph{adapt} the ``off-the-shelf'' solution to fit our needs.  
If we take the number, multiply it by 100, we get 4387.1 which we can
now round to the nearest whole number, giving us 4387.  We can then 
divide by 100 to get a number that has been rounded to the nearest 100th!

\mintinline{php}{$monthlyPayment = round($monthlyPayment * 100) / 100;}

We can use the same trick to round the monthly interest payment and any
other number expected to be whole cents.  To output our numbers, we use
\mintinline{php}{printf()} and take care to align our columns to make make it look 
nice.  To finish our adaptation, we handle the final month separately to account
for an over/under payment due to rounding.  The full solution can be found
in Code Sample \ref{code:php:loanAmortization}.

\begin{listing}
\begin{minted}[fontsize=\footnotesize]{php}
<?php
  if($argc != 4) {
    printf("Usage: %s principle apr terms\n", $argv[0]);
    exit(1);
  }

  $principle = floatval($argv[1]);
  $apr = floatval($argv[2]);
  $n = intval($argv[3]);

  $balance = $principle;
  $monthlyInterestRate = $apr / 12;

  //monthly payment
  $monthlyPayment = ($monthlyInterestRate * $principle) /
         (1 - pow( (1 + $monthlyInterestRate), -$n));
  //round to the nearest cent
  $monthlyPayment = round($monthlyPayment * 100) / 100;

  printf("Principle: $%.2f\n", $principle);
  printf("APR: %.4f%%\n", $apr * 100.0);
  printf("Months: %d\n", $n);
  printf("Monthly Payment: $%.2f\n", $monthlyPayment);

  //for the first n-1 payments in a loop:
  for($i=1; $i<$n; $i++) {
    //  compute the monthly interest, rounded:
    $monthlyInterest =
      round( ($balance * $monthlyInterestRate) * 100) / 100;
    //  compute the monthly principle payment
    $monthlyPrinciplePayment = $monthlyPayment - $monthlyInterest;
    //  update the balance
    $balance = $balance - $monthlyPrinciplePayment;
    //  print i, monthly interest, monthly principle, new balance
    printf("%d\t$%10.2f  $%10.2f  $%10.2f\n", $i, $monthlyInterest,
                         $monthlyPrinciplePayment, $balance);
  }

  //handle the last month and last payment separately
  $lastInterest = round( ($balance * $monthlyInterestRate) * 100) / 100;
  $lastPayment = $balance + $lastInterest;

  printf("Last payment = $%.2f\n", $lastPayment);
?>
\end{minted}
\caption{Loan Amortization Program in PHP}
\label{code:php:loanAmortization}
\end{listing}


