%!TEX root = ComputerScienceOne.tex

%%Chapter: Error Handling in Java

Java supports error handling through the use of exceptions.
Java has many different predefined types of exceptions that
you can freely use in your own code.  It also allows you to
define your own exception types by creating new classes
that \emph{inherit} from the predefined classes.  Java uses
the standard \mintinline{java}{try-catch-finally} control structure
to handle exceptions and allows you to \mintinline{java}{throw}
your own exceptions.

\section{Exceptions}

Java defines a base class named \mintinline{java}{Throwable}
that an object type that can be thrown using the keyword
\mintinline{java}{throw}.  There are two majors subtypes of
\mintinline{java}{Throwable}: \mintinline{java}{Error} and 
\mintinline{java}{Exception}.  The \mintinline{java}{Error} class
is used primarily for \emph{fatal} errors such as the \gls{jvmLabel}
running out of memory or some other extreme case that your
code cannot reasonably be expected to recover from.  

There are dozens of types of exceptions that are subclasses
of the standard Java \mintinline{java}{Exception} class defined by
the \gls{jdkLabel} including \mintinline{java}{IOException} 
(and its subclasses such as \mintinline{java}{FileNotFoundException})
or \mintinline{java}{SQLException} (when working with \gls{sqlLabel}
databases).

An important subclass of \mintinline{java}{Exception} is 
\mintinline{java}{RuntimeException} which represent \emph{unchecked}
exceptions that do \emph{not} need to be explicitly caught
(see Section \ref{section:java:checkedExceptions} below for 
further details).  We'll mostly focus on this type of exception.

\subsection{Catching Exceptions}

To catch an exception in Java you can use the standard
\mintinline{java}{try-catch} control block (and optionally
use the \mintinline{java}{finally} block to clean up any
resources).  Let's take, for example, the simple task of
reading input from a user using \mintinline{java}{Scanner}
and manually parsing its value into an integer.  If the
user enters a non-numeric value, parsing will fail
and result in a \mintinline{java}{NumberFormatException}
that we can then catch and handle. For example, 

\begin{minted}{java}
Scanner s = new Scanner(System.in);
int n = 0;
		
try {
  String input = s.next();
  n = Integer.parseInt(input);
} catch (NumberFormatException nfe) {
  System.err.println("You entered invalid data!");
  System.exit(1);
}
\end{minted}

In this example, we've simply displayed an error message to
the standard error output and exited the program.  That is, 
we've made the design decision that this error should be fatal.
We could have chosen to handle this error differently in the
\mintinline{java}{catch} block.

The code above could have resulted in other exceptions.  For
example if the \mintinline{java}{Scanner} failed to read the
next token from the standard input, it would have thrown a
\mintinline{java}{NoSuchElementException}.  We can add
as many \mintinline{java}{catch} blocks as we want to handle
each exception differently.

\begin{minted}{java}
Scanner s = new Scanner(System.in);
int n = 0;
		
try {
  String input = s.next();
  n = Integer.parseInt(input);
} catch (NumberFormatException nfe) {
  System.err.println("You entered invalid data!");
  System.exit(1);
} catch (NoSuchElementException nsee) {
  System.err.println("Input reading failed, using default...");
  n = 20; //a default value
} catch(Exception e) {
  System.err.println("A general exception occurred");
  e.printStackTrace();
  System.exit(1);
}
\end{minted}

Each \mintinline{java}{catch} block catches a different 
type of exception.  Thus, the name of the variable
that holds each exception must be different in the chain
of \mintinline{java}{catch} blocks; \mintinline{java}{nfe}, 
\mintinline{java}{nsee}, \mintinline{java}{e}.

Note that the last \mintinline{java}{catch} block was written
to catch a generic \mintinline{java}{Exception}.  This last
block will essentially catch any other type of exception.  
Much like an \mintinline{java}{if-else-if} statement, the
first type of exception that is caught is the block that will
be executed and they are all mutually exclusive.  Thus, a
``catch all' block like this should always be the last 
\mintinline{java}{catch} block.  The most specific types
of exceptions should be caught first and the most general
types should be caught last.

\subsection{Throwing Exceptions}

We can also manually \mintinline{java}{throw} an exception
if we need to.  For example, we can \mintinline{java}{throw}
a generic \mintinline{java}{RuntimeException} using the following.

\begin{minted}{java}
throw new RuntimeException("Something went wrong");
\end{minted}

By using a generic \mintinline{java}{RuntimeException}, we
can only attach a message to the exception (which can be
printed by code that catches the exception).  If we want
more fine-grained control over the type of exceptions, we
need to define our own exceptions.

\subsection{Creating Custom Exceptions}

To create your own exceptions, you need to create a new
class to represent the exception and make it \emph{extends} 
\mintinline{java}{RuntimeException} by using the keyword
\mintinline{java}{extends}.  This makes your exception a 
subclass or \emph{subtype} of the \mintinline{java}{RuntimeException}.
This is a concept known as \emph{inheritance} in \gls{oopLabel}.

Consider the example in the previous chapter of computing
the roots of a quadratic polynomial.  One possible error 
situation is when the roots are complex numbers.  We could
define a new Java class as follows.

\begin{minted}{java}
public class ComplexRootException extends RuntimeException {

  /**
   * Constructor that takes an error message
   */
  public ComplexRootException(String errorMessage) {
    super(errorMessage);
  }
}
\end{minted}

Now in our code we can catch and even throw this new type
of exception.

\begin{minted}{java}
//throw this exception:
if( b*b - 4*a*c < 0) {
  throw new ComplexRootException("Cannot Handle complex roots");
}
\end{minted}

\begin{minted}{java}
try {
  r1 = getComplexRoot01(a, b, c);
} catch(ComplexRootException cre) {
  //handle the exception here...
}
\end{minted}

\subsection{Checked Exceptions}
\label{section:java:checkedExceptions}

A \emph{checked} exception in Java is an exception that \emph{must}
be explicitly caught and handled.  For example, the generic 
\mintinline{java}{Exception} class is a checked exception (others 
include \mintinline{java}{IOException}, \mintinline{java}{SQLException}, 
etc.).  If a checked exception is thrown within a block of code such 
as a method then it must either be caught and handled within that
block of code or the (say) method must be specified to explicitly 
throw the exception.  For example, the method

\begin{minted}{java}
public static void processFile() {
  Scanner s = new Scanner(new File("data.csv"));
}
\end{minted}

would not actually compile because \mintinline{java}{Scanner}
throws a checked \mintinline{java}{FileNotFoundException}.  Either
we would have to explicitly \mintinline{java}{catch} the exception:

\begin{minted}{java}
public static void processFile() {
  Scanner s = null;
  try {
    s = new Scanner(new File("data.csv"));
  } catch (FileNotFoundException e) {
    //handle the exception here
  }
}
\end{minted}

or we would need to specify that the method \mintinline{java}{processFile()}
explicitly \mintinline{java}{throws} the exception:

\begin{minted}{java}
public static void processFile() throws FileNotFoundException {
  Scanner s = new Scanner(new File("data.csv"));
}
\end{minted}

Doing this, however, would force any code that called the 
\mintinline{java}{processFile()} method to surround it in a 
\mintinline{java}{try-catch} block and explicitly handle it (or
once again, throw it back to the calling method).  

The point of a checked exception is to force code to deal with
potential issues that can be reasonably anticipated (such as
the unavailability of a file).  However, from another point of 
view checked exceptions represent the exact opposite goal
of error handling.  Namely, that a function or code block can
and should \emph{inform} the calling function that an error
has occurred, but \emph{not} explicitly make a decision on
how to handle the error.  A checked exception doesn't make
the full decision for the calling function, but it does eliminate 
\emph{ignoring} the error as an option from the calling function.

Java also supports \emph{unchecked} exceptions which do
\emph{not} need to be explicitly caught.  For example, 
\mintinline{java}{NumberFormatException} or \mintinline{java}{NullPointerException}
are unchecked exceptions.  If an unchecked 
exception is thrown and not caught, it bubbles up through the
call stack until some piece of code does catch it.  If no code
catches it, it results in a fatal error and terminates the execution
of the \gls{jvmLabel}.

The \mintinline{java}{RuntimeException} class and any of its
subclasses are unchecked exceptions.  In our  \mintinline{java}{ComplexRootException}
example above, because we extended  \mintinline{java}{RuntimeException}
we made it an unchecked exception, allowing the calling
function to decide not only \emph{how} to handle it, but also
whether or not to handle it \emph{at all}.  If we had instead 
decided to extend \mintinline{java}{Exception} we would
have made our exception a checked exception.

There is considerable debate as to whether or not checked 
exceptions are a good thing (and as to whether or not unchecked
exceptions are a good thing).  Many feel (the author included) that 
checked exceptions were a mistake and their usage should be
avoided.  The rationale behind checked exceptions is summed up
in the following quote from the Java documentation \cite{javaCheckedExceptions}.

\begin{quote}
Here's the bottom line guideline: If a client can reasonably be expected 
to recover from an exception, make it a checked exception. If a client 
cannot do anything to recover from the exception, make it an unchecked 
exception
\end{quote}

The problem is that the \gls{jdkLabel}'s own design violates this principle.
For example, \mintinline{java}{FileNotFound} is a checked exception; the
reasoning being that a program could \emph{reprompt} the user for a different
file.  The problem is the assumption that the program we are writing is 
always interactive.  In fact most software is \emph{not} interactive and 
is instead designed to interact with other software.  Reprompting is \emph{not}
an option in the vast majority of cases.  

As another example, consider Java's SQL library which allows you to 
programmatically connect to an \gls{sqlLabel} database.  Nearly every
method in the \gls{apiLabel} explicitly throws a checked \mintinline{java}{SQLException}.
It stretches the imagination to understand how our code or even a
user would be able to recover (programmatically at least) from 
a lost internet connection or a bad password, etc.  

In general and in the opinion of this author, you should use
unchecked exceptions.

\section{Enumerated Types}

TODO: move this to a miscellaneous section?

Java allows you to define a special class called an 
\gls{enumerated type} which allow you to define a fixed
list of possible values.  For example, the days of the
week or months of the year are possible values used in a date.
However, they are more-or-less fixed (no one will be adding a new
day of the week or month any time soon).  An enumerated 
class allows us to define these values using human-readable 
keywords.

To create an enumerated type class we use the keywords 
\mintinline{java}{enum} (short for enumeration).  Since an
enumerated type is a class, it must follow the same rules.
It must be in a \mintinline{text}{.java} source file of the same
name.  Inside the class we provide a comma-delimited 
list of keywords to define our enumeration.  
Consider the following example.

\begin{minted}{java}
public enum Day {
  SUNDAY,
  MONDAY,
  TUESDAY,
  WEDNESDAY,
  THURSDAY,
  FRIDAY,
  SATURDAY;
}
\end{minted}

In the example, since the name of the enumeration is 
\mintinline{java}{Day} this declaration must be in a source file
named \mintinline{text}{Day.java}.  We can now declare 
variables of this type.  The possible \emph{values} it can 
take are restricted to \mintinline{c}{SUNDAY}, 
\mintinline{c}{MONDAY}, etc. and we can use these keywords 
in our program.  However these values belong to the
class \mintinline{java}{Day} and must be accessed statically.
For example,

\begin{minted}{java}
Day today = Day.MONDAY;

if(today == Day.SUNDAY || today == Day.SATURDAY) {
  System.out.println("Its the weekend!");
}
\end{minted}

Note the naming conventions: the name of the enumerated type
follows the same upper camel casing that all classes use while 
the enumerated values follow an upper case underscore convention.  
Though our example does not contain a value with multiple words, 
if it had, we would have used an underscore to separate them.  

Using an enumerated type has two advantages.  First, it enforces
that only a particular set of predefined values can be used.  You
would not be able to assign a value to a \mintinline{java}{Day} 
variable that was not already defined by the \mintinline{java}{Day}
enumeration.

Second, without an enumerated type we'd be forced
to use a collection of \glspl{magic number} to indicate values.  
Even for something as simple as the days of the week we'd be
constantly trying to remember: which day is Wednesday again?
I forget, do our weeks start with Monday or Sunday?  Etc.  By
using an enumerated type these questions are mostly moot as
we can use the more human-readable keywords and eliminate
the guess work.

\subsection{More Tricks}

Every \mintinline{java}{enum} type has some additional built-in
methods that you can use.  For example, there is a \mintinline{java}{values()}
method that can be called on the \mintinline{java}{enum} that
returns an array of the \mintinline{java}{enum}'s values.  You
can use an enhanced for loop to iterate over them, for example, 

\begin{minted}{java}
for(Day d : Day.values() {
  System.out.println(d.name());
}
\end{minted}

In the example above, we used another feature: each 
\mintinline{java}{enum} value has a \mintinline{java}{name()} 
method that returns the value as a \mintinline{java}{String}.
This example would end up printing the following.

\begin{minted}{text}
SUNDAY
MONDAY
TUESDAY
WEDNESDAY
THURSDAY
FRIDAY
SATURDAY
\end{minted}

Of course, we may want more human-oriented representations.  
To do this we could override the class's \mintinline{java}{toString()}
method to return a better string representation.  For example:

\begin{minted}{java}
public String toString() {
  if(this == SUNDAY) {
    return "Sunday";
  } else if(this == MONDAY) {
    return "Monday";
  }
  ...
  } else {
    return "Saturday";
  }
}
\end{minted}

Because \mintinline{java}{enum} types are full classes in Java, 
many more tricks can be used that leverage the power of classes
including using additional state and constructors.  We will cover
these topics later.







