%\documentclass[12pt]{scrbook}
%
%\usepackage{tikz}
%\usepackage{minted}
%\usetikzlibrary{decorations.pathreplacing,arrows}
%\usetikzlibrary{arrows,decorations.pathmorphing,backgrounds,positioning,fit,petri}
%
%\usepackage{fullpage}
%\usepackage{subfigure}
%\begin{document}
%
%
%Lorem Ipsum is simply dummy text of the printing and typesetting industry. Lorem Ipsum has been the industry's standard dummy text ever since the 1500s, when an unknown printer took a galley of type and scrambled it to make a type specimen book. It has survived not only five centuries, but also the leap into electronic typesetting, remaining essentially unchanged. It was popularised in the 1960s with the release of Letraset sheets containing Lorem Ipsum passages, and more recently with desktop publishing software like Aldus PageMaker including versions of Lorem Ipsum.
%




\begin{figure}
\centering

\subfigure[First iteration.  We insert 4 in front of 42, requiring 1 
comparison.]{

\begin{tikzpicture}[scale=.65,transform shape]

\tikzset{>=stealth',shorten <=.2cm,>=stealth',shorten >=.2cm}
% size of each node
\def\sz{9mm}
% node style definition
\tikzstyle{block} = [
	draw, fill=black!10, rectangle,
	minimum height=\sz, minimum width=\sz ];
\tikzstyle{plain} = [draw=none,fill=none];

%\node[plain] at (-1.75, 1) { index };
%\node[plain] at (0*\sz,1.0) { 0 };
%\node[plain] at (1*\sz,1.0) { 1 };
%\node[plain] at (2*\sz,1.0) { 2 };
%\node[plain] at (3*\sz,1.0) { 3 };
%\node[plain] at (4*\sz,1.0) { 4 };
%\node[plain] at (5*\sz,1.0) { 5 };
%\node[plain] at (6*\sz,1.0) { 6 };
%\node[plain] at (7*\sz,1.0) { 7 };
%\node[plain] at (8*\sz,1.0) { 8 };
%\node[plain] at (-1.75, 0) { contents };
%

\node[block,fill=black!50] (a0) at (0*\sz,0) { 42 };
\node[block,fill=green!30] (a1) at (1*\sz,0) { 4 };
\node[block] (a2) at (2*\sz,0) { 9 };
\node[block] (a3) at (3*\sz,0) { 4 };
\node[block] (a4) at (4*\sz,0) { 102 };
\node[block] (a5) at (5*\sz,0) { 34 };
\node[block] (a6) at (6*\sz,0) { 12 };
\node[block] (a7) at (7*\sz,0) { 2 };
\node[block] (a8) at (8*\sz,0) { 0 };

\draw[<-] (a0.north) to [in=90,out=90,looseness=3] node[pos=.5,above] {} (a1.north);

\end{tikzpicture}~~~~~\begin{tikzpicture}[scale=.65,transform shape]

\tikzset{>=stealth',shorten <=.2cm,>=stealth',shorten >=.2cm}
\def\sz{9mm}
\tikzstyle{block} = [
	draw, fill=black!10, rectangle,
	minimum height=\sz, minimum width=\sz ];
\tikzstyle{plain} = [draw=none,fill=none];

\node[block,fill=black!50] (a0) at (0*\sz,0) { 4 };
\node[block,fill=black!50] (a1) at (1*\sz,0) { 42 };
\node[block] (a2) at (2*\sz,0) { 9 };
\node[block] (a3) at (3*\sz,0) { 4 };
\node[block] (a4) at (4*\sz,0) { 102 };
\node[block] (a5) at (5*\sz,0) { 34 };
\node[block] (a6) at (6*\sz,0) { 12 };
\node[block] (a7) at (7*\sz,0) { 2 };
\node[block] (a8) at (8*\sz,0) { 0 };

\end{tikzpicture}

}

\subfigure[Second iteration.  The first two elements are sorted, we insert
9 by making two comparisons: to find that it is less than 42, bug greater
than 4.  At the end of the iteration, the first three elements are
sorted.]{

\begin{tikzpicture}[scale=.65,transform shape]

\tikzset{>=stealth',shorten <=.2cm,>=stealth',shorten >=.2cm}
% size of each node
\def\sz{9mm}
% node style definition
\tikzstyle{block} = [
	draw, fill=black!10, rectangle,
	minimum height=\sz, minimum width=\sz ];
\tikzstyle{plain} = [draw=none,fill=none];

%\node[plain] at (-1.75, 1) { index };
%\node[plain] at (0*\sz,1.0) { 0 };
%\node[plain] at (1*\sz,1.0) { 1 };
%\node[plain] at (2*\sz,1.0) { 2 };
%\node[plain] at (3*\sz,1.0) { 3 };
%\node[plain] at (4*\sz,1.0) { 4 };
%\node[plain] at (5*\sz,1.0) { 5 };
%\node[plain] at (6*\sz,1.0) { 6 };
%\node[plain] at (7*\sz,1.0) { 7 };
%\node[plain] at (8*\sz,1.0) { 8 };
%\node[plain] at (-1.75, 0) { contents };
%

\node[block,fill=black!50] (a0) at (0*\sz,0) { 4 };
\node[block,fill=black!50] (a1) at (1*\sz,0) { 42 };
\node[block,fill=green!30] (a2) at (2*\sz,0) { 9 };
\node[block] (a3) at (3*\sz,0) { 4 };
\node[block] (a4) at (4*\sz,0) { 102 };
\node[block] (a5) at (5*\sz,0) { 34 };
\node[block] (a6) at (6*\sz,0) { 12 };
\node[block] (a7) at (7*\sz,0) { 2 };
\node[block] (a8) at (8*\sz,0) { 0 };

\draw[<-,dotted] (a0.north) to [in=90,out=90,looseness=3] node[pos=.5,above] {2}(a2.north);
\draw[<-] (a1.north) to [in=90,out=90,looseness=3] node[pos=.5,above] {1} (a2.north);

\end{tikzpicture}~~~~~\begin{tikzpicture}[scale=.65,transform shape]

\tikzset{>=stealth',shorten <=.2cm,>=stealth',shorten >=.2cm}
\def\sz{9mm}
\tikzstyle{block} = [
	draw, fill=black!10, rectangle,
	minimum height=\sz, minimum width=\sz ];
\tikzstyle{plain} = [draw=none,fill=none];

\node[block,fill=black!50] (a0) at (0*\sz,0) { 4 };
\node[block,fill=black!50] (a1) at (1*\sz,0) { 9 };
\node[block,fill=black!50] (a2) at (2*\sz,0) { 42 };
\node[block] (a3) at (3*\sz,0) { 4 };
\node[block] (a4) at (4*\sz,0) { 102 };
\node[block] (a5) at (5*\sz,0) { 34 };
\node[block] (a6) at (6*\sz,0) { 12 };
\node[block] (a7) at (7*\sz,0) { 2 };
\node[block] (a8) at (8*\sz,0) { 0 };

\end{tikzpicture}

}

\subfigure[Third iteration.  The first three elements are sorted, we
insert the second four by making 3 comparisons.]{

\begin{tikzpicture}[scale=.65,transform shape]

\tikzset{>=stealth',shorten <=.2cm,>=stealth',shorten >=.2cm}
% size of each node
\def\sz{9mm}
% node style definition
\tikzstyle{block} = [
	draw, fill=black!10, rectangle,
	minimum height=\sz, minimum width=\sz ];
\tikzstyle{plain} = [draw=none,fill=none];

%\node[plain] at (-1.75, 1) { index };
%\node[plain] at (0*\sz,1.0) { 0 };
%\node[plain] at (1*\sz,1.0) { 1 };
%\node[plain] at (2*\sz,1.0) { 2 };
%\node[plain] at (3*\sz,1.0) { 3 };
%\node[plain] at (4*\sz,1.0) { 4 };
%\node[plain] at (5*\sz,1.0) { 5 };
%\node[plain] at (6*\sz,1.0) { 6 };
%\node[plain] at (7*\sz,1.0) { 7 };
%\node[plain] at (8*\sz,1.0) { 8 };
%\node[plain] at (-1.75, 0) { contents };
%

\node[block,fill=black!50] (a0) at (0*\sz,0) { 4 };
\node[block,fill=black!50] (a1) at (1*\sz,0) { 9 };
\node[block,fill=black!50] (a2) at (2*\sz,0) { 42 };
\node[block,fill=green!50] (a3) at (3*\sz,0) { 4 };
\node[block] (a4) at (4*\sz,0) { 102 };
\node[block] (a5) at (5*\sz,0) { 34 };
\node[block] (a6) at (6*\sz,0) { 12 };
\node[block] (a7) at (7*\sz,0) { 2 };
\node[block] (a8) at (8*\sz,0) { 0 };

\draw[<-,dotted] (a0.north) to [in=90,out=90,looseness=3] node[pos=.5,above] {3}(a3.north);
\draw[<-] (a1.north) to [in=90,out=90,looseness=3] node[pos=.5,above] {2}(a3.north);
\draw[<-] (a2.north) to [in=90,out=90,looseness=3] node[pos=.5,above] {1} (a3.north);

\end{tikzpicture}~~~~~\begin{tikzpicture}[scale=.65,transform shape]

\tikzset{>=stealth',shorten <=.2cm,>=stealth',shorten >=.2cm}
\def\sz{9mm}
\tikzstyle{block} = [
	draw, fill=black!10, rectangle,
	minimum height=\sz, minimum width=\sz ];
\tikzstyle{plain} = [draw=none,fill=none];

\node[block,fill=black!50] (a0) at (0*\sz,0) { 4 };
\node[block,fill=black!50] (a1) at (1*\sz,0) { 4 };
\node[block,fill=black!50] (a2) at (2*\sz,0) { 9 };
\node[block,fill=black!50] (a3) at (3*\sz,0) { 42 };
\node[block] (a4) at (4*\sz,0) { 102 };
\node[block] (a5) at (5*\sz,0) { 34 };
\node[block] (a6) at (6*\sz,0) { 12 };
\node[block] (a7) at (7*\sz,0) { 2 };
\node[block] (a8) at (8*\sz,0) { 0 };

\end{tikzpicture}

}

\subfigure[Fourth iteration.  Here, only one comparison is necessary
to find that 102 is already where it needs to be.]{

\begin{tikzpicture}[scale=.65,transform shape]

\tikzset{>=stealth',shorten <=.2cm,>=stealth',shorten >=.2cm}
% size of each node
\def\sz{9mm}
% node style definition
\tikzstyle{block} = [
	draw, fill=black!10, rectangle,
	minimum height=\sz, minimum width=\sz ];
\tikzstyle{plain} = [draw=none,fill=none];

%\node[plain] at (-1.75, 1) { index };
%\node[plain] at (0*\sz,1.0) { 0 };
%\node[plain] at (1*\sz,1.0) { 1 };
%\node[plain] at (2*\sz,1.0) { 2 };
%\node[plain] at (3*\sz,1.0) { 3 };
%\node[plain] at (4*\sz,1.0) { 4 };
%\node[plain] at (5*\sz,1.0) { 5 };
%\node[plain] at (6*\sz,1.0) { 6 };
%\node[plain] at (7*\sz,1.0) { 7 };
%\node[plain] at (8*\sz,1.0) { 8 };
%\node[plain] at (-1.75, 0) { contents };
%

\node[block,fill=black!50] (a0) at (0*\sz,0) { 4 };
\node[block,fill=black!50] (a1) at (1*\sz,0) { 4 };
\node[block,fill=black!50] (a2) at (2*\sz,0) { 9 };
\node[block,fill=black!50] (a3) at (3*\sz,0) { 42 };
\node[block,fill=green!50] (a4) at (4*\sz,0) { 102 };
\node[block] (a5) at (5*\sz,0) { 34 };
\node[block] (a6) at (6*\sz,0) { 12 };
\node[block] (a7) at (7*\sz,0) { 2 };
\node[block] (a8) at (8*\sz,0) { 0 };

\draw[<-] (a3.north) to [in=90,out=90,looseness=3] node[pos=.5,above] {1} (a4.north);

\end{tikzpicture}~~~~~\begin{tikzpicture}[scale=.65,transform shape]

\tikzset{>=stealth',shorten <=.2cm,>=stealth',shorten >=.2cm}
\def\sz{9mm}
\tikzstyle{block} = [
	draw, fill=black!10, rectangle,
	minimum height=\sz, minimum width=\sz ];
\tikzstyle{plain} = [draw=none,fill=none];

\node[block,fill=black!50] (a0) at (0*\sz,0) { 4 };
\node[block,fill=black!50] (a1) at (1*\sz,0) { 4 };
\node[block,fill=black!50] (a2) at (2*\sz,0) { 9 };
\node[block,fill=black!50] (a3) at (3*\sz,0) { 42 };
\node[block,fill=black!50] (a4) at (4*\sz,0) { 102 };
\node[block] (a5) at (5*\sz,0) { 34 };
\node[block] (a6) at (6*\sz,0) { 12 };
\node[block] (a7) at (7*\sz,0) { 2 };
\node[block] (a8) at (8*\sz,0) { 0 };

\end{tikzpicture}

}

\subfigure[Fifth iteration.  Here, 3 comparisons are necessary to insert
34 between 9 and 42.]{

\begin{tikzpicture}[scale=.65,transform shape]

\tikzset{>=stealth',shorten <=.2cm,>=stealth',shorten >=.2cm}
% size of each node
\def\sz{9mm}
% node style definition
\tikzstyle{block} = [
	draw, fill=black!10, rectangle,
	minimum height=\sz, minimum width=\sz ];
\tikzstyle{plain} = [draw=none,fill=none];

%\node[plain] at (-1.75, 1) { index };
%\node[plain] at (0*\sz,1.0) { 0 };
%\node[plain] at (1*\sz,1.0) { 1 };
%\node[plain] at (2*\sz,1.0) { 2 };
%\node[plain] at (3*\sz,1.0) { 3 };
%\node[plain] at (4*\sz,1.0) { 4 };
%\node[plain] at (5*\sz,1.0) { 5 };
%\node[plain] at (6*\sz,1.0) { 6 };
%\node[plain] at (7*\sz,1.0) { 7 };
%\node[plain] at (8*\sz,1.0) { 8 };
%\node[plain] at (-1.75, 0) { contents };
%

\node[block,fill=black!50] (a0) at (0*\sz,0) { 4 };
\node[block,fill=black!50] (a1) at (1*\sz,0) { 4 };
\node[block,fill=black!50] (a2) at (2*\sz,0) { 9 };
\node[block,fill=black!50] (a3) at (3*\sz,0) { 42 };
\node[block,fill=black!50] (a4) at (4*\sz,0) { 102 };
\node[block,fill=green!50] (a5) at (5*\sz,0) { 34 };
\node[block] (a6) at (6*\sz,0) { 12 };
\node[block] (a7) at (7*\sz,0) { 2 };
\node[block] (a8) at (8*\sz,0) { 0 };

\draw[<-] (a4.north) to [in=90,out=90,looseness=2] node[pos=.5,above] {1} (a5.north);
\draw[<-] (a3.north) to [in=90,out=90,looseness=2] node[pos=.5,above] {2} (a5.north);
\draw[<-,dotted] (a2.north) to [in=90,out=90,looseness=2] node[pos=.5,above] {3} (a5.north);

\end{tikzpicture}~~~~~\begin{tikzpicture}[scale=.65,transform shape]

\tikzset{>=stealth',shorten <=.2cm,>=stealth',shorten >=.2cm}
\def\sz{9mm}
\tikzstyle{block} = [
	draw, fill=black!10, rectangle,
	minimum height=\sz, minimum width=\sz ];
\tikzstyle{plain} = [draw=none,fill=none];

\node[block,fill=black!50] (a0) at (0*\sz,0) { 4 };
\node[block,fill=black!50] (a1) at (1*\sz,0) { 4 };
\node[block,fill=black!50] (a2) at (2*\sz,0) { 9 };
\node[block,fill=black!50] (a3) at (3*\sz,0) { 34 };
\node[block,fill=black!50] (a4) at (4*\sz,0) { 42 };
\node[block,fill=black!50] (a5) at (5*\sz,0) { 102 };
\node[block] (a6) at (6*\sz,0) { 12 };
\node[block] (a7) at (7*\sz,0) { 2 };
\node[block] (a8) at (8*\sz,0) { 0 };

\end{tikzpicture}

}


\subfigure[Sixth iteration.  Here, 4 comparisons are necessary to insert
12 between 9 and 34.]{

\begin{tikzpicture}[scale=.65,transform shape]

\tikzset{>=stealth',shorten <=.2cm,>=stealth',shorten >=.2cm}
% size of each node
\def\sz{9mm}
% node style definition
\tikzstyle{block} = [
	draw, fill=black!10, rectangle,
	minimum height=\sz, minimum width=\sz ];
\tikzstyle{plain} = [draw=none,fill=none];

%\node[plain] at (-1.75, 1) { index };
%\node[plain] at (0*\sz,1.0) { 0 };
%\node[plain] at (1*\sz,1.0) { 1 };
%\node[plain] at (2*\sz,1.0) { 2 };
%\node[plain] at (3*\sz,1.0) { 3 };
%\node[plain] at (4*\sz,1.0) { 4 };
%\node[plain] at (5*\sz,1.0) { 5 };
%\node[plain] at (6*\sz,1.0) { 6 };
%\node[plain] at (7*\sz,1.0) { 7 };
%\node[plain] at (8*\sz,1.0) { 8 };
%\node[plain] at (-1.75, 0) { contents };
%

\node[block,fill=black!50] (a0) at (0*\sz,0) { 4 };
\node[block,fill=black!50] (a1) at (1*\sz,0) { 4 };
\node[block,fill=black!50] (a2) at (2*\sz,0) { 9 };
\node[block,fill=black!50] (a3) at (3*\sz,0) { 34 };
\node[block,fill=black!50] (a4) at (4*\sz,0) { 42 };
\node[block,fill=black!50] (a5) at (5*\sz,0) { 102 };
\node[block,fill=green!50] (a6) at (6*\sz,0) { 12 };
\node[block] (a7) at (7*\sz,0) { 2 };
\node[block] (a8) at (8*\sz,0) { 0 };

\draw[<-] (a5.north) to [in=90,out=90,looseness=1.5] node[pos=.5,above] {1} (a6.north);
\draw[<-] (a4.north) to [in=90,out=90,looseness=1] node[pos=.5,above] {2} (a6.north);
\draw[<-] (a3.north) to [in=90,out=90,looseness=1] node[pos=.5,above] {3} (a6.north);
\draw[<-,dotted] (a2.north) to [in=90,out=90,looseness=1] node[pos=.5,above] {4} (a6.north);

\end{tikzpicture}~~~~~\begin{tikzpicture}[scale=.65,transform shape]

\tikzset{>=stealth',shorten <=.2cm,>=stealth',shorten >=.2cm}
\def\sz{9mm}
\tikzstyle{block} = [
	draw, fill=black!10, rectangle,
	minimum height=\sz, minimum width=\sz ];
\tikzstyle{plain} = [draw=none,fill=none];

\node[block,fill=black!50] (a0) at (0*\sz,0) { 4 };
\node[block,fill=black!50] (a1) at (1*\sz,0) { 4 };
\node[block,fill=black!50] (a2) at (2*\sz,0) { 9 };
\node[block,fill=black!50] (a3) at (3*\sz,0) { 12 };
\node[block,fill=black!50] (a4) at (4*\sz,0) { 34 };
\node[block,fill=black!50] (a5) at (5*\sz,0) { 42 };
\node[block,fill=black!50] (a6) at (6*\sz,0) { 102 };
\node[block] (a7) at (7*\sz,0) { 2 };
\node[block] (a8) at (8*\sz,0) { 0 };

\end{tikzpicture}

}


\subfigure[Seventh iteration.  Here, 4 comparisons are necessary to insert
12 between 9 and 34.]{

\begin{tikzpicture}[scale=.65,transform shape]

\tikzset{>=stealth',shorten <=.2cm,>=stealth',shorten >=.2cm}
% size of each node
\def\sz{9mm}
% node style definition
\tikzstyle{block} = [
	draw, fill=black!10, rectangle,
	minimum height=\sz, minimum width=\sz ];
\tikzstyle{plain} = [draw=none,fill=none];

%\node[plain] at (-1.75, 1) { index };
%\node[plain] at (0*\sz,1.0) { 0 };
%\node[plain] at (1*\sz,1.0) { 1 };
%\node[plain] at (2*\sz,1.0) { 2 };
%\node[plain] at (3*\sz,1.0) { 3 };
%\node[plain] at (4*\sz,1.0) { 4 };
%\node[plain] at (5*\sz,1.0) { 5 };
%\node[plain] at (6*\sz,1.0) { 6 };
%\node[plain] at (7*\sz,1.0) { 7 };
%\node[plain] at (8*\sz,1.0) { 8 };
%\node[plain] at (-1.75, 0) { contents };
%

\node[block,fill=black!50] (a0) at (0*\sz,0) { 4 };
\node[block,fill=black!50] (a1) at (1*\sz,0) { 4 };
\node[block,fill=black!50] (a2) at (2*\sz,0) { 9 };
\node[block,fill=black!50] (a3) at (3*\sz,0) { 12 };
\node[block,fill=black!50] (a4) at (4*\sz,0) { 34 };
\node[block,fill=black!50] (a5) at (5*\sz,0) { 42 };
\node[block,fill=black!50] (a6) at (6*\sz,0) { 102 };
\node[block,fill=green!50] (a7) at (7*\sz,0) { 2 };
\node[block] (a8) at (8*\sz,0) { 0 };

\draw[<-] (a6.north) to [in=90,out=90,looseness=1.5] node[pos=.5,above] {1} (a7.north);
\draw[<-] (a5.north) to [in=90,out=90,looseness=1] node[pos=.5,above] {2} (a7.north);
\draw[<-] (a4.north) to [in=90,out=90,looseness=1] node[pos=.5,above] {3} (a7.north);
\draw[<-] (a3.north) to [in=90,out=90,looseness=1] node[pos=.5,above] {4} (a7.north);
\draw[<-] (a2.north) to [in=90,out=90,looseness=1] node[pos=.5,above] {5} (a7.north);
\draw[<-] (a1.north) to [in=90,out=90,looseness=1] node[pos=.5,above] {6} (a7.north);
\draw[<-,dotted] (a0.north) to [in=90,out=90,looseness=1] node[pos=.5,above] {7} (a7.north);

\end{tikzpicture}~~~~~\begin{tikzpicture}[scale=.65,transform shape]

\tikzset{>=stealth',shorten <=.2cm,>=stealth',shorten >=.2cm}
\def\sz{9mm}
\tikzstyle{block} = [
	draw, fill=black!10, rectangle,
	minimum height=\sz, minimum width=\sz ];
\tikzstyle{plain} = [draw=none,fill=none];

\node[block,fill=black!50] (a0) at (0*\sz,0) { 2 };
\node[block,fill=black!50] (a1) at (1*\sz,0) { 4 };
\node[block,fill=black!50] (a2) at (2*\sz,0) { 4 };
\node[block,fill=black!50] (a3) at (3*\sz,0) { 9 };
\node[block,fill=black!50] (a4) at (4*\sz,0) { 12 };
\node[block,fill=black!50] (a5) at (5*\sz,0) { 34 };
\node[block,fill=black!50] (a6) at (6*\sz,0) { 42 };
\node[block,fill=black!50] (a7) at (7*\sz,0) { 102 };
\node[block] (a8) at (8*\sz,0) { 0 };

\end{tikzpicture}

}

\caption[Insertion Sort Example]{Example execution of Insertion Sort.
Each iteration depicts the comparisons to previous elements; the last
comparison is dashed indicating a comparison was made, but not a swap.
The final iteration is omitted for space, but would require 8 comparisons
to insert 0 at the front of the collection.}
\label{figure:insertionSortExample}

\end{figure}


%\end{document}

