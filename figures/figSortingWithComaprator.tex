%\documentclass{article}
%\usepackage{fullpage}
%\usepackage{subfigure}
%\usepackage{tikz}
%\usetikzlibrary{arrows,arrows.meta,backgrounds,calc,trees,decorations,decorations.pathmorphing}
%
%\begin{document}

\begin{figure}
\centering

\begin{tikzpicture}[scale=1.0,transform shape,
fringeedge/.style={green!50!yellow!50!black,thick},
edge/.style={->,thick},
place/.style={rectangle,draw=black,fill=red!20,inner sep=6pt,minimum size=6mm},
transition/.style={rectangle,draw=black!50,fill=black!20,thick,inner sep=0pt,minimum size=4mm}] 

\draw[fill=blue!5,thick,dotted] (0,0) rectangle (3,3);
\node[] at (1.5,1.5) {
\begin{tabular}{c}
Sorting \\
Algorithm
\end{tabular}
};

\node (INPUT) at (-2,1.5) [place] {array};
\draw[-{Latex[length=2mm]}] (INPUT) -- node[pos=.5,below] {Input} (0, 1.5);

\node (OUTPUT) at (6,1.5) [place] {sorted array};
\draw[-{Latex[length=2mm]}] (3,1.5) -- node[pos=.5,below] {Output} (OUTPUT);

\draw[fill=green!5,thick,dotted] (.25,-3.5) rectangle (2.75,-1.5);
\node[] at (1.5,-2.5) {Comparator};


\draw[-{Latex[length=2mm]}] (1,0) -- node[left,pos=.5] {$a, b$} (1,-1.5);

\draw[{Latex[length=2mm]}-] (2,0) -- node[right,pos=.5] {
$\textrm{order} = \left\{
\begin{array}{rl}
< 0 & \textrm{if } a < b \\
0 & \textrm{if } a = b \\
> 0 & \textrm{if } a > b
\end{array}\right.$} (2,-1.5);

\end{tikzpicture}

\caption[Generalized Sorting with a Comparator]{Generalized Sorting with a Comparator.  A sorting algorithm doesn't need to know \emph{what} it is sorting or how they are \emph{ordered} as long as it has access to a \emph{comparator} that \emph{does} know how to order elements.  By using a comparator, the sorting function can be kept general and generic so that one implementation can be used for any type of data.}
\label{fig:sortingAlgorithmWithComparator}
\end{figure}





%\end{document}