

\begin{figure}
\centering

{
\setmintedinline{bgcolor={}}

\begin{tikzpicture}
% size of each node
\def\sz{10mm}
% node style definition
\tikzstyle{block} = [
	draw, fill=black!10, rectangle,
	minimum height=\sz, minimum width=\sz ];
\tikzstyle{plain} = [draw=none,fill=none];
% array element definition
\def\arr{\mintinline[bgcolor={}]{text}{H}, 
         \mintinline{text}{e}, 
         \mintinline{text}{l}, 
         \mintinline{text}{l}, 
         \mintinline{text}{o}, 
         \mintinline{text}{\0},
         \mintinline{text}{?}, 
         \mintinline{text}{?}, 
         \mintinline{text}{?}, 
         \mintinline{text}{?}
         };
%\def\x{0}; % x pos of arr
%\def\y{0}; % y pos of arr
%\newcounter{ind};
\setcounter{ind}{0};
\node[plain] at (-1.75, 1) { index };
\node[plain] (a) at (-2.75, 0) { \mintinline{c}{char *s} };
\foreach \item in \arr
{
	\node[block] (a\theind) at (\theind*\sz,0) { \item };
	\node[plain] at (\theind*\sz,1.0) { \theind };
	\addtocounter{ind}{1};
}
\draw[->,>=stealth'] (a) -- (a0);
\end{tikzpicture}
}

\caption[Example of a character array (string) in C.]{A string in
C is achieved by using a \mintinline{c}{char} array.  However, the
string is terminated by a null-terminating character, \mintinline{text}{\0}.
Though an array may have space for additional characters, they are
irrelevant if the null terminator precedes them.}
\label{figure:cStringExample}
\end{figure}
